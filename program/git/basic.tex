\documentclass[12pt,a4paper]{article}
%\usepackage{fontspec, xunicode, xltxtra}  
%\setmainfont{Hiragino Sans GB}  
\usepackage{xeCJK}
%\setCJKmainfont[BoldFont=STZhongsong, ItalicFont=STKaiti]{STSong}
%\setCJKsansfont[BoldFont=STHeiti]{STXihei}
%\setCJKmonofont{STFangsong}

%使用Xelatex编译

% 设置页面
%==================================================
\linespread{2} %行距
% \usepackage[top=1in,bottom=1in,left=1.25in,right=1.25in]{geometry}
% \headsep=2cm
% \textwidth=16cm \textheight=24.2cm
%==================================================

% 其它需要使用的宏包
%==================================================
\usepackage[colorlinks,linkcolor=blue,anchorcolor=red,citecolor=green,urlcolor=blue]{hyperref} 
\usepackage{tabularx}
\usepackage{authblk}         % 作者信息
\usepackage{algorithm}     % 算法排版
\usepackage{amsmath}     % 数学符号与公式
\usepackage{amsfonts}     % 数学符号与字体
\usepackage{mathrsfs}      % 花体
\usepackage{amssymb}
\usepackage[framemethod=TikZ]{mdframed}

\usepackage{graphicx} 
\usepackage{graphics}
\usepackage{color}
\usepackage{xcolor}
\usepackage{tcolorbox}
\usepackage{lipsum}
\usepackage{empheq}

\usepackage{fancyhdr}       % 设置页眉页脚
\usepackage{fancyvrb}       % 抄录环境
\usepackage{float}              % 管理浮动体
\usepackage{geometry}     % 定制页面格式
\usepackage{hyperref}       % 为PDF文档创建超链接
\usepackage{lineno}          % 生成行号
\usepackage{listings}        % 插入程序源代码
\usepackage{multicol}       % 多栏排版
%\usepackage{natbib}         % 管理文献引用
\usepackage{rotating}       % 旋转文字,图形,表格
\usepackage{subfigure}    % 排版子图形
\usepackage{titlesec}       % 改变章节标题格式
\usepackage{moresize}   % 更多字体大小
\usepackage{anysize}
\usepackage{indentfirst}  % 首段缩进
\usepackage{booktabs}   % 使用\multicolumn
\usepackage{multirow}    % 使用\multirow

\usepackage{wrapfig}
\usepackage{titlesec}     % 改变标题样式
\usepackage{enumitem}
\usepackage{aas_macros}

\newcommand{\myvec}[1]%
   {\stackrel{\raisebox{-2pt}[0pt][0pt]{\small$\rightharpoonup$}}{#1}}  %矢量符号
\renewcommand{\vec}[1]{\boldsymbol{#1}}
\newcommand{\me}{\mathrm{e}}
\newcommand{\mi}{\mathrm{i}}
\newcommand{\dif}{\mathrm{d}}
\newcommand{\tabincell}[2]{\begin{tabular}{@{}#1@{}}#2\end{tabular}}

\def\kpc{{\rm kpc}}
\def\km{{\rm km}}
\def\cm{{\rm cm}}
\def\TeV{{\rm TeV}}
\def\GeV{{\rm GeV}}
\def\MeV{{\rm MeV}}
\def\GV{{\rm GV}}
\def\MV{{\rm MV}}
\def\yr{{\rm yr}}
\def\s{{\rm s}}
\def\ns{{\rm ns}}
\def\GHz{{\rm GHz}}
\def\muGs{{\rm \mu Gs}}
\def\arcsec{{\rm arcsec}}
\def\K{{\rm K}}
\def\microK{\mu{\rm K}}
\def\sr{{\rm sr}}
\newcolumntype{p}{D{,}{\pm}{-1}}

\renewcommand{\figurename}{Fig.}
\renewcommand{\tablename}{Tab.}

\renewcommand{\arraystretch}{1.5}

\setlength{\parindent}{0pt}  %取消每段开头的空格

\title{Basic}
\author{}
\date{\today}
\begin{document}

\maketitle

\section{Preparation}

\subsection{初始设置}
对本地计算机里安装的Git进行设置

设置使用 Git 时的姓名和邮箱地址
\begin{tcolorbox}[colback=green!5,colframe=green!40!black,title= ]
$\$$ \textcolor{red}{git config $--$global user.name ``Firstname Lastname"} \\
$\$$ \textcolor{red}{git config $--$global user.email ``$\rm your\_email@example.com$"}
\end{tcolorbox}
这个命令,会在``$\rm \sim/.gitconfig$”中以如下形式输出设置文件
\begin{tcolorbox}[colback=green!5,colframe=green!40!black,title= ]
[user] \\
name = Firstname Lastname \\
email = $\rm your\_email@example.com$
\end{tcolorbox}
想更改这些信息时,可以直接编辑这个设置文件。这里设置的姓名和邮箱地址会用在 Git 的提交日志中。由于在 GitHub 上公开仓库时,这里的姓名和邮箱地址也会随着提交日志一同被公开,所以请不要使用不便公开的隐私信息。

将 color.ui 设置为 auto 可以让命令的输出拥有更高的可读性
\begin{tcolorbox}[colback=green!5,colframe=green!40!black,title= ]
$\$$ git config $--$global color.ui auto 
\end{tcolorbox}
``$\rm \sim/.gitconfig$”中会增加下面一行。
\begin{tcolorbox}[colback=green!5,colframe=green!40!black,title= ]
[color] \\
ui = auto
\end{tcolorbox}

\subsection{设置 SSH Key}
GitHub 上连接已有仓库时的认证,是通过使用了 SSH 的公开密钥认证方式进行的。创建公开密钥认证所需的 SSH Key,并将其添加至 GitHub。运行下面的命令创建 SSH Key
\begin{tcolorbox}[colback=green!5,colframe=green!40!black,title= ]
$\$$ ssh-keygen -t rsa -C ``$\rm your\_email@example.com$" \\
Generating public/private rsa key pair. \\
Enter file in which to save the key \\
$\rm (/Users/your\_user\_directory/.ssh/id\_rsa)$: 按回车键 \\
Enter passphrase (empty for no passphrase): 输入密码 \\
Enter same passphrase again: 再次输入密码
\end{tcolorbox}
``$\rm your\_email@example.com$”的部分请改成您在创建账户时用的邮箱地址。密码需要在认证时输入,请选择复杂度高并且容易记忆的组合。输入密码后会出现以下结果
\begin{tcolorbox}[colback=green!5,colframe=green!40!black,title= ]
Your identification has been saved in $\rm /Users/your\_user\_directory/.ssh/id\_rsa$. \\
Your public key has been saved in $\rm /Users/your\_user\_directory/.ssh/id\_rsa.pub$. \\
The key fingerprint is: \\
fingerprint值 $\rm your\_email@example.com$ \\
The key's randomart image is:
\end{tcolorbox}
$\rm id\_rsa$ 文件是私有密钥,$\rm id\_rsa.pub$ 是公开密钥。

在 GitHub 中添加公开密钥,今后就可以用私有密钥进行认证了。点击右上角的账户设定按钮( Account Settings),选择 SSH Keys 菜单。点击 Add SSH Key 之后,会出现输入框。在 Title 中输入适当的密钥名称。 Key 部分请粘贴 $\rm id\_rsa.pub$ 文件里的内容。 $\rm id\_rsa.pub$的内容可以用如下方法查看
\begin{tcolorbox}[colback=green!5,colframe=green!40!black,title= ]
$\$$ cat $\rm \sim/.ssh/id\_rsa.pub$ \\
ssh-rsa 公开密钥的内容 $\rm your\_email@example.com$
\end{tcolorbox}
添加成功之后,创建账户时所用的邮箱会接到一封提示“公共密钥添加完成”的邮件。完成以上设置后,就可以用手中的私人密钥与 GitHub 进行认证和通信了。
\begin{tcolorbox}[colback=green!5,colframe=green!40!black,title= ]
$\$$ ssh -T git@github.com \\
The authenticity of host 'github.com (207.97.227.239)' can't be established. \\
RSA key fingerprint is fingerprint值 . \\
Are you sure you want to continue connecting (yes/no)? 输入yes 
\end{tcolorbox}
出现如下结果即为成功 
\begin{tcolorbox}[colback=green!5,colframe=green!40!black,title= ]
Hi hirocastest! You've successfully authenticated, but GitHub does not provide shell access.
\end{tcolorbox}
创建一个公开的仓库。点击右上角工具栏里的 New repository 图标,创建新的仓库。

在 Initialize this repository with a README 选项上打钩,随后 GitHub 会自动初始化仓库并设置 README 文件,让用户可以立刻clone 这个仓库。如果想向 GitHub 添加手中已有的 Git 仓库,建议不要勾选,直接手动 push。

下方左侧的下拉菜单非常方便,通过它可以在\textcolor{yellow}{初始化时自动生成 .gitignore 文件\footnote{该文件用来描述 Git 仓库中不需管理的文件与目录。}}。这个设定会帮我们把\textcolor{yellow}{不需要在 Git 仓库中进行版本管理的文件记录在 .gitignore 文件中},省去了每次根据框架进行设置的麻烦。下拉菜单中包含了主要的语言及框架,选择今后将要使用的即可。

右侧的下拉菜单可以选择要添加的许可协议文件。如果这个仓库中包含的代码已经确定了许可协议,那么请在这里进行选择。随后将自动生成包含许可协议内容的 LICENSE 文件,用来表明该仓库内容的许可协议。

输入选择都完成后,点击 Create repository 按钮,完成仓库的创建。

下面这个 URL 便是刚刚创建的仓库的页面
\begin{tcolorbox}[colback=green!5,colframe=green!40!black,title= ]
https://github.com/用户名/Hello-Word
\end{tcolorbox}

README.md 在初始化时已经生成好了。README.md 文件的内容会自动显示在仓库的首页当中。因此,人们一般会在这个文件中标明本仓库所包含的软件的概要、使用流程、许可协议等信息。如果使用Markdown 语法进行描述,还可以添加标记,提高可读性。

在 GitHub 上进行交流时用到的 Issue、评论、 Wiki,都可以用Markdown 语法表述,从而进行标记。准确地说应该是 GitHub Flavored Markdown( GFM)语法。该语法虽然是 GitHub 在 Markdown 语法基础上扩充而来的,但一般情况下只要按照原本的 Markdown 语法进行描述就可以。使用 GitHub 后,很多文档都需要用 Markdown 来书写。

将已有仓库 clone 到身边的开发环境中。
\begin{tcolorbox}[colback=green!5,colframe=green!40!black,title= ]
$\$$ git clone git@github.com:hirocastest/Hello-World.git \\
Cloning into 'Hello-World'... \\
remote: Counting objects: 3, done. \\
remote: Total 3 (delta 0), reused 0 (delta 0) \\
Receiving objects: $100\% (3/3)$, done.
\end{tcolorbox}
这里会要求输入 GitHub 上设置的公开密钥的密码。认证成功后,仓库便会被 clone 至仓库名后的目录中。将想要公开的代码提交至这个仓库再 push 到 GitHub 的仓库中,代码便会被公开。

添加至 Git 仓库的文件显示为 Untracked files。通过 \textcolor{red}{git add} 命令将文件加入暂存区 A,再通过 \textcolor{red}{git commit} 命令提交。添加成功后,可以通过 \textcolor{red}{git log} 命令查看提交日志。之后只要执行 \textcolor{red}{git push},GitHub 上的仓库就会被更新。



\section{基本操作}
\cite{demaree2016git} Git commands :
\begin{tcolorbox}[colback=green!5,colframe=green!40!black,title= ]
\textcolor{blue}{(master) $\$:$ git commandname parameter1 parameter2 $--$option}
\end{tcolorbox}
The command name (commandname in the example) is one of over $100$ individual functions that Git can perform. Behind the scenes, each of these commands is a separate program responsible for its own specific job. 

Options are special parameters that are denoted by at least one leading dash character. Many options have both a \textcolor{orange}{long form}, like \textcolor{blue}{$--$global}, and a \textcolor{orange}{shortcut form}, like \textcolor{blue}{$-$g}. There are also options that take values, like \textcolor{blue}{git commit $--$message=``hello world"}.

There are two that it absolutely needs in order to function: \textcolor{red}{your name} and \textcolor{red}{email address}. \textcolor{orange}{Git adds an Author attribute to every commit you make} that includes both your name and email address, so that your collaborators on a project can know who made a given change. The name you enter will be used to identify you in change logs and any other place where Git shows who made a particular change, while your email address
not only tells people how to reach you, but also tells a hosted service like GitHub who you are on their service.

Use the \textcolor{blue}{git config} command to tell Git who you are. Unlike most Git commands, which \textcolor{orange}{only work inside of a Git project}, these can be \textcolor{orange}{run from any directory}. 

要使用 Git 进行版本管理,必须先初始化仓库。 Git 是使用 \textcolor{red}{\bf git init} 命令\textcolor{blue}{进行初始化}的。建立一个目录并初始化仓库。如果初始化成功,执行了 git init命令的目录下就会生成 \textcolor{red}{.git 目录}。这个 .git 目录里\textcolor{blue}{存储着管理当前目录内容所需的仓库数据}。

在 Git 中,将这个目录的内容称为``\textcolor{blue}{附属于该仓库的工作树}”。文件的编辑等操作在工作树中进行,然后记录到仓库中,以此管理文件的历史快照。如果想将文件恢复到原先的状态,可以从仓库中调取之前的快照,在工作树中打开。开发者可以通过这种方式获取以往的文件。

 \textcolor{red}{\bf git status} 命令用于显示 Git 仓库的状态。工作树和仓库在被操作的过程中,状态会不断发生变化。在 Git 操
作过程中时常用 git status 命令查看当前状态。
\begin{tcolorbox}[colback=green!5,colframe=green!40!black,title= ]
$\$$ git status \\
$\#$ On branch master \\
$\#$  \\
$\#$  Initial commit \\
$\#$  \\
nothing to commit (create/copy files and use ``git add" to track) 
\end{tcolorbox}
结果显示了当前正处于 master 分支下。接着还显示了没有可提交的内容。所谓\textcolor{red}{提交(Commit)},是指``\textcolor{orange}{记录工作树中所有文件的当前状态}"。尚没有可提交的内容,就是说当前建立的这个仓库中还没有记录任何文件的任何状态。只要对 Git 的工作树或仓库进行操作, git status 命令的显示结果就会发生变化。

如果只是用 Git 仓库的工作树创建了文件,那么该文件并不会被记入 Git 仓库的版本管理对象当中。要想让文件成为 Git 仓库的管理对象,就需要用 \textcolor{red}{\bf git add} 命令将其\textcolor{orange}{加入暂存区}(Stage 或者 Index)中。\textcolor{blue}{暂存区是提交之前的一个临时区域}。
\begin{tcolorbox}[colback=green!5,colframe=green!40!black,title= ]
$\$$ git add README.md \\
$\$$ git status \\
$\#$ On branch master \\
$\#$ \\
$\#$ Initial commit \\
$\#$ \\
$\#$ Changes to be committed: \\
$\#$ (use ``git rm $--$cached <file>..." to unstage) \\
$\#$ \\
$\#$ new file: README.md 
\end{tcolorbox}
README.md 文件显示在 Changes to be committed 中了。

\textcolor{red}{\bf git commit }命令可以\textcolor{blue}{将当前暂存区中的文件实际保存到仓库的历史记录中。通过这些记录,可以在工作树中复原文件。}
\begin{tcolorbox}[colback=green!5,colframe=green!40!black,title= ]
$\$$ git commit $\rm -m$ ``First commit" \\
$\rm [master (root-commit) 9f129ba]$ First commit \\
1 file changed, 0 insertions(+), 0 deletions(-) \\
create mode 100644 README.md 
\end{tcolorbox}
\textcolor{orange}{-m 参数后}的 ``First commit"称作\textcolor{red}{提交信息},\textcolor{orange}{是对这个提交的概述}。想要记述得更加详细,\textcolor{orange}{不加 -m},\textcolor{red}{直接执行 git commit命令}。执行后编辑器就会启动,并显示如下结果
\begin{tcolorbox}[colback=green!5,colframe=green!40!black,title= ]
$\#$ Please enter the commit message for your changes. Lines starting \\
$\#$ with '$\#$' will be ignored, and an empty message aborts the commit. \\
$\#$ On branch master \\
$\#$ \\
$\#$ Initial commit \\
$\#$ \\
$\#$ Changes to be committed: \\
$\#$ (use ``git rm $--$cached <file>..." to unstage) \\
$\#$ \\
$\#$ new file: README.md 
\end{tcolorbox}
在编辑器中\textcolor{orange}{记述提交信息的格式}如下: \\
第一行:用一行文字简述提交的更改内容 \\
第二行:空行 \\
第三行以后:记述更改的原因和详细内容 

只要按照上面的格式输入,今后可以通过确认日志的命令或工具看到这些记录。在\textcolor{orange}{以 $\#$(井号)标为注释的 Changes to be committed(要提交的更改)栏中,可以查看本次提交中包含的文件}。将提交信息按格式记述完毕后,请保存并关闭编辑器,以$\#$(井号)标为注释的行不必删除。随后,刚才记述的提交信息就会被提交。

如果在编辑器启动后想中止提交,请将提交信息留空并直接关闭编辑器,随后提交就会被中止。

当前工作树处于刚刚完成提交的最新状态,所以结果显示没有更改。

\textcolor{red}{\bf git log} 命令可以\textcolor{blue}{查看以往仓库中提交的日志}。包括可以查看什么人在什么时候进行了提交或合并,以及操作前后有怎样的差别。
\begin{tcolorbox}[colback=green!5,colframe=green!40!black,title= ]
$\$$ git log \\
commit 9f129bae19b2c82fb4e98cde5890e52a6c546922 \\
Author: hirocaster <hohtsuka@gmail.com> \\
Date: Sun May 5 16:06:49 2013 +0900 \\
First commit 
\end{tcolorbox}
commit 栏旁边显示的``9f129b……"是\textcolor{blue}{指向这个提交的哈希值}。 Git 的其他命令中,在指向提交时会用到这个哈希值。Author 栏中显示 Git 设置的用户名和邮箱地址。 Date 栏中显示提交执行的日期和时间。再往下就是该提交的提交信息。

如果只想让程序显示第一行简述信息,可以在 git log命令后加上 \textcolor{violet}{$--$pretty=short}。
\begin{tcolorbox}[colback=green!5,colframe=green!40!black,title= ]
$\$$ git log --pretty=short \\
commit 9f129bae19b2c82fb4e98cde5890e52a6c546922 \\
Author: hirocaster <hohtsuka@gmail.com> \\
First commit
\end{tcolorbox}
只要在 \textcolor{blue}{git log 命令后加上目录名},便会\textcolor{blue}{只显示该目录下的日志}。如果\textcolor{blue}{加的是文件名},就会\textcolor{blue}{只显示与该文件相关的日志}。
\begin{tcolorbox}[colback=green!5,colframe=green!40!black,title= ]
$\$$ git log README.md
\end{tcolorbox}

如果想\textcolor{blue}{查看提交所带来的改动},可以加上 \textcolor{red}{\bf -p} 参数,\textcolor{blue}{文件的前后差别就会显示在提交信息之后}。
\begin{tcolorbox}[colback=green!5,colframe=green!40!black,title= ]
$\$$ git log -p
\end{tcolorbox}

只查看 README.md 文件的提交日志以及提交前后的差别
\begin{tcolorbox}[colback=green!5,colframe=green!40!black,title= ]
$\$$ git log -p README.md
\end{tcolorbox}
\textcolor{red}{\bf git diff} 命令可以\textcolor{blue}{查看工作树、暂存区、最新提交之间的差别}。如果尚未用 git add命令向暂存区添加任何东西,所以程序只会显示工作树与最新提交状态之间的差别。\textcolor{red}{``+”号}标出的是\textcolor{blue}{新添加的行},\textcolor{blue}{被删除的行}则用\textcolor{red}{``-”号}标出。可以看到,这次只添加了一行。
\begin{tcolorbox}[colback=green!5,colframe=green!40!black,title= ]
$\$$ git diff \\
diff $--$git a/README.md b/README.md \\
index e69de29..cb5dc9f 100644 \\
$---$ a/README.md \\
+++ b/README.md \\
@@ -0,0 +1 @@ \\
+$\#$ Git教程
\end{tcolorbox}
如果现在执行 git diff 命令,由于工作树和暂存区的状态并无差别,结果什么都不会显示。要查看与最新提交的差别,执行以下命令
\begin{tcolorbox}[colback=green!5,colframe=green!40!black,title= ]
$\$$ git diff HEAD \\
diff $--$git a/README.md b/README.md \\
index e69de29..cb5dc9f 100644 \\
$---$ a/README.md \\
+++ b/README.md \\
@@ -0,0 +1 @@ \\
+$\#$ Git教程
\end{tcolorbox}
\textcolor{violet}{在执行 git commit 命令之前先执行git diff HEAD命令,查看本次提交与上次提交之间有什么差别,等确认完毕后再进行提交}。这里的 \textcolor{blue}{HEAD 是指向当前分支中最新一次提交的指针}。


\section{分支的操作}
在进行多个并行作业时,会用到分支。在这类并行开发的过程中,往往同时存在多个最新代码状态。从 master 分支创
建 feature-A 分支和 fix-B 分支后,每个分支中都拥有自己的最新代码。master 分支是 Git 默认创建的分支,因此基本上所有开发都是以这个分支为中心进行的。不同分支中,可以同时进行完全不同的作业。等该分支的作业完成之后再与 master 分支合并。比如 feature-A 分支的作业结束后与 master 合并。

\textcolor{red}{\bf git branch} 命令可以\textcolor{blue}{将分支名列表显示},同时可以 \textcolor{blue}{确认当前所在分支}。
\begin{tcolorbox}[colback=green!5,colframe=green!40!black,title= ]
$\$$ git branch \\
$\ast$ master
\end{tcolorbox}
master 分支左侧标有``$\ast$"(星号),表示当前所在的分支。也就是说,正在 master 分支下进行开发。结果中没有显示其他分支名,表示本地仓库中只存在 master 一个分支。

如果想以当前的 master 分支为基础创建新的分支,需要用到 \textcolor{red}{\bf git checkout -b} 命令。创建名为 feature-A 的分支
\begin{tcolorbox}[colback=green!5,colframe=green!40!black,title= ]
$\$$ git checkout -b feature-A \\
Switched to a new branch `feature-A'
\end{tcolorbox}
连续执行下面两条命令也能收到同样效果
\begin{tcolorbox}[colback=green!5,colframe=green!40!black,title= ]
$\$$ \textcolor{green}{git branch feature-A} \\
$\$$ \textcolor{green}{git checkout feature-A}
\end{tcolorbox}
创建 feature-A 分支,并将当前分支切换为 feature-A 分支。这时查看分支列表,会显示处于 feature-A 分支下。
\begin{tcolorbox}[colback=green!5,colframe=green!40!black,title= ]
$\$$ git branch \\
$\ast$ feature-A \\
master 
\end{tcolorbox}
feature-A 分支左侧标有``$\ast$”,表示当前分支为 feature-A。在这个状态下像正常开发那样修改代码、执行 git add命令并进行提交的话,代码就会提交至 feature-A 分支。 像这样 \textcolor{blue}{不断对一个分支(例如feature-A)进行提交的操作},称为`` \textcolor{red}{培育分支}”。

切换至master 分支
\begin{tcolorbox}[colback=green!5,colframe=green!40!black,title= ]
$\$$ git checkout master \\
Switched to branch `master'
\end{tcolorbox}
查看 README.md 文件,会发现 README.md 文件仍然保持原先的状态,并没有被添加文字。 feature-A 分支的更改不会影响到 master 分支,这正是在开发中创建分支的优点。只要创建多个分支,就可以在不互相影响的情况下同时进行多个功能的开发。
\begin{tcolorbox}[colback=green!5,colframe=green!40!black,title= ]
$\$$ \textcolor{green}{git checkout $-$} \\
Switched to branch `feature-A'
\end{tcolorbox}
用\textcolor{red}{``$-$"(连字符)代替分支名},就可以切换至上一个分支。当然,将``$-$"替换成 feature-A 同样可以切换到 feature-A 分支。

\textcolor{red}{特性分支},是集中实现单一特性(主题),除此之外不进行任何作业的分支。在日常开发中,往往会创建数个特性分支,同时在此之外再保留一个随时可以发布软件的稳定分支。稳定分支的角色通常由 master 分支担当。

之前创建了 feature-A 分支,这一分支主要实现 feature-A,除 feature-A 的实现之外不进行任何作业。即便在开发过程中发现了 BUG,也需要再创建新的分支,在新分支中进行修正。基于特定主题的作业在特性分支中进行,主题完成后再与 master 分支合并。只要保持这样一个开发流程,就能保证 master 分支可以随时供人查看。这样一来,其他开发者也可以放心大胆地从 master 分支创建新的特性分支。


主干分支是特性分支的原点,同时也是合并的终点。通常人们会用 master 分支作为主干分支。主干分支中并没有开发到一半的代码,可以随时供他人查看。

假设 feature-A 已经实现完毕,想要将它合并到主干分支 master 中。首先切换到 master 分支
\begin{tcolorbox}[colback=green!5,colframe=green!40!black,title= ]
$\$$ git checkout master \\
Switched to branch `master'
\end{tcolorbox}
然后合并 feature-A 分支。为了在历史记录中明确记录下本次分支合并,需要创建合并提交。因此,在合并时加上 \textcolor{red}{\bf $--$no-ff} 参数。
\begin{tcolorbox}[colback=green!5,colframe=green!40!black,title= ]
$\$$ git merge $--$no-ff feature-A
\end{tcolorbox}
随后编辑器会启动,用于录入合并提交的信息
\begin{tcolorbox}[colback=green!5,colframe=green!40!black,title= ]
Merge branch `feature-A' \\
$\#$ Please enter a commit message to explain why this merge is necessary, especially if it merges an updated upstream into a topic branch. \\
$\#$ \\
$\#$ Lines starting with `$\#$' will be ignored, and an empty message aborts the commit.
\end{tcolorbox}
默认信息中已经包含了是从 feature-A 分支合并过来的相关内容,所以可不必做任何更改。将编辑器中显示的内容保存,关闭编辑器,然后就会看到
\begin{tcolorbox}[colback=green!5,colframe=green!40!black,title= ]
Merge made by the `recursive' strategy. \\
README.md | 2 ++ \\
1 file changed, 2 insertions(+)
\end{tcolorbox}
用 \textcolor{red}{git log $--$graph} 命令进行查看的话,能很清楚地看到特性分支(feature-A)提交的内容已被合并。除此以外,特性分支的创建以及合并也都清楚明了。
\begin{tcolorbox}[colback=green!5,colframe=green!40!black,title= ]
$\$$ git log $--$graph \\
$\ast$ commit 83b0b94268675cb715ac6c8a5bc1965938c15f62 \\
|\ Merge: fd0cbf0 8a6c8b9 \\
| | Author: hirocaster <hohtsuka@gmail.com> \\
| | Date: Sun May 5 16:37:57 2013 +0900 \\
| | \\
| | Merge branch `feature-A' \\
| | \\
| $\ast$ commit 8a6c8b97c8962cd44afb69c65f26d6e1a6c088d8 \\
|/ Author: hirocaster <hohtsuka@gmail.com> \\
| Date: Sun May 5 16:22:02 2013 +0900 \\
| \\
| Add feature-A \\
| \\
$\ast$ commit fd0cbf0d4a25f747230694d95cac1be72d33441d \\
| Author: hirocaster <hohtsuka@gmail.com> \\
| Date: Sun May 5 16:10:15 2013 +0900 \\
| \\
| Add index \\
| \\
$\ast$ commit 9f129bae19b2c82fb4e98cde5890e52a6c546922 \\
Author: hirocaster <hohtsuka@gmail.com> \\
Date: Sun May 5 16:06:49 2013 +0900 \\
First commit 
\end{tcolorbox}
git log $--$graph 命令可以用图表形式输出提交日志。

回溯历史版本,创建一个名为 fix-B 的特性分支。要让仓库的 HEAD、暂存区、当前工作树回溯到指定状态,需要用到 \textcolor{red}{\bf git rest --hard} 命令。只要提供目标时间点的哈希值\footnote{哈希值在每个环境中各不相同,需查看当前环境中 Add index 的哈希值,进行替换。},就可以完全恢复至该时间点的状态。
\begin{tcolorbox}[colback=green!5,colframe=green!40!black,title= ]
$\$$ git reset $--$hard fd0cbf0d4a25f747230694d95cac1be72d33441d \\
HEAD is now at fd0cbf0 Add index
\end{tcolorbox}
回溯到特性分支(feature-A)创建之前的状态。由于所有文件都回溯到了指定哈希值对应的时间点上, README.md 文件的内容也恢复到了当时的状态。
\begin{tcolorbox}[colback=green!5,colframe=green!40!black,title= ]
$\$$ git checkout -b fix-B \\
Switched to a new branch `fix-B'
\end{tcolorbox}
恢复到 feature-A 分支合并后的状态。不妨称这一操作为“推进历史”。

git log 命令只能查看以当前状态为终点的历史日志。所以这里要使用 \textcolor{red}{\bf git reflog} 命令,\textcolor{blue}{查看当前仓库的操作日志}。在日志中找出回溯历史之前的哈希值,通过 \textcolor{red}{\bf git reset --hard} 命令恢复到回溯历史前的状态。

执行 git reflog 命令,查看当前仓库执行过的操作的日志。
\begin{tcolorbox}[colback=green!5,colframe=green!40!black,title= ]
$\$$ git reflog \\
4096d9e HEAD@$\{0\}$: commit: Fix B \\
fd0cbf0 HEAD@$\{1\}$: checkout: moving from master to fix-B \\
fd0cbf0 HEAD@$\{2\}$: reset: moving to fd0cbf0d4a25f747230694d95cac1be72d33441d \\
83b0b94 HEAD@$\{3\}$: merge feature-A: Merge made by the 'recursive' strategy. \\
fd0cbf0 HEAD@$\{4\}$: checkout: moving from feature-A to master \\
8a6c8b9 HEAD@$\{5\}$: checkout: moving from master to feature-A \\
fd0cbf0 HEAD@$\{6\}$: checkout: moving from feature-A to master \\
8a6c8b9 HEAD@$\{7\}$: commit: Add feature-A \\
fd0cbf0 HEAD@$\{8\}$: checkout: moving from master to feature-A \\
fd0cbf0 HEAD@$\{9\}$: commit: Add index \\
9f129ba HEAD@$\{10\}$
\end{tcolorbox}

在日志中,可以看到 commit、 checkout、 reset、 merge 等 Git 命令的执行记录。只要不进行 Git 的 GC(Garbage Collection,垃圾回收),就可以通过日志随意调取近期的历史状态,就像给时间机器指定一个时间点,在过去未来中自由穿梭一般。即便开发者错误执行了 Git 操作,基本也都可以利用 git reflog命令恢复到原先的状态。


要修改上一条提交信息,可以使用 \textcolor{red}{\bf git commit $--$amend} 命令。


\section{推送至远程仓库}
为防止与其他仓库混淆,仓库名与本地仓库保持一致。创建时不要勾选 Initialize this repository with a README 选项。因为一旦勾选该选项, GitHub 一侧的仓库就会自动生成 README 文件,从创建之初便与本地仓库失去了整合性。虽然到时也可以强制覆盖,但为防止这一情况发生还是建议不要勾选该选项,直接点击 Create repository 创建仓库。

在 GitHub 上创建的仓库路径为``git@github.com:用户名/git-tutorial.git"。现在我们用 \textcolor{red}{\bf git remote add} 命令将它设置成本地仓库的远程仓库 A。
\begin{tcolorbox}[colback=green!5,colframe=green!40!black,title= ]
$\$$ git remote add origin git@github.com:github-book/git-tutorial.git
\end{tcolorbox}
按照上述格式执行 git remote add 命令之后, Git 会自动将 git@github.com:github-book/git-tutorial.git 远程仓库的名称设置为 \textcolor{yellow}{origin(标识符)}。

将当前分支下本地仓库中的内容推送给远程仓库,需要用到 \textcolor{red}{\bf git push} 命令。假定在 master 分支下进行操作
\begin{tcolorbox}[colback=green!5,colframe=green!40!black,title= ]
$\$$ git push -u origin master \\
Counting objects: $20$, done. \\
Delta compression using up to 8 threads. \\
Compressing objects: $100\%$ ($10/10$), done. \\
Writing objects: $100\%$ ($20/20$), 1.60 KiB, done. \\
Total $20$ (delta $3$), reused $0$ (delta $0$) \\
To git@github.com:github-book/git-tutorial.git \\
$\ast$ [new branch] master -> master \\
Branch master set up to track remote branch master from origin.
\end{tcolorbox}
像这样执行 git push命令,当前分支的内容就会被推送给远程仓库 origin 的 master 分支。\textcolor{red}{\bf -u}参数可以在推送的同时,\textcolor{yellow}{将 origin 仓库的 master 分支设置为本地仓库当前分支的 upstream(上游)。添加了这个参数,将来运行 git pull命令从远程仓库获取内容时,本地仓库的这个分支就可以直接从 origin 的 master 分支获取内容,省去了另外添加参数的麻烦。}

执行该操作后,当前本地仓库 master 分支的内容将会被推送到GitHub 的远程仓库中。在 GitHub 上也可以确认远程 master 分支的内容和本地 master 分支相同。

除了 master 分支之外,远程仓库也可以创建其他分支。在本地仓库中创建 feature-D 分支,并将它以同名形式 push 至远程仓库。
\begin{tcolorbox}[colback=green!5,colframe=green!40!black,title= ]
$\$$ git checkout -b feature-D \\
Switched to a new branch `feature-D'
\end{tcolorbox}
\begin{tcolorbox}[colback=green!5,colframe=green!40!black,title= ]
$\$$ git push -u origin feature-D \\
Total 0 (delta 0), reused 0 (delta 0) \\
To git@github.com:github-book/git-tutorial.git \\
$\ast$ [new branch] feature-D -> feature-D \\
Branch feature-D set up to track remote branch feature-D from origin.
\end{tcolorbox}

\cite{narębski2016mastering} To see which remote repositories you have configured, you can run the \textcolor{red}{\bf git remote} command. It lists the shortnames of each remote you've got. In a cloned repository you will have at least one remote: origin.
\begin{tcolorbox}[colback=green!5,colframe=green!40!black,title= ]
$\$$ git remote \\
origin
\end{tcolorbox}

To see the URL together with remotes, you can use \textcolor{red}{\bf $-$v / $--$verbose} option
\begin{tcolorbox}[colback=green!5,colframe=green!40!black,title= ]
$\$$ git remote $--$verbose \\
origin git://git.kernel.org/pub/scm/git/git.git (fetch) \\
origin git://git.kernel.org/pub/scm/git/git.git (push)
\end{tcolorbox}

If you want to inspect remotes to see more information about a particular remote, use the \textcolor{red}{\bf git remote show <remote>} subcommand
\begin{tcolorbox}[colback=green!5,colframe=green!40!black,title= ]
$\$$ git remote show origin \\
$\ast$ remote origin \\
Fetch URL: git://git.kernel.org/pub/scm/git/git.git \\
Push URL: git://git.kernel.org/pub/scm/git/git.git \\
HEAD branch: master \\
Remote branches: \\
maint tracked \\
master tracked \\
next tracked \\
pu tracked \\
todo tracked \\
Local branch configured for `git pull': \\
master merges with remote master \\
Local ref configured for `git push': \\
master pushes to master (up-to-date)
\end{tcolorbox}

Git will consult the remote configuration, the branch configuration, and the remote repository (for an up-to-date status). If you want to skip contacting the remote repository and use cached information instead, add the \textcolor{red}{\bf $-$n} option to \textcolor{red}{\bf git remote show}.




\section{从远程仓库获取}
将 GitHub 上的仓库 clone 到本地。
\begin{tcolorbox}[colback=green!5,colframe=green!40!black,title= ]
$\$$ git clone git@github.com:github-book/git-tutorial.git \\
Cloning into `git-tutorial'... \\
remote: Counting objects: $20$, done. \\
remote: Compressing objects: $100\%$ ($7/7$), done. \\
remote: Total $20$ (delta $3$), reused $20$ (delta $3$) \\
Receiving objects: $100\%$ ($20/20$), done. \\
Resolving deltas: $100\%$ ($3/3$), done. \\
\end{tcolorbox}
\textcolor{orange}{执行 git clone命令后会默认处于 master 分支下,同时系统会自动将 origin 设置成该远程仓库的标识符}。也就是说,\textcolor{orange}{当前本地仓库的 master 分支与 GitHub 端远程仓库(origin)的 master 分支在内容上是完全相同的}。
\begin{tcolorbox}[colback=green!5,colframe=green!40!black,title= ]
$\$$ git branch -a \\
$\ast$ master \\
remotes/origin/HEAD -> origin/master \\
remotes/origin/feature-D \\
remotes/origin/master 
\end{tcolorbox}
用 \textcolor{red}{\bf git branch -a} 命令查看当前分支的相关信息。添加 \textcolor{red}{\bf -a} 参数可以同时显示本地仓库和远程仓库的分支信息。结果中显示了 remotes/origin/feature-D,证明远程仓库中已经有了 feature-D 分支。

将 feature-D 分支获取至本地仓库:
\begin{tcolorbox}[colback=green!5,colframe=green!40!black,title= ]
$\$$ git checkout -b feature-D origin/feature-D \\
Branch feature-D set up to track remote branch feature-D from origin. \\
Switched to a new branch `feature-D'
\end{tcolorbox}
 \textcolor{red}{\bf -b} 参数的后面是\textcolor{orange}{本地仓库中新建分支的名称}。新建分支名称后面是获取来源的分支名称。例子中指定了 origin/feature-D,就是说以名为 origin 的仓库(这里指 GitHub 端的仓库)的 feature-D 分支为来源,在本地仓库中创建 feature-D 分支。

假定是另一名开发者,要做一个新的提交。在 README.md 文件中添加一行文字,查看更改。
\begin{tcolorbox}[colback=green!5,colframe=green!40!black,title= ]
$\$$ git diff \\
diff $--$git a/README.md b/README.md \\
index af647fd..30378c9 100644 \\
$---$ a/README.md \\
+++ b/README.md \\
@@ $-3,3 +3,4$ @@ \\
$-$ feature-A \\
$-$ fix-B \\
$-$ feature-C \\
$+ -$ feature-D 
\end{tcolorbox}
\begin{tcolorbox}[colback=green!5,colframe=green!40!black,title= ]
$\$$ git commit -am ``Add feature-D" \\
$\rm [feature-D ed9721e]$ Add feature-D \\
1 file changed, 1 insertion(+)
\end{tcolorbox}
推送 feature-D 分支
\begin{tcolorbox}[colback=green!5,colframe=green!40!black,title= ]
$\$$ git push \\
Counting objects: 5, done. \\
Delta compression using up to 8 threads. \\
Compressing objects: $100\%$ ($2/2$), done. \\
Writing objects: $100\%$ ($3/3$), 281 bytes, done. \\
Total 3 (delta 1), reused 0 (delta 0) \\
To git@github.com:github-book/git-tutorial.git \\
ca0f98b..ed9721e feature-D -> feature-D 
\end{tcolorbox}
从远程仓库获取 feature-D 分支,在本地仓库中提交更改,再将feature-D 分支推送回远程仓库,通过这一系列操作,就可以与其他开发者相互合作,共同培育 feature-D 分支,实现某些功能。

使用  \textcolor{red}{\bf git pull} 命令,将本地的 feature-D 分支更新到最新状态。当前分支为 feature-D 分支。
\begin{tcolorbox}[colback=green!5,colframe=green!40!black,title= ]
$\$$ git pull origin feature-D \\
remote: Counting objects: 5, done. \\
remote: Compressing objects: $100\%$ ($1/1$), done. \\
remote: Total 3 (delta 1), reused 3 (delta 1) \\
Unpacking objects: $100\%$ ($3/3$), done. \\
From github.com:github-book/git-tutorial \\
$\ast$ branch feature-D -> FETCH$\_$HEAD \\
First, rewinding head to replay your work on top of it... \\
Fast-forwarded feature-D to ed9721e686f8c588e55ec6b8071b669f411486b8. 
\end{tcolorbox}

GitHub 端远程仓库中的 feature-D 分支是最新状态,所以本地仓库中的 feature-D 分支就得到了更新。今后只需要像平常一样在本地进行提交再 push 给远程仓库,就可以与其他开发者同时在同一个分支中进行作业,不断给 feature-D 增加新功能。

如果两人同时修改了同一部分的源代码, push 时就很容易发生冲突。所以多名开发者在同一个分支中进行作业时,为减少冲突情况的发生,建议更频繁地进行 push 和 pull 操作。






%%%%%%%%%%%%%%%%%%%%%%%%%%%%%%%%%%%%%%%%%%%%%%%%%%%%%%%%%%%%%%%%%%%%%%
\bibliographystyle{unsrt_update}
\bibliography{ref}
%%%%%%%%%%%%%%%%%%%%%%%%%%%%%%%%%%%%%%%%%%%%%%%%%%%%%%%%%%%%%%%%%%%%%%

\end{document}