\documentclass[12pt,a4paper]{article}
%\usepackage{fontspec, xunicode, xltxtra}  
%\setmainfont{Hiragino Sans GB}  
%\usepackage{xeCJK}
%\setCJKmainfont[BoldFont=STZhongsong, ItalicFont=STKaiti]{STSong}
%\setCJKsansfont[BoldFont=STHeiti]{STXihei}
%\setCJKmonofont{STFangsong}

%使用Xelatex编译

% 设置页面
%==================================================
\linespread{2} %行距
% \usepackage[top=1in,bottom=1in,left=1.25in,right=1.25in]{geometry}
% \headsep=2cm
% \textwidth=16cm \textheight=24.2cm
%==================================================

% 其它需要使用的宏包
%==================================================
\usepackage[colorlinks,linkcolor=blue,anchorcolor=red,citecolor=green,urlcolor=blue]{hyperref} 
\usepackage{tabularx}
\usepackage{authblk}         % 作者信息
\usepackage{algorithm}     % 算法排版
\usepackage{amsmath}     % 数学符号与公式
\usepackage{amsfonts}     % 数学符号与字体
\usepackage{amssymb}

\usepackage{graphicx} 
\usepackage{graphics}
\usepackage{xcolor}
\usepackage{color}

\usepackage{fancyhdr}       % 设置页眉页脚
\usepackage{fancyvrb}       % 抄录环境
\usepackage{float}              % 管理浮动体
\usepackage{geometry}     % 定制页面格式
\usepackage{hyperref}       % 为PDF文档创建超链接
\usepackage{lineno}          % 生成行号
\usepackage{listings}        % 插入程序源代码
\usepackage{multicol}       % 多栏排版
\usepackage{natbib}         % 管理文献引用
\usepackage{rotating}       % 旋转文字,图形,表格
\usepackage{subfigure}    % 排版子图形
\usepackage{titlesec}       % 改变章节标题格式
\usepackage{moresize}   % 更多字体大小
\usepackage{anysize}
\usepackage{indentfirst}  % 首段缩进
\usepackage{booktabs}   % 使用\multicolumn
\usepackage{multirow}    % 使用\multirow
\usepackage{wrapfig}

\usepackage{titlesec}     % 改变标题样式
\usepackage{enumitem}

\renewcommand{\vec}[1]{\boldsymbol{#1}}
\newcommand{\me}{\mathrm{e}}
\newcommand{\mi}{\mathrm{i}}
\newcommand{\dif}{\mathrm{d}}
\newcommand{\tabincell}[2]{\begin{tabular}{@{}#1@{}}#2\end{tabular}}

\def\kpc{{\rm kpc}}
\def\km{{\rm km}}
\def\cm{{\rm cm}}
\def\TeV{{\rm TeV}}
\def\GeV{{\rm GeV}}
\def\MeV{{\rm MeV}}
\def\GV{{\rm GV}}
\def\MV{{\rm MV}}
\def\yr{{\rm yr}}
\def\s{{\rm s}}
\def\ns{{\rm ns}}
\def\GHz{{\rm GHz}}
\def\muGs{{\rm \mu Gs}}
\def\arcsec{{\rm arcsec}}
\def\K{{\rm K}}
\def\microK{\mu{\rm K}}
\def\sr{{\rm sr}}
\newcolumntype{p}{D{,}{\pm}{-1}}

\renewcommand{\figurename}{Fig.}
\renewcommand{\tablename}{Tab.}

\renewcommand{\arraystretch}{1.5}

\setlength{\parindent}{0pt}  %取消每段开头的空格

\title{Reference}
\author{}
\date{\today}
\begin{document}

\maketitle
\section{Compound Types}
A compound type is a type that is \textcolor{cyan}{defined in terms of another type}. Simple declarations consist of a type followed by a list of variable names. More generally, a declaration is a \textcolor{red}{base type} followed by a list of \textcolor{red}{declarators}. \textcolor{cyan}{Each declarator names a variable and gives the variable a type that is related to the base type}.

\section{References}
Technically speaking, when we use the term reference, we mean ``\textcolor{red}{lvalue reference}”.

A reference defines an alternative name for an object. A reference type ``refers to” another type. We \textcolor{red}{define a reference type} by writing a declarator of the form \textcolor{red}{$\&$d}, where \textcolor{red}{d is the name being declared}.

int ival = 1024; \\
\textcolor{orange}{int $\&$refVal = ival;} // refVal refers to (is another name for) ival \\
\textcolor{orange}{int $\&$refVal2;} // error: a reference must be initialized

Ordinarily, when we initialize a variable, the value of the initializer is copied into the object we are creating. \textcolor{red}{When we define a reference, instead of copying the initializer’s value, we bind the reference to its initializer. Once initialized, a reference remains bound to its initial object. There is no way to rebind a reference to refer to a different object. Because there is no way to rebind a reference, references must be initialized.}

A Reference Is an Alias

A reference is not an object. Instead, a reference is just another name for an already existing object. After a reference has been defined, all operations on that reference are actually operations on the object to which the reference is bound. 

refVal = 2; // assigns 2 to the object to which refVal refers, i.e., to ival \\
int ii = refVal; // same as ii = ival

When we assign to a reference, we are assigning to the object to which the reference is bound. When we fetch the value of a reference, we are really fetching the value of the object to which the reference is bound. When we use a reference as an initializer, we are really using the object to which the reference is bound.

// ok: refVal3 is bound to the object to which refVal is bound, i.e., to ival \\
int $\&$refVal3 = refVal;

Because \textcolor{cyan}{references are not objects}, we may \textcolor{cyan}{not define a reference to a reference}.

Reference Definitions

We can \textcolor{cyan}{define multiple references in a single definition}. Each identifier that is a reference must be preceded by the \textcolor{red}{$\&$} symbol.

With two exceptions, the type of a reference and the object to which the reference refers must match exactly. \textcolor{red}{A reference may be bound only to an object}, not to a literal or to the result of a more general expression

int $\&$refVal4 = 10; // error: initializer must be an object \\
double dval = 3.14; \\
int $\&$refVal5 = dval; // error: initializer must be an int object

\subsection{rvalue reference}
These references are primarily intended for use inside classes. 

















































\end{document}