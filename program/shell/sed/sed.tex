\documentclass[12pt,a4paper]{article}
%\usepackage{fontspec, xunicode, xltxtra}  
%\setmainfont{Hiragino Sans GB}  
\usepackage{xeCJK}
%\setCJKmainfont[BoldFont=STZhongsong, ItalicFont=STKaiti]{STSong}
%\setCJKsansfont[BoldFont=STHeiti]{STXihei}
%\setCJKmonofont{STFangsong}

%使用Xelatex编译

% 设置页面
%==================================================
\linespread{2} %行距
% \usepackage[top=1in,bottom=1in,left=1.25in,right=1.25in]{geometry}
% \headsep=2cm
% \textwidth=16cm \textheight=24.2cm
%==================================================

% 其它需要使用的宏包
%==================================================
\usepackage[colorlinks,linkcolor=blue,anchorcolor=red,citecolor=green,urlcolor=blue]{hyperref} 
\usepackage{tabularx}
\usepackage{authblk}         % 作者信息
\usepackage{algorithm}     % 算法排版
\usepackage{amsmath}     % 数学符号与公式
\usepackage{amsfonts}     % 数学符号与字体
\usepackage{amssymb}
\usepackage{amsthm}
\usepackage{mathrsfs}      % 花体

\usepackage[framemethod=TikZ]{mdframed}
\usepackage{graphicx} 
\usepackage{graphics}
\usepackage{xcolor}
\usepackage{color}

\usepackage{fancyhdr}       % 设置页眉页脚
\usepackage{fancyvrb}       % 抄录环境
\usepackage{float}              % 管理浮动体
\usepackage{geometry}     % 定制页面格式
\usepackage{hyperref}       % 为PDF文档创建超链接
\usepackage{lineno}          % 生成行号
\usepackage{listings}        % 插入程序源代码
\usepackage{multicol}       % 多栏排版
\usepackage{natbib}         % 管理文献引用
\usepackage{rotating}       % 旋转文字,图形,表格
\usepackage{subfigure}    % 排版子图形
\usepackage{titlesec}       % 改变章节标题格式
\usepackage{moresize}   % 更多字体大小
\usepackage{anysize}
\usepackage{indentfirst}  % 首段缩进
\usepackage{booktabs}   % 使用\multicolumn
\usepackage{multirow}    % 使用\multirow
\usepackage{wrapfig}

\usepackage{titlesec}     % 改变标题样式
\usepackage{enumitem}

\renewcommand{\vec}[1]{\boldsymbol{#1}}
\newcommand{\me}{\mathrm{e}}
\newcommand{\mi}{\mathrm{i}}
\newcommand{\dif}{\mathrm{d}}
\newcommand{\tabincell}[2]{\begin{tabular}{@{}#1@{}}#2\end{tabular}}

\def\kpc{{\rm kpc}}
\def\km{{\rm km}}
\def\cm{{\rm cm}}
\def\TeV{{\rm TeV}}
\def\GeV{{\rm GeV}}
\def\MeV{{\rm MeV}}
\def\GV{{\rm GV}}
\def\MV{{\rm MV}}
\def\yr{{\rm yr}}
\def\s{{\rm s}}
\def\ns{{\rm ns}}
\def\GHz{{\rm GHz}}
\def\muGs{{\rm \mu Gs}}
\def\arcsec{{\rm arcsec}}
\def\K{{\rm K}}
\def\microK{\mu{\rm K}}
\def\sr{{\rm sr}}
\newcolumntype{p}{D{,}{\pm}{-1}}

\renewcommand{\figurename}{Fig.}
\renewcommand{\tablename}{Tab.}

\renewcommand{\arraystretch}{1.5}

\setlength{\parindent}{0pt}  %取消每段开头的空格


\newcounter{theo}[section]\setcounter{theo}{0}
\renewcommand{\thetheo}{\arabic{section}.\arabic{theo}}
\newenvironment{theo}[2][]{%
\refstepcounter{theo}%
\ifstrempty{#1}%
{\mdfsetup{%
frametitle={%
\tikz[baseline=(current bounding box.east),outer sep=0pt]
\node[anchor=east,rectangle,fill=blue!20]
{\strut Theorem~\thetheo};}}
}%
{\mdfsetup{%
frametitle={%
\tikz[baseline=(current bounding box.east),outer sep=0pt]
\node[anchor=east,rectangle,fill=blue!20]
{\strut ~\thetheo:~#1};}}%
}%
\mdfsetup{innertopmargin=10pt,linecolor=blue!20,%
linewidth=2pt,topline=true,%
frametitleaboveskip=\dimexpr-\ht\strutbox\relax
}
\begin{mdframed}[]\relax%
\label{#2}}{\end{mdframed}}

\title{sed}
\author{}
\date{\today}
\begin{document}

\maketitle
sed是一种流编辑器(stream editor),它在处理时,把当前处理的行存储在临时缓冲区中,称为``模式空间"(pattern space),接着用sed命令处理缓冲区中的内容,处理完成后,把缓冲区的内容送往屏幕。接着处理下一行,这样不断重复,直到文件末尾。文件内容并没有 改变,除非你使用重定向存储输出。sed主要用来自动编辑一个或多个文件;简化对文件的反复操作;编写转换程序等。

sed是非交互式的编辑器。它\textcolor{cyan}{不会修改文件},除非\textcolor{cyan}{使用shell重定向来保存结果}。默认情况下,\textcolor{cyan}{所有的输出行都被打印到屏幕上}。

sed编辑器逐行处理文件(或输入),并将结果发送到屏幕。具体过程如下:首先sed把当前正在处理的行保存在一个临时缓存区中(也称为模式空间),然后处理临时缓冲区中的行,完成后把该行发送到屏幕上。sed每处理完一行就将其从临时缓冲区删除,然后将下一行读入,进行处理和显示。处理完输入文件的最后一行后,sed便结束运行。sed把每一行都存在临时缓冲区中,对这个副本进行编辑,所以不会修改原文件。
 
\textcolor{red}{地址}用于决定对哪些行进行编辑。地址的形式可以是数字、正则表达式、或二者的结合。如果没有指定地址,sed将处理输入文件的所有行。
 
地址是一个数字,则表示行号;是``$\$$"符号,则表示最后一行。例如: 
sed -n `3p' datafile \\
只打印第三行
 
sed -n `$\$$p' datafile \\
只打印最后一行
 
 只显示指定行范围的文件内容,例如:\\
$\#$ 只查看文件的第$100$行到第$200$行 \\
sed -n `100,200p' mysql$\_$slow$\_$query.log

流编辑器可以对从管道这样的标准输入接收的数据进行编辑。无需将要编辑的数据存储在磁盘上的文件中。

sed通过对输入数据执行人任意数量用户指定的编辑操作(命令)。sed是基于行的,按顺序对每一行执行命令。sed将其结果写入标准输出(stdout),不修改任何输入文件。

\section{sed的选项、命令、替换标记}

\begin{theo}[命令格式]{}
sed [options] `command' file(s) \\
sed [options] -f scriptfile file(s)
\end{theo}

\begin{theo}[选项]{}
-e<script>或--expression=<script>:以选项中的指定的script来处理输入的文本文件;直接在指令列模式上进行sed的动作编辑;-e选项可以支持sed进行多点编辑处理,使用多个scripts或者expression时,之间使用; 分号隔开\\
-f<script文件>或--file=<script文件>:以选项中的指定的script文件来处理输入的文本文件;直接将sed的动作写在一个档案内,-f filename则可以执行filename内的sed动作;\\
-h或--help:显示帮助;\\
-n或--quiet或—silent:仅显示script处理后的结果;在一般 sed 的用法中,所有来自stdin的资料一般都会被列出到屏幕上。加上-n参数后,则只有经过sed处理的那一行(或者动作)才会被列出来;\\
-r:sed的动作支援的是延伸型正规表示法的语法。(预设是基础正规表示法语法) \\
-V或--version:显示版本信息。\\
-i:\textcolor{red}{直接修改读取的档案内容,而不是由屏幕输出}。
\end{theo}

参数

文件:指定待处理的文本文件列表。

\begin{theo}[sed命令]{}
a$\backslash$ 在当前行下面插入文本。 \\
i$\backslash$ 在当前行上面插入文本。 \\
c$\backslash$ 把选定的行改为新的文本。\\ 
d 删除,删除选择的行。 \\
D 删除模板块的第一行。 \\
s 替换指定字符。\\
h 拷贝模板块的内容到内存中的缓冲区。 \\
H 追加模板块的内容到内存中的缓冲区。 \\
g 获得内存缓冲区的内容,并替代当前模板块中的文本。\\
G 获得内存缓冲区的内容,并追加到当前模板块文本的后面。\\
l 列表不能打印字符的清单。\\
n 读取下一个输入行,用下一个命令处理新的行而不是用第一个命令。\\
N 追加下一个输入行到模板块后面并在二者间嵌入一个新行,改变当前行号码。\\
p 打印模板块的行。\\
P(大写) 打印模板块的第一行。\\
q 退出sed。\\
b lable 分支到脚本中带有标记的地方,如果分支不存在则分支到脚本的末尾。\\
r file 从file中读行。\\
t label if分支,从最后一行开始,条件一旦满足或者T,t命令,将导致分支到带有标号的命令处,或者到脚本的末尾。\\
T label 错误分支,从最后一行开始,一旦发生错误或者T,t命令,将导致分支到带有标号的命令处,或者到脚本的末尾。\\
w file 写并追加模板块到file末尾。\\
W file 写并追加模板块的第一行到file末尾。\\
! 表示后面的命令对所有没有被选定的行发生作用。\\
= 打印当前行号码。\\
$\#$ 把注释扩展到下一个换行符以前。 
\end{theo}

\begin{theo}[sed替换标记]{}
g 表示行内全面替换。\\
p 表示打印行。\\
w 表示把行写入一个文件。\\
x 表示互换模板块中的文本和缓冲区中的文本。\\
y 表示把一个字符翻译为另外的字符(但是不用于正则表达式)\\
$\backslash$1 子串匹配标记$\&$已匹配字符串标记
\end{theo}

\begin{theo}[sed元字符集]{}
$\hat{}$ 匹配行开始,如:/ $\hat{}~$sed/匹配所有以sed开头的行。\\
$\$$ 匹配行结束,如:/sed$\$$/匹配所有以sed结尾的行。\\
. 匹配一个非换行符的任意字符,如:/s.d/匹配s后接一个任意字符,最后是d。\\
$\star$ 匹配0个或多个字符,如:/${}^{\star}{\rm sed}$/匹配所有模板是一个或多个空格后紧跟sed的行。\\
$[ ]$ 匹配一个指定范围内的字符,如/[ss]ed/匹配sed和Sed。\\
$[~\hat{}~ ]$ 匹配一个不在指定范围内的字符,如:/$[~\hat{}~ {\rm A-RT-Z}]$ed/匹配不包含A-R和T-Z的一个字母开头,紧跟ed的行。\\
$\backslash(\cdots\backslash)$ 匹配子串,保存匹配的字符,如s/$\backslash(love\backslash)$able/$\backslash 1$rs,loveable被替换成lovers。\\
$\&$ 保存搜索字符用来替换其他字符,如s/love/$**\&**$/,love这成**love**。\\
$\backslash<$ 匹配单词的开始,如:/$\backslash<$love/匹配包含以love开头的单词的行。\\
$\backslash>$ 匹配单词的结束,如/love$\backslash>$/匹配包含以love结尾的单词的行。\\
x$\backslash\{ {\rm m}\backslash\}$ 重复字符x,$m$次,如:/0$\backslash\{ 5\backslash\}$/匹配包含5个0的行。\\
x$\backslash\{ {\rm m}, \backslash\}$ 重复字符x,至少$m$次,如:/0$\backslash\{ 5, \backslash\}$/匹配至少有5个0的行。\\
x$\backslash\{ {\rm m, n}\backslash\}$ 重复字符x,至少$m$次,不多于$n$次,如:/0$\backslash\{ 5, 10\backslash\}$/匹配$5\sim 10$个$0$的行。
\end{theo}



\section{规则表达式}










\section{实例}

\subsection{替换操作:s命令}
替换文本中的字符串:\\
sed `s/book/books/' file

-n选项和p命令一起使用表示只打印那些发生替换的行:\\
sed -n `s/test/TEST/p' file

直接编辑文件选项\textcolor{red}{-i},会匹配file文件中每一行的第一个book替换为books:\\
sed -i `s/book/books/g' file

\subsection{全面替换标记g}
使用后缀\textcolor{red}{/g}标记会\textcolor{cyan}{替换每一行中的所有匹配}:\\
sed `s/book/books/g' file 

当需要从第N处匹配开始替换时,可以使用 /Ng:\\
echo sksksksksksk | sed 's/sk/SK/2g' \\
skSKSKSKSKSK 

echo sksksksksksk | sed 's/sk/SK/3g' \\
skskSKSKSKSK 

echo sksksksksksk | sed 's/sk/SK/4g' \\
skskskSKSKSK

\subsection{定界符}
以上命令中字符/在sed中作为定界符使用,也可以使用任意的定界符:\\
sed `s:test:TEXT:g' \\
sed `s|test|TEXT|g' 

定界符出现在样式内部时,需要进行转义:\\
sed `s/$\backslash$/bin/$\backslash$/usr$\backslash$/local$\backslash$/bin/g'


\subsection{删除操作:d命令}
删除空白行:\\
sed `/ $\hat{}~ \$$/d' file 

删除文件的第2行:\\
sed `2d' file 

删除文件的第2行到末尾所有行:\\
sed `2, $\$$d' file

删除文件最后一行:\\
sed `$\$$d' file 

删除文件中所有开头是test的行:
sed `/ $\hat{}$~ test/'d file

\subsection{已匹配字符串标记$\&$}
正则表达式 $\backslash$ w$\backslash +$ 匹配每一个单词,使用 $[\&]$ 替换它,$\&$对应于之前所匹配到的单词:\\
echo this is a test line | sed `s/$\backslash$w$\backslash +$/$[\&]$/g' \\
$[$this$]$ $[$is$]$ $[$a$]$ $[$test$]$ $[$line$]$ 

所有以192.168.0.1开头的行都会被替换成它自已加localhost:\\
sed `s/ $\hat{}$ 192.168.0.1/$\&$localhost/' file \\
192.168.0.1localhost

\subsection{子串匹配标记$\backslash 1$}
匹配给定样式的其中一部分:\\
echo this is digit 7 in a number | sed `s/digit $\backslash([0-9]\backslash)$/$\backslash 1$/' \\
this is 7 in a number \\

命令中digit $7$,被替换成了$7$。样式匹配到的子串是$7$,$\backslash(\cdots \backslash)$用于匹配子串,对于匹配到的第一个子串就标记为$\backslash 1$,依此类推匹配到的第二个结果就是$\backslash 2$,例如:\\
echo aaa BBB | sed `s/$\backslash ([a-z]\backslash +\backslash) \backslash([A-Z]\backslash +\backslash)/\backslash 2 \backslash 1/$' \\
BBB aaa

love被标记为$1$,所有loveable会被替换成lovers,并打印出来:\\
sed -n `${\rm s}/\backslash ({\rm love}\backslash) {\rm able}/\backslash 1 {\rm rs/p}$' file

\subsection{组合多个表达式}
sed `表达式' | sed `表达式'\\
等价于:\\
sed `表达式; 表达式'

\subsection{引用}
sed表达式可以使用单引号来引用,但是如果表达式内部包含变量字符串,就需要使用双引号。
test=hello\\
echo hello WORLD | sed ``s/$\$$test/HELLO" \\
HELLO WORLD 

\subsection{选定行的范围:,(逗号)}
所有在模板test和check所确定的范围内的行都被打印:
sed -n '/test/,/check/p' file 

打印从第5行开始到第一个包含以test开始的行之间的所有行:\\
sed -n `5,/$\hat{}$ test/p' file \\
对于模板test和west之间的行,每行的末尾用字符串aaa bbb替换:\\
sed `/test/,/west/s/$\$$/aaa bbb/' file 

\subsection{多点编辑:e命令}
-e选项允许在同一行里执行多条命令:\\
sed -e `1,5d' -e 's/test/check/' file 

上面sed表达式的第一条命令删除1至5行,第二条命令用check替换test。命令的执行顺序对结果有影响。如果两个命令都是替换命令,那么第一个替换命令将影响第二个替换命令的结果。 

和 -e 等价的命令是 --expression: \\
sed --expression='s/test/check/' --expression='/love/d' file 

\subsection{从文件读入:r命令}
file里的内容被读进来,显示在与test匹配的行后面,如果匹配多行,则file的内容将显示在所有匹配行的下面:\\
sed `/test/r file' filename 

\subsection{写入文件:w命令}
在example中所有包含test的行都被写入file里:\\
sed -n `/test/w file' example 

\subsection{追加(行下):a$\backslash$命令}
将 this is a test line 追加到 以test 开头的行后面:\\
sed `/$\hat{}$ test/a$\backslash$this is a test line' file 

在 test.conf 文件第2行之后插入\\
this is a test line: sed -i `2a$\backslash$this is a test line' test.conf 

\subsection{插入(行上):i$\backslash$命令}
将 this is a test line 追加到以test开头的行前面:\\
sed `/$\hat{}$ test/i$\backslash$this is a test line' file 

在test.conf文件第5行之前插入this is a test line: \\
sed -i `5i$\backslash$ this is a test line' test.conf 

\subsection{下一个:n命令}
如果test被匹配,则移动到匹配行的下一行,替换这一行的aa,变为bb,并打印该行,然后继续:\\
sed `/test/$\{$ n; s/aa/bb/; $\}$' file 

\subsection{变形:y命令}
把1~10行内所有abcde转变为大写,注意,正则表达式元字符不能使用这个命令:\\
sed `1,10y/abcde/ABCDE/' file 

\subsection{退出:q命令}
打印完第10行后,退出sed \\
sed `10q' file 

\subsection{保持和获取:h命令和G命令}
在sed处理文件的时候,每一行都被保存在一个叫模式空间的临时缓冲区中,除非行被删除或者输出被取消,否则所有被处理的行都将 打印在屏幕上。接着模式空间被清空,并存入新的一行等待处理。\\
sed -e `/test/h' -e `$\$$G' file 

在这个例子里,匹配test的行被找到后,将存入模式空间,h命令将其复制并存入一个称为保持缓存区的特殊缓冲区内。第二条语句的意思是,当到达最后一行后,G命令取出保持缓冲区的行,然后把它放回模式空间中,且追加到现在已经存在于模式空间中的行的末尾。在这个例子中就是追加到最后一行。简单来说,任何包含test的行都被复制并追加到该文件的末尾。 

\subsection{保持和互换:h命令和x命令}
互换模式空间和保持缓冲区的内容。也就是把包含test与check的行互换:\\
sed -e `/test/h' -e `/check/x' file

\subsection{脚本scriptfile}
sed脚本是一个sed的命令清单,启动Sed时以-f选项引导脚本文件名。sed对于脚本中输入的命令非常挑剔,在命令的末尾不能有任何空白或文本,如果在一行中有多个命令,要用分号分隔。以$\#$开头的行为注释行,且不能跨行。 \\
sed $[$options$]$ -f scriptfile file(s) 

\subsection{打印奇数行或偶数行} 
方法1:\\
sed -n `p;n' test.txt $\#$奇数行 \\
sed -n `n;p' test.txt $\#$偶数行 

方法2:\\
sed -n `$1\sim 2$p' test.txt $\#$奇数行 \\
sed -n `$2\sim 2$p' test.txt $\#$偶数行 


\subsection{打印匹配字符串的下一行}
grep -A 1 SCC URFILE \\
sed -n `/SCC/$\{$n;p$\}$' URFILE \\
awk `/SCC/$\{$getline; print$\}$' URFILE













\end{document}