\documentclass[11pt,a4paper]{article}
%\usepackage{fontspec, xunicode, xltxtra}  
%\setmainfont{Hiragino Sans GB}  
%\usepackage{xeCJK}
%\setCJKmainfont[BoldFont=STZhongsong, ItalicFont=STKaiti]{STSong}
%\setCJKsansfont[BoldFont=STHeiti]{STXihei}
%\setCJKmonofont{STFangsong}

%使用Xelatex编译

% 设置页面
%==================================================
\linespread{2} %行距
% \usepackage[top=1in,bottom=1in,left=1.25in,right=1.25in]{geometry}
% \headsep=2cm
% \textwidth=16cm \textheight=24.2cm
%==================================================

% 其它需要使用的宏包
%==================================================
\usepackage[colorlinks,linkcolor=blue,anchorcolor=red,citecolor=green,urlcolor=blue]{hyperref} 
\usepackage{tabularx}
\usepackage{authblk}         % 作者信息
\usepackage{algorithm}     % 算法排版
\usepackage{amsmath}     % 数学符号与公式
\usepackage{amsfonts}     % 数学符号与字体
\usepackage{mathrsfs}      % 花体
\usepackage[framemethod=TikZ]{mdframed}

\usepackage{graphicx} 
\usepackage{graphics}
\usepackage{color}
\usepackage{xcolor}
\usepackage{tcolorbox}
\usepackage{lipsum}
\usepackage{empheq}

\usepackage{fancyhdr}       % 设置页眉页脚
\usepackage{fancyvrb}       % 抄录环境
\usepackage{float}              % 管理浮动体
\usepackage{geometry}     % 定制页面格式
\usepackage{hyperref}       % 为PDF文档创建超链接
\usepackage{lineno}          % 生成行号
\usepackage{listings}        % 插入程序源代码
\usepackage{multicol}       % 多栏排版
%\usepackage{natbib}         % 管理文献引用
\usepackage{rotating}       % 旋转文字,图形,表格
\usepackage{subfigure}    % 排版子图形
\usepackage{titlesec}       % 改变章节标题格式
\usepackage{moresize}   % 更多字体大小
\usepackage{anysize}
\usepackage{indentfirst}  % 首段缩进
\usepackage{booktabs}   % 使用\multicolumn
\usepackage{multirow}    % 使用\multirow

\usepackage{wrapfig}
\usepackage{titlesec}     % 改变标题样式
\usepackage{enumitem}
\usepackage{aas_macros}


\newcommand{\myvec}[1]%
   {\stackrel{\raisebox{-2pt}[0pt][0pt]{\small$\rightharpoonup$}}{#1}}  %矢量符号
\renewcommand{\vec}[1]{\boldsymbol{#1}}
\newcommand{\me}{\mathrm{e}}
\newcommand{\mi}{\mathrm{i}}
\newcommand{\dif}{\mathrm{d}}
\newcommand{\tabincell}[2]{\begin{tabular}{@{}#1@{}}#2\end{tabular}}

\def\kpc{{\rm kpc}}
\def\km{{\rm km}}
\def\cm{{\rm cm}}
\def\TeV{{\rm TeV}}
\def\GeV{{\rm GeV}}
\def\MeV{{\rm MeV}}
\def\GV{{\rm GV}}
\def\MV{{\rm MV}}
\def\yr{{\rm yr}}
\def\s{{\rm s}}
\def\ns{{\rm ns}}
\def\GHz{{\rm GHz}}
\def\muGs{{\rm \mu Gs}}
\def\arcsec{{\rm arcsec}}
\def\K{{\rm K}}
\def\microK{\mu{\rm K}}
\def\sr{{\rm sr}}
\newcolumntype{p}{D{,}{\pm}{-1}}

\renewcommand{\figurename}{Fig.}
\renewcommand{\tablename}{Tab.}

\renewcommand{\arraystretch}{1.5}

\setlength{\parindent}{0pt}  %取消每段开头的空格

\newcounter{theo}[section]\setcounter{theo}{0}
\renewcommand{\thetheo}{\arabic{section}.\arabic{theo}}
\newenvironment{theo}[2][]{%
\refstepcounter{theo}%
\ifstrempty{#1}%
{\mdfsetup{%
frametitle={%
\tikz[baseline=(current bounding box.east),outer sep=0pt]
\node[anchor=east,rectangle,fill=blue!20]
{\strut Theorem~\thetheo};}}
}%
{\mdfsetup{%
frametitle={%
\tikz[baseline=(current bounding box.east),outer sep=0pt]
\node[anchor=east,rectangle,fill=blue!20]
{\strut Theorem~\thetheo:~#1};}}%
}%
\mdfsetup{innertopmargin=10pt,linecolor=blue!20,%
linewidth=2pt,topline=true,%
frametitleaboveskip=\dimexpr-\ht\strutbox\relax
}
\begin{mdframed}[]\relax%
\label{#2}}{\end{mdframed}}

\newcommand*\widefbox[1]{\fbox{\hspace{2em}#1\hspace{2em}}}



\title{TENSOR AND DIFFERENTIAL FORMS}
\author{}
\date{\today}
\begin{document}

\maketitle

An isotropic distribution of mass would be a spherical distribution. No matter what direction one looks, the distribution looks the same. We had been discussed in the Friedmann's cosmological model of the universe. A simpler model is a spherical ball. However, an ellipsoidal distribution is slightly non-isotropic. In general, the distribution of matter can be described using the inertia tensor, a $3\times 3$ matrix
\begin{equation*}
\renewcommand{\arraystretch}{0.7}
\begin{pmatrix}
I_{xx} & I_{xy} & I_{xz} \\
I_{yx} & I_{yy} & I_{yz} \\
I_{zx} & I_{zy} & I_{zz} 
\end{pmatrix}
\end{equation*}
The entries are the moments of inertia about the origin for a continuous distribution of mass:
\begin{align*}
& I_{xx} = \int y^2 +z^2 \dif m ~, ~~~I_{xy} = I_{yx} = -\int xy \dif m ~,\\
& I_{yy} = \int x^2 +z^2 \dif m ~, ~~~I_{yz} = I_{zy} = -\int yz \dif m ~,\\
& I_{zz} = \int x^2 +y^2 \dif m ~, ~~~I_{zx} = I_{xz} = -\int xz \dif m ~, 
\end{align*}
The total angular momentum is $\vec{L} = I \vec{\Omega}$, where $\vec{\Omega}$ is a column vector of the components of angular velocity. The components of the moment of inertia can be written more compactly using index notation as
\begin{equation*}
I_{ij} = \int (r^2 \delta_{ij} -x_i x_j) \rho \dif V ~.
\end{equation*}
Under a rotation of the coordinate system, the angular momentum equation becomes $\vec{L}^\prime = I^\prime \vec{\Omega}^\prime$. The relation to the original quantities is 
\begin{align*}
& \vec{L}^\prime = I^\prime \vec{\Omega}^\prime ~, \\
& \hat{R}_\theta \vec{L} = I^\prime \hat{R}_\theta \vec{\Omega} ~, \\
& \vec{L} =  \hat{R}^{-1}_\theta I^\prime \hat{R}_\theta \vec{\Omega} ~.
\end{align*}
The moment of inertia changes under a transformation,
\begin{equation*}
I = \hat{R}^{-1}_\theta I^\prime \hat{R}_\theta ~. 
\end{equation*}
If the resulting matrix is diagonal, the moment of inertia tensor have been diagonalized,
\begin{equation*}
I = 
\renewcommand{\arraystretch}{0.7}
\begin{pmatrix}
I_{x} & 0 & 0 \\
0 & I_{y} & 0 \\
0 & 0 & I_{z} 
\end{pmatrix}
\end{equation*}
The diagonal entries are called the principal moments of inertia.

\section{Tensor Analysis}
\subsection{Covariant and Contravariant Tensors}
\cite{arfken} The rotational transformation of a vector $\vec{A} = A_1\vec{\hat{e}}_1 + A_2\vec{\hat{e}}_2 +A_3\vec{\hat{e}}_3$ from the Cartesian system defined by $\vec{\hat{e}}_i (i = 1, 2, 3)$ into a rotated coordinate system defined by $\vec{\hat{e}}^\prime_i$, with the same vector $\vec{A}$ then represented as $\vec{A}^\prime = A^\prime_1\vec{\hat{e}}^\prime_1 + A^\prime_2\vec{\hat{e}}^\prime_2 + A^\prime_3\vec{\hat{e}}^\prime_3$. The components of $\vec{A}$ and $\vec{A}^\prime$ are related by
\begin{equation}
A^\prime_i = \sum_j (\vec{\hat{e}}^\prime_i \cdot \vec{\hat{e}}^\prime_j) A_j ~,
\end{equation}
where the coefficients $(\vec{\hat{e}}^\prime_i \cdot \vec{\hat{e}}^\prime_j)$ are the projections of $\vec{\hat{e}}^\prime_i$ in the $\vec{\hat{e}}^\prime_j$ directions. Because the $\vec{\hat{e}}_i$ and the $\vec{\hat{e}}_j$ are linearly related,
\begin{equation}
A^\prime_i = \sum_j \dfrac{\partial x_i^\prime}{\partial x_j} A_j ~.
\label{cotrava}
\end{equation}
The gradient of a scalar $\varphi$ has in the unrotated Cartesian coordinates the components $(\nabla \varphi)_j = \dfrac{\partial \varphi}{\partial x_j} \vec{\hat{e}}_j$, i.e. in a rotated system
\begin{equation}
(\nabla \varphi)^\prime_i \equiv \dfrac{\partial \varphi}{\partial x_i^\prime} = \sum_j \dfrac{\partial x_j}{\partial x^\prime_i} \dfrac{\partial \varphi}{\partial x_j} ~,
\label{cova}
\end{equation}
Quantities transforming according to Eq. (\ref{cotrava}) are called \textcolor{red}{contravariant vectors}, while those transforming according to Eq. (\ref{cova}) are termed \textcolor{red}{covariant vectors}. 

\begin{align}
& (A^\prime)^i = \sum_j \dfrac{\partial (x^\prime)^i}{\partial x^j} A^j  ~~, \text{$\vec{A}$ a contravariant vector} \\
& A^\prime_i = \sum_j \dfrac{\partial x^j}{\partial (x^\prime)^i} A^j  ~~, \text{$\vec{A}$ a covariant vector} 
\end{align}

\subsection{Tensors of Rank $2$}
Define contravariant, mixed, and covariant tensors of rank 2 by the following equations for their components under coordinate transformations:
\begin{align}
(A^\prime)^{ij} = \sum_{kl} \dfrac{\partial (x^\prime)^i}{\partial x^k} \dfrac{\partial (x^\prime)^j}{\partial x^l} A^{kl} ~, \\ 
(B^\prime)^{i}_j = \sum_{kl} \dfrac{\partial (x^\prime)^i}{\partial x^k} \dfrac{\partial x^l}{\partial (x^\prime)^j} B^{k}_l ~, \\ 
(C^\prime)_{ij} = \sum_{kl} \dfrac{\partial x^k}{\partial (x^\prime)^i} \dfrac{\partial x^l}{\partial (x^\prime)^j} C_{kl} ~, 
\end{align}
The second-rank tensor $A$ (with components $A^{kl}$) may be represented by writing out its components in a square array
\begin{equation}
A = 
\renewcommand{\arraystretch}{0.7}
\begin{pmatrix}
A^{11} & A^{12} & A^{13} \\
A^{21} & A^{22} & A^{23} \\
A^{31} & A^{32} & A^{33}
\end{pmatrix}
\end{equation}
For $A$, it takes the form
\begin{align}
& \color{red} (A^\prime)^{ij} = \sum_{kl} S_{ik} A^{kl} (S^T)_{lj}, \\
& \color{red} A^\prime = S A S^T ~,
\end{align}
which is known as a \textcolor{red}{similarity transformation}. 





\subsection{Isotropic Tensors}




\subsection{Contraction}





\subsection{Direct Product}




\subsection{Inverse Transformation}







\subsection{Quotient Rule}


























%%%%%%%%%%%%%%%%%%%%%%%%%%%%%%%%%%%%%%%%%%%%%%%%%%%%%%%%%%%%%%%%%%%%%%
\bibliographystyle{unsrt_update}
\bibliography{ref}
%%%%%%%%%%%%%%%%%%%%%%%%%%%%%%%%%%%%%%%%%%%%%%%%%%%%%%%%%%%%%%%%%%%%%%

\end{document}