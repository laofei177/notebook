\documentclass[11pt,a4paper]{article}
%\usepackage{fontspec, xunicode, xltxtra}  
%\setmainfont{Hiragino Sans GB}  
\usepackage{xeCJK}
%\setCJKmainfont[BoldFont=STZhongsong, ItalicFont=STKaiti]{STSong}
%\setCJKsansfont[BoldFont=STHeiti]{STXihei}
%\setCJKmonofont{STFangsong}

%使用Xelatex编译

% 设置页面
%==================================================
\linespread{1.5} %行距
% \usepackage[top=1in,bottom=1in,left=1.25in,right=1.25in]{geometry}
% \headsep=2cm
% \textwidth=16cm \textheight=24.2cm
%==================================================

% 其它需要使用的宏包
%==================================================
\usepackage[colorlinks,linkcolor=blue,anchorcolor=red,citecolor=green,urlcolor=blue]{hyperref} 
\usepackage{tabularx}
\usepackage{authblk}         % 作者信息
\usepackage{algorithm}     % 算法排版
\usepackage{amsmath}     % 数学符号与公式
\usepackage{amsfonts}     % 数学符号与字体
\usepackage{amssymb}

\usepackage{graphics}
\usepackage{color}
\usepackage{fancyhdr}       % 设置页眉页脚
\usepackage{fancyvrb}       % 抄录环境
\usepackage{float}              % 管理浮动体
\usepackage{geometry}     % 定制页面格式
\usepackage{hyperref}       % 为PDF文档创建超链接
\usepackage{lineno}          % 生成行号
\usepackage{listings}        % 插入程序源代码
\usepackage{multicol}       % 多栏排版
%\usepackage{natbib}         % 管理文献引用
\usepackage{rotating}       % 旋转文字,图形,表格
\usepackage{subfigure}    % 排版子图形
\usepackage{titlesec}       % 改变章节标题格式
\usepackage{moresize}   % 更多字体大小
\usepackage{anysize}
\usepackage{indentfirst}  % 首段缩进
\usepackage{booktabs}   % 使用\multicolumn
\usepackage{multirow}    % 使用\multirow
\usepackage{graphicx} 
\usepackage{wrapfig}
\usepackage{xcolor}
\usepackage{titlesec}     % 改变标题样式
\usepackage{enumitem}

\renewcommand{\vec}[1]{\boldsymbol{#1}}
\newcommand{\me}{\mathrm{e}}
\newcommand{\mi}{\mathrm{i}}
\newcommand{\dif}{\mathrm{d}}
\newcommand{\tabincell}[2]{\begin{tabular}{@{}#1@{}}#2\end{tabular}}

\def\kpc{{\rm kpc}}
\def\km{{\rm km}}
\def\cm{{\rm cm}}
\def\TeV{{\rm TeV}}
\def\GeV{{\rm GeV}}
\def\MeV{{\rm MeV}}
\def\GV{{\rm GV}}
\def\MV{{\rm MV}}
\def\yr{{\rm yr}}
\def\s{{\rm s}}
\def\ns{{\rm ns}}
\def\GHz{{\rm GHz}}
\def\muGs{{\rm \mu Gs}}
\def\arcsec{{\rm arcsec}}
\def\K{{\rm K}}
\def\microK{\mu{\rm K}}
\def\sr{{\rm sr}}
\newcolumntype{p}{D{,}{\pm}{-1}}

\renewcommand{\figurename}{Fig.}
\renewcommand{\tablename}{Tab.}

\renewcommand{\arraystretch}{1.5}

\title{Green Function}
\author{}
\date{\today}
\begin{document}

\maketitle

\section{泊松方程}

\subsection{1-D}
\cite{Rother:2256622} The equation
\begin{equation}
\dfrac{\dif^2 G(x, x^\prime)}{\dif x^2} = \delta(x-x^\prime) ~,
\end{equation}
It has no effect on our final observation if the source and observation points are interchanged. The observation (the measurable quantity) depends only on the spatial distance between source- and observation point. This symmetry is called Reciprocity, but now with respect to position. The Reciprocity condition
\begin{equation}
G(x, x^\prime) = G( x^\prime, x)
\end{equation}
This condition is obviously fulfilled if
\begin{equation}
G(x, x^\prime) = F(|x-x^\prime|) = F(u) ~,
\end{equation}
is used as an ansatz with the so far unknown function $F$. 
\begin{equation}
\dfrac{\dif F(u)}{\dif x} = F_x = F_u \cdot u_x = F_u \cdot [H(x-x^\prime) -H(x^\prime-x) ] ~,
\end{equation}
and
\begin{equation}
F_{xx} = F_{uu} \cdot u_x^2 + 2\cdot F_u \cdot \delta(x-x^\prime) = F_{uu} +2\cdot F_u\cdot \delta(x-x^\prime) = \delta(x-x^\prime) ~.
\end{equation}
Require that the function $F$ must be a solution of the homogeneous equation
\begin{align}
F_{uu} &= 0 ~, \\
F(u) &= C_1\cdot u +C_2 ~.
\end{align}
The additional condition is
\begin{equation}
[F_u]_{u = 0 } = \dfrac{1}{2} ~.
\end{equation}
The Green's function is
\begin{equation}
\left[\dfrac{\dif G(x, x^\prime)}{\dif x} = \right]_{x=x^\prime-\epsilon}^{x=x^\prime+\epsilon} = 1 ~.
\end{equation}
As known from potential theory, potentials are determined except for an arbitrary constant. Let us therefore choose $C_2 = 0$ and consider
\begin{equation}
G(x, x^\prime) = F(u) = \dfrac{1}{2}u =  \dfrac{1}{2} \cdot |x-x^\prime| ~,
\end{equation}
as the free-space Green's function of the $1$-dim. Poisson equation.



\subsection{2-D}
\cite{Rother:2256622} Assuming that the source point agrees with the origin of the polar coordinate system the Poisson equation for the Green’s function $G^{(2)}(R)$ reads
\begin{equation}
G^{(2)}_{RR}(R) +\dfrac{1}{R} \cdot G^{(2)}(R) = 2\delta_p (R) = \dfrac{1}{\pi R} \delta(R) ~.
\end{equation}
Use
\begin{equation}
G^{(2)}(R) = F(R) \cdot H(R) ~,
\end{equation}
as an appropriate ansatz. It seems as if the Heaviside function $H(R)$ can be omitted since $R \leqslant 0$. But according to our classical method we have to calculate the first and second derivative of the Green's function. In so doing $H(R)$ produces the required Dirac's delta function. However, $H(R)$ can be omitted in the final result, i.e., once we have determined $F(R)$. 
\begin{equation}
\left[F_{RR}(R) +\dfrac{F_{R}(R)}{R}  \right] \cdot H(R) + 2\cdot F_R(R) \cdot \delta(R) = \dfrac{1}{\pi R}\delta(R) ~.
\end{equation}
The unknown function $F(R)$ can be determined by looking for the general solution of the homogeneous equation in the square brackets, i.e., of the ordinary differential equation
\begin{align}
& F_{RR}(R) +\dfrac{F_{R}(R)}{R}  = 0 ~.
& F(R) = C_1 \cdot \ln(R) +C_2 ~.
\end{align}
Constant $C_1$ can now be determined by applying condition
\begin{equation}
\left[ R\cdot \dfrac{\dif G^{(2)}(R)}{\dif R} \right]_{R_\epsilon} = \dfrac{1}{2\pi} ~,
\end{equation}
\begin{equation}
C_1 = \dfrac{1}{2\pi} ~.
\end{equation}
If $C_2$ is again set to zero, as already done in the $1$-dim. case, the free-space Green's function of the $2$-dim. Poisson equation is given by
\begin{equation}
G^{(2)}(R) =  \dfrac{1}{2\pi} \cdot \ln (R) \cdot H(R) ~.
\end{equation}
But it becomes also clear that only the inhomogeneity $2 \delta_p(R)$ produces a solution that is in correspondence with condition of the unit source. On the other hand, using the inhomogeneity $\delta_p(R)$ would result in
\begin{equation}
G^{(2)}(R) =  \dfrac{1}{4\pi} \cdot \ln (R) \cdot H(R) ~.
\end{equation}
Replace $R$ by $|\vec{r} - \vec{r}_0|$ if the source point does not agrees with the origin of the coordinate system. Since
\begin{equation}
|\vec{r} - \vec{r}_0| = [(x-x^\prime)^2 +(y-y^\prime)^2]^{1/2} ~,
\end{equation}
\begin{equation}
G^{(2)}(R) =  \dfrac{1}{2\pi} \cdot \ln \left\{[R^2 +R^{\prime 2} -2R R^\prime\cdot \cos (\phi -\phi^\prime)]^{1/2} \right\}
\end{equation}
for the free-space Green's function of the $2$-dim. Poisson equation.




\cite{Herman:1742872} Consider 
\begin{align}
\nabla^2 u &= f ~, ~~~ \text{in} ~~ D, \\
u &= g ~, ~~~ \text{on} ~~ C, 
\end{align}
The Green‘s Function satisfies the differential equation and homogeneous boundary conditions. The associated problem is given by
\begin{align}
\nabla^2 G &= \delta(\xi -x, \eta -y) ~, ~~~ \text{in} ~~ D, \\
G &\equiv 0 ~, ~~~ \text{on} ~~ C. 
\end{align}
Green's Function is symmetric in its arguments. However, this is not always the case and depends on things such as the self-adjointedness of the problem. we will assume that the Green's Function satisfies
\begin{equation}
\nabla_{r^\prime}^2 G = \delta(\xi -x, \eta -y) ~,
\end{equation}
where the notation $\nabla_{r^\prime}$ means differentiation with respect to the variables $\xi$ and $\eta$.
\begin{align}
\nonumber \int_D (u \nabla_{r^\prime}^2 G - G \nabla_{r^\prime}^2 u) \dif A^\prime &= \int_C (u \nabla_{r^\prime} G - G \nabla_{r^\prime} u) \cdot \dif \vec{s}^\prime ~. \\
\nonumber \int_D (u \nabla_{r^\prime}^2 G - G \nabla_{r^\prime}^2 u) \dif A^\prime &= \int_D(u(\xi, \eta) \delta(\xi-x, \eta- y) - G(x,y;\xi,\eta)f(\xi, \eta) ) \dif \xi \dif \eta \\
\nonumber &= u(x,y) - \int_D G(x,y;\xi,\eta)f(\xi, \eta) \dif \xi \dif \eta ~.
\end{align}
Using the boundary conditions, $u(\xi, \eta) = g(\xi, \eta)$ on $C$ and $G(x, y; \xi, \eta) = 0$ on $C$, 
\begin{equation}
\int_C \int_C (u \nabla_{r^\prime} G - G \nabla_{r^\prime} u) \cdot \dif \vec{s}^\prime = \int_C g(\xi, \eta) \nabla_{r^\prime} G \cdot \dif \vec{s}^\prime ~.
\end{equation}
\begin{equation}
u(x,y) = \int_D G(x,y;\xi,\eta)f(\xi, \eta) \dif \xi \dif \eta + \int_C g(\xi, \eta) \nabla_{r^\prime} G \cdot \dif \vec{s}^\prime
\end{equation}

For the Laplacian in polar coordinates, the Green's function is
\begin{equation}
v_{rr} + \dfrac{1}{r} v_r = \delta(r) ~.
\end{equation}
For $r \neq 0$, this is a Cauchy-Euler type of differential equation. The general solution is $v(r) = A \ln r + B$.

Due to the singularity at $r = 0$, we integrate over a domain in which a small circle of radius e is cut form the plane and apply the two-dimensional Divergence Theorem. 
\begin{align}
\nonumber 1 &= \int_{D_\epsilon} \delta(r) \dif A \\
\nonumber &= \int_{D_\epsilon} \nabla^2 v \dif A \\
\nonumber &= \int_{C_\epsilon} \nabla v\cdot \dif \vec{s} \\
&= \int_{C_\epsilon} \dfrac{\partial v}{\partial r}\dif S = 2\pi A ~.
\end{align}
$A = 1/2\pi$.  $B$ is arbitrary, so we will take $B = 0$.

The Green's function for Poisson's Equation is
\begin{equation}
G(\vec{r}, \vec{r}^\prime) = \dfrac{1}{2\pi} \ln |\vec{r} - \vec{r}^\prime| ~.
\end{equation}





























\subsection{3-D}
\cite{2010数学物理方法, 2015数学物理方法, 2006数学物理方法} 确定了$G$,就能利用积分表示式求得泊松方程边值问题的解。虽然求格林函数的问题本身也是边值问题,但这是特殊的边值问题。无界区域的格林函数称为相应方程的基本解。将一个一般边值问题的格林函数$G$分成两部分
\begin{equation}
G = G_0 +G_1 ~.
\end{equation}
其中$G_0$是基本解。对于三维泊松方程,$G_0$满足
\begin{equation}
\Delta G_0 = \delta(\vec{r} -\vec{r}_0) ~.
\label{eq:green_3D}
\end{equation}
$G_1$满足相应的齐次方程(Laplace方程)
\begin{equation}
\Delta G_1 = 0 ~。
\end{equation}
及相应的边界条件。方程\ref{eq:green_3D}描述的是位于点$\vec{r}_0$,电量$-\varepsilon_0$的点电荷在无界空间中所产生电场在$\vec{r}$的电势,即$G_0 = -1/4\pi|\vec{r} -\vec{r}_0|$。

假设点源位于坐标原点,由于区域是无界的,点源产生的场与方向无关,选取球坐标$(r, \theta, \varphi)$,则$G_0$只是$r$的函数,方程变为常微分方程,当$r\neq 0$时,$G_0$满足Laplace方程
\begin{equation}
\dfrac{1}{r^2} \dfrac{\dif }{\dif r} \left(r^2 \dfrac{\dif G_0}{\dif r} \right) = 0 ~,
\end{equation}
解为
\begin{equation}
G_0 = -\dfrac{C_1}{r} +C_2 ~.
\end{equation}
令无穷远处$G_0 = 0$,$C_2 = 0$。将方程\ref{eq:green_3D}在包含$r_0 = 0$的区域作体积分,区域可取为以$r_0 = 0$为球心,半径为$\varepsilon$的小球$K_\varepsilon$,其边界面为$\Sigma_\varepsilon$,
\begin{equation}
\iiint\limits_{K_\varepsilon} \Delta G_0 \dif V = 1 ~.
\end{equation}
\begin{equation}
\iiint\limits_{K_\varepsilon} \Delta G_0 \dif V = \iint\limits_{\Sigma_\varepsilon} \dfrac{\partial G_0}{\partial r} \dif S = \int_0^{2\pi} \int_{0}^{\pi} \dfrac{\partial }{\partial r} \left(-\dfrac{C_1}{r} \right) r^2 \sin \theta \dif \theta \dif \varphi = 4\pi C_1 ~.
\end{equation}
则$C_1 = \dfrac{1}{4\pi}$,
\begin{equation}
G_0(r)= -\dfrac{1}{4\pi r} ~.
\end{equation}
若电荷位于任意点$\vec{r}_0$,则
\begin{equation}
G_0(r)= -\dfrac{1}{4\pi}  \dfrac{1}{|\vec{r} - \vec{r}_0|} ~.
\end{equation}

\cite{Herman:1742872} Poisson's Equation for the electric potential is
\begin{equation}
\nabla^2 \phi = -\dfrac{\rho}{\varepsilon_0} ~.
\end{equation}
The Poisson's Equation also arises in Newton's Theory of Gravitation for the gravitational potential in the form $\nabla^2 \phi = -4\pi G \rho$, where $\rho$ is the matter density.

Consider Poisson's Equation
\begin{equation}
\nabla^2 \phi(\vec{r}) = -4\pi f(\vec{r}) ~,
\end{equation}
for $\vec{r}$ defined throughout all space. The Fourier transform can be generalized to three dimensions as
\begin{equation}
\hat{\phi}(\vec{k}) = \int_V \phi(\vec{r}) {\rm e}^{i\vec{k}\cdot \vec{r}}\dif^3 r ~,
\end{equation}
where the integration is over all space, $V$. The inverse Fourier transform is
\begin{equation}
\phi(\vec{r}) = \dfrac{1}{(2\pi)^3} \int_{V_k} \hat{\phi}(\vec{k}) {\rm e}^{-i\vec{k}\cdot \vec{r}}\dif^3 k ~.
\end{equation}

The Fourier transform of the Laplacian follows from computing Fourier transforms of any derivatives that are present. Assuming that $\phi$ and its gradient vanish for large distances, then
\begin{equation}
\mathcal F[\nabla^2 \phi] = -(k_x^2 +k_y^2 +k_z^2) \hat{\phi}(\vec{k}) ~.
\end{equation}
Poisson's Equation becomes
\begin{align}
& k^2 \hat{\phi}(\vec{k}) = 4\pi \hat{f}(\vec{k}) ~.\\
& \hat{\phi}(\vec{k}) = \dfrac{4\pi}{k^2}  \hat{f}(\vec{k}) ~.
\end{align}
The solution to Poisson's Equation is then determined from the inverse Fourier transform,
\begin{equation}
\phi(\vec{r}) = \dfrac{4\pi}{(2\pi)^3} \int_{V_k} \hat{\phi}(\vec{k})\dfrac{{\rm e}^{-i\vec{k}\cdot \vec{r}}}{k^2} \dif^3 k ~.
\end{equation}
Set $f(\vec{r}) = f_0 \delta^3(\vec{r})$ in order to represent a point source. For a unit point charge, $f_0 = 1/4\pi \epsilon_0$.


\cite{Rother:2256622} 



























































\section{Helmholtz Equation}






















\section{Modified Helmholtz Equation}
































%%%%%%%%%%%%%%%%%%%%%%%%%%%%%%%%%%%%%%%%%%%%%%%%%%%%%%%%%%%%%%%%%%%%%%
\bibliographystyle{unsrt_update}
\bibliography{ref}
%%%%%%%%%%%%%%%%%%%%%%%%%%%%%%%%%%%%%%%%%%%%%%%%%%%%%%%%%%%%%%%%%%%%%%


\end{document}