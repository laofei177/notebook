\documentclass[11pt,a4paper]{article}
%\usepackage{fontspec, xunicode, xltxtra}  
%\setmainfont{Hiragino Sans GB}  
\usepackage{xeCJK}
%\setCJKmainfont[BoldFont=STZhongsong, ItalicFont=STKaiti]{STSong}
%\setCJKsansfont[BoldFont=STHeiti]{STXihei}
%\setCJKmonofont{STFangsong}

%使用Xelatex编译

% 设置页面
%==================================================
\linespread{2} %行距
% \usepackage[top=1in,bottom=1in,left=1.25in,right=1.25in]{geometry}
% \headsep=2cm
% \textwidth=16cm \textheight=24.2cm
%==================================================

% 其它需要使用的宏包
%==================================================
\usepackage[colorlinks,linkcolor=blue,anchorcolor=red,citecolor=green,urlcolor=blue]{hyperref} 
\usepackage{tabularx}
\usepackage{authblk}         % 作者信息
\usepackage{algorithm}     % 算法排版
\usepackage{amsmath}     % 数学符号与公式
\usepackage{amsfonts}     % 数学符号与字体
\usepackage{mathrsfs}      % 花体
\usepackage{graphics}
\usepackage{color}
\usepackage{fancyhdr}       % 设置页眉页脚
\usepackage{fancyvrb}       % 抄录环境
\usepackage{float}              % 管理浮动体
\usepackage{geometry}     % 定制页面格式
\usepackage{hyperref}       % 为PDF文档创建超链接
\usepackage{lineno}          % 生成行号
\usepackage{listings}        % 插入程序源代码
\usepackage{multicol}       % 多栏排版
\usepackage{natbib}         % 管理文献引用
\usepackage{rotating}       % 旋转文字,图形,表格
\usepackage{subfigure}    % 排版子图形
\usepackage{titlesec}       % 改变章节标题格式
\usepackage{moresize}   % 更多字体大小
\usepackage{anysize}
\usepackage{indentfirst}  % 首段缩进
\usepackage{booktabs}   % 使用\multicolumn
\usepackage{multirow}    % 使用\multirow
\usepackage{graphicx} 
\usepackage{wrapfig}
\usepackage{xcolor}
\usepackage{titlesec}     % 改变标题样式
\usepackage{enumitem}

\renewcommand{\vec}[1]{\boldsymbol{#1}}
\newcommand{\me}{\mathrm{e}}
\newcommand{\mi}{\mathrm{i}}
\newcommand{\dif}{\mathrm{d}}
\newcommand{\tabincell}[2]{\begin{tabular}{@{}#1@{}}#2\end{tabular}}

\def\kpc{{\rm kpc}}
\def\km{{\rm km}}
\def\cm{{\rm cm}}
\def\TeV{{\rm TeV}}
\def\GeV{{\rm GeV}}
\def\MeV{{\rm MeV}}
\def\GV{{\rm GV}}
\def\MV{{\rm MV}}
\def\yr{{\rm yr}}
\def\s{{\rm s}}
\def\ns{{\rm ns}}
\def\GHz{{\rm GHz}}
\def\muGs{{\rm \mu Gs}}
\def\arcsec{{\rm arcsec}}
\def\K{{\rm K}}
\def\microK{\mu{\rm K}}
\def\sr{{\rm sr}}
\newcolumntype{p}{D{,}{\pm}{-1}}

\renewcommand{\figurename}{Fig.}
\renewcommand{\tablename}{Tab.}

\renewcommand{\arraystretch}{1.5}

\setlength{\parindent}{0pt}  %取消每段开头的空格

\title{狭义相对论——动力学}
\author{}
\date{\today}
\begin{document}

\maketitle

\section{相对论能动量}
粒子的\textcolor{red}{动量}
\begin{equation}
\vec{p} = \gamma m \vec{u} = \frac{m\vec{u}}{\sqrt{1-\frac{u^2}{c^2}}}
\end{equation}
粒子的\textcolor{red}{能量} 
\begin{equation}
E = \gamma m c^2 = \frac{m c^2}{\sqrt{1-\frac{u^2}{c^2}}}
\end{equation}
粒子的\textcolor{red}{静能}
\begin{equation}
E = m c^2
\end{equation}
粒子的\textcolor{red}{动能}
\begin{equation}
T = E - mc^2
\end{equation}
\begin{equation}
E^2 = p^2 c^2 +m^2 c^4
\end{equation}
从任何小于$c$的有限速率增加到或者超过光速是不可能的;若粒子生来就具有大于光速的速率,并不破坏这一说法;

一个自由粒子的能量、动量与速度的关系:
\begin{equation}
\vec{u} = \frac{c^2 \vec{p}}{E}
\end{equation}

\section{能动量四维矢量的洛伦兹变换}
the Lorentz transformation for momentum and energy
\begin{eqnarray}
\nonumber p^{\prime}_x &=& \gamma(p_x -vE/c^2) ~, \\
\nonumber p^{\prime}_y &=& p_y ~, \\
\nonumber p^{\prime}_z &=& p_z ~, \\
E^{\prime} &=& \gamma(E -vp_x)
\end{eqnarray}
the inverse transformation is
\begin{eqnarray}
\nonumber p_x &=& \gamma(p^{\prime}_x +vE^{\prime}/c^2) ~, \\
\nonumber p_y &=& p^{\prime}_y ~, \\
\nonumber p_z &=& p^{\prime}_z ~, \\
E &=& \gamma(E^{\prime} +vp^{\prime}_x) 
\end{eqnarray}

洛伦兹不变式
\begin{equation}
E^2 -\vec{p}^2 c^2 = {E^{\prime}}^2 -\vec{p}^2 c^2 = {\rm Const.}
\end{equation}
若系统为单一粒子,其静质量为$m_0$,将$x^{\prime}, y^{\prime}, z^{\prime}$系统固定在$m_0$上,即$m_0$粒子在$x^{\prime}, y^{\prime}, z^{\prime}$系内是静止的,其动量$\vec{p}^{\prime} = 0$,则不变式的常量为$m_0^2$
\begin{equation}
E^2 -\vec{p}^2 c^2 = {E^{\prime}}^2 = m_0^2
\end{equation}
对于多粒子系统,其总能量和总动量组成四矢量,在彼此间无相互作用时,洛伦兹不变式为
\begin{eqnarray}
\left(\sum_i E_i  \right)^2 -\left(\sum_i \vec{p}_i  \right)^2 = {\rm Const.}
\end{eqnarray}
在质心系统中,$\sum_i \vec{p}_i^* = 0$,此常量等于$E_{\rm cm}^2 = {E^*}^2 = S$。\textcolor{red}{在一般情况下,此常量并不等于系统的总质量$M_0$}。但若这个多粒子系统是\textcolor{red}{由单一母粒子$M_0$衰变产生的},则这一常量等于\textcolor{red}{母粒子总静止能量的平方$M_0^2$},$M_0$称为这个\textcolor{red}{多粒子系统的不变质量}。


\section{分布函数的变换}


\section{碰撞的相对论运动学}
把质心系适当地推广为洛伦兹参照系,在该参照系内,所有粒子的总的空间线动量对于$0$;这样一个洛伦兹参照系总是能够找到的,因为一个质点系的总四维动量矢量是类时的;

总的四维动量
\begin{equation}
P_{\mu} = \sum_r p_{r\mu}
\end{equation}
\begin{eqnarray}
\nonumber P_{\mu} P_{\mu} = \sum_{r,s} p_{r\mu} p_{s\mu} &=& -\sum_{r} m_r^2 c^2 + \sum_{r \neq s} p_{r\mu} p_{s\mu} ~, \\
&=& -\sum_{r} m_r^2 c^2 + \sum_{r \neq s} m_r m_s \gamma_r \gamma_s (c^2 -\vec{v}_r \cdot \vec{v}_s)
\end{eqnarray}
由于实物粒子的速度始终小于$c$,所以$P_{\mu}$的平方始终是负值;一定会有某种洛伦兹坐标系,它保证被变换矢量$P_{\mu}$的空间分量全部为$0$,这种坐标系称为\textcolor{red}{动量中心系},或者不太严格地称为\textcolor{red}{质心系},“C-O-M”系;

碰撞前后总的四维动量矢量守恒;意味着空间线动量的守恒和总能量(包括静止质能)的守恒;变换到动量中心系和起始于动量中心系的洛伦兹变换;同时构成一些在所有洛伦兹系统内都有相同数值的洛伦兹不变量(世量标量);

考虑由两个粒子引起的反应,它产生出另一组质量为$m_r (r = 3, 4, 5, \cdots)$的粒子,在“C-O-M”系内,变换后的总动量$P^{\prime}_{\mu}$具有零值空间分量以及一个第四分量$iT^{\prime}/c$。把“C-O-M”系看作是质量为$M=T^{\prime}/c^2$的复合粒子的固有系统或静止系统;$P_{\mu}$量值的平方在所有洛伦兹系统内必定是不变量(并在反应中守恒)。因此
\begin{equation}
P_{\mu} P_{\mu} = P^{\prime}_{\mu}P^{\prime}_{\mu} = -\frac{{E^{\prime}}^2}{c^2} = -M^2 c^2
\end{equation}
对原有粒子,
\begin{equation}
P_{\mu} P_{\mu} = -(m_1^2 +m_2^2) c^2 + 2p_{1\mu} p_{2\mu}
\end{equation}
在“C-O-M”系内的能量或等效质量$M$,可依据入射粒子表达成
\begin{equation}
{E^{\prime}}^2 \equiv M^2 c^4 = (m_1^2 +m_2^2) c^4 +2(E_1 E_2 -c^2 \vec{p}_1 \cdot \vec{p}_2 )
\end{equation}
假设有一个粒子,如粒子2,在实验室系下是静止的,则$\vec{p}_2 = 0$和$T_2 = m_2 c^2$,C-O-M能量为
\begin{equation}
{E^{\prime}}^2 \equiv M^2 c^4 = (m_1^2 +m_2^2) c^4 +2m_2 c^2 E_1 =  (m_1^2 +m_2^2)^2 c^4 +2m_2 c^2 T_1
\end{equation}
C-O-M系内的有效能量仅随入射动能缓慢增加,甚至在运动动能远大于静止质能的“超相对论”区域内,$E^{\prime}$也仅按$T_1$的平方根增加;

在C-O-M系中适用的小数量入射能量按比例增加的效应,是用阈能表示的;有可能引起反应(除了弹性散射)的最低能量是反应物静止在C-O-M系内时的能量。任何有限动能都要求一个较高的$E^{\prime}$,或者说要求一个较高的入射能量;反应后,C-O-M系内的总的四维动量$P_{\mu}^{\prime \prime}$,它在阈值处的量值决定于
\begin{equation}
P_{\mu}^{\prime \prime} P_{\mu}^{\prime \prime} = -c^4 \left(\sum_r m_r \right)^2 
\end{equation}
对于静止靶,阈值处入射运动能量决定于
\begin{equation}
\frac{T_1}{m_1 c^2} = \frac{\left(\sum\limits_r m_r \right)^2 -(m_1 +m_2)^2}{2m_1 m_2}
\end{equation}
若反应的$Q$定义为
\begin{equation}
Q  = \sum_r m_r -(m_1 +m_2)
\end{equation}
则
\begin{equation}
\frac{T_1}{m_1 c^2} = \frac{Q^2 +2Q(m_1 +m_2)}{2m_1 m_2}
\end{equation}

实验室系内反应产物在阈值处的能量;C-O-M系是质量$M$的静止系统,它的$P^{\prime}_4 = i M c$,在任何其它系统中,四维矢量的第四分量是$P_4 = i M c \gamma$,但在实验室系内则为
\begin{equation}
P_4 = \frac{i}{c} (E_1 +E_2) = \frac{i}{c} (E_1 +m_2 c^2)
\end{equation}
后一式只对静止靶粒子成立;因此,C-O-M系统相对于实验室系的运动应使
\begin{equation}
\gamma = \frac{E_1 +m_2 c^2}{Mc^2}
\end{equation}
但在阈值处,所有的反应产物在C-O-M系内都是静止的,因此$M = \sum\limits_r m_r$,这时有
\begin{equation}
\gamma = \frac{T_1 +(m_1 +M_2) c^2}{\sum\limits_r m_r c^2}   ~~~~~\text{阈值}
\end{equation}
在实验室系内,第$s$个反应产物的动能是
\begin{equation}
T_s = m_s c^2 (\gamma -1)
\end{equation}




\section{粒子间的弹性碰撞}













\end{document}