\documentclass[11pt,a4paper]{article}
%\usepackage{fontspec, xunicode, xltxtra}  
%\setmainfont{Hiragino Sans GB}  
\usepackage{xeCJK}
%\setCJKmainfont[BoldFont=STZhongsong, ItalicFont=STKaiti]{STSong}
%\setCJKsansfont[BoldFont=STHeiti]{STXihei}
%\setCJKmonofont{STFangsong}

%使用Xelatex编译

% 设置页面
%==================================================
\linespread{2} %行距
% \usepackage[top=1in,bottom=1in,left=1.25in,right=1.25in]{geometry}
% \headsep=2cm
% \textwidth=16cm \textheight=24.2cm
%==================================================

% 其它需要使用的宏包
%==================================================
\usepackage[colorlinks,linkcolor=blue,anchorcolor=red,citecolor=green,urlcolor=blue]{hyperref} 
\usepackage{tabularx}
\usepackage{authblk}         % 作者信息
\usepackage{algorithm}     % 算法排版
\usepackage{amsmath}     % 数学符号与公式
\usepackage{amsfonts}     % 数学符号与字体
\usepackage{mathrsfs}      % 花体
\usepackage{amssymb}
\usepackage[framemethod=TikZ]{mdframed}

\usepackage{graphicx} 
\usepackage{graphics}
\usepackage{color}
\usepackage{xcolor}
\usepackage{tcolorbox}
\usepackage{lipsum}
\usepackage{empheq}

\usepackage{fancyhdr}       % 设置页眉页脚
\usepackage{fancyvrb}       % 抄录环境
\usepackage{float}              % 管理浮动体
\usepackage{geometry}     % 定制页面格式
\usepackage{hyperref}       % 为PDF文档创建超链接
\usepackage{lineno}          % 生成行号
\usepackage{listings}        % 插入程序源代码
\usepackage{multicol}       % 多栏排版
%\usepackage{natbib}         % 管理文献引用
\usepackage{rotating}       % 旋转文字,图形,表格
\usepackage{subfigure}    % 排版子图形
\usepackage{titlesec}       % 改变章节标题格式
\usepackage{moresize}   % 更多字体大小
\usepackage{anysize}
\usepackage{indentfirst}  % 首段缩进
\usepackage{booktabs}   % 使用\multicolumn
\usepackage{multirow}    % 使用\multirow

\usepackage{wrapfig}
\usepackage{titlesec}     % 改变标题样式
\usepackage{enumitem}
\usepackage{aas_macros}
\usepackage{bigints}

\renewcommand{\vec}[1]{\boldsymbol{#1}}
\newcommand{\me}{\mathrm{e}}
\newcommand{\mi}{\mathrm{i}}
\newcommand{\dif}{\mathrm{d}}
\newcommand{\tabincell}[2]{\begin{tabular}{@{}#1@{}}#2\end{tabular}}

\def\kpc{{\rm kpc}}
\def\km{{\rm km}}
\def\cm{{\rm cm}}
\def\TeV{{\rm TeV}}
\def\GeV{{\rm GeV}}
\def\MeV{{\rm MeV}}
\def\GV{{\rm GV}}
\def\MV{{\rm MV}}
\def\yr{{\rm yr}}
\def\s{{\rm s}}
\def\ns{{\rm ns}}
\def\GHz{{\rm GHz}}
\def\muGs{{\rm \mu Gs}}
\def\arcsec{{\rm arcsec}}
\def\K{{\rm K}}
\def\microK{\mu{\rm K}}
\def\sr{{\rm sr}}
\newcolumntype{p}{D{,}{\pm}{-1}}

\renewcommand{\figurename}{Fig.}
\renewcommand{\tablename}{Tab.}

\renewcommand{\arraystretch}{1.5}

\setlength{\parindent}{0pt}  %取消每段开头的空格

\newcounter{theo}[section]\setcounter{theo}{0}
\renewcommand{\thetheo}{\arabic{section}.\arabic{theo}}
\newenvironment{theo}[2][]{%
\refstepcounter{theo}%
\ifstrempty{#1}%
{\mdfsetup{%
frametitle={%
\tikz[baseline=(current bounding box.east),outer sep=0pt]
\node[anchor=east,rectangle,fill=blue!20]
{\strut Theorem~\thetheo};}}
}%
{\mdfsetup{%
frametitle={%
\tikz[baseline=(current bounding box.east),outer sep=0pt]
\node[anchor=east,rectangle,fill=blue!20]
{\strut Theorem~\thetheo:~#1};}}%
}%
\mdfsetup{innertopmargin=10pt,linecolor=blue!20,%
linewidth=2pt,topline=true,%
frametitleaboveskip=\dimexpr-\ht\strutbox\relax
}
\begin{mdframed}[]\relax%
\label{#2}}{\end{mdframed}}

\newcommand*\widefbox[1]{\fbox{\hspace{2em}#1\hspace{2em}}}


\title{微振动}
\author{}
\date{\today}
\begin{document}

\maketitle

\section{一维自由振动}
\cite{2007理论物理学教程} 在稳定平衡位置附近的运动称为微振动。稳定平衡位置是指势能$U(q)$取极小值的位置,偏离该位置会导致产生力$-\dfrac{\dif U}{\dif q}$,它力图使系统返回平衡位置。用$q_0$表示广义坐标$q$在平衡位置的值。在偏离平衡位置很小的情况下,在$U(q)-U(q_0)$按$q-q_0$的幂次展开,
\begin{equation*}
U(q) - U(q_0) \approx \dfrac{k}{2} (q-q_0)^2 ~,
\end{equation*}
$k$是二阶导数$U^{\prime \prime}(q)$在$q =q_0$处的值,是正数。





\section{强迫振动}
\cite{2007理论物理学教程} 可变外力场作用下系统的振动,称为\textcolor{red}{强迫振动}。


强迫力是频率为$\gamma$的简单时间周期函数,即
\begin{equation}
F(t) = f \cos(\gamma t +\beta) ~.
\end{equation}

\begin{equation}
x = a\cos (\omega t +\alpha) +\dfrac{f}{m(\omega^2 -\gamma^2)} \cos(\gamma t +\beta) ~.
\end{equation}
任意积分常数$a$和$\alpha$由初始条件确定。在周期性强迫力作用下,系统的运动是两个振动的合成,两个振动的频率分别为系统的固有频率$\omega$和强迫力的频率$\gamma$。以上不适用于共振情况,即强迫力的频率$\gamma$与固有频率$\omega$相等。
\begin{equation}
x = a\cos (\omega t +\alpha) +\dfrac{f}{m(\omega^2 -\gamma^2)} [ \cos(\gamma t +\beta) -\cos(\omega t +\beta)] ~.
\end{equation}
当$\gamma \rightarrow \omega$,可得
\begin{equation}
x = a\cos (\omega t +\alpha) +\dfrac{f}{2m \omega} t \sin(\omega t +\beta) ~.
\end{equation}
在共振情况下,振动的振幅随时间线性增长。





\section{多自由度系统振动}
\cite{2007理论物理学教程} 






\section{分子振动}
\cite{2007理论物理学教程} 













\section{阻尼振动}
\cite{2007理论物理学教程} 



若速度足够小,可以将摩擦力按速度的幂次展开。作用在广义坐标为$x$的一维微振动系统的广义摩擦力$f_{\rm fr} $写成
\begin{equation}
f_{\rm fr} = -\alpha \dot{x} ~,
\end{equation}
$\alpha$为正的系数,

\begin{align}
& m \ddot{x} = -kx -\alpha \dot{x} ~, \\
& \dfrac{k}{m} = \omega_0^2 ~, \dfrac{\alpha}{m} = 2 \lambda ~.
\end{align}
$\omega_0$是没有摩擦力时系统自由振动的频率。$\lambda$称为\textcolor{red}{阻尼系数},或\textcolor{red}{阻尼衰减率}。$\lambda T$($T = 2\pi/\omega$是周期)称为\textcolor{red}{对数阻尼衰减率}。

\begin{equation}
\ddot{x} + 2\lambda \dot{x} +\omega_0^2 x = 0 ~.
\end{equation}
假设$x = e^{rt}$可得关于$r$的特征方程
\begin{equation*}
r^2 +2\lambda r + \omega_0^2 = 0 ~.
\end{equation*}
通解为
\begin{align*}
& x = c_1 e^{r_1 t} +c_2 e^{r_2 t} ~, \\
& r_{1, 2} = -\lambda \pm \sqrt{\lambda^2 -\omega_0^2} ~.
\end{align*}








\section{有摩擦的强迫振动}
\cite{2007理论物理学教程} 



















\section{参变振动}
\cite{2007理论物理学教程} 















\section{非简谐振动}
\cite{2007理论物理学教程} 









\section{非线性振动中的共振}
\cite{2007理论物理学教程} 







\section{快速振动场中的运动}
\cite{2007理论物理学教程} 























































































%%%%%%%%%%%%%%%%%%%%%%%%%%%%%%%%%%%%%%%%%%%%%%%%%%%%%%%%%%%%%%%%%%%%%%
\bibliographystyle{unsrt_update}
\bibliography{ref}
%%%%%%%%%%%%%%%%%%%%%%%%%%%%%%%%%%%%%%%%%%%%%%%%%%%%%%%%%%%%%%%%%%%%%%


\end{document}