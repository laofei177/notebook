\documentclass[12pt,a4paper]{article}
%\usepackage{fontspec, xunicode, xltxtra}  
%\setmainfont{Hiragino Sans GB}  
%\usepackage{xeCJK}
%\setCJKmainfont[BoldFont=STZhongsong, ItalicFont=STKaiti]{STSong}
%\setCJKsansfont[BoldFont=STHeiti]{STXihei}
%\setCJKmonofont{STFangsong}

%使用Xelatex编译

% 设置页面
%==================================================
\linespread{2} %行距
% \usepackage[top=1in,bottom=1in,left=1.25in,right=1.25in]{geometry}
% \headsep=2cm
% \textwidth=16cm \textheight=24.2cm
%==================================================

% 其它需要使用的宏包
%==================================================
\usepackage[colorlinks,linkcolor=blue,anchorcolor=red,citecolor=green,urlcolor=blue]{hyperref} 
\usepackage{tabularx}
\usepackage{authblk}         % 作者信息
\usepackage{algorithm}     % 算法排版
\usepackage{amsmath}     % 数学符号与公式
\usepackage{amsfonts}     % 数学符号与字体
\usepackage{mathrsfs}      % 花体
\usepackage{graphics}
\usepackage{xcolor,amsmath}
\usepackage{color}
\usepackage{fancyhdr}       % 设置页眉页脚
\usepackage{fancyvrb}       % 抄录环境
\usepackage{float}              % 管理浮动体
\usepackage{geometry}     % 定制页面格式
\usepackage{hyperref}       % 为PDF文档创建超链接
\usepackage{lineno}          % 生成行号
\usepackage{listings}        % 插入程序源代码
\usepackage{multicol}       % 多栏排版
\usepackage{natbib}         % 管理文献引用
\usepackage{rotating}       % 旋转文字,图形,表格
\usepackage{subfigure}    % 排版子图形
\usepackage{titlesec}       % 改变章节标题格式
\usepackage{moresize}   % 更多字体大小
\usepackage{anysize}
\usepackage{indentfirst}  % 首段缩进
\usepackage{booktabs}   % 使用\multicolumn
\usepackage{multirow}    % 使用\multirow
\usepackage{graphicx} 
\usepackage{wrapfig}
\usepackage{xcolor}
\usepackage{titlesec}     % 改变标题样式
\usepackage{enumitem}


\newcommand{\myvec}[1]%
   {\stackrel{\raisebox{-2pt}[0pt][0pt]{\small$\rightharpoonup$}}{#1}}  %矢量符号
\renewcommand{\vec}[1]{\boldsymbol{#1}}
\newcommand{\me}{\mathrm{e}}
\newcommand{\mi}{\mathrm{i}}
\newcommand{\dif}{\mathrm{d}}
\newcommand{\tabincell}[2]{\begin{tabular}{@{}#1@{}}#2\end{tabular}}

\def\kpc{{\rm kpc}}
\def\km{{\rm km}}
\def\cm{{\rm cm}}
\def\TeV{{\rm TeV}}
\def\GeV{{\rm GeV}}
\def\MeV{{\rm MeV}}
\def\GV{{\rm GV}}
\def\MV{{\rm MV}}
\def\yr{{\rm yr}}
\def\s{{\rm s}}
\def\ns{{\rm ns}}
\def\GHz{{\rm GHz}}
\def\muGs{{\rm \mu Gs}}
\def\arcsec{{\rm arcsec}}
\def\K{{\rm K}}
\def\microK{\mu{\rm K}}
\def\sr{{\rm sr}}
\newcolumntype{p}{D{,}{\pm}{-1}}

\renewcommand{\figurename}{Fig.}
\renewcommand{\tablename}{Tab.}

\renewcommand{\arraystretch}{1.5}

\setlength{\parindent}{0pt}  %取消每段开头的空格




\title{Special Relativity}
\author{}
\date{\today}
\begin{document}

\maketitle


\section{Invariance of Electric Charge}
Lorentz force equation for a particle of charge $q$,
\begin{equation}
\frac{\dif \vec{p}}{\dif t} = q\left(\vec{E} +\frac{\vec{v}}{c} \times \vec{B} \right)
\end{equation}
$\vec{p}$ transforms as the space part of the $4$-vector of energy and momentum, 
\begin{equation*}
p^\alpha = (p_0, \vec{p}) = m (U_0, \vec{U})
\end{equation*}
where $p_0 = E/c$ and $U^\alpha$ is the $4$-velocity. 

\section{Covariance of Electrodaynamics}
the electric and magnetic fields are the elements of a \textcolor{red}{second-rank, antisymmetric field-strength tensor},
\begin{equation}
F^{\alpha \beta} = \partial^{\alpha} A^{\beta} -\partial^{\beta} A^{\alpha}
\end{equation}
Explicitly, the \textcolor{red}{field-strength tensor} is 
\renewcommand{\arraystretch}{1}
\begin{gather*}
F^{\alpha \beta} = 
\begin{pmatrix} 
0 &-E_x  &-E_y &-E_z \\
E_x &0 &-B_z &B_y \\
E_y &B_z &0 &-B_x \\
E_z &-B_y &B_x &0
\end{pmatrix}
\end{gather*}
\renewcommand{\arraystretch}{1.5}
the field-strength tensor with two \textcolor{red}{covariant indices}
\renewcommand{\arraystretch}{1}
\begin{gather*}
F_{\alpha \beta} = \mathrm{g}_{\alpha \gamma} F^{\gamma \delta} \mathrm{g}_{\delta \beta} =
\begin{pmatrix} 
0 &E_x  &E_y &E_z \\
-E_x &0 &-B_z &B_y \\
-E_y &B_z &0 &-B_x \\
-E_z &-B_y &B_x &0
\end{pmatrix}
\end{gather*}
\renewcommand{\arraystretch}{1.5}
The elements of $F_{\alpha \beta}$ are obtained from $F^{\alpha \beta}$ by putting $\vec{E} \rightarrow -\vec{E}$.

The \textcolor{red}{dual field-strength tensor $\mathscr{F}^{\alpha \beta}$}
\renewcommand{\arraystretch}{1}
\begin{gather*}
\mathscr{F}^{\alpha \beta} = \frac{1}{2} \epsilon^{\alpha\beta\gamma\delta} F_{\gamma \delta} =
\begin{pmatrix} 
0     & -B_x  & -B_y & -B_z \\
B_x &      0  &  E_z & -E_y \\
B_y & -E_z  &      0 &  E_x \\
B_z &  E_y  & -E_x &      0
\end{pmatrix}
\end{gather*}
\renewcommand{\arraystretch}{1.5}
The elements of the dual tensor $\mathscr{F}^{\alpha \beta}$ are obtained from $F^{\alpha \beta}$ by putting $\vec{E} \rightarrow \vec{B}$ and $\vec{B} \rightarrow -\vec{E}$.
\textcolor{red}{$\epsilon^{\alpha\beta\gamma\delta}$} is the \textcolor{red}{totally antisymmetric fourth-rank tensor}:
\begin{equation}
\epsilon^{\alpha\beta\gamma\delta} =
\begin{cases}
+1 & \text{for} \alpha = 0, \beta = 1, \gamma = 2, \delta = 3, \text{and any even permutation} \\
-1  & \text{for any odd permutation} \\
0   & \text{if any two indices are equal}
\end{cases}
\end{equation}
The nonvanishing elements all have one time and three(different) space indices and that $\epsilon_{\alpha\beta\gamma\delta} = -\epsilon^{\alpha\beta\gamma\delta}$. The tensor $\epsilon^{\alpha\beta\gamma\delta}$ is a \textcolor{red}{pseudotensor under spatial inversions}.
 
Write Maxwell equations in an explicitly covariant form:
\begin{eqnarray*}
\nabla \cdot \vec{E} &=& 4\pi \rho ~\\
\nabla \times \vec{B} -\frac{1}{c} \frac{\partial \vec{E}}{\partial t} &=& \frac{4\pi}{c} \vec{J}
\end{eqnarray*}
In terms of $F^{\alpha \beta}$ and the 4-current $J^{\alpha}$,
\begin{equation}
\color{red} \partial_{\alpha} F^{\alpha \beta} = \frac{4\pi}{c} J^{\beta}
\end{equation}

\begin{eqnarray*}
\nabla \cdot \vec{B} &=& 0 ~ \\
\nabla \times \vec{E} + \frac{1}{c} \frac{\partial \vec{B}}{\partial t} &=& 0
\end{eqnarray*}
{\color{red} 
\begin{equation}
\partial_{\alpha} \mathscr{F}^{\alpha \beta} = 0
\end{equation} }
or in terms of $F^{\alpha \beta}$, rather than $\mathscr{F}^{\alpha \beta}$,
{\color{red}
\begin{equation}
\partial_{\alpha} F_{\beta \gamma} +\partial_{\beta} F_{\gamma \alpha} +\partial_{\gamma} F_{\alpha \beta} = 0
\end{equation} }

\section{transformation of electromagnetic fields}
The fields $\vec{E}^{\prime}$ and $\vec{B}^{\prime}$ in one inertial frame $K^{\prime}$ can be  expressed in terms of $\vec{E}$ and $\vec{B}$ in another inertial frame $K$ according to
\begin{equation}
F^{\alpha \beta} = \frac{\partial x^{\prime \alpha}}{\partial x^{\gamma}}  \frac{\partial x^{\prime \beta}}{\partial x^{\delta}} F^{\gamma \delta} 
\end{equation}
i.e.
\begin{equation}
F^{\prime} = A F \tilde{A}
\end{equation}
$A$ is the Lorentz transformation matrix.

A boost along the $x_1$ axis with speed $c\beta$ from the unprinted frame to the primed frame, the explicit equations of transformation are 
\begin{subequations}
\renewcommand{\theequation}{\theparentequation.\alph{equation}}
\begin{center}
\begin{minipage}[b]{9em}
\begin{equation*}
E_1^{\prime} = E_1
\end{equation*}
\end{minipage}\qquad
\begin{minipage}[b]{9em}
\begin{equation*}
B_1^{\prime} = B_1
\end{equation*}
\end{minipage} \\
\begin{minipage}[b]{9em}
\begin{equation*}
E_2^{\prime} = \gamma(E_2 -\beta B_3)
\end{equation*}
\end{minipage}\qquad
\begin{minipage}[b]{9em}
\begin{equation*}
B_2^{\prime} = \gamma(B_2 +\beta E_3)
\end{equation*}
\end{minipage} \\
\begin{minipage}[b]{9em}
\begin{equation*}
E_3^{\prime} = \gamma(E_3 +\beta B_2)
\end{equation*}
\end{minipage}\qquad
\begin{minipage}[b]{9em}
\begin{equation*}
B_3^{\prime} = \gamma(B_3 -\beta E_2)
\end{equation*}
\end{minipage}
\end{center}
\end{subequations}
For a general Lorentz transformation from $K$ to a system $K^{\prime}$ moving with velocity $\vec{v}$ relative to $K$, the transformation of the fields is
{\color{red}
\begin{eqnarray}
\vec{E}^{\prime} &=& \gamma(\vec{E} +\vec{\beta} \times \vec{B}) -\frac{\gamma^2}{\gamma+1} \vec{\beta}(\vec{\beta} \cdot \vec{E}) ~,\\
\vec{B}^{\prime} &=& \gamma(\vec{B} -\vec{\beta} \times \vec{E}) -\frac{\gamma^2}{\gamma+1} \vec{\beta}(\vec{\beta} \cdot \vec{B})
\end{eqnarray} }
$\vec{E}$ and $\vec{B}$ have no independent existence. A purely electric or magnetic field in one coordinate system will appear as a mixture of electric and magnetic fields in another coordinate frame.


































\end{document}