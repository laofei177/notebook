\documentclass[12pt,a4paper]{article}
%\usepackage{fontspec, xunicode, xltxtra}
%\setmainfont{Hiragino Sans GB}
%\usepackage{xeCJK}
%\setCJKmainfont[BoldFont=STZhongsong, ItalicFont=STKaiti]{STSong}
%\setCJKsansfont[BoldFont=STHeiti]{STXihei}
%\setCJKmonofont{STFangsong}

%使用Xelatex编译

% 设置页面
%==================================================
\linespread{2} %行距
% \usepackage[top=1in,bottom=1in,left=1.25in,right=1.25in]{geometry}
% \headsep=2cm
% \textwidth=16cm \textheight=24.2cm
%==================================================

% 其它需要使用的宏包
%==================================================
\usepackage[colorlinks,linkcolor=blue,anchorcolor=red,citecolor=green,urlcolor=blue]{hyperref}
\usepackage{tabularx}
\usepackage{authblk}         % 作者信息
\usepackage{algorithm}     % 算法排版
\usepackage{amsmath}     % 数学符号与公式
\usepackage{amsfonts}     % 数学符号与字体
\usepackage{mathrsfs}      % 花体
\usepackage{graphics}
\usepackage{color}
\usepackage{fancyhdr}       % 设置页眉页脚
\usepackage{fancyvrb}       % 抄录环境
\usepackage{float}              % 管理浮动体
\usepackage{geometry}     % 定制页面格式
\usepackage{hyperref}       % 为PDF文档创建超链接
\usepackage{lineno}          % 生成行号
\usepackage{listings}        % 插入程序源代码
\usepackage{multicol}       % 多栏排版
%\usepackage{natbib}         % 管理文献引用
\usepackage{rotating}       % 旋转文字,图形,表格
\usepackage{subfigure}    % 排版子图形
\usepackage{titlesec}       % 改变章节标题格式
\usepackage{moresize}   % 更多字体大小
\usepackage{anysize}
\usepackage{indentfirst}  % 首段缩进
\usepackage{booktabs}   % 使用\multicolumn
\usepackage{multirow}    % 使用\multirow
\usepackage{graphicx}
\usepackage{wrapfig}
\usepackage{xcolor}
\usepackage{titlesec}     % 改变标题样式
\usepackage{enumitem}

\newcommand{\myvec}[1]%
   {\stackrel{\raisebox{-2pt}[0pt][0pt]{\small$\rightharpoonup$}}{#1}}  % 矢量符号
\renewcommand{\vec}[1]{\boldsymbol{#1}}
\newcommand{\me}{\mathrm{e}}
\newcommand{\mi}{\mathrm{i}}
\newcommand{\dif}{\mathrm{d}}
\newcommand{\tabincell}[2]{\begin{tabular}{@{}#1@{}}#2\end{tabular}}

\def\kpc{{\rm kpc}}
\def\km{{\rm km}}
\def\cm{{\rm cm}}
\def\TeV{{\rm TeV}}
\def\GeV{{\rm GeV}}
\def\MeV{{\rm MeV}}
\def\GV{{\rm GV}}
\def\MV{{\rm MV}}
\def\yr{{\rm yr}}
\def\s{{\rm s}}
\def\ns{{\rm ns}}
\def\GHz{{\rm GHz}}
\def\muGs{{\rm \mu Gs}}
\def\arcsec{{\rm arcsec}}
\def\K{{\rm K}}
\def\microK{\mu{\rm K}}
\def\sr{{\rm sr}}
\newcolumntype{p}{D{,}{\pm}{-1}}

\renewcommand{\figurename}{Fig.}
\renewcommand{\tablename}{Tab.}

\renewcommand{\arraystretch}{1.5}

\setlength{\parindent}{0pt}  %取消每段开头的空格

\title{Synchrotron Radiation}
\author{}
\date{\today}
\begin{document}

\maketitle

\cite{1996PhyU...39..155G} A particle (of charge $e$ and mass $m$) with the total energy $E$ moves in a homogeneous magnetic field of strength $\vec{H}$ along a spiral with the angular frequency
\begin{equation}
\omega_H = \dfrac{|e|H}{mc} \dfrac{mc^2}{E} = 1.76 \times 10^7 H \dfrac{mc^2}{E} ~,
\end{equation}
the particles are electrons, and the field $H$ is measured in oersteds (gausses).

If the particle is ultrarelativistic, it emits waves in the direction of its own velocity within the angle on the order of $mc^2/E$. If the angle $\chi$ between $\vec{v}$ and $\vec{H}$ is not too small, i.e. $\chi \gg mc^2/E$, the particle emits waves with many frequencies which are overtones of $\omega_H/\sin^2 \chi$. At $E \gg mc^2$, the spectrum is practically continuous, and the radiation intensity maximum corresponds to a frequency
\begin{align}
\nonumber \nu_{\rm m} = \dfrac{\omega_m}{2\pi} &= 0.07 \dfrac{|e| H_\perp}{mc} \left(\dfrac{E}{mc^2} \right)^2 \\
\nonumber &= 1.2\times 10^6 H_\perp \left(\dfrac{E}{mc^2} \right)^2 \\
\nonumber &= 1.8\times 10^{18} H_\perp [E(\rm erg)]^2 \\
&= 4.6\times 10^{-6} H_\perp [E(\rm eV)]^2 ~Hz ~, 
\end{align}
\begin{equation}
\hbar \omega_{\rm m} = 1.9 \times 10^{-20} H_\perp [E(\rm eV)]^2 ~eV ~,
\end{equation}
where $H_\perp = H \sin \chi$ is a component of the field $\vec{H}$ perpendicular to particle velocity $\vec{v}$.

In a typical interstellar field $H \sim H_\perp \sim 3 \times 10^{-6}$ Oe for
particles with the energy $E \sim 10^9 - 10^{11}$ eV, the frequency $\nu_{\rm m}$ falls in the range of $10^7 - 10^{11}$ Hz which corresponds to the wavelength $\lambda_{\rm m} =c/\nu_{\rm m} \sim 30 ~{\rm m} \pm 0.3 ~{\rm cm}$. Thus, the electron component of cosmic rays with $E > 10^9$ eV largely radiates in the radio-frequency range when in interstellar fields.

At frequency $\nu_{\rm m}$, the spectral density of the radioemission power is
\begin{align}
p_{\rm m} \equiv p(\nu_{\rm m}) = 1.6 \times \dfrac{|e|^3 H_\perp}{mc^2} = 2.16 \times 10^{-22} H_\perp ~\rm \dfrac{erg}{s \cdot  Hz} ~.
\end{align}
If we consider a region in which emitting electrons at a concentration $N_{\rm e}$ are isotropically distributed by velocity, the corresponding emissivity (the spectral power of radiation per unit volume and unit solid angle) is given by
\begin{align}
\varepsilon_{\nu_{\rm m}} = \dfrac{p_{\rm m} N_{\rm e}}{4\pi} = 1.7 \times 10^{-23} H_\perp N_{\rm e} ~\rm \dfrac{erg}{cm^3 \cdot s \cdot sr \cdot Hz} ~.
\end{align}
where $H_\perp$ is the mean value of a field component perpendicular to the particle velocity for the radiating volume. The maximum radiation intensity for monochromatic electrons is
\begin{align}
J_{\nu_{\rm m}} = \int \varepsilon_{\nu_{\rm m}} \dif l = 1.7 \times 10^{-23} H_\perp N_{\rm e} L ~\rm \dfrac{erg}{cm^2 \cdot s \cdot sr \cdot Hz} ~,
\end{align}
where $L$ is the characteristic size of the region emitting radiowaves along the line of sight (i.e. $N_{\rm e}L = \int N_{\rm e} \dif l$).

In the case of a discrete source of radioemission (e.g., a supernova remnant) which is at a distance R from us, the radiation flux is
\begin{align}
\Phi_{\nu_{\rm m}} = \dfrac{p_{\rm m} N_{\rm e} V}{4\pi R^2} = 1.7 \times 10^{-23} \dfrac{H_\perp N_{\rm e} V}{R^2} ~\rm \dfrac{erg}{cm^2 \cdot s \cdot sr \cdot Hz} ~,
\end{align}
$V$ is the source volume.

The electron energy $E = 0.75 \times 10^{-9} \sqrt{\nu_{\rm m} /H_\perp}$ erg. The total electron energy in the source is
\begin{align}
W_{\rm e} = EN_{\rm e} V = 4.4 \times 10^{13} \dfrac{\nu_{\rm m}^{1/2} \Phi_{\nu_{\rm m}} R^2}{H_\perp^{3/2}} ~.
\end{align}
Consider the electron spectrum of the power-law type, when the concentration of electrons in the interval $\dif E$ has the form
\begin{align}
N_{\rm e}(E) \dif E = K_{\rm e} E^{-\gamma_{\rm e}} \dif E ~,
\end{align}
the intensity is
\begin{align}
J_{\nu} = {\rm const} \cdot K_{\rm e} LH^{(\gamma_{\rm e}+1)/2} \nu^{(1-\gamma_{\rm e})/2} ~,
\end{align}
where $H$ is a certain average value of the magnetic field along the line of sight. Measuring the dependence of $J_{\nu}$ on frequency $\nu$ immediately yields index $\gamma_{\rm e}$ while the value of $J_{\nu}$ itself allows coefficient $K_{\rm e}$ to be determined if $L$ and $H$ are known, that is to find the electron spectrum in the source and then the associated energy density
\begin{align}
w_{\rm cr, e} = \int  E N_{\rm e}(E) \dif E ~,
\end{align}














%%%%%%%%%%%%%%%%%%%%%%%%%%%%%%%%%%%%%%%%%%%%%%%%%%%%%%%%%%%%%%%%%%%%%%
\bibliographystyle{unsrt_update}
\bibliography{ref}
%%%%%%%%%%%%%%%%%%%%%%%%%%%%%%%%%%%%%%%%%%%%%%%%%%%%%%%%%%%%%%%%%%%%%%


\end{document}
