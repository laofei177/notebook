\documentclass[12pt,a4paper]{article}
%\usepackage{fontspec, xunicode, xltxtra}  
%\setmainfont{Hiragino Sans GB}  
\usepackage{xeCJK}
%\setCJKmainfont[BoldFont=STZhongsong, ItalicFont=STKaiti]{STSong}
%\setCJKsansfont[BoldFont=STHeiti]{STXihei}
%\setCJKmonofont{STFangsong}

%使用Xelatex编译

% 设置页面
%==================================================
\linespread{2} %行距
% \usepackage[top=1in,bottom=1in,left=1.25in,right=1.25in]{geometry}
% \headsep=2cm
% \textwidth=16cm \textheight=24.2cm
%==================================================

% 其它需要使用的宏包
%==================================================
\usepackage[colorlinks,linkcolor=blue,anchorcolor=red,citecolor=green,urlcolor=blue]{hyperref} 
\usepackage{tabularx}
\usepackage{authblk}         % 作者信息
\usepackage{algorithm}     % 算法排版
\usepackage{amsmath}     % 数学符号与公式
\usepackage{amsfonts}     % 数学符号与字体
\usepackage{mathrsfs}      % 花体
\usepackage{graphics}
\usepackage{color}
\usepackage{fancyhdr}       % 设置页眉页脚
\usepackage{fancyvrb}       % 抄录环境
\usepackage{float}              % 管理浮动体
\usepackage{geometry}     % 定制页面格式
\usepackage{hyperref}       % 为PDF文档创建超链接
\usepackage{lineno}          % 生成行号
\usepackage{listings}        % 插入程序源代码
\usepackage{multicol}       % 多栏排版
\usepackage{natbib}         % 管理文献引用
\usepackage{rotating}       % 旋转文字,图形,表格
\usepackage{subfigure}    % 排版子图形
\usepackage{titlesec}       % 改变章节标题格式
\usepackage{moresize}   % 更多字体大小
\usepackage{anysize}
\usepackage{indentfirst}  % 首段缩进
\usepackage{booktabs}   % 使用\multicolumn
\usepackage{multirow}    % 使用\multirow
\usepackage{graphicx} 
\usepackage{wrapfig}
\usepackage{xcolor}
\usepackage{titlesec}     % 改变标题样式
\usepackage{enumitem}
\usepackage{harpoon}   %矢量符号

\newcommand{\myvec}[1]%
   {\stackrel{\raisebox{-2pt}[0pt][0pt]{\small$\rightharpoonup$}}{#1}}  %矢量符号
\renewcommand{\vec}[1]{\boldsymbol{#1}}
\newcommand{\me}{\mathrm{e}}
\newcommand{\mi}{\mathrm{i}}
\newcommand{\dif}{\mathrm{d}}
\newcommand{\tabincell}[2]{\begin{tabular}{@{}#1@{}}#2\end{tabular}}

\def\kpc{{\rm kpc}}
\def\km{{\rm km}}
\def\cm{{\rm cm}}
\def\TeV{{\rm TeV}}
\def\GeV{{\rm GeV}}
\def\MeV{{\rm MeV}}
\def\GV{{\rm GV}}
\def\MV{{\rm MV}}
\def\yr{{\rm yr}}
\def\s{{\rm s}}
\def\ns{{\rm ns}}
\def\GHz{{\rm GHz}}
\def\muGs{{\rm \mu Gs}}
\def\arcsec{{\rm arcsec}}
\def\K{{\rm K}}
\def\microK{\mu{\rm K}}
\def\sr{{\rm sr}}
\newcolumntype{p}{D{,}{\pm}{-1}}

\renewcommand{\figurename}{Fig.}
\renewcommand{\tablename}{Tab.}

\renewcommand{\arraystretch}{1.5}

\setlength{\parindent}{0pt}  %取消每段开头的空格

\title{热力学基本概念}
\author{}
\date{\today}
\begin{document}

\maketitle

热力学系统:宏观物体,亦即由大量微观粒子所组成的;

外界:可以对系统发生影响的外部环境;

绝热壁:不允许它两边的物体发生任何形式的热交换;

导热壁:

刚性壁:不允许物体发生位移;用刚性壁包围的固体不可能发生形变;外界对物体不可能作机械功;

热接触:两边的物体彼此处于热接触;由刚性、导热壁分开的两个物体,彼此只允许发生传热,而不允许发生力的或电磁的相互作用,也不可能发生物质交换;

\textcolor{red}{孤立系}:如果系统由绝热且刚性的壁与环境分开,系统不会受到外界的任何影响,即不可能发生任何能量与物质交换;

\textcolor{red}{闭系}:系统与外界不发生物质交换;允许系统与外界有能量交换(通过做功与传热);

\textcolor{red}{开系}:粒子数可变的系统;

\section{平衡态}

在没有外界影响的条件下,物体各部分的性质长时间内不发生任何变化的状态;

Note:\\
1) 若把平衡态简单定义为“物体各部分的性质长时间内不发生任何变化的状态”是不充分的;非平衡定态或稳恒态:不随时间变化的非平衡态;\\
2) “没有外界影响”:物体与外界之间没有宏观的能量与物质交换;

动态平衡:平衡态只是宏观性质不随时间变化,微观上分子仍在不停地运动;存在涨落;

弛豫时间:在一定的条件下,初始不处于平衡态的系统,经过一段时间,必将趋近于平衡态;\\
Note:“一定条件”:1) 孤立系;2) 不变的外界条件;

热源、热库:恒定温度的外界;与物体发生有限数量的热量交换对热库的影响可以忽略;\\
物体处于恒定压强的外界环境中;\\
系统与大粒子源或粒子库接触\\
趋于平衡是依靠粒子间的相互作用实现的;

\textcolor{red}{状态变量}:宏观变量;系统平衡态由状态变量描写;

均匀系:一个物体各部分性质完全相同;单相系;

非均匀系:各部分性质不相同;复相系;每一个均匀部分称为一个相;

\textcolor{red}{广延量}:与系统的总质量成正比;可加性;摩尔数,体积,内能,熵,$\dots$

\textcolor{red}{强度量}:代表物质的内在性质;不可加;具有局域的性质;与总质量无关;压强,温度,密度,内能密度,熵密度,$\dots$

\textcolor{red}{局域平衡近似}:描写非平衡态;将系统分成许多小块,每一块宏观上足够小,微观上足够大;每一小块近似地看成是均匀的;强度变量是坐标$\myvec{r}$和时间$t$的函数;

准静态绝热过程

\subsection{绝热膨胀}


\subsection{节流过程、Joule--Thomson效应}



\section{态函数}

\subsection{温度}

可以直接测量;常用作状态变量;


\subsection{内能}

\subsection{热容}
\textcolor{red}{热量}是在过程中传递的一种能量,是\textcolor{red}{与过程有关}的。一个系统在某一过程中温度升高$1$ K所吸收的热量,称为系统在该过程的热容量。以$\Delta Q$表示系统在某一过程中温度升高$\Delta T$所吸收的热量,则系统在该过程的热容量为
\begin{equation}
C = \underset{\Delta T \rightarrow 0}\lim \frac{\Delta Q}{\Delta T}
\end{equation}
热容量的单位为焦耳每开尔文$J\cdot K^{-1}$。系统在某一过程中的热容量不仅取决于物质的固有特性,且与系统的质量成正比,是一个广延量。

摩尔热容量$C_m$:$1$ mol 物质的热容量。摩尔热容量除与过程有关外,只与物质的固有属性有关,是一个强度量。

系统的热容量$C$与摩尔热容量$C_m$的关系为$C = n C_m$,$n$为系统的物质的量。单位质量的物质在某一过程的热容量称为物质在该过程的\textcolor{red}{比热容}。

\subsubsection{等容过程}
系统的体积不变,外界对系统不做功,$W = 0$,所以$Q = \Delta U$,定容热容
\begin{equation}
C_V = \underset{\Delta T \rightarrow 0}\lim \left(\frac{\Delta Q}{\Delta T} \right)_V = \underset{\Delta T \rightarrow 0}\lim \left(\frac{\Delta U}{\Delta T} \right)_V = \left(\frac{\partial U}{\partial T} \right)_V
\end{equation}
$\left(\dfrac{\partial U}{\partial T} \right)_V$表示体积不变的情况下内能随温度的变化率。

一般的简单系统,$U$是$T$、$V$的函数,$C_V$也是$T$、$V$的函数。

\subsubsection{等压过程}
外界对系统作功为$W = -p\Delta V$,$Q = \Delta U +p\Delta V$
\begin{eqnarray}
\nonumber C_p &=& \underset{\Delta T \rightarrow 0}\lim \left(\frac{\Delta Q}{\Delta T} \right)_p \\
\nonumber &=& \underset{\Delta T \rightarrow 0}\lim \left(\frac{\Delta U +p\Delta V}{\Delta T} \right)_p \\
&=& \left(\frac{\partial U}{\partial T} \right)_p +p\left(\frac{\partial V}{\partial T} \right)_p
\end{eqnarray}
等压过程中焓的变化为
\begin{equation}
\Delta H= \Delta U +p\Delta V
\end{equation}
即是等压过程中系统从外界吸收的热量。定压热容
\begin{equation}
C_p =  \left(\frac{\partial H}{\partial T} \right)_p
\end{equation}

一般的简单系统,$C_p$是$T$、$p$的函数。

\subsection{焓}
状态函数
\begin{equation}
H = U +pV
\end{equation}



\subsection{吉布斯函数}

\subsection{自由能}



\section{物态方程}
\begin{equation}
T = f(p, V)
\end{equation}
或者
\begin{equation}
p = p(T, V)
\end{equation}
或者
\begin{equation}
V = V(p, T)
\end{equation}
或者
\begin{equation}
g(p, V, T) = 0
\end{equation}

\textcolor{red}{膨胀系数$\alpha$} \\
压强不变时,体积随温度的相对变化率;
\begin{equation}
\alpha \equiv \frac{1}{V} \left( \frac{\partial V}{\partial T} \right)_p
\end{equation}

\textcolor{red}{压强系数$\beta$} \\
体积不变时,压强随温度的相对变化率;
\begin{equation}
\beta \equiv \frac{1}{p} \left( \frac{\partial p}{\partial T} \right)_V
\end{equation}

\textcolor{red}{等温压缩系数}(简称:\textcolor{red}{压缩系数}) \textcolor{red}{$\beta$} \\
温度不变时,体积随压强的相对变化率;
\begin{equation}
\kappa_T \equiv -\frac{1}{V} \left( \frac{\partial V}{\partial p} \right)_T
\end{equation}

\begin{equation}
\alpha = \kappa_T \beta p
\end{equation}
可以由
\begin{equation}
\color{red} \left( \frac{\partial V}{\partial T} \right)_p  \left( \frac{\partial T}{\partial p} \right)_V \left( \frac{\partial p}{\partial V} \right)_T = -1
\end{equation}
导出。

\subsection{理想气体}
它是实际气体在压强$p\rightarrow 0$时的极限;\\
可以作为实际气体在温度不太低、密度足够稀薄时的近似;
\begin{equation}
pV = NRT
\end{equation}
$N$: 气体的摩尔数,\\
$T$:气体温标,\\
$R = 8.3145$ J/(mol  K):摩尔气体常数

\subsection{范德瓦耳斯气体}
考虑到分子之间的相互作用引起的修正
\begin{equation}
\left( p+\frac{N^2 a}{V^2} \right) (V -Nb) = NRT
\end{equation}
$N^2 a/V^2$: 分子之间的吸引力引起的修正;\\
$Nb$: 分子之间的排斥力引起的修正;
若气体密度足够低,使$N^2 a/V^2$和$Nb$可以忽略时,范德瓦耳斯方程回到理想气体方程。

\subsection{昂尼斯方程}
按压强的级数展开作为实际气体的物态方程;
\begin{equation}
pV = NRT\{1 +A_2 p + A_3 p^2 +A_4 p^3 + \dots  \}
\end{equation}
$A_2, A_3, A_4, \dots$都是温度的函数,分别称为第二、第三、第四、$\dots$位力系数。

按体积的负幂次展开
\begin{equation}
pV = NRT\left \{1 +\frac{B_2}{V} + \frac{B_3}{V^2} +\frac{B_4}{V^3} + \dots \right \}
\end{equation}
$B_2, B_3, B_4, \dots$都是温度的函数,分别称为第二、第三、第四、$\dots$位力系数。

\subsection{流体和各项同性固体}


\subsection{顺磁固体}
顺磁物质在没有外加磁场时,不表现出磁性;\\
当外加磁场$\myvec{\mathscr{H}}$时,才表现出磁性。 \\
对各项同性顺磁固体,其磁化强度$\myvec{\mathscr{M}}$(即单位体积的总磁矩)的方向与$\myvec{\mathscr{H}}$相同,可取为标量$\mathscr{M}$和$\mathscr{H}$。\\
描写顺磁固体的平衡态的独立状态变量:$(T, V,  \mathscr{H})$ \\
顺磁固体的物态方程遵从居里定律,即
\begin{equation}
\mathscr{M} = \frac{C}{T} \mathscr{H}
\end{equation}
$C$:与物质有关的正常数;\\
居里定律只有在$\mathscr{H}/T$的比值很小时(弱场与高温)下才适用。

\overrightharp{A}










\end{document}