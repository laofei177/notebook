\documentclass[12pt,a4paper]{article}
%\usepackage{fontspec, xunicode, xltxtra}  
%\setmainfont{Hiragino Sans GB}  
%\usepackage{xeCJK}
%\setCJKmainfont[BoldFont=STZhongsong, ItalicFont=STKaiti]{STSong}
%\setCJKsansfont[BoldFont=STHeiti]{STXihei}
%\setCJKmonofont{STFangsong}

%使用Xelatex编译

% 设置页面
%==================================================
\linespread{2} %行距
% \usepackage[top=1in,bottom=1in,left=1.25in,right=1.25in]{geometry}
% \headsep=2cm
% \textwidth=16cm \textheight=24.2cm
%==================================================

% 其它需要使用的宏包
%==================================================
\usepackage[colorlinks,linkcolor=blue,anchorcolor=red,citecolor=green,urlcolor=blue]{hyperref} 
\usepackage{tabularx}
\usepackage{authblk}         % 作者信息
\usepackage{algorithm}     % 算法排版
\usepackage{amsmath}     % 数学符号与公式
\usepackage{amsfonts}     % 数学符号与字体
\usepackage{mathrsfs}      % 花体
\usepackage{amssymb}

\usepackage{graphicx} 
\usepackage{graphics}
\usepackage{xcolor}
\usepackage{color}

\usepackage{fancyhdr}       % 设置页眉页脚
\usepackage{fancyvrb}       % 抄录环境
\usepackage{float}              % 管理浮动体
\usepackage{geometry}     % 定制页面格式
\usepackage{hyperref}       % 为PDF文档创建超链接
\usepackage{lineno}          % 生成行号
\usepackage{listings}        % 插入程序源代码
\usepackage{multicol}       % 多栏排版
\usepackage{natbib}         % 管理文献引用
\usepackage{rotating}       % 旋转文字,图形,表格
\usepackage{subfigure}    % 排版子图形
\usepackage{titlesec}       % 改变章节标题格式
\usepackage{moresize}   % 更多字体大小
\usepackage{anysize}
\usepackage{indentfirst}  % 首段缩进
\usepackage{booktabs}   % 使用\multicolumn
\usepackage{multirow}    % 使用\multirow
\usepackage{wrapfig}

\usepackage{titlesec}     % 改变标题样式
\usepackage{enumitem}

\newcommand{\myvec}[1]%
   {\stackrel{\raisebox{-2pt}[0pt][0pt]{\small$\rightharpoonup$}}{#1}}  %矢量符号
\renewcommand{\vec}[1]{\boldsymbol{#1}}
\newcommand{\me}{\mathrm{e}}
\newcommand{\mi}{\mathrm{i}}
\newcommand{\dif}{\mathrm{d}}
\newcommand{\tabincell}[2]{\begin{tabular}{@{}#1@{}}#2\end{tabular}}

\def\kpc{{\rm kpc}}
\def\km{{\rm km}}
\def\cm{{\rm cm}}
\def\TeV{{\rm TeV}}
\def\GeV{{\rm GeV}}
\def\MeV{{\rm MeV}}
\def\GV{{\rm GV}}
\def\MV{{\rm MV}}
\def\yr{{\rm yr}}
\def\s{{\rm s}}
\def\ns{{\rm ns}}
\def\GHz{{\rm GHz}}
\def\muGs{{\rm \mu Gs}}
\def\arcsec{{\rm arcsec}}
\def\K{{\rm K}}
\def\microK{\mu{\rm K}}
\def\sr{{\rm sr}}
\newcolumntype{p}{D{,}{\pm}{-1}}

\renewcommand{\figurename}{Fig.}
\renewcommand{\tablename}{Tab.}

\renewcommand{\arraystretch}{1.5}

\setlength{\parindent}{0pt}  %取消每段开头的空格

\title{The Theory of Simple Gases}
\author{}
\date{\today}
\begin{document}

\maketitle



\section{An ideal gas in a quantum-mechanical microcanonical ensemble}
Consider a gaseous system of $N$ noninteracting, indistinguishable particles confined to a space of volume $V$ and sharing a given energy $E$. $\Omega(N, V, E)$ denotes the number of distinct microstates accessible to the system under the macrostate $(N,V,E)$.

For large $V$, the single-particle energy levels in the system are very close to one another, we may divide the energy spectrum into a large number of ``groups of levels", which may be referred to as \textcolor{orange}{energy cells}. Let $\varepsilon_i$ denote the average energy of a level, and \textcolor{orange}{$g_i$ the (arbitrary) number of levels}, in the $i$th cell; we assume that \textcolor{red}{all $g_i \gg 1$}. The distribution set $\{n_i\}$ must conform to the conditions
\begin{equation}
\sum_i n_i = N 
\label{con1}
\end{equation}
and 
\begin{equation}
\sum_i n_i \varepsilon_i = E ~.
\label{con2}
\end{equation}
\begin{equation}
\Omega(N, V, E) = \sum_{\{n_i\}} W\{n_i\} ~,
\end{equation}
where $W\{n_i\}$ is the number of distinct microstates associated with the distribution set $\{n_i\}$, in which summation goes over all distribution sets that conform to conditions (\ref{con1}) and (\ref{con2}).
\begin{equation}
W\{n_i\} = \prod_i w(i) ~,
\end{equation}
where $w(i)$ is the number of distinct microstates associated with the $i$th cell of the spectrum (the cell that contains $n_i$ particles, to be accommodated among $g_i$ levels) while the product goes over all the cells in the spectrum. $w(i)$ is the number of distinct ways in which the $n_i$ identical, and indistinguishable, particles can be distributed among the $g_i$ levels of the $i$th cell. 


the \textcolor{orange}{entropy of the system} is
\begin{equation}
\color{red} S(N, V, E) = k\ln \Omega(N, V, E) = k \ln \left[\sum_{\{n_i\}} W\{n_i\} \right] ~.
\end{equation}
The logarithm of the sum can be approximated by the logarithm of the largest term in the sum, therefore
\begin{equation}
S(N, V, E) \approx k\ln W\{n_i^\ast \}
\end{equation}
where $\{n_i^\ast \}$ is the distribution set that maximizes the number $W\{n_i\}$; the numbers $n_i^\ast$ are the most probable values of the distribution numbers $n_i$. The maximization is to be carried out under the restrictions that the quantities $N$ and $E$ remain constant.


\section{An ideal gas in other quantum-mechanical ensembles}




\section{Statistics of the occupation numbers}



\section{Kinetic considerations}





\end{document}