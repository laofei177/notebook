\documentclass[12pt,a4paper]{article}
%\usepackage{fontspec, xunicode, xltxtra}  
%\setmainfont{Hiragino Sans GB}  
\usepackage{xeCJK}
%\setCJKmainfont[BoldFont=STZhongsong, ItalicFont=STKaiti]{STSong}
%\setCJKsansfont[BoldFont=STHeiti]{STXihei}
%\setCJKmonofont{STFangsong}

%使用Xelatex编译

% 设置页面
%==================================================
\linespread{2} %行距
% \usepackage[top=1in,bottom=1in,left=1.25in,right=1.25in]{geometry}
% \headsep=2cm
% \textwidth=16cm \textheight=24.2cm
%==================================================

% 其它需要使用的宏包
%==================================================
\usepackage[colorlinks,linkcolor=blue,anchorcolor=red,citecolor=green,urlcolor=blue]{hyperref} 
\usepackage{tabularx}
\usepackage{authblk}         % 作者信息
\usepackage{algorithm}     % 算法排版
\usepackage{amsmath}     % 数学符号与公式
\usepackage{amsfonts}     % 数学符号与字体
\usepackage{mathrsfs}      % 花体
\usepackage{amssymb}
\usepackage[framemethod=TikZ]{mdframed}

\usepackage{graphicx} 
\usepackage{graphics}
\usepackage{color}
\usepackage{xcolor}
\usepackage{tcolorbox}
\usepackage{lipsum}
\usepackage{empheq}

\usepackage{fancyhdr}       % 设置页眉页脚
\usepackage{fancyvrb}       % 抄录环境
\usepackage{float}              % 管理浮动体
\usepackage{geometry}     % 定制页面格式
\usepackage{hyperref}       % 为PDF文档创建超链接
\usepackage{lineno}          % 生成行号
\usepackage{listings}        % 插入程序源代码
\usepackage{multicol}       % 多栏排版
%\usepackage{natbib}         % 管理文献引用
\usepackage{rotating}       % 旋转文字,图形,表格
\usepackage{subfigure}    % 排版子图形
\usepackage{titlesec}       % 改变章节标题格式
\usepackage{moresize}   % 更多字体大小
\usepackage{anysize}
\usepackage{indentfirst}  % 首段缩进
\usepackage{booktabs}   % 使用\multicolumn
\usepackage{multirow}    % 使用\multirow

\usepackage{wrapfig}
\usepackage{titlesec}     % 改变标题样式
\usepackage{enumitem}
\usepackage{aas_macros}
\usepackage{bigints}

\newcommand{\myvec}[1]%
   {\stackrel{\raisebox{-2pt}[0pt][0pt]{\small$\rightharpoonup$}}{#1}}  %矢量符号
\renewcommand{\vec}[1]{\boldsymbol{#1}}
\newcommand{\me}{\mathrm{e}}
\newcommand{\mi}{\mathrm{i}}
\newcommand{\dif}{\mathrm{d}}
\newcommand{\tabincell}[2]{\begin{tabular}{@{}#1@{}}#2\end{tabular}}

\def\kpc{{\rm kpc}}
\def\km{{\rm km}}
\def\cm{{\rm cm}}
\def\TeV{{\rm TeV}}
\def\GeV{{\rm GeV}}
\def\MeV{{\rm MeV}}
\def\GV{{\rm GV}}
\def\MV{{\rm MV}}
\def\yr{{\rm yr}}
\def\s{{\rm s}}
\def\ns{{\rm ns}}
\def\GHz{{\rm GHz}}
\def\muGs{{\rm \mu Gs}}
\def\arcsec{{\rm arcsec}}
\def\K{{\rm K}}
\def\microK{\mu{\rm K}}
\def\sr{{\rm sr}}
\newcolumntype{p}{D{,}{\pm}{-1}}

\renewcommand{\figurename}{Fig.}
\renewcommand{\tablename}{Tab.}

\renewcommand{\arraystretch}{1.5}

\setlength{\parindent}{0pt}  %取消每段开头的空格

\newcounter{theo}[section]\setcounter{theo}{0}
\renewcommand{\thetheo}{\arabic{section}.\arabic{theo}}
\newenvironment{theo}[2][]{%
\refstepcounter{theo}%
\ifstrempty{#1}%
{\mdfsetup{%
frametitle={%
\tikz[baseline=(current bounding box.east),outer sep=0pt]
\node[anchor=east,rectangle,fill=blue!20]
{\strut Theorem~\thetheo};}}
}%
{\mdfsetup{%
frametitle={%
\tikz[baseline=(current bounding box.east),outer sep=0pt]
\node[anchor=east,rectangle,fill=blue!20]
{\strut Theorem~\thetheo:~#1};}}%
}%
\mdfsetup{innertopmargin=10pt,linecolor=blue!20,%
linewidth=2pt,topline=true,%
frametitleaboveskip=\dimexpr-\ht\strutbox\relax
}
\begin{mdframed}[]\relax%
\label{#2}}{\end{mdframed}}

\newcommand*\widefbox[1]{\fbox{\hspace{2em}#1\hspace{2em}}}


\title{玻色--爱因斯坦凝聚}
\author{}
\date{\today}
\begin{document}

\maketitle
\cite{2013热力学} 当理想玻色气体的$n\lambda^3$等于或大于$2.612$的临界值时,会出现\textcolor{red}{玻色-爱因斯坦凝聚}。

考虑由$N$个全同、近独立的玻色子组成的系统,温度为$T$、体积为$V$。设粒子的自旋为$0$。根据玻色分布给出,温度为$T$时,处在能级$e_l$上的粒子数为
\begin{equation}
a_l = \frac{\omega_l}{e^{\frac{\varepsilon_l -\mu}{kT}} -1}
\end{equation}
由于处在任一能级上的粒子数不能取负值,因此$e^{\frac{\varepsilon_l -\mu}{kT}} > 1$。设$\varepsilon_0$为离子的最低能级,则
\begin{equation}
\varepsilon_0 > \mu
\end{equation}
即理想玻色气体的化学势须低于粒子最低能级的能量。若取最低能级为能量的零点,$\varepsilon_0 = 0$,则
\begin{equation}
\mu < 0 ~.
\end{equation}
化学势$\mu$由
\begin{equation}
\frac{1}{V} \sum_i \frac{\omega_l}{e^{\frac{\varepsilon_l -\mu}{kT}} -1} = \frac{N}{V} = n
\end{equation}
确定为温度$T$、粒子数密度$n=N/V$的函数。$\varepsilon_l$和$\omega_l$都与温度无关,当粒子数密度$n$给定,温度愈低,该式确定的$\mu$值越高($|\mu|$愈小)。若将求和用积分代替,
\begin{equation}
\frac{2\pi}{h^3} (2m)^{3/2} \int_0^\infty \frac{\varepsilon^{1/2} \dif \varepsilon}{e^{\frac{\varepsilon -\mu}{kT}} -1} = n ~,
\end{equation}
该式适用于热力学极限或能级间距远小于$kT$的情形。化学势随温度的降低而升高,当温度降到某一临界温度$T_c$时,$\mu$将趋于$-0$。$e^{-\frac{\mu}{kT_c}}$趋于$1$。临界温度$T_c$由
\begin{equation}
\frac{2\pi}{h^3} (2m)^{3/2} \int_0^\infty \frac{\varepsilon^{1/2} \dif \varepsilon}{e^{\frac{\varepsilon}{kT}} -1} = n ~,
\end{equation}
令$x = \varepsilon/kT$,则
\begin{equation}
\frac{2\pi}{h^3} (2m kT_c)^{3/2} \int_0^\infty \frac{x^{1/2} \dif x}{e^{x} -1} = n ~,
\end{equation}
\begin{equation*}
\int_0^\infty \frac{x^{1/2} \dif x}{e^{x} -1} = \frac{2.612 \sqrt{\pi}}{2} 
\end{equation*}
对于给定的粒子数密度$n$,临界温度$T_c$为
\begin{equation}
T_c = \frac{2\pi}{(2.612)^{2/3}} \frac{\hbar^2}{mk} n^{2/3} 
\end{equation}
温度愈低,$\mu$值越高,但在任何温度下$\mu$值必须是负的。当$T < T_c$时,$\mu$仍趋于$-0$。在$T_c$以上,$\mu$值为负的有限值时,处在能级$\varepsilon = 0$的粒子数与总粒子数相比是一个小量,用积分代替求和引起的误差是可以忽略的。

在$T < T_c$时,
\begin{equation}
n_0(T) +\frac{2\pi}{h^3} (2m)^{3/2} \int_0^\infty \frac{\varepsilon^{1/2} \dif \varepsilon}{e^{\frac{\varepsilon}{kT}} -1} = n ~,
\end{equation}
$n_0(T)$是温度为$T$时,处在能级$\varepsilon = 0$的粒子数密度;第二项是处在激发能级$\varepsilon > 0$的粒子数密度$n_{\varepsilon > 0}$,已取极限$\mu \rightarrow -0$。

\begin{eqnarray*}
n_{\varepsilon > 0} = \frac{2\pi}{h^3} (2m)^{3/2} \int_0^\infty \frac{\varepsilon^{1/2} \dif \varepsilon}{e^{\frac{\varepsilon}{kT}} -1}  = \frac{2\pi}{h^3} (2m kT)^{3/2} \int_0^\infty \frac{x^{1/2} \dif x}{e^{x} -1} = n\left(\frac{T}{T_c} \right)^{3/2}
\end{eqnarray*}
温度为$T$时,处在最低能级$\varepsilon = 0$的粒子数密度为
\begin{equation}
n_0(T) = n \left[1- \left(\frac{T}{T_c} \right)^{3/2}\right]
\end{equation}
即,在$T_c$以下,$n_0$与$n$具有相同的量级。

在绝对零度下,粒子将尽可能占据能量最低的状态。对于玻色粒子,一个量子态所能容纳的粒子数目不受限制,因此绝对零度下玻色粒子将全部处在$\varepsilon = 0$的最低能级。在$T < T_c$时,有宏观量级的粒子在$\varepsilon = 0$的能级凝聚,称为\textcolor{red}{玻色-爱因斯坦凝聚}。\textcolor{red}{$T_c$为凝聚温度,凝聚在$\varepsilon_0$的粒子集合称为玻色凝聚体}。凝聚体能量和动量为$0$,对压强没有贡献。由于凝聚体的微观状态完全确定,熵也为$0$。

在$T < T_c$时,理想玻色气体的内能为处在能级$\varepsilon > 0$的粒子能量的统计平均值
\begin{eqnarray*}
U &=& \frac{2\pi V}{h^3} (2m)^{3/2} \int_0^\infty \frac{\varepsilon^{3/2} \dif \varepsilon}{e^{\frac{\varepsilon}{kT}} -1} \\
&=& \frac{2\pi V}{h^3} (2m)^{3/2} (kT)^{5/2} \int_0^\infty \frac{x^{3/2} \dif x}{e^{x} -1} \\
&=& 0.770 NkT \left(\frac{T}{T_c} \right)^{3/2}
\end{eqnarray*}
其中$x = \varepsilon/kT$。定容热容量为
\begin{equation}
C_V = \left(\frac{\partial U}{\partial T} \right)_V = \frac{5U}{2T} = 1.925 Nk \left(\frac{T}{T_c} \right)^{3/2}
\end{equation}
在$T < T_c$时,理想玻色气体的$C_V$与$T^{3/2}$成正比。到$T = T_c$时,$C_V$达到极大值$C_V = 1.925 Nk$,高温时应趋于经典值$\dfrac{3}{2} Nk$。在$T = T_c$的尖峰处,$C_V$连续,但$C_V$对$T$的偏导数存在突变。
\begin{equation}
n \left(\frac{h}{\sqrt{2\pi m k T_c} }  \right)^3 = n \lambda^3 = 2.612
\end{equation}
是理想玻色气体出现凝聚的临界条件。出现凝聚体的条件为
\begin{equation}
n \lambda^3 \geqslant 2.612
\end{equation}
满足上式时,原子的热波长大于原子的平均间距,量子统计关联起决定性作用。可以通过降低温度和增加气体粒子数密度的方法来实现波色凝聚。


\cite{2007热力学与统计物理学} 强简并区,$z$接近$1$,但仍小于$1$或$e^\alpha$接近$1$但仍大于$1$($z \lesssim 1$ 或 $e^\alpha > 1$)。对于理想玻色气体,$z \geqslant 1$或$e^\alpha \leqslant$是不允许的。因为任何一个能级上的粒子数不可能是负值,由$\bar{a}_\lambda = g_\lambda/({\rm e}^{\alpha +\beta \varepsilon_\lambda} -1)$,必须有${\rm e}^{\alpha +\beta \varepsilon_\lambda} > 1$(对一切$\lambda$)。粒子平均动能的最低能级$\varepsilon_0$可能取为$0$(粒子平动动能的最低能级的量级为
\begin{equation*}
\varepsilon_0 \sim \dfrac{2\pi^2 \hbar^2}{mL^2} ~,
\end{equation*}
若取$m \sim 10^{-24}$ g,$L \sim 1$ cm,得到
\begin{equation*}
\varepsilon_0 \sim 10^{-30} {\rm erg} \sim 10^{-42} ~ {\rm eV} ~.
\end{equation*}
称它是零能量态或者零动量态。),故${\rm e}^{\alpha} > 1$或$z = {\rm e}^{-\alpha} < 1$。理想玻色气体在强简并条件下将发生一种新的相变,称为玻色-爱因斯坦凝聚。 

\begin{align}
\label{N}
& N = \dfrac{V}{h^3} (2\pi m kT)^{3/2} g_{3/2}(z) ~, \\
& g_{3/2}(z) = \sum_{\lambda =1}^\infty \dfrac{z^\lambda}{\lambda^{3/2}} = z + \dfrac{1}{2^{3/2}} z^2 +\dfrac{1}{3^{3/2}} z^3 +\cdots
\end{align}
$g_{3/2}(z) $在$0\leqslant z \leqslant 1$的范围内都是收敛的,它随$z$的增加而单调连续的增加,在$z=1$处达到最大值$g_{3/2}(1) = \zeta \left(\dfrac{3}{2}\right) \approx 2.612$,其中$\zeta$为黎曼$\zeta$函数。当$N, V$给定时,随着温度的下降,$g_{3/2}(z) $的值增加。由于$g_{3/2}(z) $有上限,必定存在某一非零温度$T_c$,使得
\begin{align}
& N = \dfrac{V}{h^3} (2\pi m kT_c)^{3/2} g_{3/2}(1) ~, \\
& T_c = \dfrac{h^2}{2\pi mk} \left[\dfrac{n}{g_{3/2}(1)} \right]^{2/3} ~.
\end{align}
当$T < T_c$时,
\begin{equation}
N > \dfrac{V}{h^3} (2\pi m kT)^{3/2} g_{3/2}(1) ~, ~(T < T_c) ~,
\end{equation}
表明式(\ref{N})不再成立。

在计算$\ln { \Xi}$时,对子系量子态求和近似地用子相体积的积分代替。由于态密度$D(\varepsilon) \sim \varepsilon^{1/2}$,当$\varepsilon = 0$时,$D(0) = 0$。在积分中把$\varepsilon = 0$态的贡献完全丢掉了。这样做对$T \leqslant T_c$是合理的,但对$T < T_c$则不行。

\begin{align}
\nonumber \ln { \Xi} &= -\sum_\lambda g_\lambda \ln(1 -{\rm e}^{-\alpha -\beta \varepsilon_\lambda}) \\
&= -\ln(1-{\rm e}^{-\alpha}) -\sum_{\varepsilon_\lambda \geqslant \varepsilon_1} g_\lambda \ln (1 -{\rm e}^{-\alpha -\beta \varepsilon_\lambda}) ~,
\end{align}
右边第一项代表粒子最低能级(即粒子的基态$\varepsilon_0 = 0$,已设$g_0 = 1$)的贡献,第二项的求和代表所有粒子激发态的贡献,$\varepsilon_1$为第一激发能级。对于宏观系统,$\dfrac{\Delta\varepsilon}{kT} \ll 1 $,仍可用对子相体积的积分来代替,
\begin{align}
\nonumber \ln { \Xi} &= -\ln(1-{\rm e}^{-\alpha}) - \int_{\varepsilon_1}^\infty \ln (1 -{\rm e}^{-\alpha -\beta \varepsilon}) D(\varepsilon) \dif \varepsilon ~, \\
&= \ln(1-{\rm e}^{-\alpha}) - \int_{0}^\infty \ln (1 -{\rm e}^{-\alpha -\beta \varepsilon}) D(\varepsilon) \dif \varepsilon
\end{align}
由$D(\varepsilon) = 0$将第二项积分的下限改成$0$。

\begin{align}
& N = -\dfrac{\partial }{\partial \alpha} \ln { \Xi} = \dfrac{1}{{\rm e}^{\alpha} -1} + \int_0^\infty \dfrac{D(\varepsilon) \dif \varepsilon}{{\rm e}^{\alpha +\beta \varepsilon} -1} = \overline{N}_0 +\overline{N}_{\rm exc} ~, \\
& \overline{N}_0 = \dfrac{1}{{\rm e}^{\alpha} -1} ~, \\
& \overline{N}_{\rm exc} = \int_0^\infty \dfrac{D(\varepsilon) \dif \varepsilon}{{\rm e}^{\alpha +\beta \varepsilon} -1} ~,
\end{align}
$\overline{N}_0$和$\overline{N}_{\rm exc}$分别代表基态和所有激发态上占据的粒子数。由$z = {\rm e}^{-\alpha}$,
\begin{align}
& \overline{N}_0 = \dfrac{z}{1-z} ~, \\
& \overline{N}_{\rm exc} = \dfrac{V}{\lambda^3_T} g_{3/2}(z) = \dfrac{V}{h^3} (2\pi m kT)^{3/2} g_{3/2}(z) ~,
\end{align}

当$T > T_c$时,尽管就单个能级,$\overline{N}_0$比任何单个激发能级上占据的粒子数都多,但由于绝大多数的粒子都占据在激发能级上,以致$\overline{N}_{\rm exc} \approx N$,与$N\sim 10^{20}$相比,$\overline{N}_0$完全可以忽略。这对$T \geqslant T_c$的一切温度都成立。所有激发态上占据的粒子数的最大值(在忽略$\overline{N}_0$后等于$N$)为
\begin{equation}
(\overline{N}_{\rm exc})_{\rm max} = \dfrac{V}{h^3} (2\pi m kT_c)^{3/2} g_{3/2}(1) ~. 
\end{equation}
当$T < T_c$时,$\overline{N}_{\rm exc} < (\overline{N}_{\rm exc})_{\rm max} $,且$\overline{N}_{\rm exc}$随$T \rightarrow 0$而趋于$0$。$\overline{N}_0$随$T \rightarrow 0$而趋于$N$。对$T < T_c$,
\begin{align}
& \dfrac{(\overline{N}_{\rm exc})}{N} = \left( \dfrac{T}{T_c} \right)^{3/2} ~~ (T < T_c) ~, \\
& \overline{N}_0 = N -\overline{N}_{\rm exc}  = N \left[1 - \left( \dfrac{T}{T_c} \right)^{3/2}\right] ~.
\end{align}
当温度降至$T_c$以下时,将有宏观数量的粒子从激发态聚集到基态上去,这一现象称为玻色-爱因斯坦凝聚。

若把凝聚到基态(零能量亦即零动量态)上的粒子看成凝聚相,而把其余处在激发态上的粒子看成与凝聚相达到平衡的``气相",发生BEC的系统很像气-液相变。BEC与通常的气-液相变存在两点不同:\\
1. 气-液相变中,气相与液相在实空间是分开的,分子从气相转变到液相是实空间中的凝聚,而在BEC中,粒子从激发态($\varepsilon \neq 0$,动量也不为$0$)转变到基态(零能量与零动量态)是动量空间从$p \neq 0$的态转变到$p =0$的态,是动量空间的凝聚。BEC中激发态的粒子与零动量态的粒子占据实空间中相同的区域,并不分成实空间中的两个部分。

2. 通常的气-液相变必须存在分子之间的相互作用力,没有相互作用,相变是不可能的。BEC是对理想玻色气体,尽管理想气体分子之间的相互作用力可以忽略,但由于玻色子之间的量子起源的有效吸引,导致相变成为可能。这是由量子力学起源的相互作用,即统计关联。


\begin{align}
\nonumber \ln { \Xi} = -\ln(1-{\rm e}^{-\alpha}) + \dfrac{V}{\lambda^{3/2}_T} g_{5/2}(z) ~,
\end{align}
右边第一项(来自基态上的粒子的贡献)只依赖于$\alpha$,而与$V$和$\beta$均无关,故第一项对压强和内能均无贡献,即
\begin{align}
& p = \dfrac{1}{\beta} \dfrac{\partial }{\partial V} \ln { \Xi} =  \dfrac{1}{\beta \lambda^{3/2}_T} g_{5/2}(z) ~, \\
& \overline{E} = -\dfrac{\partial }{\partial \beta} \ln { \Xi} = \dfrac{3}{2} k T  \dfrac{V}{\lambda^{3/2}_T} g_{5/2}(z) ~.
\end{align}
\begin{equation}
\dfrac{pV}{kT} = \dfrac{\overline{E}}{\dfrac{3}{2} kT} = \dfrac{V}{\lambda^{3/2}_T} g_{5/2}(z) ~,
\end{equation}
占据在基态(零动量态)上的粒子,对压强和内能均无贡献。当$T < T_c$时,由于$g_{5/2}(z)$在$0 \leqslant z \leqslant 1$是连续收敛函数,而在$T < T_c$区,$z$接近于$1$,
\begin{equation}
\dfrac{pV}{kT} = \dfrac{\overline{E}}{\dfrac{3}{2} kT} = \dfrac{V}{h^3} (2\pi m kT)^{3/2} g_{5/2}(1) ~,
\end{equation}
其中$g_{5/2}(1) = \zeta \left(\dfrac{5}{2} \right) \approx 1.341$。
\begin{align}
& \dfrac{pV}{NkT} = \dfrac{\overline{E}}{\dfrac{3}{2} NkT} = \dfrac{g_{5/2}(1)}{g_{3/2}(1)} \left( \dfrac{T}{T_c} \right)^{3/2} ~, \\
& C_V = \left( \dfrac{\partial \overline{E} }{\partial T} \right)_V = \dfrac{15 g_{5/2}(1)}{4 g_{3/2}(1)} Nk \left( \dfrac{T}{T_c} \right)^{3/2} ~.
\end{align}
随$T \rightarrow 0$,$p, \overline{E}, C_V$分别以$T^{5/2}$和$T^{3/2}$趋于$0$。理想玻色气体的$C_V$随$T$趋于$0$符合热力学第三定律。

当$T \rightarrow \infty$($\lambda_T \rightarrow 0$)时,$\dfrac{C_V}{Nk} \rightarrow 1.5$,即趋于经典极限值。随着温度降低,热容将向着增大的方向偏离经典值,直到$T= T_c$达到最大值。$T < T_c$,它将随温度下降而趋于$0$。在$T= T_c$函数是一个尖点,即$C_V$对$T$的微商在$T_c$点不连续,但$C_V$本身在$T_c$点连续。

\begin{align}
& N = \dfrac{z}{1-z} + \dfrac{V}{\lambda^{3}_T} g_{3/2}(z) ~, \\
& 1 = \dfrac{1}{N} \dfrac{z}{1-z} + \dfrac{1}{ n \lambda^{3}_T} g_{3/2}(z) ~.
\end{align}
当$T \rightarrow 0$,
\begin{align}
N = \dfrac{z}{1-z} ~, \\
z  = \dfrac{N}{1+N} ~.
\end{align}
表明$T \rightarrow 0, z \rightarrow 1$,但$z$不可能恒等于$1$。























































%%%%%%%%%%%%%%%%%%%%%%%%%%%%%%%%%%%%%%%%%%%%%%%%%%%%%%%%%%%%%%%%%%%%%%
\bibliographystyle{unsrt_update}
\bibliography{ref}
%%%%%%%%%%%%%%%%%%%%%%%%%%%%%%%%%%%%%%%%%%%%%%%%%%%%%%%%%%%%%%%%%%%%%%


\end{document}