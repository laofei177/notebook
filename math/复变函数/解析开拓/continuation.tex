\documentclass[12pt,a4paper]{article}
%\usepackage{fontspec, xunicode, xltxtra}
%\setmainfont{Hiragino Sans GB}
\usepackage{xeCJK}
%\setCJKmainfont[BoldFont=STZhongsong, ItalicFont=STKaiti]{STSong}
%\setCJKsansfont[BoldFont=STHeiti]{STXihei}
%\setCJKmonofont{STFangsong}

%使用Xelatex编译

% 设置页面
%==================================================
\linespread{2} %行距
% \usepackage[top=1in,bottom=1in,left=1.25in,right=1.25in]{geometry}
% \headsep=2cm
% \textwidth=16cm \textheight=24.2cm
%==================================================

% 其它需要使用的宏包
%==================================================
\usepackage[colorlinks,linkcolor=blue,anchorcolor=red,citecolor=green,urlcolor=blue]{hyperref} 
\usepackage{tabularx}
\usepackage{authblk}         % 作者信息
\usepackage{algorithm}     % 算法排版
\usepackage{amsmath}     % 数学符号与公式
\usepackage{amsfonts}     % 数学符号与字体
\usepackage{mathrsfs}      % 花体
\usepackage{amssymb}
\usepackage[framemethod=TikZ]{mdframed}

\usepackage{graphicx} 
\usepackage{graphics}
\usepackage{color}
\usepackage{xcolor}
\usepackage{tcolorbox}
\usepackage{lipsum}
\usepackage{empheq}

\usepackage{fancyhdr}       % 设置页眉页脚
\usepackage{fancyvrb}       % 抄录环境
\usepackage{float}              % 管理浮动体
\usepackage{geometry}     % 定制页面格式
\usepackage{hyperref}       % 为PDF文档创建超链接
\usepackage{lineno}          % 生成行号
\usepackage{listings}        % 插入程序源代码
\usepackage{multicol}       % 多栏排版
%\usepackage{natbib}         % 管理文献引用
\usepackage{rotating}       % 旋转文字,图形,表格
\usepackage{subfigure}    % 排版子图形
\usepackage{titlesec}       % 改变章节标题格式
\usepackage{moresize}   % 更多字体大小
\usepackage{anysize}
\usepackage{indentfirst}  % 首段缩进
\usepackage{booktabs}   % 使用\multicolumn
\usepackage{multirow}    % 使用\multirow

\usepackage{wrapfig}
\usepackage{titlesec}     % 改变标题样式
\usepackage{enumitem}
\usepackage{aas_macros}

\renewcommand{\vec}[1]{\boldsymbol{#1}}
\newcommand{\me}{\mathrm{e}}
\newcommand{\mi}{\mathrm{i}}
\newcommand{\dif}{\mathrm{d}}
\newcommand{\tabincell}[2]{\begin{tabular}{@{}#1@{}}#2\end{tabular}}

\def\kpc{{\rm kpc}}
\def\km{{\rm km}}
\def\cm{{\rm cm}}
\def\TeV{{\rm TeV}}
\def\GeV{{\rm GeV}}
\def\MeV{{\rm MeV}}
\def\GV{{\rm GV}}
\def\MV{{\rm MV}}
\def\yr{{\rm yr}}
\def\s{{\rm s}}
\def\ns{{\rm ns}}
\def\GHz{{\rm GHz}}
\def\muGs{{\rm \mu Gs}}
\def\arcsec{{\rm arcsec}}
\def\K{{\rm K}}
\def\microK{\mu{\rm K}}
\def\sr{{\rm sr}}
\newcolumntype{p}{D{,}{\pm}{-1}}

\renewcommand{\figurename}{Fig.}
\renewcommand{\tablename}{Tab.}

\renewcommand{\arraystretch}{1.5}

\setlength{\parindent}{0pt}  %取消每段开头的空格

\newcounter{theo}[section]\setcounter{theo}{0}
\renewcommand{\thetheo}{\arabic{section}.\arabic{theo}}
\newenvironment{theo}[2][]{%
\refstepcounter{theo}%
\ifstrempty{#1}%
{\mdfsetup{%
frametitle={%
\tikz[baseline=(current bounding box.east),outer sep=0pt]
\node[anchor=east,rectangle,fill=blue!20]
{\strut Theorem~\thetheo};}}
}%
{\mdfsetup{%
frametitle={%
\tikz[baseline=(current bounding box.east),outer sep=0pt]
\node[anchor=east,rectangle,fill=blue!20]
{\strut Theorem~\thetheo:~#1};}}%
}%
\mdfsetup{innertopmargin=10pt,linecolor=blue!20,%
linewidth=2pt,topline=true,%
frametitleaboveskip=\dimexpr-\ht\strutbox\relax
}
\begin{mdframed}[]\relax%
\label{#2}}{\end{mdframed}}

\newcommand*\widefbox[1]{\fbox{\hspace{2em}#1\hspace{2em}}}

\newcommand*{\mysqrt}[4]{\sqrt[\leftroot{#1}\uproot{#2}#3]{#4}}

\title{解析开拓}
\author{}
\date{\today}
\begin{document}

\maketitle
\section{解析开拓的概念与方法}
\subsection{解析开拓的概念}
\begin{tcolorbox}[colback=green!5,colframe=green!40!black,title= 定义]
设函数$f(z)$在集合$E$上有定义。若存在区域$D \supset E$及区域$D$上的解析函数$F(z)$,使得在集合$E$上有$F(z) = f(z)$,则称函数$F(z)$是$f(z)$从$E$开拓到$D$的解析函数,简称$F(z)$是$f(z)$的\textcolor{red}{解析开拓}。
\end{tcolorbox}



\begin{tcolorbox}[colback=green!5,colframe=green!40!black,title= Theorem]
设平面上的区域$D_1$与$D_2$有一个公共部分$d$,函数$f_1(z)$在$D_1$解析;函数$f_2(z)$在$D_2$解析,且在$d = D_1\cap D_2$上$f_1 = f_2$,则函数
\begin{equation}
F(z) = \begin{cases}
f_1(z) ~, & z \in D_1\setminus d ~, \\
f_2(z) ~, & z \in D_2\setminus d ~, \\ 
f_1(z) = f_2(z) ~, &z \in d
\end{cases}
\end{equation}
是区域$D = D_1 +D_2$上的单值解析函数。
\end{tcolorbox}


\begin{tcolorbox}[colback=green!5,colframe=green!40!black,title= Theorem]
设区域$D_1$与$D_2$有公共边界$\Gamma_{12}$,它是一条逐段光滑曲线。设函数$f_1(z)$在$D_1$解析,在$D_1+\Gamma_{12}$上连续;函数$f_2(z)$在$D_2$内解析,在$D_2 +\Gamma_{12}$上连续,且满足$f_1(z) = f_2(z), z \in \Gamma_{12}$,则函数
\begin{equation}
F(z) = \begin{cases}
f_1(z) ~, & z \in D_1 ~, \\
f_1(z) = f_2(z) ~, & z \in \Gamma_{12} ~, \\ 
f_2(z) ~, &z \in D_2
\end{cases}
\end{equation}
是区域$D = D_1 +\Gamma_{12} +D_2$上的单值解析函数。
\end{tcolorbox}




\subsection{解析开拓的具体方法}

\begin{tcolorbox}[colback=green!5,colframe=green!40!black,title= 黎曼-施瓦茨对称定理]
设区域$D_1$在上半平面上,它有一段边界$\Gamma_1$在实轴上,函数$f_1(z)$在$D_1$内解析,在$D_1 +\Gamma_{1}$上连续,且在$\Gamma_1$取实数值,则可以构成一个区域$D_2$,它与$D_1$关于实轴对称,函数$f_2(z)$:
\begin{equation}
f_2(z) = \overline{f_1(\bar{z})} ~, ~~ z \in D_2 + \Gamma_1 ~,
\end{equation}
在$D_2$解析,在$D_2 +\Gamma_{1}$上连续,$f_2(z) = f_1(z), z \in \Gamma_1$,且函数
\begin{equation}
F(z) = \begin{cases}
f_1(z) ~, & z \in D_1 ~, \\
f_1(z) = f_2(z) ~, & z \in \Gamma_{1} ~, \\ 
f_2(z) ~, &z \in D_2
\end{cases}
\end{equation}
就是区域$D = D_1 +\Gamma_{1} +D_2$上的单值解析函数。
\end{tcolorbox}



幂级数开拓法:设函数$f_1(z)$在区域$D_1$解析,$z_0 \in D$,$z_0$到$D$的边界$\partial D$的距离为
\begin{equation*}
\rho(z_0) = \underset{z \in \partial D}{\rm min} |z-z_0| ~,
\end{equation*}





\begin{tcolorbox}[colback=green!5,colframe=green!40!black,title= 定义]
设函数$f(z)$在区域$D$内解析。对于区域$D$的边界$\partial D$上的任意点$z_0$。若存在一个在$z=z_0$的领域$|z-z_0| < \rho$上的解析函数$\varphi_{z_0}(z)$,它在区域$(|z-z_0| < \rho) \cap D$上取值等于$f(z)$,则称$z_0$是函数$f(z)$的\textcolor{red}{正则点};若这样的解析函数$\varphi_{z_0}(z)$保存在,则称$z_0$是函数$f(z)$的\textcolor{red}{奇点}。
\end{tcolorbox}


对于区域内点,若函数$f(z)$在$z=z_0$解析,则称$z_0$是它的正则点。若$z_0$是$f(z)$的孤立奇点,则当$z_0$是可去奇点时,它就可以看做是正则点;若$z_0$是极点或本性奇点时,则$z_0$是奇点。


对于在圆内的解析函数,有幂级数开拓的定理
\begin{tcolorbox}[colback=green!5,colframe=green!40!black,title= 定义]
设函数$f(z)$在圆$|z-z_0| < \rho$,$z_1 \in |z-z_0| < \rho$,
\begin{equation}
f(z) = \sum\limits_{n=0}^\infty \dfrac{f^{(n)}(z_1)}{n!} (z-z_1)^n ~.
\end{equation}
要使该幂级数的收敛半径$R > \rho - |z_1 -z_0|$的充要条件是点$z_2 = z_0 + \rho e^{i {\rm arg}(z_1 -z_0)}$是正则点。
\end{tcolorbox}



\begin{tcolorbox}[colback=green!5,colframe=green!40!black,title= 推论]
要使点$z_2 = z_0 + \rho e^{i {\rm arg}(z_1 -z_0)}$是奇点的充要条件是
\begin{equation}
R = \dfrac{1}{\underset{n \rightarrow \infty}\lim \sqrt[\leftroot{-2}\uproot{10}n]{\dfrac{|f^{(n)} (z_1) |}{n!}} }  = \rho - |z_1 -z_0| ~.
\end{equation}
\end{tcolorbox}


\begin{tcolorbox}[colback=green!5,colframe=green!40!black,title= 定理]
设幂级数
\begin{equation}
\sum\limits_{n=0}^\infty a_n(z-z_0)^n = f(z) ~, ~~ |z-z_0| < \rho
\end{equation}
的收敛半径是$\rho$,则函数$f(z)$在收敛圆$|z-z_0| = \rho$上必有奇点。
\end{tcolorbox}




\section{完全解析函数与黎曼曲面}







\section{利用多值函数进行积分计算}











\end{document}