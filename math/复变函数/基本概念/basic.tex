\documentclass[12pt,a4paper]{article}
%\usepackage{fontspec, xunicode, xltxtra}
%\setmainfont{Hiragino Sans GB}
\usepackage{xeCJK}
%\setCJKmainfont[BoldFont=STZhongsong, ItalicFont=STKaiti]{STSong}
%\setCJKsansfont[BoldFont=STHeiti]{STXihei}
%\setCJKmonofont{STFangsong}

%使用Xelatex编译

% 设置页面
%==================================================
\linespread{2} %行距
% \usepackage[top=1in,bottom=1in,left=1.25in,right=1.25in]{geometry}
% \headsep=2cm
% \textwidth=16cm \textheight=24.2cm
%==================================================

% 其它需要使用的宏包
%==================================================
\usepackage[colorlinks,linkcolor=blue,anchorcolor=red,citecolor=green,urlcolor=blue]{hyperref}
\usepackage{tabularx}
\usepackage{authblk}         % 作者信息
\usepackage{algorithm}     % 算法排版
\usepackage{amsmath}     % 数学符号与公式
\usepackage{amsfonts}     % 数学符号与字体
\usepackage{mathrsfs}      % 花体
\usepackage{graphics}
\usepackage{color}
\usepackage{fancyhdr}       % 设置页眉页脚
\usepackage{fancyvrb}       % 抄录环境
\usepackage{float}              % 管理浮动体
\usepackage{geometry}     % 定制页面格式
\usepackage{hyperref}       % 为PDF文档创建超链接
\usepackage{lineno}          % 生成行号
\usepackage{listings}        % 插入程序源代码
\usepackage{multicol}       % 多栏排版
\usepackage{natbib}         % 管理文献引用
\usepackage{rotating}       % 旋转文字,图形,表格
\usepackage{subfigure}    % 排版子图形
\usepackage{titlesec}       % 改变章节标题格式
\usepackage{moresize}   % 更多字体大小
\usepackage{anysize}
\usepackage{indentfirst}  % 首段缩进
\usepackage{booktabs}   % 使用\multicolumn
\usepackage{multirow}    % 使用\multirow
\usepackage{graphicx}
\usepackage{wrapfig}
\usepackage{xcolor}
\usepackage{titlesec}     % 改变标题样式
\usepackage{enumitem}

\renewcommand{\vec}[1]{\boldsymbol{#1}}
\newcommand{\me}{\mathrm{e}}
\newcommand{\mi}{\mathrm{i}}
\newcommand{\dif}{\mathrm{d}}
\newcommand{\tabincell}[2]{\begin{tabular}{@{}#1@{}}#2\end{tabular}}

\def\kpc{{\rm kpc}}
\def\km{{\rm km}}
\def\cm{{\rm cm}}
\def\TeV{{\rm TeV}}
\def\GeV{{\rm GeV}}
\def\MeV{{\rm MeV}}
\def\GV{{\rm GV}}
\def\MV{{\rm MV}}
\def\yr{{\rm yr}}
\def\s{{\rm s}}
\def\ns{{\rm ns}}
\def\GHz{{\rm GHz}}
\def\muGs{{\rm \mu Gs}}
\def\arcsec{{\rm arcsec}}
\def\K{{\rm K}}
\def\microK{\mu{\rm K}}
\def\sr{{\rm sr}}
\newcolumntype{p}{D{,}{\pm}{-1}}

\renewcommand{\figurename}{Fig.}
\renewcommand{\tablename}{Tab.}

\renewcommand{\arraystretch}{1.5}

\setlength{\parindent}{0pt}  %取消每段开头的空格

\title{基本概念}
\author{}
\date{\today}
\begin{document}

\maketitle




\section{复数的运算}
\begin{equation}
z^n = r^n e^{in\theta} = r^n (\cos n\theta +i\sin n\theta)
\end{equation}

\textcolor{red}{De Moivre公式}
\begin{equation}
(\cos \theta +i \sin \theta)^n = \cos n\theta +i \sin n\theta
\end{equation}


\subsection{复数运算的规律}
1. 和的封闭性:设$z_1$和$z_2$是复数,则$z_1 +z_2$也是复数;

2. 加法的交换律和结合律:设$z_1$、$z_2$及$z_3$都是复数,则
\begin{eqnarray}
\nonumber z_1 +(z_2 +z_3) &=& (z_1 +z_2) +z_3 \\
z_1 + z_2 &=& z_2 + z_1
\end{eqnarray}

3. $0 = 0 +i0$是复数,且对于任意的复数$z$,都有$0+z = z+0 = z$,称为关于加法有主元素$0$,也称$0$为零元素。

4. 对于任意一个复数$z$,有一个复数$-z$,使得$z+(-z) = (-z) + z = 0$,称为关于加法有逆元素$-z$。

代数上把满足以上四个性质的数系称为构成一个\textcolor{red}{加法群(交换群)}。

5. 设$z_1$及$z_2$是复数,则$z_1z_2$也是复数,称为乘法的封闭性。

6. 乘法的结合律和交换律:设$z_1$、$z_2$及$z_3$都是复数,则
\begin{eqnarray}
\nonumber z_1(z_2z_3) &=& (z_1z_2)z_3 \\
z_1z_2 &=& z_2z_1
\end{eqnarray}

7. $1$是复数,对于任意的复数$z$,都有$1\cdot z = z\cdot 1 = z$,称为关于乘法有主元素$1$,也称$1$为单位元素。

8. 对于任意一个非零复数$z$,都有一个复数$1/z$,使得
\begin{equation}
z\cdot \frac{1}{z} = \frac{1}{z} \cdot z = 1
\end{equation}
称为关于乘法有逆元素$1/z$。

代数上把满足以上四个性质的数系称为构成一个\textcolor{red}{乘法群(交换群)}

9. 分配律:设$z_1$、$z_2$及$z_3$都是复数,则
\begin{equation}
(z_1 +z_2)z_3 = z_1z_3 +z_2z_3
\end{equation}











































\end{document}
