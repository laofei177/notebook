\documentclass[12pt,a4paper]{article}
%\usepackage{fontspec, xunicode, xltxtra}  
%\setmainfont{Hiragino Sans GB}  
\usepackage{xeCJK}
\setCJKmainfont[BoldFont=STZhongsong, ItalicFont=STKaiti]{STSong}
\setCJKsansfont[BoldFont=STHeiti]{STXihei}
\setCJKmonofont{STFangsong}

%使用Xelatex编译

% 设置页面
%==================================================
\linespread{2} %行距
% \usepackage[top=1in,bottom=1in,left=1.25in,right=1.25in]{geometry}
% \headsep=2cm
% \textwidth=16cm \textheight=24.2cm
%==================================================

% 其它需要使用的宏包
%==================================================
\usepackage[colorlinks,linkcolor=blue,anchorcolor=red,citecolor=green,urlcolor=blue]{hyperref} 
\usepackage{tabularx}
\usepackage{authblk}         % 作者信息
\usepackage{algorithm}     % 算法排版
\usepackage{amsmath}     % 数学符号与公式
\usepackage{amsfonts}     % 数学符号与字体
\usepackage{mathrsfs}      % 花体
\usepackage{graphics}
\usepackage{color}
\usepackage{fancyhdr}       % 设置页眉页脚
\usepackage{fancyvrb}       % 抄录环境
\usepackage{float}              % 管理浮动体
\usepackage{geometry}     % 定制页面格式
\usepackage{hyperref}       % 为PDF文档创建超链接
\usepackage{lineno}          % 生成行号
\usepackage{listings}        % 插入程序源代码
\usepackage{multicol}       % 多栏排版
\usepackage{natbib}         % 管理文献引用
\usepackage{rotating}       % 旋转文字,图形,表格
\usepackage{subfigure}    % 排版子图形
\usepackage{titlesec}       % 改变章节标题格式
\usepackage{moresize}   % 更多字体大小
\usepackage{anysize}
\usepackage{indentfirst}  % 首段缩进
\usepackage{booktabs}   % 使用\multicolumn
\usepackage{multirow}    % 使用\multirow
\usepackage{graphicx} 
\usepackage{wrapfig}
\usepackage{xcolor}
\usepackage{titlesec}     % 改变标题样式
\usepackage{enumitem}
\usepackage{harpoon}   %矢量符号

\newcommand{\myvec}[1]%
   {\stackrel{\raisebox{-2pt}[0pt][0pt]{\small$\rightharpoonup$}}{#1}}  %矢量符号
\renewcommand{\vec}[1]{\boldsymbol{#1}}
\newcommand{\me}{\mathrm{e}}
\newcommand{\mi}{\mathrm{i}}
\newcommand{\dif}{\mathrm{d}}
\newcommand{\tabincell}[2]{\begin{tabular}{@{}#1@{}}#2\end{tabular}}

\def\kpc{{\rm kpc}}
\def\km{{\rm km}}
\def\cm{{\rm cm}}
\def\TeV{{\rm TeV}}
\def\GeV{{\rm GeV}}
\def\MeV{{\rm MeV}}
\def\GV{{\rm GV}}
\def\MV{{\rm MV}}
\def\yr{{\rm yr}}
\def\s{{\rm s}}
\def\ns{{\rm ns}}
\def\GHz{{\rm GHz}}
\def\muGs{{\rm \mu Gs}}
\def\arcsec{{\rm arcsec}}
\def\K{{\rm K}}
\def\microK{\mu{\rm K}}
\def\sr{{\rm sr}}
\newcolumntype{p}{D{,}{\pm}{-1}}

\renewcommand{\figurename}{Fig.}
\renewcommand{\tablename}{Tab.}

\renewcommand{\arraystretch}{1.5}

\setlength{\parindent}{0pt}  %取消每段开头的空格

\title{傅里叶级数和傅里叶积分}
\author{}
\date{\today}
\begin{document}

\maketitle
周期现象

周期函数

与所考虑的周期线性有关的各种量,在经历周期$T$后,重新取得它们的原值
\begin{equation}
\varphi(t+T) = \varphi(t)
\end{equation}

周期函数的图解可以由一系列正弦型量的图解得来;
\begin{equation}
\varphi(t) = A_0 +\sum_{n=1}^{\infty} A_n \sin(n\omega t +\alpha_n)
\label{sin_exp}
\end{equation}
其中
\begin{equation}
\omega = \frac{2\pi}{T}
\end{equation}

由函数$\varphi(t)$表示的复杂振动可以分解成各别的调和振动;

函数$\varphi(t)$的\textcolor{red}{调和成分}或\textcolor{red}{调和素}:组成展开式(\ref{sin_exp})的各个正弦型量;

\textcolor{red}{调和分析}:将周期函数分解成调和素的手续;

\section{傅里叶级数}
假定函数$f(x)$在区间$[-\pi,\pi]$上按常义或者非常义可积分,对后一情形,假定函数绝对可积;设展开式
\begin{equation}
f(x) = a_0 + \sum_{n=1}^{\infty} (a_n \cos nx +b_n \sin nx)
\end{equation}
成立;

欧拉—傅里叶公式
\begin{equation}
a_0 = \frac{1}{2\pi} \int_{-\pi}^{\pi} f(x) \dif x
\end{equation}

\begin{equation}
a_m = \frac{1}{\pi} \int_{-\pi}^{\pi} f(x) \cos mx \dif x ~~ (m = 1,2,3,\cdots)
\end{equation}

\begin{equation}
b_m = \frac{1}{\pi} \int_{-\pi}^{\pi} f(x) \sin mx \dif x ~~(m = 1,2,3,\cdots)
\end{equation}

如果周期是$2\pi$的函数$f(x)$可以展开成一致收敛的三角级数,则这级数一定是$f(x)$的傅里叶级数;

两个函数在这区间上\textcolor{red}{正交}:如果在区间$[a,b]$上所定义的两函数$\varphi(x)$和$\psi(x)$的乘积,其积分为$0$,
\begin{equation}
\int_{a}^{b} \varphi(x) \psi(x) \dif x = 0
\end{equation}

\textcolor{red}{正交函数系}:考虑定义在区间$[a,b]$上的函数系$\{\varphi_n(x) \}$。设系中各函数与它们的平方在$[a,b]$上皆可积分,则它们的两两乘积在同一区间上也可积分。如果系中各函数两两正交,即
\begin{equation}
\int_{a}^{b} \varphi_n(x) \varphi_m(x) \dif x = 0 ~~(n,m = 0,1,2,3,\cdots; n\neq m)
\end{equation}

假定
\begin{equation}
\int_{a}^{b} \varphi^2_n(x) \dif x = \lambda_n > 0
\end{equation}
若$\lambda_n = 1~~(n = 0,1,2,\cdots)$,函数系是\textcolor{red}{规范}的;

\textcolor{red}{加权$p(x)$的正交性}:
\begin{equation}
\int_{a}^{b} p(x) \varphi(x) \psi(x) \dif x = 0
\end{equation}
如:函数系$\{J_0(\xi_n x)\}$是加权$x$正交的;

已给函数对于函数系$\{\varphi_n(x)\}$的\textcolor{red}{(广义)傅里叶级数}

设在区间$[a,b]$上已给任一正交系$\{\varphi_n(x)\}$。将定义在$[a,b]$上的函数$f(x)$展开成函数$\varphi$的级数,
\begin{equation}
f(x) = c_0 \varphi_0(x) +c_1 \varphi_1(x) +\cdots +c_n \varphi_n(x) +\cdots 
\end{equation}
其中
\begin{equation}
c_m = \frac{1}{\lambda_m} \int_a^b f(x) \varphi_m(x) \dif x ~~(m = 0,1,2,\cdots)
\end{equation}

三角插值法

用三角多项式
\begin{equation}
\sigma_n(x) = \alpha_0 + \sum_{k=1}^{n} (\alpha_k \cos kx +\beta_k \sin kx)
\label{triang_poly}
\end{equation}
作为函数$f(x)$的近似式,使三角多项式与函数在许多点上有相同的值。

选取$n$阶三角多项式(\ref{triang_poly})的$2n+1$个系数:$\alpha_0,\alpha_1,\beta_1,\cdots,\alpha_n,\beta_n$,使得在区间$(-\pi,\pi)$内预先指定的$2n+1$个点处,如
\begin{equation}
\xi_i = i \lambda ~~(i = -n, -n+1, \cdots, -1, 0, 1, \cdots, n-1, n)
\end{equation}
各点处,
\begin{equation}
 \lambda = \frac{2\pi}{2n+1},
\end{equation}
三角多项式的值与函数$f(x)$的值相等。

\begin{equation}
\alpha_0 + \sum_{k=1}^{n} (\alpha_k \cos k\xi_i +\beta_k \sin k\xi_i) = f(\xi_i) ~~(i = -n, -n+1, \cdots, n)
\end{equation}


\subsection{函数的傅里叶级数展开式}

\subsection{非周期函数}

\subsection{只含余弦或正弦}


\subsection{傅里叶级数的收敛性}



\subsection{傅里叶级数的逐项积分法}
假定函数$f(x)$在区间$[-\pi,\pi]$上绝对可积,设它的傅里叶级数
\begin{equation}
f(x) \sim \frac{a_0}{2} + \sum_{n=1}^{\infty} (a_n \cos nx +b_n \sin nx)
\label{fourier}
\end{equation}

任意区间$[x', x'']$(其中$-\pi \leq x' < x'' \leq \pi$)
\begin{equation}
\int_{x'}^{x''} f(x) \dif x = \int_{x'}^{x''} \frac{a_0}{2} \dif x +\sum_{n=1}^{\infty} \int_{x'}^{x''} [a_n \cos nx +b_n \sin nx ] \dif x
\end{equation}
函数$f(x)$的积分可将与它相应的傅里叶级数逐项积分而求得。即令不假定傅里叶级数(\ref{fourier})本身收敛于函数$f(x)$,还恒可将它逐项积分

\subsection{傅里叶级数的逐项微分法}


\section{傅里叶积分}



设函数$f(x)$在点$x$处连续,如果不连续,则设
\begin{equation}
f(x) = \frac{f(x+0) +f(x-0)}{2}
\end{equation}
成立。

在主值意义下一定存在,且
\begin{equation}
\text{V.p.} \int_{-\infty}^{+\infty} \dif z \int_{-\infty}^{+\infty} f(u) \sin z(u-x) \dif u = 0
\end{equation}

\begin{equation}
F(z) = \frac{1}{\sqrt{2\pi} } \int_{-\infty}^{+\infty} f(u) e^{izu} \dif u
\end{equation}

\begin{equation}
f(x) = \frac{1}{\sqrt{2\pi} } \int_{-\infty}^{+\infty} F(z) e^{-ixz} \dif z 
\end{equation}



























\end{document}