\documentclass[12pt,a4paper]{article}
%\usepackage{fontspec, xunicode, xltxtra}  
%\setmainfont{Hiragino Sans GB}  
\usepackage{xeCJK}
%\setCJKmainfont[BoldFont=STZhongsong, ItalicFont=STKaiti]{STSong}
%\setCJKsansfont[BoldFont=STHeiti]{STXihei}
%\setCJKmonofont{STFangsong}

%使用Xelatex编译

% 设置页面
%==================================================
\linespread{2} %行距
% \usepackage[top=1in,bottom=1in,left=1.25in,right=1.25in]{geometry}
% \headsep=2cm
% \textwidth=16cm \textheight=24.2cm
%==================================================

% 其它需要使用的宏包
%==================================================
\usepackage[colorlinks,linkcolor=blue,anchorcolor=red,citecolor=green,urlcolor=blue]{hyperref} 
\usepackage{tabularx}
\usepackage{authblk}         % 作者信息
\usepackage{algorithm}     % 算法排版
\usepackage{amsmath}     % 数学符号与公式
\usepackage{amsfonts}     % 数学符号与字体
\usepackage{mathrsfs}      % 花体
\usepackage{amssymb}
\usepackage[framemethod=TikZ]{mdframed}

\usepackage{graphicx} 
\usepackage{graphics}
\usepackage{color}
\usepackage{xcolor}
\usepackage{tcolorbox}
\usepackage{lipsum}
\usepackage{empheq}

\usepackage{fancyhdr}       % 设置页眉页脚
\usepackage{fancyvrb}       % 抄录环境
\usepackage{float}              % 管理浮动体
\usepackage{geometry}     % 定制页面格式
\usepackage{hyperref}       % 为PDF文档创建超链接
\usepackage{lineno}          % 生成行号
\usepackage{listings}        % 插入程序源代码
\usepackage{multicol}       % 多栏排版
%\usepackage{natbib}         % 管理文献引用
\usepackage{rotating}       % 旋转文字,图形,表格
\usepackage{subfigure}    % 排版子图形
\usepackage{titlesec}       % 改变章节标题格式
\usepackage{moresize}   % 更多字体大小
\usepackage{anysize}
\usepackage{indentfirst}  % 首段缩进
\usepackage{booktabs}   % 使用\multicolumn
\usepackage{multirow}    % 使用\multirow

\usepackage{wrapfig}
\usepackage{titlesec}     % 改变标题样式
\usepackage{enumitem}
\usepackage{aas_macros}
\usepackage{bigints}
\usepackage{harpoon}   %矢量符号
\usepackage{extarrows}

\newcommand{\myvec}[1]%
   {\stackrel{\raisebox{-2pt}[0pt][0pt]{\small$\rightharpoonup$}}{#1}}  %矢量符号
\renewcommand{\vec}[1]{\boldsymbol{#1}}
\newcommand{\me}{\mathrm{e}}
\newcommand{\mi}{\mathrm{i}}
\newcommand{\dif}{\mathrm{d}}
\newcommand{\tabincell}[2]{\begin{tabular}{@{}#1@{}}#2\end{tabular}}

\def\kpc{{\rm kpc}}
\def\km{{\rm km}}
\def\cm{{\rm cm}}
\def\TeV{{\rm TeV}}
\def\GeV{{\rm GeV}}
\def\MeV{{\rm MeV}}
\def\GV{{\rm GV}}
\def\MV{{\rm MV}}
\def\yr{{\rm yr}}
\def\s{{\rm s}}
\def\ns{{\rm ns}}
\def\GHz{{\rm GHz}}
\def\muGs{{\rm \mu Gs}}
\def\arcsec{{\rm arcsec}}
\def\K{{\rm K}}
\def\microK{\mu{\rm K}}
\def\sr{{\rm sr}}
\newcolumntype{p}{D{,}{\pm}{-1}}

\renewcommand{\figurename}{Fig.}
\renewcommand{\tablename}{Tab.}

\renewcommand{\arraystretch}{1.5}

\setlength{\parindent}{0pt}  %取消每段开头的空格

\newcounter{theo}[section]\setcounter{theo}{0}
\renewcommand{\thetheo}{\arabic{section}.\arabic{theo}}
\newenvironment{theo}[2][]{%
\refstepcounter{theo}%
\ifstrempty{#1}%
{\mdfsetup{%
frametitle={%
\tikz[baseline=(current bounding box.east),outer sep=0pt]
\node[anchor=east,rectangle,fill=blue!20]
{\strut Theorem~\thetheo};}}
}%
{\mdfsetup{%
frametitle={%
\tikz[baseline=(current bounding box.east),outer sep=0pt]
\node[anchor=east,rectangle,fill=blue!20]
{\strut Theorem~\thetheo:~#1};}}%
}%
\mdfsetup{innertopmargin=10pt,linecolor=blue!20,%
linewidth=2pt,topline=true,%
frametitleaboveskip=\dimexpr-\ht\strutbox\relax
}
\begin{mdframed}[]\relax%
\label{#2}}{\end{mdframed}}

\newcommand*\widefbox[1]{\fbox{\hspace{2em}#1\hspace{2em}}}


\title{Markov链}
\author{}
\date{\today}
\begin{document}

\maketitle

\section{Markov过程及其概率分布}

由时刻$t_0$系统或过程所处的状态,可以决定系统或过程在时刻$t > t_0$所处的状态,而无需借助于$t_0$以前系统或过程所处状态的历史资料。

\textcolor{red}{马尔可夫性}或\textcolor{red}{无后效性}:过程(或系统)在时刻$t_0$所处的状态为已知的条件下,过程在时刻$t > t_0$所处状态的条件分布与过程在时刻$t_0$之前所处的状态无关。

设随机过程$\{X(t), t\in T \}$的状态空间为$I$。若对时间$t$的任意$n$个数值$t_1 < t_2 < \cdots < t_n, n \geqslant 3, t_i \in T$,在条件$X(t_i) = x_i, x_i \in I, i = 1, 2, \cdots, n -1$下,$X(t_n)$的条件分布函数恰等于在条件$X(t_{n-1}) = x_{n-1}$下$X(t_n)$的条件分布函数,即
\begin{align}
\nonumber P\{X(t_n) &\leqslant x_n | X(t_1) = x_1, X(t_2) = x_2, \cdots, X(t_{n-1}) = x_{n-1} \} \\
& = P\{X(t_n) \leqslant x_n | X(t_{n-1}) = x_{n-1} \} ~, x_n \in \vec{R} ~,
\end{align}
或
\begin{equation*}
F_{t_n|t_1 \cdots t_{n-1}}(x_n, t_n | x_1, x_2, \cdots, x_{n-1}; t_1, t_2, \cdots, t_{n-1}) = F_{t_n|t_{n-1}} (x_n, t_n | x_{n-1}, t_{n-1})
\end{equation*}
称过程$\{X(t), t\in T \}$具有马尔可夫性或无后效性,并称此过程为\textcolor{red}{马尔可夫过程}。

时间和状态都是离散的马尔可夫过程称为\textcolor{red}{马尔可夫链},简称马氏链,记为$\{X_n = X(n), n = 0, 1, 2, \cdots \}$,它可以看作在时间集$T_1 = \{0, 1, 2, \cdots \}$上对离散状态的马氏过程相继观察的结果。记链的状态空间为$I = \{a_1, a_2, \cdots \}, a_i \in \vec{R}$。在链的情况下,马尔可夫性用条件分布律表示,即对任意的正整数 $n, r$和$0 \leqslant t_1 < t_2 < \cdots < t_r < m; t_i, m , n+m \in T_1$,有
\begin{align}
& P\{X_{m+n} = a_j | X_{t_1} = a_{i_1}, X_{t_2} = a_{i_2}, \cdots, X_{t_r} = a_{i_r}, X_{m} = a_{i} \} \\
& = P\{X_{m+n} = a_j | X_m = a_i \}  = P_{ij}(m, m+n) ~,
\end{align}
其中$a_i \in I$。称条件概率
\begin{equation*}
 P_{ij}(m, m+n) = P\{X_{m+n} = a_j | X_m = a_i \} ~,
\end{equation*}
为马氏链在时刻$m$处于状态$a_i$条件下,在时刻$m+n$转移到状态$a_j$的\textcolor{red}{转移概率}。

由于链在时刻$m$从任何一个状态$a_i$出发,到另一时刻$m+n$,必然转移到$a_1, a_2, \cdots$诸状态中的某一个,
\begin{equation}
\sum_{j=1}^{+\infty} = P_{ij}(m, m+n) = 1, ~~ i = 1, 2, \cdots ~.
\end{equation}
由转移概率组成的矩阵$\vec{P}(m, m+n) = (P_{ij}(m, m+n) )$称为马氏链的\textcolor{red}{转移概率矩阵}。此矩阵的每一行元之和等于$1$。

当转移概率$P_{ij}(m, m+n)$只与$i, j$以及时间间距$n$有关时,记为$P_{ij}(n)$,即
\begin{equation}
P_{ij}(m, m+n) = P_{ij}(n) ~,
\end{equation}
称此转移概率具有\textcolor{red}{平稳性}。也称此链是\textcolor{red}{齐次的},或\textcolor{red}{时齐的}。

在齐次马氏链情况下,转移概率
\begin{equation}
P_{ij}(n) = P\{X_{m+n} = a_j | X_m = a_i \} ~,
\end{equation}
称为马氏链的$n$步转移概率,$\vec{P}(n) = (P_{ij}(n))$为$n$步转移概率矩阵。

一步转移概率
\begin{equation}
P_{ij} = P_{ij}(1) = P\{X_{m+1} = a_j | X_m = a_i \} ~,
\end{equation}
一步转移概率矩阵
\begin{equation*}
\renewcommand{\arraystretch}{0.7}
\begin{pmatrix}
p_{11} & p_{12} & \cdots & p_{1j} & \cdots \\
p_{21} & p_{22} & \cdots & p_{2j} & \cdots \\
\vdots   & \vdots  &            & \vdots \\
p_{i1} & p_{i2} & \cdots & p_{ij} & \cdots \\ 
\vdots   & \vdots  &            & \vdots \\
\end{pmatrix}
 = \vec{P}(1) = \vec{P} ~.
\end{equation*}









齐次马氏链的有限维分布:记
\begin{equation*}
p_j(0) = P\{X_0 = a_j \} ~, ~~ a_j \in I, ~ j = 1, 2, \cdots, 
\end{equation*}
称它为马氏链的初始分布。马氏链在任一时刻$n \in T_1$的一维分布:
\begin{equation*}
p_j(n) = P\{X_n = a_j \} ~, ~~ a_j \in I, ~ j = 1, 2, \cdots.
\end{equation*}


写成行向量
\begin{equation}
\vec{p}(n) = (p_1(n), p_2(n), \cdots, p_j(n), \cdots)
\end{equation}

\begin{equation}
\vec{p}(n) = \vec{p}(0) \vec{P}(n)  ~.
\end{equation}
马氏链在任一时刻$n \in T_1$时的一维分布由初始分布$\vec{p}(0)$和$n$步转移概率矩阵所确定。


马氏链的有限维分布同样完全由初始分布和转移概率所确定。

转移概率决定了马氏链运动的统计规律。


\section{多步转移概率的确定}
设$\{X(n), n = 0, 1, 2, \cdots \}$是一齐次马氏链,则对任意的$u, v \in T$,有
\begin{equation}
P_{ij} (u +v) = \sum_{k=1}^{+\infty} P_{ik}(u) P_{kj}(v) ~, ~~ i, j = 1, 2, \cdots 
\end{equation}
此为Chapman-Kolmogorov方程,简称C-K方程。C-K方程基于下述事实,即``从时刻$s$所处的状态$a_i$,即$X(s) = a_i$出发,经时段$u+v$转移到状态$a_j$,即$X(s+u+v) = a_j$"这一事件可分解为``从$X(s) = a_i$出发,先经时段$u$转移到中间状态$a_k(k = 1, 2, \cdots)$,再从$a_k$经时段$v$转移到状态$a_j$"这样一些事件的和事件。

C-K方程也可以写成矩阵形式:
\begin{equation}
\vec{P} (u +v) = \vec{P} (u) \vec{P} (v) ~.
\end{equation}
令$u = 1, v = n-1$,
\begin{equation}
\vec{P} (n) = \vec{P} (1) \vec{P} (n-1) = \vec{P}\vec{P}(n-1) = \cdots = \vec{P}^n ~.
\end{equation}
对于齐次马氏链,$n$步转移概率矩阵是一步转移概率矩阵的$n$次方。










































\section{遍历性}



一般,设齐次马氏链的状态空间为$I$,若对于所有$a_i, a_j \in I$,转移概率$P_{ij}(n)$存在极限
\begin{equation*}
\underset{n\rightarrow \infty}\lim P_{ij}(n) = \pi_j  ~~~ (\text{不依赖于}~ i)
\end{equation*}
或
\begin{equation*}
\vec{P} (n) = \vec{P}^n  \xlongrightarrow{(n\rightarrow +\infty)} \renewcommand{\arraystretch}{0.7}
\begin{pmatrix}
\pi_{1} & \pi_{2} & \cdots & \pi_{j} & \cdots \\
\pi_{1} & \pi_{2} & \cdots & \pi_{j} & \cdots \\
\vdots   & \vdots  &            & \vdots \\
\pi_{1} & \pi_{2} & \cdots & \pi_{j} & \cdots \\ 
\vdots   & \vdots  &            & \vdots \\
\end{pmatrix}
\end{equation*}
则称此链具有\textcolor{red}{遍历性}。又若$\sum_j \pi_j = 1$,则同时称$\vec{\pi} = (\pi_1, \pi_2, \cdots)$为链的\textcolor{red}{极限分布}。

有限链的遍历性的一个充分条件
\begin{tcolorbox}[colback=green!5,colframe=green!40!black,title= 定理]
设齐次马氏链$\{X_n, n \geqslant 1 \}$的状态空间为$I = \{a_1, a_2, \cdots, a_N \}$,$\vec{P}$是它的一步转移概率矩阵,若存在正整数$m$,使对任意的$a_i, a_j \in I$,都有
\begin{equation}
P_{ij} (m) > 0 ~, ~~~ i , j = 1, 2, \cdots, N ~,
\end{equation}
则此链具有遍历性,且有极限分布$\vec{\pi} = (\pi_1, \pi_2, \cdots, \pi_N)$,它是方程组
\begin{equation}
\vec{\pi} = \vec{\pi} \vec{P} ~~\text{或即} ~~ \pi_j = \sum_{i=1}^N \pi_i p_{ij} ~, ~~ j = 1, 2, \cdots, N ~
\end{equation}
的满足条件
\begin{equation}
\pi_j > 0 ~, \sum_{j=1}^N = 1 
\end{equation}
的唯一解。
\end{tcolorbox}




%%%%%%%%%%%%%%%%%%%%%%%%%%%%%%%%%%%%%%%%%%%%%%%%%%%%%%%%%%%%%%%%%%%%%%
\bibliographystyle{unsrt_update}
\bibliography{ref}
%%%%%%%%%%%%%%%%%%%%%%%%%%%%%%%%%%%%%%%%%%%%%%%%%%%%%%%%%%%%%%%%%%%%%%


\end{document}