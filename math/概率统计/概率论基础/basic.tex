\documentclass[12pt,a4paper]{article}
%\usepackage{fontspec, xunicode, xltxtra}  
%\setmainfont{Hiragino Sans GB}  
\usepackage{xeCJK}
%\setCJKmainfont[BoldFont=STZhongsong, ItalicFont=STKaiti]{STSong}
%\setCJKsansfont[BoldFont=STHeiti]{STXihei}
%\setCJKmonofont{STFangsong}

%使用Xelatex编译

% 设置页面
%==================================================
\linespread{2} %行距
% \usepackage[top=1in,bottom=1in,left=1.25in,right=1.25in]{geometry}
% \headsep=2cm
% \textwidth=16cm \textheight=24.2cm
%==================================================

% 其它需要使用的宏包
%==================================================
\usepackage[colorlinks,linkcolor=blue,anchorcolor=red,citecolor=green,urlcolor=blue]{hyperref} 
\usepackage{tabularx}
\usepackage{authblk}         % 作者信息
\usepackage{algorithm}     % 算法排版
\usepackage{amsmath}     % 数学符号与公式
\usepackage{amssymb}
\usepackage{amsfonts}     % 数学符号与字体
\usepackage{mathrsfs}      % 花体
\usepackage{graphics}
\usepackage{color}
\usepackage{fancyhdr}       % 设置页眉页脚
\usepackage{fancyvrb}       % 抄录环境
\usepackage{float}              % 管理浮动体
\usepackage{geometry}     % 定制页面格式
\usepackage{hyperref}       % 为PDF文档创建超链接
\usepackage{lineno}          % 生成行号
\usepackage{listings}        % 插入程序源代码
\usepackage{multicol}       % 多栏排版
\usepackage{natbib}         % 管理文献引用
\usepackage{rotating}       % 旋转文字,图形,表格
\usepackage{subfigure}    % 排版子图形
\usepackage{titlesec}        % 改变章节标题格式
\usepackage{moresize}    % 更多字体大小
\usepackage{anysize}
\usepackage{indentfirst}   % 首段缩进
\usepackage{booktabs}    % 使用\multicolumn
\usepackage{multirow}     % 使用\multirow
\usepackage{graphicx} 
\usepackage{wrapfig}
\usepackage{xcolor}
\usepackage{titlesec}       % 改变标题样式
\usepackage{enumitem}
\usepackage{harpoon}    %矢量符号
\usepackage{xcolor}        % 高亮




\newcommand{\myvec}[1]%
   {\stackrel{\raisebox{-2pt}[0pt][0pt]{\small$\rightharpoonup$}}{#1}}  %矢量符号
\renewcommand{\vec}[1]{\boldsymbol{#1}}
\newcommand{\me}{\mathrm{e}}
\newcommand{\mi}{\mathrm{i}}
\newcommand{\dif}{\mathrm{d}}
\newcommand{\tabincell}[2]{\begin{tabular}{@{}#1@{}}#2\end{tabular}}

\def\kpc{{\rm kpc}}
\def\km{{\rm km}}
\def\cm{{\rm cm}}
\def\TeV{{\rm TeV}}
\def\GeV{{\rm GeV}}
\def\MeV{{\rm MeV}}
\def\GV{{\rm GV}}
\def\MV{{\rm MV}}
\def\yr{{\rm yr}}
\def\s{{\rm s}}
\def\ns{{\rm ns}}
\def\GHz{{\rm GHz}}
\def\muGs{{\rm \mu Gs}}
\def\arcsec{{\rm arcsec}}
\def\K{{\rm K}}
\def\microK{\mu{\rm K}}
\def\sr{{\rm sr}}
\newcolumntype{p}{D{,}{\pm}{-1}}

\renewcommand{\figurename}{Fig.}
\renewcommand{\tablename}{Tab.}

\renewcommand{\arraystretch}{1.5}

\setlength{\parindent}{0pt}  %取消每段开头的空格



\title{概率论概念}
\author{}
\date{\today}
\begin{document}

\maketitle

随机现象

\section{随机试验}

E的\textcolor{red}{样本空间}:随机试验E的所有可能结果组成的集合,记为$S$;

\textcolor{red}{样本点}:样本空间的元素,即E的每一个结果;

E的\textcolor{red}{随机事件}:试验E的样本空间$S$的子集,简称\textcolor{red}{事件};事件是指S中满足某些条件的子集;当$S$是由有限个元素或由可列无限个元素组成时,每个子集都是一个事件;若S是由不可列无限个元素组成时,某些子集必须排除在外;

在每次试验中,当且仅当这一子集中的一个样本点出现时,称这一事件发生;

基本事件,由一个样本点组成的单点集;

样本空间$S$包好所有的样本点,它是$S$自身的子集,在每次试验中它总是发生的,称为必然事件;

空集$\emptyset$不包含任何样本点,不可能事件;

\subsection{事件间的关系}
设试验E的样本空间为$S$,而$A$,$B$,$A_k (k = 1, 2, \cdots)$是S的子集,

1. 若$A\subset B$,则称事件$B$\textcolor{red}{包含}事件$A$,事件$A$的发生必然导致事件$B$发生;若$A\subset B$且$B\subset A$,即$A=B$,事件$A$和事件$B$\textcolor{red}{相等};

2. \textcolor{red}{和事件}:事件$A\bigcup B = \{x | x \in A ~\text{或} ~ x \in B\}$;当且仅当$A$,$B$中至少有一个发生时,事件$A\bigcup B$发生;

3. \textcolor{red}{积事件}:事件$A\bigcap B = \{x | x \in A ~\text{且} ~ x \in B\}$;当且仅当$A$,$B$同时发生时,事件$A\bigcap B$发生;也记作$AB$;

4. \textcolor{red}{差事件}:事件$A - B = \{x | x \in A ~\text{且} ~ x \notin B\}$;当且仅当$A$发生,$B$不发生时,事件$A- B$发生;

5. \textcolor{red}{互不相容的},\textcolor{red}{互斥的}:事件$A\bigcap B = \emptyset$;事件$A$和事件$B$不能同时发生;

6. 互为\textcolor{red}{逆事件}:若事件$A\bigcup B = S ~\text{且} ~ A\bigcap B = \emptyset$;事件$A$和事件$B$互为\textcolor{red}{对立事件};对每次试验,事件$A$和事件$B$必有一个发生,且仅有一个发生;事件$A$的对立事件记为$\bar{A} = S - A$;


\subsection{事件的运算}
设$A$,$B$,$C$为事件,

交换律:$A\bigcup B = B\bigcup A$;$A\bigcap B = B\bigcap A$;

结合律:$A\bigcup (B \bigcup C) = (A\bigcup B) \bigcup C$;$A\bigcap (B \bigcap C) = (A\bigcap B) \bigcap C$;

分配律:$A\bigcup (B \bigcap C) = (A\bigcup B) \bigcap (A\bigcup C)$;$A\bigcap (B \bigcup C) = (A\bigcap B) \bigcup (A\bigcap C)$;

德摩根律:$\overline{(A\bigcup B)} = \overline{A} \bigcap \overline{B}$;$\overline{(A\bigcap B)} = \overline{A} \bigcup \overline{B}$

\section{概率}

在相同条件下,进行了$n$次试验,其中事件$A$发生的次数$n_A$称为事件$A$发生的\textcolor{red}{频数};比值$n_A/n$为事件A发生的\textcolor{red}{频率},记成\textcolor{red}{$f_n(A)$};

基本性质:

1. $0 \leqslant f_n(A) \leqslant 1$;

2. $f_n(S) = 1$;

3. 若$A_1, A_2, \cdots, A_k$是两两互不相容的事件,则
\begin{equation}
f_n(A_1 \bigcup A_2 \bigcup \cdots \bigcup A_k) = f_n(A_1) + f_n(A_2) + \cdots +f_n(A_k)
\end{equation}

设$E$是随机试验,$S$是它的样本空间。对于$E$的每一事件$A$赋于一个实数,记为\textcolor{red}{$P(A)$},称为事件$A$的\textcolor{red}{概率},如果集合函数$P(\cdot)$满足下列条件:

1. 非负性: 对于每一个事件,有$P(A) \geqslant 0$;

2. 规范性:对于必然事件,有$P(S) = 1$;

3. 可列可加性:设$A_1, A_2, \cdots$是两两互不相容的事件,即对于$i \neq j, A_i A_j = \varnothing, i, j = 1, 2, \cdots$,则有
\begin{equation}
P(A_1 \bigcup A_2 \bigcup \cdots) = P(A_1) + P(A_2) + \cdots 
\end{equation}
\colorbox{yellow}{第五章将证明,当$n \rightarrow \infty$,频率$f_n(A)$在一定意义下接近于概率$P(A)$。}

\subsection{概率的性质}
1. $P(\varnothing) = 0$;

2. 有限可加性:若$A_1, A_2, \cdots, A_n$是两两互不相容的事件,则有
\begin{equation}
P(A_1 \bigcup A_2 \bigcup \cdots \bigcup A_n) = P(A_1) + P(A_2) + \cdots +P(A_n)
\end{equation}

3. 设$A, B$是两个事件,若$A \subset B$,则有
\begin{eqnarray}
P(B-A) &=& P(B) -P(A); \\
P(B) &\geqslant& P(A);
\end{eqnarray}

4. 对于任一事件$A$,
\begin{equation}
P(A) \leqslant 1;
\end{equation}

5. 逆事件的概率:对于任一事件$A$,
\begin{equation}
P(\bar{A}) = 1- P(A); 
\end{equation}

6. 加法公式:对于任意两事件$A, B$,
\begin{equation}
P(A\bigcup B) = P(A) +P(B) -P(A\bigcap B);
\end{equation}
\begin{eqnarray}
\nonumber P(A_1\bigcup A_2 \bigcup A_3) &=& P(A_1) +P(A_2) +P(A_3) -P(A_1A_2)\\ 
&& -P(A_2A_3) -P(A_1A_3) +P(A_1A_2A_3)
\end{eqnarray}

推广到多个事件的情况,对于任意$n$个事件$A_1, A_2, \cdots, A_n$,
\begin{eqnarray}
\nonumber P(A_1\bigcup A_2 \bigcup \cdots \bigcup A_n) &=& \sum_{i=1}^n P(A_i) -\sum_{1\leqslant i < j \leqslant n} P(A_iA_j) \\
\nonumber && +\sum_{1\leqslant i < j < k \leqslant n} P(A_i A_j A_k) + \cdots \\
&& +(-1)^{n-1} P(A_1 A_2 \cdots A_n)
\end{eqnarray}


\section{等可能概型(古典概型)}
1. 试验的样本空间只包含有限个元素;

2. 试验中每个基本事件发生的可能性相同;

设试验的样本空间为$S = \{e_1, e_2, \cdots, e_n\}$。由于试验中每个基本事件发生的可能性相同,
\begin{equation}
P(\{e_1\}) = P(\{e_2\}) = \cdots = P(\{e_n\});
\end{equation}
又由于基本事件两两互不相容,
\begin{eqnarray}
\nonumber 1 = P(S) &=& P(\{e_1\} \bigcup \{e_2\} \bigcup \cdots \bigcup \{e_n\}) \\
\nonumber &=& P(\{e_1\}) +P(\{e_2\}) + \cdots +P(\{e_n\}) \\ 
\nonumber &=& nP(\{e_i\}),
\end{eqnarray}
\begin{equation}
\nonumber P(\{e_i\}) = \frac{1}{n}, i = 1, 2, \cdots, n
\end{equation}
若事件$A$包含$k$个基本事件,即$A = \{e_{i_1}\} \bigcup \{e_{i_2}\} \bigcup \cdots \bigcup \{e_{i_k}\}$,$i_1, i_2, \cdots, i_k$是$1, 2, \cdots, n$中某$k$个不同的数,则有
\begin{equation}
P(A) = \sum_{j = 1}^k P(\{e_{i_j}\}) = \frac{k}{n} = \frac{A\text{包含的基本事件数}}{S\text{中基本事件的总数}}
\end{equation}

放回抽样、不放回抽样




\section{条件概率}
事件$A$已经发生的条件下,事件$B$发生的概率$P(B|A)$


设$A, B$是两个事件,且$P(A) > 0$,称
\begin{equation}
P(B|A) = \frac{P(AB)}{P(A)}
\end{equation}
为在事件$A$发生的条件下事件$B$发生的\textcolor{red}{条件概率};

条件概率$P(\cdot|A)$符合概率定义的三个条件:

1. 非负性:对于每一事件$B$,$P(B|A) \geqslant 0$;

2. 规范性:对于必然事件$S$,$P(S|A) = 1$;

3. 可列可加性:设$B_1, B_2, \cdots$是两两互不相容事件,则
\begin{equation}
P\left(\bigcup_{i=1}^{\infty} B_i|A \right) = \sum_{i=1}^{\infty} P(B_i|A)
\end{equation}

\subsection{乘法定理}
设$P(A) > 0$,则
\begin{equation}
P(AB) = P(B|A) P(A)
\end{equation}
\begin{equation}
P(ABC) = P(C|AB) P(B|A) P(A)
\end{equation}

推广到多个事件,设$A_1, A_2, \cdots, A_n$为$n$个事件,$n \geqslant 2$,且$P(A_1 A_2 \cdots A_{n-1}) \geqslant 0$,则
\begin{eqnarray}
\nonumber P(A_1 A_2 \cdots A_{n}) &=& P(A_n|A_1 A_2 \cdots A_{n-1}) P(A_{n-1}|A_1 A_2 \cdots A_{n-2}) \cdots \\
&& P(A_2|A_1) P(A_1)
\end{eqnarray}


\subsection{全概率公式}
设$S$为试验$E$的样本空间,$B_1, B_2, \cdots, B_n$为$E$的一组事件,若

1. $B_iB_j = \varnothing, i \neq j, i, j = 1, 2, \cdots, n$;

2. $B_1 \bigcup B_2 \bigcup \cdots \bigcup B_n = S$;

则称$B_1, B_2, \cdots, B_n$为样本空间$S$的一个\textcolor{red}{划分};

若$B_1, B_2, \cdots, B_n$为样本空间的一个划分,则对每次试验,事件$B_1, B_2, \cdots, B_n$中必有一个且仅有一个发生;

\textcolor{red}{全概率公式}

设试验$E$的样本空间为$S$,$A$为$E$的事件,$B_1, B_2, \cdots, B_n$为$S$的一个划分,且$P(B_i) > 0 (i = 1, 2, \cdots, n)$,则
\begin{equation}
P(A) = P(A|B_1) P(B_1) + P(A|B_2) P(B_2) + \cdots +P(A|B_n) P(B_n)
\end{equation}


\subsection{贝叶斯公式}
设试验$E$的样本空间为$S$,$A$为$E$的事件,$B_1, B_2, \cdots, B_n$为$S$的一个划分,且$P(A) > 0, P(B_i) > 0 (i = 1, 2, \cdots, n)$,则
\begin{equation}
P(B_i|A) = \frac{P(A|B_i)P(B_i)}{\sum\limits_{j=1}^n P(A|B_j)P(B_j)}, ~i =1, 2, \cdots, n
\end{equation}


\textcolor{red}{先验概率}:由以往的数据分析得到的;

\textcolor{red}{后验概率}:在得到信息后再重新加以修正的概率;

\section{独立性}
设$A, B$是两事件,若满足等式
\begin{equation}
P(AB) = P(A) P(B)
\end{equation}
称事件$A, B$\textcolor{red}{相互独立},简称$A, B$\textcolor{red}{独立}。

两事件相互独立的含义是它们中一个已发生,不影响另一个发生的概率。

\colorbox{yellow}{若$P(A) > 0, P(B) > 0$,则$A, B$相互独立与$A, B$互不相容不能同时成立。}

定理一:

设$A, B$是两事件,且$P(A) > 0$。若$A, B$相互独立,则$P(B|A) = P(B)$。反之亦然。

定理二:

若事件$A$与$B$相互独立,则下列各对事件也相互独立:$A$与$\bar{B}$,$\bar{A}$与$B$,$\bar{A}$与$\bar{B}$

推广到三个事件:

设$A, B, C$是三个事件,若满足等式
\begin{eqnarray}
\nonumber P(AB) &=& P(A) P(B) \\
\nonumber P(BC) &=& P(B) P(C) \\
\nonumber P(AC) &=& P(A) P(C) \\
P(ABC) &=& P(A) P(B) P(C) 
\end{eqnarray}
称事件$A, B, C$相互独立。

设$A_1, A_2, \cdots, A_n$是$n(n \geqslant 2)$个事件,若对于其中任意$2$个,任意$3$个,$\cdots$,任意$n$个事件的积事件的概率,都等于各事件概率之积,称事件$A_1, A_2, \cdots, A_n$相互独立。

推论:

1. 若事件$A_1, A_2, \cdots, A_n (n \geqslant 2)$相互独立,则其中任意$k(2 \leqslant k \leqslant n)$个事件也是相互独立的。

2. 若$n$个事件$A_1, A_2, \cdots, A_n (n \geqslant 2)$相互独立,则将$A_1, A_2, \cdots, A_n$中任意多个事件换成它们的对立事件,所得的$n$个事件仍相互独立。






































































\end{document}