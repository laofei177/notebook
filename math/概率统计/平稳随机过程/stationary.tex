\documentclass[12pt,a4paper]{article}
%\usepackage{fontspec, xunicode, xltxtra}  
%\setmainfont{Hiragino Sans GB}  
\usepackage{xeCJK}
%\setCJKmainfont[BoldFont=STZhongsong, ItalicFont=STKaiti]{STSong}
%\setCJKsansfont[BoldFont=STHeiti]{STXihei}
%\setCJKmonofont{STFangsong}

%使用Xelatex编译

% 设置页面
%==================================================
\linespread{2} %行距
% \usepackage[top=1in,bottom=1in,left=1.25in,right=1.25in]{geometry}
% \headsep=2cm
% \textwidth=16cm \textheight=24.2cm
%==================================================

% 其它需要使用的宏包
%==================================================
\usepackage[colorlinks,linkcolor=blue,anchorcolor=red,citecolor=green,urlcolor=blue]{hyperref} 
\usepackage{tabularx}
\usepackage{authblk}         % 作者信息
\usepackage{algorithm}     % 算法排版
\usepackage{amsmath}     % 数学符号与公式
\usepackage{amsfonts}     % 数学符号与字体
\usepackage{mathrsfs}      % 花体
\usepackage{amssymb}
\usepackage[framemethod=TikZ]{mdframed}

\usepackage{graphicx} 
\usepackage{graphics}
\usepackage{color}
\usepackage{xcolor}
\usepackage{tcolorbox}
\usepackage{lipsum}
\usepackage{empheq}

\usepackage{fancyhdr}       % 设置页眉页脚
\usepackage{fancyvrb}       % 抄录环境
\usepackage{float}              % 管理浮动体
\usepackage{geometry}     % 定制页面格式
\usepackage{hyperref}       % 为PDF文档创建超链接
\usepackage{lineno}          % 生成行号
\usepackage{listings}        % 插入程序源代码
\usepackage{multicol}       % 多栏排版
\usepackage{natbib}         % 管理文献引用
\usepackage{rotating}       % 旋转文字,图形,表格
\usepackage{subfigure}    % 排版子图形
\usepackage{titlesec}       % 改变章节标题格式
\usepackage{moresize}   % 更多字体大小
\usepackage{anysize}
\usepackage{indentfirst}  % 首段缩进
\usepackage{booktabs}   % 使用\multicolumn
\usepackage{multirow}    % 使用\multirow
\usepackage{graphicx} 
\usepackage{wrapfig}
\usepackage{xcolor}
\usepackage{titlesec}     % 改变标题样式
\usepackage{enumitem}
\usepackage{harpoon}   %矢量符号
\usepackage{leftidx}

\newcommand{\myvec}[1]%
   {\stackrel{\raisebox{-2pt}[0pt][0pt]{\small$\rightharpoonup$}}{#1}}  %矢量符号
\renewcommand{\vec}[1]{\boldsymbol{#1}}
\newcommand{\me}{\mathrm{e}}
\newcommand{\mi}{\mathrm{i}}
\newcommand{\dif}{\mathrm{d}}
\newcommand{\tabincell}[2]{\begin{tabular}{@{}#1@{}}#2\end{tabular}}

\def\kpc{{\rm kpc}}
\def\km{{\rm km}}
\def\cm{{\rm cm}}
\def\TeV{{\rm TeV}}
\def\GeV{{\rm GeV}}
\def\MeV{{\rm MeV}}
\def\GV{{\rm GV}}
\def\MV{{\rm MV}}
\def\yr{{\rm yr}}
\def\s{{\rm s}}
\def\ns{{\rm ns}}
\def\GHz{{\rm GHz}}
\def\muGs{{\rm \mu Gs}}
\def\arcsec{{\rm arcsec}}
\def\K{{\rm K}}
\def\microK{\mu{\rm K}}
\def\sr{{\rm sr}}
\newcolumntype{p}{D{,}{\pm}{-1}}

\renewcommand{\figurename}{Fig.}
\renewcommand{\tablename}{Tab.}

\renewcommand{\arraystretch}{1.5}

\setlength{\parindent}{0pt}  %取消每段开头的空格

\title{平稳随机过程}
\author{}
\date{\today}
\begin{document}

\maketitle

\section{平稳随机过程的概念}
若对于任意的$n(= 1, 2, \cdots)$,$t_1, t_2, \cdots, t_n \in T$和任意实数$h$,当$t_1 +h, t_2 +h, \cdots, t_n +h \in T$时,$n$维随机变量
\begin{equation}
(X(t_1), X(t_2), \cdots, X(t_n)~)
\end{equation}
和
\begin{equation}
(X(t_1+h), X(t_2+h), \cdots, X(t_n+h)~)
\end{equation}
具有相同的分布函数,则称随机过程$\{X(t), t\in T\}$具有平稳性,称此过程为\textcolor{red}{平稳随机过程}、\textcolor{red}{平稳过程};\textcolor{red}{严平稳过程}、\textcolor{red}{狭义平稳过程};

过程的统计特征性不随时间的推移而变化;

若前后的环境和主要条件都不随时间的推移而变化,一般就认为是平稳的;

当定义在离散参数集上时,也称为\textcolor{red}{平稳随机序列}、\textcolor{red}{平稳时间序列};

\textcolor{blue}{非平稳过程}:一般,随机过程处于过渡阶段时总是非平稳的;

设平稳过程$X(t)$的均值函数$E[X(t)]$存在,则均值函数$\mu_X$、均方值函数$\Psi^2_X$和方差函数$\sigma^2_X$均为常数;

若自相关函数$R_X(t_1, t_2) = E[X(t_1)\cdot X(t_2)]$存在,二维随机变量$(X(t_1), X(t_2) )$与$(X(0), X(t_2-t_1) )$同分布,
\begin{equation}
E[X(t_1)X(t_2)] = E[X(0)X(t_2-t_1)] 
\end{equation}

$R_X(t_1,t_2) = R_X(t_2-t_1)$,或者$R_X(t,t+\tau) = E[X(t)X(t+\tau)]  = R_X(\tau)$

平稳过程的自相关函数仅是时间差$\tau = t_2-t_1$的单变量函数;

协方差函数
\begin{equation}
C_X(\tau) = E\{[X(t)-\mu_X][X(t+\tau)-\mu_X] \} = R_X(\tau) -\mu_X^2
\end{equation}
令$\tau = 0$,则
\begin{equation}
\sigma_X^2 = C_X(0) = R_X(0) -\mu_X^2
\end{equation}

给定二阶矩过程$\{X(t), t \in T\}$,若对任意$t, t+\tau \in T$,
\begin{eqnarray}
\nonumber E[X(t)] = \mu_X(\text{常数}), \\
E[X(t)X(t+\tau)] = R_X(\tau), 
\end{eqnarray}
称$\{X(t), t \in T\}$为\textcolor{red}{宽平稳过程}、\textcolor{red}{广义平稳过程};

正态过程的概率密度由均值函数和自相关函数完全确定,若均值函数和自相关函数不随时间变化,则概率密度不随时间变化,宽平稳正态过程也必定是严平稳的;

考虑两个平稳过程$X(t)$和$Y(t)$,若它们的互相关函数也只是时间差的单变量函数,$R_{XY}(\tau)$,即
\begin{equation}
R_{XY}(t, t+\tau) = E[X(t)Y(t+\tau)] = R_{XY}(\tau)
\end{equation}
称$X(t)$和$Y(t)$是\textcolor{red}{平稳相关}的,或这两个过程是\textcolor{red}{联合(宽)平稳}的;

\section{各态历经性}
根据实验记录确定平稳过程的均值和自相关函数的理论依据、方法

给定二阶矩过程$\{X(t), t \in T\}$,若它的每一个样本函数在$[a, b] \subset T$上的积分都存在,就说随机过程$X(t)$在$[a, b]$上的积分存在,记为
\begin{equation}
Y = \int_a^b X(t) \dif t
\end{equation}

在某些情况下,对于随机过程的所有样本函数,在$[a, b]$上的积分未必全都存在;引入均方意义下的积分,即考虑$[a, b]$内的一组分点:
\begin{equation}
a = t_0 < t_1 < t_2 < \cdots < t_n = b, 
\end{equation}
且记
\begin{equation}
\Delta t_i = t_i -t_{i-1}, ~ t_{i-1} \leq \tau_i \leq t_i, ~ i = 1, 2, \cdots, n
\end{equation}
若有满足
\begin{equation}
\lim_{\max \Delta t_i \rightarrow 0} E\{[Y -\sum_{i=1}^n X(\tau_i) \Delta t_i]^2 \} = 0
\end{equation}
的随机变量$Y$存在,就称$Y$为$X(t)$在$[a, b]$上的\textcolor{red}{均方积分};

设$X_1, X_2, \cdots, X_n, \cdots$是一随机变量序列,若存在随机变量$X_0$,使
\begin{equation}
\lim_{n\rightarrow \infty} E\{[X_n -X_0]^2\} = 0
\end{equation}
则称$X_0$是$X_n$的均方极限,记为${\rm l.i.m.} X_n = X_0$


二阶矩过程$X(t)$在$[a, b]$上均方积分存在的充分条件是自相关函数的二重积分,即
\begin{equation}
\int_a^b \int_a^b R_X(s, t) \dif s \dif t
\end{equation}
存在,且此时还成立有
\begin{equation}
E[Y] = \int_a^b E[X(t)] \dif t
\end{equation}
过程$X(t)$的积分均值等于过程的均值函数的积分;

\textcolor{red}{时间均值函数}
\begin{equation}
\left\langle X(t) \right\rangle = \lim_{T\rightarrow +\infty} \frac{1}{2T} \int_{-T}^{T} X(t) \dif t
\end{equation}

\textcolor{red}{时间相关函数}
\begin{equation}
\left\langle X(t) X(t+\tau) \right\rangle = \lim_{T\rightarrow +\infty} \frac{1}{2T} \int_{-T}^{T} X(t) X(t+\tau)  \dif t
\end{equation}

设$X(t)$是一平稳过程,
1. 若
\begin{equation}
\left\langle X(t) \right\rangle = E[X(t)] = \mu_X
\end{equation}
以概率$1$成立,则称过程$X(t)$的\textcolor{red}{均值具有各态历经性}。
2. 若对任意实数$\tau$,
\begin{equation}
\left\langle X(t) X(t+\tau) \right\rangle = E[X(t)X(t+\tau)] = R_X(\tau)
\end{equation}
以概率$1$成立,则称过程$X(t)$的\textcolor{red}{自相关函数具有各态历经性}。
当$\tau = 0$时,称\textcolor{red}{均方值具有各态历经性}。
3. 若$X(t)$的均值和自相关函数都具有各态历经性,则称$X(t)$是(宽)\textcolor{red}{各态历经过程},或者是\textcolor{red}{各态历经}的;

“以概率$1$成立”,对$X(t)$的所有样本函数;

\textcolor{red}{遍历性},\textcolor{red}{ergodicity}



\begin{tcolorbox}[colback=green!5,colframe=green!40!black,title= 均值各态历经定理]
平稳过程$X(t)$的均值具有各态历经性的充要条件是
\begin{equation}
\lim_{T\rightarrow +\infty} \dfrac{1}{T} \int_0^{2T} \left( 1-\dfrac{\tau}{2T} \right) [R_X(\tau) - \mu_X^2 ] \dif \tau = 0 ~.
\label{eq:aver}
\end{equation}
\end{tcolorbox}




\begin{tcolorbox}[colback=green!5,colframe=green!40!black,title= 推论]
在$\lim\limits_{\tau \rightarrow +\infty} R_X(\tau)$存在条件下,若$\lim\limits_{\tau \rightarrow +\infty} R_X(\tau) = \mu_X^2$,则(\ref{eq:aver})成立,均值具有各态历经性;若$\lim\limits_{\tau \rightarrow +\infty} R_X(\tau) \neq \mu_X^2$,则(\ref{eq:aver})不成立,均值不具有各态历经性。
\end{tcolorbox}



\begin{tcolorbox}[colback=green!5,colframe=green!40!black,title= 自相关函数各态历经定理]
平稳过程$X(t)$的自相关函数$R_X(\tau)$具有各态历经性的充要条件是
\begin{equation}
\lim_{T\rightarrow +\infty} \dfrac{1}{T} \int_0^{2T} \left(1 -\dfrac{\tau_1}{T} \right)  [B(\tau_1) - R_X^2(\tau) ] \dif \tau_1 = 0 ~,
\label{eq:autocorrela}
\end{equation}
其中$B(\tau_1) = E[X(t) X(t+\tau) X(t+\tau_1) X(t+\tau+\tau_1)]$. 
\end{tcolorbox}

\begin{tcolorbox}[colback=green!5,colframe=green!40!black,title= 定理三]
\begin{equation*}
\lim_{T\rightarrow +\infty} \dfrac{1}{T} \int_0^{T} X(t) \dif t = E[X(t)] = \mu_X ~,
\end{equation*}
以概率$1$成立的充要条件是
\begin{equation}
\lim_{T\rightarrow +\infty} \dfrac{1}{T} \int_0^{T} \left(1 -\dfrac{\tau}{T} \right)  [R_X(\tau) - \mu_X^2 ] \dif \tau = 0 ~.
\label{eq:aver}
\end{equation}
\end{tcolorbox}


\begin{tcolorbox}[colback=green!5,colframe=green!40!black,title= 定理四]
\begin{equation*}
\lim_{T\rightarrow +\infty} \dfrac{1}{T} \int_0^{T} X(t)X(t+\tau) \dif t = E[X(t)X(t+\tau)] = R_X(\tau) ~,
\end{equation*}
以概率$1$成立的充要条件是
\begin{equation}
\lim_{T\rightarrow +\infty} \dfrac{1}{T} \int_0^{T} \left(1 -\dfrac{\tau_1}{T} \right)  [B(\tau_1) - R_X^2(\tau) ] \dif \tau_1 = 0 ~.
\label{eq:autocorrela_}
\end{equation}
\end{tcolorbox}

各态历经定理从理论上给出了如下保证:一个平稳过程$X(t)$,若$0<t< +\infty$,只要它满足条件(\ref{eq:aver})和(\ref{eq:autocorrela_}),便可以根据“以概率$1$成立”的含义,从一次试验所得到的样本数$x(t)$来确定出该过程的均值和自相关函数,即
\begin{equation}
\lim_{T\rightarrow +\infty} \dfrac{1}{T} \int_0^{T} x(t) \dif t = \mu_X ~,
\end{equation}
和
\begin{equation}
\lim_{T\rightarrow +\infty} \dfrac{1}{T} \int_0^{T} x(t)x(t+\tau) \dif t = R_X(\tau) ~.
\end{equation}

若试验记录$x(t)$只在时间区间$[0,T]$上给出,则相应于()和()式有以下无偏估计式:
\begin{align}
\mu_X &\approx \hat{\mu}_X = \dfrac{1}{T} \int_0^T x(t) \dif t ~, \\
R_X(\tau) &\approx \hat{R}_X(\tau) = \dfrac{1}{T -\tau} \int_0^{T-\tau} x(t) x(t+\tau) \dif t  = \dfrac{1}{T -\tau} \int_\tau^{T} x(t) x(t-\tau) \dif t ~, ~~ 0 \leqslant \tau < T ~.
\end{align}






各态历经定理的条件是比较宽的,工程中碰到的大多数平稳过程都能够满足。不过,要验证它们是否成立却是十分困难的。实践中,通常先假定所研究的平稳过程具有各态历经性,并从这个假定出发,对由此而产生的各种资料进行分析处理,看所得的结论是否与实际相符。若不符,则要修改假设。




\section{相关函数的性质}

\section{平稳随机过程的功率谱密度}

利用傅里叶变换确立平稳过程的频率结构-功率谱密度;

设有时间函数$x(t), -\infty < t < +\infty$,假如$x(t)$满足\textcolor{blue}{狄利克雷(Dirichlet)条件},且\textcolor{blue}{绝对可积},即
\begin{equation}
\int_{-\infty}^{+\infty} |x(t)| \dif t < +\infty,
\end{equation}
那么$x(t)$的傅里叶变换存在,或者说具有频谱
\begin{equation}
F_x(\omega) = \int_{-\infty}^{+\infty} x(t) e^{-i\omega t} \dif t
\end{equation}
其逆变换
\begin{equation}
x(t) = \frac{1}{2\pi} \int_{-\infty}^{+\infty} F_x(\omega) e^{i\omega t} \dif \omega
\end{equation}
$ F_x(-\omega)$一般是复数量,其共轭函数
\begin{equation}
F^*_x(\omega) = F_x(-\omega) ~.
\end{equation}
\textcolor{red}{Parseval等式}
\begin{equation}
\int_{-\infty}^{+\infty} x^2(t) \dif t = \frac{1}{2\pi} \int_{-\infty}^{+\infty} |F_x(\omega)|^2 \dif \omega
\end{equation}
左边:$x(t)$在$(-\infty, +\infty)$上的总能量;\\
右边:$|F_x(\omega)|^2$,$x(t)$的能谱密度;\\
Parseval等式可以理解为总能量的谱表示式。

假定$x(t)$在$(-\infty, +\infty)$上的平均功率,即
\begin{equation}
\lim_{T\rightarrow +\infty} \frac{1}{2T} \int_{-T}^{+T} x^2(t) \dif t 
\end{equation}
是存在的;

由给定的$x(t)$构造一个截尾函数
\begin{equation}
x_T(t) = \left\{
\begin{aligned}
x(t),  & & |t| \leqslant T, \\
0,  & & |t| > T.
\end{aligned}
\right.
\end{equation}

$x_T(t)$的傅里叶变换
\begin{equation}
F_x(\omega, T) = \int_{-\infty}^{+\infty} x_T(t) e^{-i\omega t} \dif t = \int_{-T}^{T} x(t) e^{-i\omega t} \dif t ~,
\end{equation}
它的Parseval等式
\begin{equation}
\int_{-\infty}^{+\infty} x^2_T(t) \dif t = \frac{1}{2\pi} \int_{-\infty}^{+\infty} |F_x(\omega, T)|^2 \dif \omega ~.
\end{equation}
两边除以$2T$,
\begin{equation}
\frac{1}{2T}\int_{-T}^{+T} x^2(t) \dif t = \frac{1}{4\pi T} \int_{-\infty}^{+\infty} |F_x(\omega, T)|^2 \dif \omega
\end{equation}
令$T\rightarrow +\infty$,$x(t)$在$(-\infty, +\infty)$上的\textcolor{red}{平均功率}即可表示为
\begin{equation}
\lim_{T\rightarrow +\infty} \frac{1}{2T}\int_{-\infty}^{+\infty} x^2(t) \dif t = \frac{1}{2\pi} \int_{-\infty}^{+\infty} \lim_{T\rightarrow +\infty} \frac{1}{2T} |F_x(\omega, T)|^2 \dif \omega
\label{eq:aver_power}
\end{equation}
相应于能谱密度,右端被积式称作函数$x(t)$的\textcolor{red}{平均功率谱密度}、\textcolor{red}{功率谱密度},记为
\begin{equation}
\color{red} S_x(\omega) = \lim_{T\rightarrow +\infty} \frac{1}{2T} |F_x(\omega, T)|^2
\end{equation}
(\ref{eq:aver_power})式的右端就是平均功率的谱表示式。



\subsection{平稳过程}
$X(t), -\infty < t < +\infty$
\begin{align}
F_X(\omega, T) &= \int_{-T}^{T} X(t) e^{-i\omega t} \dif t  ~, \\
\frac{1}{2T}\int_{-T}^{+T} X^2(t) \dif t &= \frac{1}{4\pi T} \int_{-\infty}^{+\infty} |F_X(\omega, T)|^2 \dif \omega ~.
\end{align}
两式的积分都是随机的。上式左端的均值的极限,即
\begin{equation}
\lim_{T\rightarrow +\infty} E\left \{ \frac{1}{2T}\int_{-T}^{+T} X^2(t) \dif t \right \}
\end{equation}
定义为\textcolor{red}{平稳过程$X(t)$的平均功率};

交换积分与均值的顺序,且平稳过程的均方值是常数;
\begin{equation}
\Psi^2_X = \lim_{T\rightarrow +\infty} E\left \{ \frac{1}{2T}\int_{-T}^{T} X^2(t) \dif t \right \} = \lim_{T\rightarrow +\infty} \frac{1}{2T} \int_{-T}^{T} E[X^2(t)] \dif t
\end{equation}
平稳过程的平均功率$=$该过程的均方值或$R_X(0)$。交换运算顺序后,
\begin{equation}
\Psi^2_X = \frac{1}{2\pi} \int_{-\infty}^{+\infty} \lim_{T\rightarrow +\infty} \frac{1}{2T} E\{ |F_x(\omega, T)|^2 \} \dif \omega = \color{red} \frac{1}{2\pi} \int_{-\infty}^{+\infty} S_{X}(\omega) \dif \omega ~.
\end{equation}
此为\textcolor{red}{平稳过程$X(t)$的平均功率的谱表示式},$X(t)$的平均功率关于频率的分布。\textcolor{red}{平稳过程的功率谱密度}
\begin{equation}
\color{red} S_{X}(\omega) = \lim_{T\rightarrow +\infty} \frac{1}{2T} E\{ |F_x(\omega, T)|^2 \} ~,
\end{equation}
也称\textcolor{red}{自谱密度}、\textcolor{red}{谱密度}\footnote{平稳过程的总能量为无限,能谱密度也不存在。},他是从频率这个角度描述$X(t)$的统计规律的最主要的数字特征。

若已知平稳过程$X(t)$的谱密度,在任何特定频率范围$(\omega_1, \omega_2)$内的谱密度对平均功率的贡献为
\begin{equation}
\leftidx{_{(\omega_1\cdot \omega_2)}^{} }{\Psi}{^2_X} = \frac{1}{2\pi} \int_{\omega_1}^{\omega_2} S_{X}(\omega) \dif \omega
\end{equation}
也称双边谱密度;

在$[0, +\infty)$上的平稳过程$X(t)$,定义单边谱密度
\begin{equation}
G_X(\omega) = \left\{
\begin{aligned}
2\lim_{T\rightarrow +\infty} \frac{1}{T} E\{ |F_x(\omega, T)|^2 \}, & & \omega \geq 0 \\
0, & & \omega < 0
\end{aligned}
\right.
\end{equation}

单边谱密度和双边谱密度的关系
\begin{equation}
G_X(\omega) = \left\{
\begin{aligned}
2S_X(\omega), & & \omega \geq 0 \\
0, &  & \omega < 0
\end{aligned}
\right.
\end{equation}


$S_X(\omega)$是$\omega$的实的、非负的偶函数;

\textcolor{red}{维纳-辛钦(Wiener-Khinchin)公式}:$S_X(\omega)$和自相关函数$R_X(\tau)$是一傅里叶变换对,即
\begin{eqnarray}
\nonumber S_X(\omega) &=& \int_{-\infty}^{+\infty}  R_X(\tau) e^{-i\omega \tau} \dif \tau \\
R_X(\tau) &=& \frac{1}{2\pi}  \int_{-\infty}^{+\infty}  S_X(\omega) e^{i\omega \tau} \dif \omega
\end{eqnarray}
\textcolor{red}{平稳过程}在\textcolor{red}{自相关函数绝对可积}的条件下,\textcolor{blue}{谱密度$S_X(\omega)$}和\textcolor{blue}{自相关函数$R_X(\tau)$}是\textcolor{red}{傅里叶变换对}。

令$\tau = 0$,
\begin{equation}
R_X(\tau) = \frac{1}{2\pi}  \int_{-\infty}^{+\infty}  S_X(\omega) \dif \omega
\end{equation}

有理谱密度的一般形式
\begin{equation}
S_X(\omega) = S_0 \dfrac{\omega^{2n} +a_{2n-2} \omega^{2n-2} +\cdots + a_0}{\omega^{2m} +b_{2m-2} \omega^{2m-2} +\cdots + b_0} ~,
\end{equation}
式中$S_0 > 0$,要求均方值有限,$m > n$,且分母应无实数根。


当自相关函数$R_X(\tau) = 1$时,谱密度$S_X(\omega) = 2\pi \delta(\omega)$。正弦函数的自相关函数$R_X(\tau) = a\cos \omega_0 \tau$的谱密度为
\begin{equation}
S_X(\omega) = a\pi [\delta(\omega -\omega_0) +\delta(\omega +\omega_0) ] ~.
\end{equation}



\subsection{白噪声}
均值为零而谱密度为正常数,即
\begin{equation*}
S_X(\omega) = S_0, ~~ -\infty < \omega < +\infty (S_0 > 0)
\end{equation*}
的平稳过程$X(t)$,称为\textcolor{red}{白噪声过程},简称\textcolor{red}{白噪声}。白噪声的自相关函数为
\begin{equation*}
R_X(\tau) = \dfrac{1}{2\pi} \int_{-\infty}^\infty S_X(\omega) {\rm e}^{i\omega \tau} \dif \omega = \dfrac{S_0}{2\pi} \int_{-\infty}^\infty {\rm e}^{i\omega \tau} \dif \omega = S_0 \delta(\tau) ~.
\end{equation*}
白噪声也可以定义为均值为零、自相关函数为$\delta$函数的随机过程,且这个过程在$t_1 \neq t_2$时,$X(t_1)$和$X(t_2)$是不相关的。

带限白噪声,其谱密度仅在某些有限频率范围内取异于零的常数。低通白噪声,
\begin{equation*}
S_X(\omega) = \left\{\begin{array}{ll}
S_0, & |\omega| \leqslant \omega_1,\\
0 , & |\omega| > \omega_1 ,
\end{array}\right.
\end{equation*}
相应的自相关函数为
\begin{align*}
R_X(\tau) &= \dfrac{1}{2\pi} \int_{-\infty}^\infty S_X(\omega) {\rm e}^{i\omega \tau} \dif \omega =  \dfrac{1}{2\pi} \int_{-\omega_1}^{\omega_1} S_0 {\rm e}^{i\omega \tau} \dif \omega \\
&=  \left\{\begin{array}{ll}
\dfrac{S_0 \omega_1}{\pi}, & \tau = 0,\\
\dfrac{S_0}{2\pi} \dfrac{{\rm e}^{i\omega \tau}}{i\tau} \Bigg|_{-\omega_1}^{\omega_1} = \dfrac{S_0 \omega_1}{\pi} \left[\dfrac{\sin(\omega_1 \tau)}{\omega_1 \tau}  \right], & \tau \neq 0 ,
\end{array}\right.
\end{align*}
当$\tau =\dfrac{k\pi}{\omega_1}, k = \pm 1, \pm 2, \cdots, \cdots$时,$R_X(\tau) = 0$。低通白噪声$X(t)$在$t_2-t_1 = \dfrac{k\pi}{\omega_1}$时,$X(t_1)$和$X(t_2)$是不相关的。


\subsection{互谱密度及其性质}
设$X(t)$和$Y(t)$是两个平稳相关的随机过程,定义
\begin{equation}
S_{XY} (\omega) = \underset{T\rightarrow +\infty}\lim \dfrac{1}{2T} E\{F_X(-\omega, T) F_Y(\omega, T)\} ~,
\end{equation}
为平稳过程$X(t)$和$Y(t)$的\textcolor{red}{互谱密度}。互谱密度不再是$\omega$的实的、正的偶函数,\\
1. $S_{XY} (\omega) = S^\ast_{YX}(\omega) $,即$S_{XY} (\omega)$和$S_{YX} (\omega)$互为共轭函数。\\
2. 在互相关函数$R_{XY}(\tau)$绝对可积的条件下,维纳-辛钦公式
\begin{align}
S_{XY}(\omega) = \int_{-\infty}^\infty R_{XY}(\tau) {\rm e}^{-i\omega \tau} \dif \tau ~, \\
R_{XY}(\tau) = \dfrac{1}{2\pi} \int_{-\infty}^\infty S_{XY}(\omega) {\rm e}^{i\omega \tau} \dif \omega ~.
\end{align}
3. ${\rm Re}[S_{XY}(\omega)]$和${\rm Re}[S_{YX}(\omega)]$是$\omega$的偶函数,${\rm Im}[S_{XY}(\omega)]$和${\rm Im}[S_{YX}(\omega)]$是$\omega$的奇函数。\\
4. 互谱密度和自谱密度之间有不等式
\begin{equation*}
|S_{XY}(\omega)|^2 \leqslant S_{X}(\omega) S_{Y}(\omega) ~.
\end{equation*}


















\end{document}