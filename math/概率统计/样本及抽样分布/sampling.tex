\documentclass[12pt,a4paper]{article}
%\usepackage{fontspec, xunicode, xltxtra}  
%\setmainfont{Hiragino Sans GB}  
\usepackage{xeCJK}
%\setCJKmainfont[BoldFont=STZhongsong, ItalicFont=STKaiti]{STSong}
%\setCJKsansfont[BoldFont=STHeiti]{STXihei}
%\setCJKmonofont{STFangsong}

%使用Xelatex编译

% 设置页面
%==================================================
\linespread{2} %行距
% \usepackage[top=1in,bottom=1in,left=1.25in,right=1.25in]{geometry}
% \headsep=2cm
% \textwidth=16cm \textheight=24.2cm
%==================================================

% 其它需要使用的宏包
%==================================================
\usepackage[colorlinks,linkcolor=blue,anchorcolor=red,citecolor=green,urlcolor=blue]{hyperref} 
\usepackage{tabularx}
\usepackage{authblk}         % 作者信息
\usepackage{algorithm}     % 算法排版
\usepackage{amsmath}     % 数学符号与公式
\usepackage{amsfonts}     % 数学符号与字体
\usepackage{mathrsfs}      % 花体
\usepackage{amssymb}

\usepackage{graphicx} 
\usepackage{graphics}
\usepackage{xcolor}
\usepackage{color}

\usepackage{fancyhdr}       % 设置页眉页脚
\usepackage{fancyvrb}       % 抄录环境
\usepackage{float}              % 管理浮动体
\usepackage{geometry}     % 定制页面格式
\usepackage{hyperref}       % 为PDF文档创建超链接
\usepackage{lineno}          % 生成行号
\usepackage{listings}        % 插入程序源代码
\usepackage{multicol}       % 多栏排版
\usepackage{natbib}         % 管理文献引用
\usepackage{rotating}       % 旋转文字,图形,表格
\usepackage{subfigure}    % 排版子图形
\usepackage{titlesec}       % 改变章节标题格式
\usepackage{moresize}   % 更多字体大小
\usepackage{anysize}
\usepackage{indentfirst}  % 首段缩进
\usepackage{booktabs}   % 使用\multicolumn
\usepackage{multirow}    % 使用\multirow

\usepackage{wrapfig}

\usepackage{titlesec}     % 改变标题样式
\usepackage{enumitem}
\usepackage{harpoon}   %矢量符号

\newcommand{\myvec}[1]%
   {\stackrel{\raisebox{-2pt}[0pt][0pt]{\small$\rightharpoonup$}}{#1}}  %矢量符号
\renewcommand{\vec}[1]{\boldsymbol{#1}}
\newcommand{\me}{\mathrm{e}}
\newcommand{\mi}{\mathrm{i}}
\newcommand{\dif}{\mathrm{d}}
\newcommand{\tabincell}[2]{\begin{tabular}{@{}#1@{}}#2\end{tabular}}

\def\kpc{{\rm kpc}}
\def\km{{\rm km}}
\def\cm{{\rm cm}}
\def\TeV{{\rm TeV}}
\def\GeV{{\rm GeV}}
\def\MeV{{\rm MeV}}
\def\GV{{\rm GV}}
\def\MV{{\rm MV}}
\def\yr{{\rm yr}}
\def\s{{\rm s}}
\def\ns{{\rm ns}}
\def\GHz{{\rm GHz}}
\def\muGs{{\rm \mu Gs}}
\def\arcsec{{\rm arcsec}}
\def\K{{\rm K}}
\def\microK{\mu{\rm K}}
\def\sr{{\rm sr}}
\newcolumntype{p}{D{,}{\pm}{-1}}

\renewcommand{\figurename}{Fig.}
\renewcommand{\tablename}{Tab.}

\renewcommand{\arraystretch}{1.5}

\setlength{\parindent}{0pt}  %取消每段开头的空格

\title{样品及抽样分布}
\author{}
\date{\today}
\begin{document}

\maketitle
\section{随机样本}
\textcolor{red}{总体}:

试验的全部可能观察值;

\textcolor{red}{个体}:

每一个可能的观察值;

\textcolor{red}{容量}:

总体中所包含的个体的数量;

总体中的每一个个体是随机试验的一个观察值,它是某一随机变量$X$的值;一个总体对应于一个随机变量;

通过从总体中抽取一部分个体,根据获得的数据对总体分布得出推断;被抽出的部分个体,叫总体的一个样本;

从总体中抽取一个个体,即对总体$X$进行一次观察并记录结果;在相同条件下,对总体$X$进行$n$次重复的、独立的观察;将$n$次观察结果按试验的次序记为$X_1, X_2, \cdots, X_n$。由于$X_1, X_2, \cdots, X_n$是对随机变量$X$观察的结果,且各次观察是在相同条件下独立进行的,则$X_1, X_2, \cdots, X_n$是相互独立的,且都是与$X$具有相同分布的随机变量$\Longrightarrow$$X_1, X_2, \cdots, X_n$称为来自总体$X$的一个简单随机样本,$n$:样本的容量;$n$次观察得到一组实数$x_1, x_2, \cdots, x_n$,依次是随机变量$X_1, X_2, \cdots, X_n$的观察值,称为样本值;

\textcolor{red}{简单随机样本}:

设$X$是具有分布函数$F$的随机变量,若$X_1, X_2, \cdots, X_n$是具有同一分布函数$F$的、相互独立的随机变量,称$X_1, X_2, \cdots, X_n$为从分布函数$F$(或总体$F$、总体$X$)得到的容量为$n$的简单随机样本,简称\textcolor{red}{样本};

\textcolor{red}{样本值}:

观察值$x_1, x_2, \cdots, x_n$;$X$的$n$个独立的观察值;\\

若$X_1, X_2, \cdots, X_n$为$F$的一个样本,则$X_1, X_2, \cdots, X_n$相互独立,且它们的分布函数都是$F$,则$(X_1, X_2, \cdots, X_n)$的分布函数:
\begin{equation}
F^*(x_1, x_2, \cdots, x_n) = \prod_{i=1}^{n} F(x_i)
\end{equation}
若$X$具有概率密度$f$,则$(X_1, X_2, \cdots, X_n)$的概率密度:
\begin{equation}
f^*(x_1, x_2, \cdots, x_n) = \prod_{i=1}^{n} f(x_i)
\end{equation}

\section{抽样分布}
\textcolor{red}{统计量}:

设$X_1, X_2, \cdots, X_n$是来自总体$X$的一个样本,\textcolor{red}{$g(X_1, X_2, \cdots, X_n)$}是$X_1, X_2, \cdots, X_n$的函数,若$g$中不含未知参数,则称$g(X_1, X_2, \cdots, X_n)$是一统计量;

因为$X_1, X_2, \cdots, X_n$是随机变量,而统计量$g(X_1, X_2, \cdots, X_n)$是随机变量的函数,因此也是一随机变量。设$x_1, x_2, \cdots, x_n$是相应于样本$X_1, X_2, \cdots, X_n$的样本值,称\textcolor{red}{$g(x_1, x_2, \cdots, x_n)$}是$g(X_1, X_2, \cdots, X_n)$的\textcolor{red}{观察值}。

\textcolor{red}{样本平均值}:
\begin{equation}
\overline{X} = \frac{1}{n} \sum_{i=1}^{n} X_i
\end{equation}

\textcolor{red}{样本方差}:
\begin{equation}
S^2 = \textcolor{blue}{\frac{1}{n-1}} \sum_{i=1}^{n} (X_i - \overline{X})^2 = \textcolor{blue}{\frac{1}{n-1}} \left( \sum_{i=1}^{n} X_i^2 - n \overline{X}^2  \right)
\end{equation}

\textcolor{red}{样本标准差}:
\begin{equation}
S = \sqrt{S^2} = \sqrt{ \textcolor{blue}{\frac{1}{n-1}} \sum_{i=1}^{n} (X_i - \overline{X})^2 }
\end{equation}

\textcolor{red}{样本$k$阶(原点)矩}
\begin{equation}
A_k = \frac{1}{n} \sum_{i=1}^{n} X_i^k, ~~k = 1,2,\cdots
\end{equation}

\textcolor{red}{样本$k$阶中心矩}
\begin{equation}
B_k = \frac{1}{n} \sum_{i=1}^{n} (X_i-\overline{X})^k, ~~k = 1,2,\cdots
\end{equation}

其观察值:
\begin{equation}
\overline{x} = \frac{1}{n} \sum_{i=1}^{n} x_i
\end{equation}

\begin{equation}
s^2 = \frac{1}{n-1} \sum_{i=1}^{n} (x_i - \overline{x})^2 = \frac{1}{n-1} \left( \sum_{i=1}^{n} x_i^2 - n \overline{x}^2  \right)
\end{equation}

\begin{equation}
s = \sqrt{ \frac{1}{n-1} \sum_{i=1}^{n} (x_i - \overline{x})^2 }
\end{equation}

\begin{equation}
a_k = \frac{1}{n} \sum_{i=1}^{n} x_i^k, ~~k = 1,2,\cdots
\end{equation}

\begin{equation}
b_k = \frac{1}{n} \sum_{i=1}^{n} (x_i-\overline{x})^k, ~~k = 1,2,\cdots
\end{equation}

若总体$X$的$k$阶矩$E(X^k) \stackrel{\text{记成}}\longrightarrow \mu_k$存在,则当$n\rightarrow \infty$时,$A_k \stackrel{P}\longrightarrow \mu_k, k =1 ,2, \cdots$。由于$X_1, X_2, \cdots, X_n$独立且与$X$同分布,所以$X_1, X_2, \cdots, X_n$独立且与$X^k$同分布。故有
\begin{equation*}
E(X^k_1) = E(X^k_2) = \cdots = E(X^k_n) = \mu_k
\end{equation*}
由辛钦定理可知
\begin{equation*}
A_k = \frac{1}{n} \sum_{i=1}^k X_i^k \stackrel{P}\longrightarrow \mu_k, k =1 ,2, \cdots ~.
\end{equation*}
由依概率收敛的序列的性质,
\begin{equation}
g(A_1, A_2, \cdots, A_k) \stackrel{P}\longrightarrow g(\mu_1, \mu_2, \cdots, \mu_k) ~.
\end{equation}
其中$g$为连续函数。这是\textcolor{purple}{矩估计法的理论依据}。

\textcolor{red}{经验分布函数}

\textcolor{cyan}{与总体分布函数$F(x)$相应的统计量};

设$X_1, X_2, \cdots, X_n$是总体$F$的一个样本,用\textcolor{cyan}{$S(x), -\infty < x < \infty$}表示$X_1, X_2, \cdots, X_n$中\textcolor{cyan}{不大于$x$的随机变量的个数}。定义经验分布函数$F_n(x)$为
\begin{equation}
\color{red} F_n(x) = \frac{1}{n} S(x),~ -\infty < x < \infty ~.
\end{equation}

经验分布函数$F_n(x)$的观察值
\begin{eqnarray*}
F_n(x) = \left\{
\begin{aligned}
0,&& x < x_{(1)} ~,\\
\dfrac{k}{n},&& x_{(k)} \leqslant x <   x_{(k+1)} ~, \\
1,&& x \geqslant x_{(n)} ~.
\end{aligned}
\right.
\end{eqnarray*}
对于任一实数$x$,当$x\rightarrow \infty$时,$F_n(x)$以概率$1$一致收敛于分布函数$F(x)$,即
\begin{equation*}
P\{\underset{n\rightarrow \infty}\lim~ \underset{-\infty < x < \infty}{\rm sup} |F_n(x) -F(x)| = 0 \} = 1 ~.
\end{equation*}
因此对于任一实数$x$,当$n$充分大时,经验分布函数的任一观察值$F_n(x)$与总体分布函数$F(x)$只有微小的差别,可当作$F(x)$来使用。

\textcolor{red}{抽样分布}:

统计量的分布;\\
使用统计量进行统计推断时需要知道它的分布;\\
当总体的分布函数已知,抽样分布是确定的;

以下是来自正态总体的统计量分布 \\
\subsection{$\chi^2$分布}

设$X_1, X_2, \cdots, X_n$是来自总体$N(0,1)$的样本,称统计量
\begin{equation}
\color{red} \chi^2 = X_1^2 + X_1^2 +\cdots + X_n^2 
\end{equation}
服从自由度为$n$的$\chi^2$分布,记为\textcolor{red}{$\chi^2 \sim \chi^2(n)$}。
自由度:独立变量的个数;

$\chi^2(n)$分布的概率密度函数:
\begin{equation}
\color{red} f(y) = \left\{
\begin{aligned}
&\frac{1}{2^{n/2} \Gamma(n/2)} y^{n/2-1} e^{-y/2},& y> 0 \\
&0,& 
\end{aligned}
\right.
\end{equation}



\subsection{$t$分布}
也称\textcolor{red}{学生氏分布}

设$X \sim N(0,1)$,$Y\sim \chi^2(n)$,且$X$,$Y$独立,称随机变量
\begin{equation}
\color{red} t = \frac{X}{\sqrt{Y/n}}
\end{equation}
服从自由度为$n$的$t$分布,记为\textcolor{red}{$t \sim t(n)$}。

$t(n)$分布的概率密度函数:
\begin{equation}
\color{red} h(t) = \frac{\Gamma[(n+1)/2]}{\sqrt{\pi n}~\Gamma(n/2)} \left(1+ \frac{t^2}{n} \right)^{-(n+1)/2}, ~~ -\infty < t < \infty
\end{equation}

\subsection{$F$分布}

设$U\sim \chi^2(n_1)$,$V\sim \chi^2(n_2)$,且$U$、$V$独立,称随机变量
\begin{equation}
\color{red} F = \frac{U/n_1}{V/n_2}
\end{equation}
服从自由度为$(n_1,n_2)$的$F$分布,记为\textcolor{red}{$F \sim F(n_1,n_2)$}。

$F$分布的概率密度函数:
\begin{equation}
\psi(y) = \left\{
\begin{aligned}
&\frac{\Gamma[(n_1 +n_2)/2] (n_1/n_2)^{n_1/2} y^{n_1/2-1}}{\Gamma(n_1/2) \Gamma(n_2/2) [1+(n_1y/n_2)]^{(n_1+n_2)/2}},& y> 0 \\
&0,& 
\end{aligned}
\right.
\end{equation}

\section{直方图和箱线图}


















\end{document}