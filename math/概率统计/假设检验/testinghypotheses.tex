\documentclass[12pt,a4paper]{article}
%\usepackage{fontspec, xunicode, xltxtra}  
%\setmainfont{Hiragino Sans GB}  
\usepackage{xeCJK}
%\setCJKmainfont[BoldFont=STZhongsong, ItalicFont=STKaiti]{STSong}
%\setCJKsansfont[BoldFont=STHeiti]{STXihei}
%\setCJKmonofont{STFangsong}

%使用Xelatex编译

% 设置页面
%==================================================
\linespread{2} %行距
% \usepackage[top=1in,bottom=1in,left=1.25in,right=1.25in]{geometry}
% \headsep=2cm
% \textwidth=16cm \textheight=24.2cm
%==================================================

% 其它需要使用的宏包
%==================================================
\usepackage[colorlinks,linkcolor=blue,anchorcolor=red,citecolor=green,urlcolor=blue]{hyperref} 
\usepackage{tabularx}
\usepackage{authblk}         % 作者信息
\usepackage{algorithm}     % 算法排版
\usepackage{amsmath}     % 数学符号与公式
\usepackage{amsfonts}     % 数学符号与字体
\usepackage{mathrsfs}      % 花体
\usepackage{amssymb}
\usepackage{bm}

\usepackage{graphics}
\usepackage{color}
\usepackage{fancyhdr}       % 设置页眉页脚
\usepackage{fancyvrb}       % 抄录环境
\usepackage{float}              % 管理浮动体
\usepackage{geometry}     % 定制页面格式
\usepackage{hyperref}       % 为PDF文档创建超链接
\usepackage{lineno}          % 生成行号
\usepackage{listings}        % 插入程序源代码
\usepackage{multicol}       % 多栏排版
%\usepackage{natbib}         % 管理文献引用
\usepackage{rotating}       % 旋转文字,图形,表格
\usepackage{subfigure}    % 排版子图形
\usepackage{titlesec}       % 改变章节标题格式
\usepackage{moresize}   % 更多字体大小
\usepackage{anysize}
\usepackage{indentfirst}  % 首段缩进
\usepackage{booktabs}   % 使用\multicolumn
\usepackage{multirow}    % 使用\multirow
\usepackage{graphicx} 
\usepackage{wrapfig}
\usepackage{xcolor}
\usepackage{titlesec}     % 改变标题样式
\usepackage{enumitem}
\usepackage{harpoon}   %矢量符号

\newcommand{\myvec}[1]%
   {\stackrel{\raisebox{-2pt}[0pt][0pt]{\small$\rightharpoonup$}}{#1}}  %矢量符号
\renewcommand{\vec}[1]{\boldsymbol{#1}}
\newcommand{\me}{\mathrm{e}}
\newcommand{\mi}{\mathrm{i}}
\newcommand{\dif}{\mathrm{d}}
\newcommand{\tabincell}[2]{\begin{tabular}{@{}#1@{}}#2\end{tabular}}

\def\kpc{{\rm kpc}}
\def\km{{\rm km}}
\def\cm{{\rm cm}}
\def\TeV{{\rm TeV}}
\def\GeV{{\rm GeV}}
\def\MeV{{\rm MeV}}
\def\GV{{\rm GV}}
\def\MV{{\rm MV}}
\def\yr{{\rm yr}}
\def\s{{\rm s}}
\def\ns{{\rm ns}}
\def\GHz{{\rm GHz}}
\def\muGs{{\rm \mu Gs}}
\def\arcsec{{\rm arcsec}}
\def\K{{\rm K}}
\def\microK{\mu{\rm K}}
\def\sr{{\rm sr}}
\newcolumntype{p}{D{,}{\pm}{-1}}

\renewcommand{\figurename}{Fig.}
\renewcommand{\tablename}{Tab.}

\renewcommand{\arraystretch}{1.5}

\setlength{\parindent}{0pt}  %取消每段开头的空格

\title{假设检验}
\author{}
\date{\today}
\begin{document}

\maketitle

\section{假设检验}
\cite{浙大概率论与数理统计} 在总体的分布函数完全未知或只知其形式、但不知其参数的情况,为了推断总体的某些未知特性,提出某些关于总体的假设。根据样本对所提出的假设作出是接受,还是拒绝的决策。


提出两个相互独立的假设
\begin{eqnarray*}
H_0 :  = \\
H_1 :  \neq ~.
\end{eqnarray*}
然后给出一个合理的法则,根据这一法则,利用已知样本作出决策是接受假设$H_0$(拒绝假设$H_1$),还是拒绝假设$H_0$(接受假设$H_1$)。由于\textcolor{cyan}{样本均值$\overline{X}$是$\mu$的无偏估计},$\overline{X}$的观察值$\overline{x}$的大小在一定程度上反映$\mu$的大小。若假设$H_0$为真,则观察值$\overline{x}$与$\mu_0$的偏差$|\overline{x}-\mu_0|$一般不应太大。若$|\overline{x}-\mu_0|$过分大,就怀疑假设$H_0$的正确性而拒绝$H_0$,并考虑到当$H_0$为真时$\dfrac{\overline{X}-\mu_0}{\sigma/\sqrt{n}} \sim N(0,1)$。衡量$|\overline{x}-\mu_0|$的大小可归结为衡量$\dfrac{|\overline{x}-\mu_0|}{\sigma/\sqrt{n}}$的大小。适当选定一正数$k$,使当观察值$\overline{x}$满足$\dfrac{|\overline{x}-\mu_0|}{\sigma/\sqrt{n}} \geqslant k$时拒绝假设$H_0$,反之,若$\dfrac{|\overline{x}-\mu_0|}{\sigma/\sqrt{n}} < k$,就接受假设$H_0$。

然而,由于作出决策的依据是一个样本,当实际上$H_0$为真时仍可能作出拒绝$H_0$的决策,这种可能性是无法消除的。这是一种错误,犯这种错误的概率记为 \\
$P\{$当$H_0$为真拒绝$H_0\}$~或~$P_{\mu_0}\{$拒绝$H_0\}$~或~$P_{\mu \in H_0}\{$拒绝$H_0\}$ \\
\textcolor{red}{$P_{\mu_0}\{\cdot\}$表示参数$\mu$取$\mu_0$时事件$\{\cdot\}$的概率},\textcolor{red}{$P_{\mu \in H_0}\{\cdot\}$表示$\mu$取$H_0$规定的值时事件$\{\cdot\}$的概率}。希望将犯这类错误的概率控制在一定限度内,即给出一个较小的整数$\alpha (0 < \alpha < 1)$,使犯这类错误的概率不超过$\alpha$,即使得
\begin{equation}
\color{red} P\{  \text{当}H_0\text{为真拒绝}H_0 \} = P_{\mu_0}\left\{ \Bigg| \dfrac{\overline{X}-\mu_0}{\sigma/\sqrt{n}} \Bigg| \geqslant k \right\} \leqslant \alpha ~.
\end{equation}
当$H_0$为真时,$Z=\dfrac{\overline{X}-\mu_0}{\sigma/\sqrt{n}} \sim N(0,1)$,由标准正态分布分位点的定义
\begin{equation}
\color{red} k = z_{\alpha/2} ~.
\end{equation}
若\textcolor{red}{$Z$的观察值}满足
\begin{equation*}
\color{red} |z| = \Bigg| \dfrac{\overline{x}-\mu_0}{\sigma/\sqrt{n}} \Bigg|  \geqslant k = z_{\alpha/2} ~,
\end{equation*}
则\textcolor{red}{拒绝$H_0$},若
\begin{equation*}
\color{red} |z| = \Bigg| \dfrac{\overline{x}-\mu_0}{\sigma/\sqrt{n}} \Bigg| < k = z_{\alpha/2} ~,
\end{equation*}
则\textcolor{red}{接受$H_0$}。
通常$\alpha$总是取得较小,$\alpha = 0.01, 0.005$。若$H_0$为真,即当$\mu = \mu_0$时,$\left\{\Bigg| \dfrac{\overline{X}-\mu_0}{\sigma/\sqrt{n}} \Bigg| \geqslant z_{\alpha/2} \right\}$是一个小概率事件,根据实际推断原理可以认为,若$H_0$为真,则由一次试验得到的观察值$\overline{x}$满足不等式$\Bigg| \dfrac{\overline{x}-\mu_0}{\sigma/\sqrt{n}} \Bigg| \geqslant z_{\alpha/2}$几乎是不会发生的。如果在一次观察中出现了满足$\Bigg| \dfrac{\overline{x}-\mu_0}{\sigma/\sqrt{n}} \Bigg| \geqslant z_{\alpha/2}$的$\overline{x}$,则有理由怀疑原先的假设$H_0$的正确性,因而拒绝$H_0$。若出现的观察值$\overline{x}$满足$\Bigg| \dfrac{\overline{x}-\mu_0}{\sigma/\sqrt{n}} \Bigg| < z_{\alpha/2}$,此时没有理由拒绝假设$H_0$,即接受假设$H_0$。

\textcolor{purple}{当样本容量固定时,选定$\alpha$后,数$k$就可以确定,然后按照统计量$Z=\dfrac{\overline{X}-\mu_0}{\sigma/\sqrt{n}}$的观察值的绝对值$|z|$大于等于$k$还是小于$k$作出决策。若$|z| = \Bigg| \dfrac{\overline{x}-\mu_0}{\sigma/\sqrt{n}} \Bigg| \geqslant k$,则称$\overline{x}$与$\mu_0$的差异是显著的,拒绝$H_0$;若$|z| = \Bigg| \dfrac{\overline{x}-\mu_0}{\sigma/\sqrt{n}} \Bigg| < k$,则称$\overline{x}$与$\mu_0$的差异是不显著的,接受$H_0$}。\textcolor{red}{$\alpha$称为显著性水平}。统计量$Z=\dfrac{\overline{X}-\mu_0}{\sigma/\sqrt{n}}$称为\textcolor{red}{检验统计量}。

检验问题通常叙述成:在显著性水平$\alpha$下,检验假设
\begin{equation}
H_0 : \mu = \mu_0 ~, ~~ H_1 : \mu \neq \mu_0 ~.
\end{equation}
也即``\textcolor{red}{在显著性水平$\alpha$下,针对$H_1$检验$H_0$}"。\textcolor{red}{$H_0$}称为\textcolor{red}{原假设}或\textcolor{red}{零假设},\textcolor{red}{$H_1$}称为\textcolor{red}{备择假设}(在原假设被拒绝后可供选择的假设)。

当检验统计量取某个区域$C$中的值时,拒绝原假设$H_0$,称\textcolor{red}{区域$C$为拒绝域},拒绝域的边界点称为\textcolor{red}{临界点}。以上问题的拒绝域为$|z| \geqslant z_{\alpha/2}$,而$z=-z_{\alpha/2}, z=z_{\alpha/2}$为临界点。在假设$H_0$实际为真时,可能犯拒绝$H_0$的错误,称为“弃真”错误,也叫第I类错误。当$H_0$实际为不真时,有可能接受$H_0$,称为“取伪”错误,也叫第II类错误。犯第II类错误的概率 \\
$P\{$当$H_0$不真接受$H_0 \}$~或~$P_{\mu \in H_1}\{$接受$H_0\}$ \\
在确定检验法则时,尽可能使犯两类错误的概率都较小。但\textcolor{cyan}{当样本容量固定时,若减少犯一类错误的概率,则犯另一类错误的概率往往增大。要使犯两类错误的概率都减小,除非增加样本容量}。在给定样本容量的情况下,总是控制犯第|类错误的概率,使它不大于$\alpha$。\textcolor{cyan}{只对犯第I类错误的概率加以控制,而不考虑犯第II类错误的概率的检验},称为\textcolor{red}{显著性检验}。
若备择假设$H_1$中,$\mu$可能大于$\mu_0$,也可能小于$\mu_0$,称为\textcolor{violet}{双边备择假设},该假设检验为\textcolor{violet}{双边假设检验}。
\begin{eqnarray*}
H_0 : \mu \leqslant \mu_0 ,~~~ H_1 : \mu > \mu_0
\end{eqnarray*}
称为\textcolor{violet}{右边检验},
\begin{eqnarray*}
H_0 : \mu \geqslant \mu_0 ,~~~ H_1 : \mu < \mu_0
\end{eqnarray*}
称为\textcolor{violet}{左边检验}。右边检验和左边检验统称\textcolor{violet}{单边检验}。



处理参数的假设检验步骤如下:\\
1. 根据实际问题的要求,提出原假设$H_0$及备择假设$H_1$;\\
2. 给定显著性水平$\alpha$,及样本容量$n$;\\
3. 确定检验统计量及拒绝域的形式;\\
4. 按$P\{$当$H_0$为真拒绝$H_0\} \leqslant \alpha$求出拒绝域;\\
5. 取样,根据样本观察值作出决策,是接受还是拒绝$H_0$。



Decide between two competing statements, \textcolor{red}{$H_0$} and \textcolor{red}{$H_a$}, called the \textcolor{red}{null hypothesis} and \textcolor{red}{alternative hypothesis}, based on the data. There are \textcolor{cyan}{two possible errors} in adjudicating between these hypotheses: 

\textcolor{cyan}{Type 1 error} Here one \textcolor{cyan}{wrongly rejects the null hypothesis $H_0$} giving a \textcolor{cyan}{false positive decision}.

\textcolor{cyan}{Type 2 error} Here one \textcolor{cyan}{fails to reject the null hypothesis} when the \textcolor{cyan}{alternative is true}, giving a \textcolor{cyan}{false negative decision}. In our example, we would incorrectly infer that a signal is absent when it truly is present.

[实验的数学处理] 选择一个统计量
\begin{equation}
\lambda = \lambda(\vec{x})
\end{equation}
叫做\textcolor{red}{检验统计量},在假设$H$成立的条件下,检验统计量$\lambda$的概率(密度)函数$p(\lambda | H)$已知。由$p(\lambda | H)$可以决定$\lambda$值的某个区域$\omega$,叫做\textcolor{red}{拒绝域}。整个区域的概率含量
\begin{equation}
P_r(\lambda \in \omega | H) = \int_{\lambda \in \omega} p(\lambda | H) \dif \lambda = \alpha 
\end{equation}
也很小($\lambda \in \omega$表示$\lambda$值属于区域$\omega$)。若假设$H$为真,则$\lambda$值落入拒绝域$\omega$内的可能性很小。若由样本$\vec{x}$所算得的统计量$\lambda$的数值$\lambda = \lambda(\vec{x})$落到了拒绝域$\omega$内,则说观测结果$\vec{x}$同假设$H$有显著的矛盾(显著水平$\alpha$),或者说在显著水平$\alpha$下拒绝假设$H$。若统计量$\lambda = \lambda(\vec{x})$落在区域$\omega$外,则样本$\vec{x}$同假设$H$没有显著的矛盾(显著水平$\alpha$),或者说在显著水平$\alpha$下接受假设$H$。

显著水平$\alpha$就是在原假设$H$成立的条件下,检验统计量$\lambda$在拒绝域$\omega$内的概率含量,即分布$p(\lambda | H)$在拒绝域$\omega$内的概率含量。








即使原假设$H$为真,$\lambda$值也可能落入拒绝域$\omega$内。此时会错误地拒绝原假设$H$,显著水平$\alpha$就是犯这种错误地概率。$\alpha$是在$H$为真的条件下,$\lambda$落入区域$\omega$内的概率。$\alpha$是区域$\omega$包含$\lambda$的置信水平。记$\lambda$的全部可能取值





\section{正态总体均值的假设检验}



\section{正态总体方差的假设检验}



\section{置信区间与假设检验之间的关系}



\section{样本容量的选取}



\section{分布拟合检验}
\subsection{单个分布的$\chi^2$拟合检验法}
设总体$X$的分布未知,$x_1, x_2, \cdots, x_n$是来自$X$的样本值。检验假设 \\
$H_0$ : 总体$X$的分布函数为$F(x)$;\\
$H_1$ : 总体$X$的分布函数不是$F(x)$;\\
设$F(x)$不含未知参数。(也常以分布律或概率密度代替$F(x)$)

将在$H_0$下$X$可能取值的全体$\Omega$分成互不相交的子集$A_1, A_2, \cdots, A_k$,以$f_i (i =1, 2, \cdots, k)$记样本观察值$x_1, x_2, \cdots, x_n$落在$A_i$的个数,表示事件$A_i =\{X$的值落在子集$A_i$内$\}$在$n$次独立试验中发生$f_i$次,在这$n$次试验中事件$A_i$发生的频率为$f_i/n$。当$H_0$为真时,根据$H_0$中所假设的$X$的分布函数来计算事件$A_i$的概率,得到$p_i = P(A_i), i = 1, 2, \cdots, k$。当$H_0$为真,且试验次数很多时,频率$f_i/n$和概率$p_i$的差异不应太大。用统计量
\begin{equation}
\sum_{i=1}^k C_i \left(\frac{f_i}{n} -p_i\right)^2
\end{equation}
来度量样本与$H_0$中所假设的分布的吻合程度,其中$C_i (i = 1, 2, \cdots, k)$为给定的常数。若选取$C_i = n/p_i, (i = 1, 2, \cdots, k)$,检验统计量为
\begin{equation}
\chi^2 = \sum_{i=1}^k \frac{n}{p_i} \left(\frac{f_i}{n} -p_i\right)^2 = \sum_{i=1}^n \frac{f_i^2}{np_i} -n ~,
\end{equation}
当$n$充分大$(n \geqslant 50)$,则当$H_0$为真时,该统计量近似服从$\chi^2 (k-1)$分布。当$H_0$为真时,$\chi^2$不应该太大,若$\chi^2$过大就拒绝$H_0$,拒绝域形式为
\begin{equation}
\chi^2 \geqslant G
\end{equation}
$G$为正常数。对于给定的显著性水平$\alpha$,确定$G$使 \\
$P\{$当$H_0$为真时拒绝$H_0\} = P_{H_0} \{ \chi^2 \geqslant G\} = \alpha$ \\
得到$G = \chi^2_{\alpha}(k-1)$。即当样本观察值使得
\begin{equation}
\chi^2 \geqslant \chi^2_{\alpha}(k-1)
\end{equation}
则在显著性水平$\alpha$下拒绝$H_0$,否则接受$H_0$。$n$不能小于$50$,$np_i \geqslant 5$,否则适当合并$A_i$。

\subsection{分布族的$\chi^2$拟合检验法}
原假设\\
$H_0$ : 总体$X$的分布函数是$F(x; \theta_1, \theta_2, \cdots, \theta_r)$,\\
$F$的形式已知,而$\bm{\theta} = (\theta_1, \theta_2, \cdots, \theta_r)$是未知参数,它们在某一个范围取值。将在$H_0$下$X$可能取值的全体$\Omega$分成$k (k > r+1)$个互不相交的子集$A_1, A_2, \cdots, A_k$,以$f_i (i = 1, 2, \cdot, k)$记样本观察值$x_1, x_2, \cdots, x_n$落在$A_i$的个数,则事件$A_i = \{X$的值落在$A_i$内$\}$的频率为$f_i/n$。当$H_0$为真时,由$H_0$所假设的分布函数来计算$P(A_i)$,得到$P(A_i) = p_i(\theta_1, \theta_2, \cdots, \theta_r) = p_i(\bm{\theta} ) = p_i$。先利用样本求出未知参数的最大似然估计(在$H_0$下),以估计值作为参数值,求出$p_i$的估计值$\hat{p}_i = \hat{P}(A_i)$。
\begin{equation}
\chi^2 = \sum_{i=1}^k \frac{f_i^2}{n \hat{p}_i} -n
\end{equation}
作为检验假设$H_0$的统计量。在某些条件下,在$H_0$为真时,近似地有
\begin{equation}
\chi^2 = \sum_{i=1}^k \frac{f_i^2}{n \hat{p}_i} -n \sim \chi^2 (k-r-1) ~.
\end{equation}
拒绝域为
\begin{equation}
\chi^2 \geqslant \chi^2_\alpha (k-r-1) ~,
\end{equation}
$\alpha$为显著性水平。

\subsection{偏度、峰度检验}
$\chi^2$拟合检验法用来检验总体的正态性时,犯||类错误的概率往往较大。

随机变量$X$的\textcolor{red}{偏度}和 \textcolor{red}{峰度}是$X$的标准变化量 \textcolor{cyan}{$[X-E(X)]/\sqrt{D(X)}$}的三阶矩和四阶矩:
\begin{eqnarray*}
\nu_1 = E\left[\left( \frac{X-E(X)}{\sqrt{D(X)} } \right)^3 \right] = \frac{E[(X-E(X))^3]}{(D(X) )^{3/2}} ~, \\
\nu_1 = E\left[\left( \frac{X-E(X)}{\sqrt{D(X)} } \right)^4 \right] = \dfrac{E[(X-E(X))^4]}{(D(X) )^{2}} ~.
\end{eqnarray*}
当随机变量$X$服从正态分布时,$\nu_1 = 0$且$\nu_2 = 3$。

设$X_1, X_2, \cdots, X_n$是来自总体$X$的样本,则$\nu_1, \nu_2$的矩估计量分别是
\begin{equation}
\color{red} G_1 = B_3/B_2^{3/2} ~, G_2 = B_4/B_2^2 ~,
\end{equation}
其中$B_k (k = 2, 3, 4)$是样本$k$阶中心矩,分布称$G_1, G_2$为\textcolor{cyan}{样本偏度}和\textcolor{cyan}{样本峰度}。

若总体$X$为正态分布,当$n$充分大时,近似有
\begin{eqnarray}
G_1 &\sim& N\left(0, \frac{6(n-2)}{(n+1)(n+3)} \right) ~, \\
G_2 &\sim& N\left(3-\frac{6}{n+1},  \frac{24n(n-2)(n-3)}{(n+1)^2(n+3)(n+5)} \right) ~.
\end{eqnarray}

检验假设 $H_0$:$X$为正态总体。记
\begin{eqnarray}
\sigma_1 &=& \sqrt{\frac{6(n-2)}{(n+1)(n+3)} }~, \\
\sigma_2 &=& \sqrt{\frac{24n(n-2)(n-3)}{(n+1)^2(n+3)(n+5)} }
\end{eqnarray}
$\mu_2 = 3-\dfrac{6}{n+1}, U_1 = G_1/\sigma_1, U_2 = (G_2-\mu_2)/\sigma_2$。当$H_0$为真且$n$充分大时,近似有
\begin{equation}
U_1 \sim N(0, 1) ~, ~~ U_2 \sim N(0, 1) ~.
\end{equation}
样本偏度$G_1$、样本峰度$G_2$分别依概率收敛于总体偏度$\nu_1$和总体峰度$\nu_2$。当$H_0$为真且$n$充分大时,$G_1$与$\nu_1 = 0$的偏离不应太大,而$G_2$与$\nu_2 = 3$的偏离不应太大。当$|U_1|$的观察值$|u_1|$或$|U_2|$的观察值$|u_2|$过大时,就拒绝$H_0$。取显著性水平为$\alpha$,$H_0$拒绝域为
\begin{equation}
|u_1| \geqslant k_1, ~~ {\rm or} ~~  |u_2| \geqslant k_2
\end{equation}
其中$k_1, k_2$由下式得到
\begin{eqnarray}
P_{H_0}\{ |U_1| \geqslant k_1 \} &=& \frac{\alpha}{2} ~, \\
P_{H_0}\{ |U_2| \geqslant k_2 \} &=& \frac{\alpha}{2} ~.
\end{eqnarray}
$P_{H_0}\{\cdot \}$表示当$H_0$为真时事件$\{\cdot \}$的概率,即有$k_1 = z_{\alpha/4}, k_2 = z_{\alpha/4}$。拒绝域为
\begin{equation}
|u_1| \geqslant z_{\alpha/4}, ~~ {\rm or} ~~  |u_2| \geqslant z_{\alpha/4}
\end{equation}

当$n$充分大时,
\begin{eqnarray}
\nonumber P\{\text{当}H_0\text{为真时拒绝}H_0\} &=& P_{H_0}\{ (|U_1| \geqslant z_{\alpha/4}) \bigcup (|U_2| \geqslant z_{\alpha/4}) \}  \\
\nonumber &\leqslant& P_{H_0}\{|U_1| \geqslant z_{\alpha/4} \} + P_{H_0}\{|U_2| \geqslant z_{\alpha/4} \} \\
&=& \frac{\alpha}{2} + \frac{\alpha}{2} = \alpha
\end{eqnarray}





















样本容量需大于$100$。


\section{秩和检验}
设有两个连续型总体,概率密度函数分别为$f_1(x), f_2(x)$,均为未知,但已知
\begin{equation}
f_1(x) = f_2(x-a) ~, ~~a\text{为未知常数} ~, 
\end{equation}
$f_1(x)$和$f_2(x)$至多只差一平移。检验下述假设
\begin{eqnarray}
H_0 : a = 0, ~~H1 : a < 0 ~. \\
H_0 : a = 0, ~~H1 : a > 0 ~. \\
H_0 : a = 0, ~~H1 : a \neq 0 ~.
\end{eqnarray}
若两总体均值$\mu_1, \mu_2$存在,由于$f_1, f_2$至多只差一平移,则
\begin{equation*}
\mu_2 = \mu_1 - a ~.
\end{equation*}
上述假设分布等价于
\begin{eqnarray}
H_0 : \mu_1 = \mu_2, ~~H1 : \mu_1 < \mu_2 ~. \\
H_0 : \mu_1 = \mu_2, ~~H1 : \mu_1 > \mu_2 ~. \\
H_0 : \mu_1 = \mu_2, ~~H1 : \mu_1 \neq \mu_2 ~.
\end{eqnarray}
设$X$为一总体,将一容量为$n$的样本观察值按自小到大的次序编号排列成
\begin{equation}
x_{(1)} < x_{(2)} < \cdots < x_{(n)} ~, 
\end{equation}
称$x_{(i)}$的足标$i$为$x_{(i)}$的\textcolor{red}{秩},$i = 1, 2, \cdots, n$。设自$1,2$两总体分布抽取容量为$n_1, n_2$的样本,且设两样本独立。假定$n_1\leqslant n_2$。将$n_1+ n_2$个观察值放在一起,按自小到大的次序排列,求出每个观察值的秩,将属于第$1$个总体的样本观察值的秩相加,其和记为$R_1$,称为第$1$样本的\textcolor{red}{秩和}。其余的观察值的秩的总和记作$R_2$,称为第$2$样本的\textcolor{red}{秩和}。$R_1, R_2$是离散型的随机变量,且
\begin{equation}
R_1+ R_2 = \frac{1}{2}(n_1 +n_2) (n_1 +n_2+1) ~.
\end{equation}
$R_1, R_2$中一个确定后另一个随之确定。当$H_0$为真时,即有$f_1(x) = f_2(x)$,两独立样本来自同一个总体。因而第$1$个样本中诸元素的秩应该随机地、分散在自然数$1\sim n_1 +n_2$中取值,不应过分集中取较小或较大的值。考虑到
\begin{equation*}
\frac{1}{2}n_1 (n_1+1) \leqslant R_1 \leqslant \frac{1}{2}n_1 (n_1 +2n_2+1) ~,
\end{equation*}
即知当$H_0$为真时秩和$R_1$不应取太靠近上述不等式两端的值。因而当$R_1$的观察值$r_1$过分大或过分小时,拒绝$H_0$。在给定显著性水平$\alpha$下,$H_0$的拒绝域为
\begin{equation*}
r_1 \leqslant C_U\left(\frac{\alpha}{2} \right) ~~\text{或}~~ r_1 \geqslant C_L\left(\frac{\alpha}{2} \right) ~,
\end{equation*}
其中临界点$C_U\left(\dfrac{\alpha}{2} \right)$是满足$P_{a=0}\left\{ R_1 \leqslant C_U\left(\dfrac{\alpha}{2} \right) \right\} \leqslant \dfrac{\alpha}{2}$的最大整数,而$C_L\left(\dfrac{\alpha}{2} \right)$是满足$P_{a=0}\left\{ R_1 \geqslant C_L\left(\dfrac{\alpha}{2} \right) \right\} \leqslant \dfrac{\alpha}{2}$的最小整数。而犯第I类错误的概率为
\begin{equation*}
P_{a=0}\left\{ R_1 \leqslant C_U\left(\dfrac{\alpha}{2} \right) \right\} +P_{a=0}\left\{ R_1 \geqslant C_L\left(\dfrac{\alpha}{2} \right) \right\} \leqslant \frac{\alpha}{2} +\frac{\alpha}{2} = \alpha ~.
\end{equation*}




\section{假设检验问题的$p$值检验法}
之前讨论的假设检验法称为\textcolor{red}{临界值法}。

假设检验问题的\textcolor{red}{$p$值}(probability value)是由\textcolor{cyan}{检验统计量的样本观察值得出}的\textcolor{cyan}{原假设可被拒绝}的\textcolor{cyan}{最小显著性水平}。任一检验问题的$p$值可以根据检验统计量的样本观察值以及检验统计量在$H_0$下一个特定的参数值(一般是$H_0$和$H_1$所规定的参数的分界点)对应的分布求出。

若显著性水平$\alpha \geqslant p$,则对应的临界值$z_\alpha \leqslant$观察值$z_0$,表示观察值落在拒绝域内,拒绝$H_0$;若显著性水平$\alpha < p$,则对应的临界值$z_\alpha >$观察值$z_0$,表示观察值不落在拒绝域内,接受$H_0$。$p$值是原假设$H_0$可被拒绝的最小显著性水平。

任一检验问题的$p$值可以根据检验统计量的样本观察值以及检验统计量在$H_0$下一个特定的参数值($H_0$和$H_1$所规定的参数的分界点)对应的分布求出。

对于任意指定的显著性水平$\alpha$,\\
(1) 若\textcolor{red}{$p$值$\leqslant \alpha$},则在显著性水平$\alpha$下\textcolor{red}{拒绝$H_0$}。\\
(2) 若\textcolor{red}{$p$值$\geqslant \alpha$},则在显著性水平$\alpha$下\textcolor{red}{接受$H_0$}。\\
用来确定$H_0$的拒绝域。利用$p$值来确定检验拒绝域的方法,称为\textcolor{red}{$p$值检验法}。








若$p$值$\leqslant 0.01$,则推断拒绝$H_0$的依据很强或检验是高度显著的;若$0.01 < p$值$\leqslant 0.05$,则推断拒绝$H_0$的依据强或检验是显著的;若$0.05 < p$值$\leqslant 0.1$,则推断拒绝$H_0$的理由是弱的,检验是不显著的;若$p$值$> 0.1$,一般来说没有理由拒绝$H_0$。




\cite{1980实验的数学处理} ``参数估计"问题是由观测值样本$\vec{x}$推断分布参数$\theta$,其中随机变量分布$p(x; \theta)$的函数形式已知。若理论上还不能给出被观测随机变量的概率分布的确切形式,或者虽然存在着某个理论分布公式,但不知道它是否适用于具体的实验观测条件,此时被采用的某个分布形式只是一个假设,它是否合理还需要根据观测得到的样本来判断,即假设检验。

\section{显著性检验}

\subsection{统计假设}

\subsection{检验统计量和显著水平}


\subsection{拟合性检验 $H: p(x; \theta) = f(x; \theta)$}
根据样本$\vec{x}$检验随机变量$\vec{x}$的分布$p(x; \theta)$是否为某种假定的分布形式$ f(x; \theta)$,即
\begin{equation*}
\text{检验} ~ H: p(x; \theta) = f(x; \theta) ~,
\end{equation*}
这一类的检验问题叫做\textcolor{red}{拟合性检验}。适用于检验任意形式的假设分布$f(x; \theta)$。但它们几乎都是大样本的检验法,所利用的各种检验统计量$\lambda$,只有在大样本的情况下才渐近服从一个与假设分布$f(x; \theta)$的形式无关的极限分布。对于一种特定的假设分布形式,可以找到比一般方法更精确的检验法---一个更适用的检验统计量,但它不能应用于其他分布形式的检验。



(一) $\chi^2$检验


(5) 计算皮尔逊(Pearson)$\chi^2$量:
\begin{equation}
\chi^2 = \sum_{i=1}^m \dfrac{(n_i -E_i)^2}{E_i} ~.
\end{equation}
当观测值个数$N \rightarrow \infty$时,皮尔逊$\chi^2$量渐近地服从自由度$\nu = m - K - 1$的$\chi^2$分布,即
\begin{equation}
p\left(\chi^2 = \sum_{i=1}^m \dfrac{(n_i-E_i)^2}{E_i} \right) \rightarrow \chi^2(\nu = m - K -1) ~.
\end{equation}
因此可以用皮尔逊$\chi^2$量作检验。

(6) 利用皮尔逊$\chi^2$量作为检验统计量,选取显著水平为$\alpha$,由$\chi^2$分布表查出$\chi^2_{1-\alpha}(m-K-1)$,则拒绝域为
\begin{equation*}
\omega = (\chi^2_{1-\alpha}(m-K-1), \infty) ~,
\end{equation*}
即




皮尔逊$\chi^2$量的大小表现了实测频数$n_i$和理论预期值$E_i$之间差异的大小。



\section{参数检验(简单假设)}
如何由观测值样本来判断关于分布参数的两个假设(原假设$H$和备择假设$H^\prime$)。参数检验问题中,观测值$x$的分布$p(x;\theta)$的函数形式已知,对于简单假设的情况,参数$\theta$只能取$\theta_0$或者$\theta_1$这两个值,即对于参数$\theta$的原假设和备择假设为
\begin{equation}
H: \theta = \theta_0, ~ H^\prime: \theta = \theta_1 ~.
\end{equation}
“检验$H: \theta = \theta_0, ~ H^\prime: \theta = \theta_1$”,就是在已知参数只可能取$\theta_0$和$\theta_1$这两个值的情况下,根据观测值样本$\vec{x} = (x_1, x_2, \cdots, x_N)$判断参数$\theta$的值是不是$\theta_0$。


选择一个检验统计量$\lambda=\lambda(\vec{x})$,然后决定一个显著水平$\alpha$的拒绝域$\omega$,若样本的$\lambda$值落入拒绝域$\omega$内,则拒绝原假设$H$。但是,即使$H$为真,$\lambda$值也有一定的概率(即显著水平$\alpha$)落入拒绝域内:
\begin{align}
\nonumber P_r(\lambda \in \omega | H) &= \int_{\lambda\in \omega} p(\lambda |H) \dif \lambda \\
\nonumber &= \int_{\lambda\in \omega} p(\lambda |\theta = \theta_0) \dif \lambda \\
&= \alpha
\end{align}
$H$为真时错误地拒绝假设$H$,叫做\textcolor{red}{假设检验的第一类错误}。出现第一类错误的概率就是\textcolor{red}{显著水平$\alpha$},又叫做\textcolor{red}{检验的损失}。

若$\lambda$值落在拒绝域$\omega$外,则接受假设$H$(根据样本$\vec{x}$没有明显的理由拒绝假设$H$)。$H$非真而$H^\prime$为真时,$\lambda$值也有一定的概率落到拒绝域之外,即落入区域$W-\omega$内,
\begin{align}
\nonumber P_r(\lambda \in W-\omega | H^\prime) &= 1-P_r(\lambda \in \omega | H^\prime) \\
\nonumber &= 1- \int_{\lambda\in \omega} p(\lambda |H^\prime) \dif \lambda \\
\nonumber &= 1- \int_{\lambda\in \omega} p(\lambda |\theta = \theta_1) \dif \lambda \\
&= \beta
\end{align}
$H$为非真而$H^\prime$为真时,错误地接受假设$H$,叫做\textcolor{red}{假设检验的第二类错误}。出现第二类错误的概率$\beta$叫做\textcolor{red}{检验的污染}。只有在污染$\beta$相当小时,所谓“接受原假设$H$”才意味着原假设很可能是正确的。

同时考虑损失和污染这两类错误的大小,提供了衡量检验方法好坏的标准:为了有效地分辨原假设$H$和备择假设$H^\prime$,要求检验的损失$\alpha$和污染$\beta$都比较小。选定检验的损失,即选定显著水平$\alpha$后,可采用各种不同的统计量作为检验统计量。对于某个检验统计量,又可以采用各种不同的区域作为拒绝域,只要在原假设$H$为真时检验统计量在该区域内的概率含量为$\alpha$即可。如果只考虑检验的损失,则存在着很多种不同的检验方法。在一定的显著水平,即一定的检验损失的要求下,应当采用污染比较小的检验方法。

\begin{equation}
1-\beta = P_r(\lambda \in \omega |H^\prime) = \int_{\lambda \in \omega} p(\lambda|\theta=\theta_1) \dif \lambda ~.
\end{equation}
$1-\beta$叫做\textcolor{red}{检验的功效}。功效即$H^\prime$被拒绝的份额。对于选定的损失$\alpha$,应当采用污染$\beta$较小,即功效$1-\beta$较大的检验方法。

\section{似然比检验}
在损失$\alpha$一定的情况下,污染$\beta$最小(功效$1-\beta$最大)的检验方法叫做\textcolor{red}{佳效检验}。对于观测值分布$p(x;\theta)$参数的简单假设
\begin{equation}
H: \theta = \theta_0, ~ H^\prime: \theta = \theta_1 ~.
\end{equation}
存在一个佳效检验,叫做\textcolor{red}{似然比检验}。





































































%%%%%%%%%%%%%%%%%%%%%%%%%%%%%%%%%%%%%%%%%%%%%%%%%%%%%%%%%%%%%%%%%%%%%%
\bibliographystyle{unsrt_update}
\bibliography{ref}
%%%%%%%%%%%%%%%%%%%%%%%%%%%%%%%%%%%%%%%%%%%%%%%%%%%%%%%%%%%%%%%%%%%%%%


\end{document}