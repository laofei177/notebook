\documentclass[12pt,a4paper]{article}
%\usepackage{fontspec, xunicode, xltxtra}  
%\setmainfont{Hiragino Sans GB}  
\usepackage{xeCJK}
%\setCJKmainfont[BoldFont=STZhongsong, ItalicFont=STKaiti]{STSong}
%\setCJKsansfont[BoldFont=STHeiti]{STXihei}
%\setCJKmonofont{STFangsong}

%使用Xelatex编译

% 设置页面
%==================================================
\linespread{2} %行距
% \usepackage[top=1in,bottom=1in,left=1.25in,right=1.25in]{geometry}
% \headsep=2cm
% \textwidth=16cm \textheight=24.2cm
%==================================================

% 其它需要使用的宏包
%==================================================
\usepackage[colorlinks,linkcolor=blue,anchorcolor=red,citecolor=green,urlcolor=blue]{hyperref} 
\usepackage{tabularx}
\usepackage{authblk}         % 作者信息
\usepackage{algorithm}     % 算法排版
\usepackage{amsmath}     % 数学符号与公式
\usepackage{amssymb}
\usepackage{amsfonts}     % 数学符号与字体
\usepackage{mathrsfs}      % 花体
\usepackage{graphics}
\usepackage{color}
\usepackage{fancyhdr}       % 设置页眉页脚
\usepackage{fancyvrb}       % 抄录环境
\usepackage{float}              % 管理浮动体
\usepackage{geometry}     % 定制页面格式
\usepackage{hyperref}       % 为PDF文档创建超链接
\usepackage{lineno}          % 生成行号
\usepackage{listings}        % 插入程序源代码
\usepackage{multicol}       % 多栏排版
%\usepackage{natbib}         % 管理文献引用
\usepackage{rotating}       % 旋转文字,图形,表格
\usepackage{subfigure}    % 排版子图形
\usepackage{titlesec}        % 改变章节标题格式
\usepackage{moresize}    % 更多字体大小
\usepackage{anysize}
\usepackage{indentfirst}   % 首段缩进
\usepackage{booktabs}    % 使用\multicolumn
\usepackage{multirow}     % 使用\multirow
\usepackage{graphicx} 
\usepackage{wrapfig}
\usepackage{xcolor}
\usepackage{titlesec}       % 改变标题样式
\usepackage{enumitem}
\usepackage{harpoon}    %矢量符号
\usepackage{xcolor}        % 高亮




\newcommand{\myvec}[1]%
   {\stackrel{\raisebox{-2pt}[0pt][0pt]{\small$\rightharpoonup$}}{#1}}  %矢量符号
\renewcommand{\vec}[1]{\boldsymbol{#1}}
\newcommand{\me}{\mathrm{e}}
\newcommand{\mi}{\mathrm{i}}
\newcommand{\dif}{\mathrm{d}}
\newcommand{\tabincell}[2]{\begin{tabular}{@{}#1@{}}#2\end{tabular}}

\def\kpc{{\rm kpc}}
\def\km{{\rm km}}
\def\cm{{\rm cm}}
\def\TeV{{\rm TeV}}
\def\GeV{{\rm GeV}}
\def\MeV{{\rm MeV}}
\def\GV{{\rm GV}}
\def\MV{{\rm MV}}
\def\yr{{\rm yr}}
\def\s{{\rm s}}
\def\ns{{\rm ns}}
\def\GHz{{\rm GHz}}
\def\muGs{{\rm \mu Gs}}
\def\arcsec{{\rm arcsec}}
\def\K{{\rm K}}
\def\microK{\mu{\rm K}}
\def\sr{{\rm sr}}
\newcolumntype{p}{D{,}{\pm}{-1}}

\renewcommand{\figurename}{Fig.}
\renewcommand{\tablename}{Tab.}

\renewcommand{\arraystretch}{1.5}

\setlength{\parindent}{0pt}  %取消每段开头的空格



\title{排列组合}
\author{}
\date{\today}
\begin{document}

\maketitle

\section[Section Title]{计数基本法则\footnote{参考自\cite{GVK330306553}} }
有两个试验,其中试验$1$有$m$种可能发生的结果,对应于试验$1$的每一个结果,试验$2$有$n$种可能发生的结果,则对这两个试验,一共有$mn$种可能结果。

推广:

一共有$r$个试验,第一个试验有$n_1$种可能结果;对于第一个试验的每一种试验结果,第二个试验有$n_2$种可能结果;对应于头两个试验的每一种试验结果,第三个试验有$n_3$种可能结果;$\ldots$那么,这$r$个试验一共有$n_1 \cdot n_2 \cdots n_r$种可能结果。

\textcolor{red}{乘法法则}:

假设必须依次完成$k$个动作。若完成第一个动作有$n_1$种方法,完成第二个动作有$n_2$种方法,完成第三个动作有$n_3$种方法,以此类推,完成第$k$个动作有$n_k$种方法,那么一起完成全部$k$个动作有$n_1 \cdot n_2 \cdot n_3 \cdot \cdots \cdot n_k$种方法。



\section{排列}
permutation

假设有$n$个元素,一共有$n(n-1)(n-2)\cdots 3\cdot 2 \cdot 1 = n!$种不同的排列方式。

$n$个元素,若其中$n_1$个元素彼此相同,另$n_2$个彼此相同,$\cdots$,$n_r$个也彼此相同,那么一共有
\begin{equation}
\frac{n!}{n_1 ! n_2 ! \cdots n_r !}
\end{equation}
种排列方式。

\textcolor{red}{有序集}:

来自于$n$个元素的一个集合称为有序的,若这个集合的每一个元素与从$1$到$n$的某个数(元素的号码)相对应,且不同的元素对应着不同的数。有序集被认为是不同的,若它们的元素,或者元素的次序有差别。

\textcolor{red}{给定集合的排列}:

从同一个集合可以得到仅区别于元素次序不同的有序集。这些有序集称为这个集合的排列。$n$个元素的集合,其排列个数等于
\begin{equation}
P_n = n!
\end{equation}

\textcolor{red}{从$n$个元素取$k$个元素的排列}:

$n$个元素的集合中的$k$个元素的有序子集称为从$n$个元素取$k$个元素的排列。从$n$个元素中取$k$个元素,排列的个数等于
\begin{equation}
A_n^k = k! \cdot C_n^k = n(n-1)\cdots (n-k+1).
\end{equation}



\section{组合}
从$n$个元素中取$r$个,一共有多少种取法?

若考虑顺序,从$n$个元素中选择$r$个组成一组一共有$n(n-1)\cdots(n-r+1)$种方式,而每个含$r$个元素的小组都被重复计算了共$r!$次,所以能组成不同的组的数目为
\begin{equation}
\frac{n(n-1)\cdots(n-r+1)}{r!} = \frac{n!}{(n-r)!r!}
\end{equation}

对$r \leqslant n$,定义
\begin{equation}
\binom n r = \frac{n!}{(n-r)! r!}
\end{equation}
表示从$n$个元素中一次取$r$个的可能组合数;从$n$个元素中一次取$r$个元素的可能取法的数目,如果不考虑抽取顺序的话。

$0!$定义为$1$,因此,
\begin{equation}
\binom n 0 = \binom n n = 1
\end{equation}
当$i < 0$或$i > n$时,也认为
\begin{equation}
\binom n i = 0.
\end{equation}

\textcolor{red}{从$n$个元素取$k$个元素的组合}:

假设集合$A$有$n$个元素,那么含有$k$个元素的$A$的子集的个数为
\begin{equation}
C_n^k = \frac{n!}{k! (n-k)!}
\end{equation}

集合$A = \{a_1, a_2, \cdots, a_n\}$的$k$个元素的子集称为从$n$个元素$\{a_1, a_2, \cdots, a_n\}$中取$k$个元素的组合。

\section{二项式定理}
\begin{equation}
(x+y)^n = \sum_{k=0}^n \binom n k x^k y^{n-k}
\end{equation}
\begin{equation}
\binom n k
\end{equation}
称为\textcolor{red}{二项式系数}。

恒等式:
\begin{equation}
\binom{n}{r} = \binom{n-1}{r-1} +\binom{n-1}{r}, ~1 \leqslant r \leqslant n
\end{equation}





\section{多项式定理}
\begin{equation}
(x_1 +x_2 +\cdots +x_r)^n = \sum_{\substack{(n_1,\cdots,n_r): \\n_1+\cdots+n_r=n} } \binom{n}{n_1, n_2, \cdots, n_r}  x_1^{n_1} x_2^{n_2}\cdots x_r^{n_r}
\end{equation}
求和号对一切满足$n_1 +n_2 +\cdots +n_r = n$的所有非负整向量$(n_1, n_2, \cdots, n_r)$求和。
\begin{equation}
\binom{n}{n_1, n_2, \cdots, n_r} 
\end{equation}
称为\textcolor{red}{多项式系数}。


有$n$个不同的元素,分成$r$组,每组分别有$n_1, n_2, \cdots, n_r$个元素,其中
\begin{equation}
\sum_{i=1}^{r} n_i = n, 
\end{equation}
一共有多少种分法?
\begin{eqnarray}
\nonumber && \binom{n}{n_1} \binom{n -n_1}{n_2} \cdots \binom{n-n_1-n_2-\cdots -n_{r-1}}{n_r} \\
\nonumber &=& \frac{n!}{(n-n_1)!n_1!}\cdot \frac{(n-n_1)!}{(n-n_1-n_2)!n_2!}\cdots \frac{(n-n_1-n_2-\cdots -n_{r-1})!}{0!n_r!} \\
&=& \frac{n!}{n_1! n_2!\cdots n_r!}
\end{eqnarray}

若$n_1+n_2+\cdots+n_r = n$,定义
\begin{equation}
\binom{n}{n_1, n_2, \cdots, n_r} = \frac{n!}{n_1! n_2!\cdots n_r!}
\end{equation}
表示$n$个不同的元素分成大小分别为$n_1, n_2, \cdots, n_r$的$r$组的方法数。

\textcolor{red}{$n$个元素的集合划分为$m$组的方法个数}:

设$k_1, k_2, \cdots, k_m$是非负整数,且$k_1 +k_2+\cdots +k_m = n$。使得$n$各元素的集合$A$成为分别含有$k_1, \cdots, k_m$个元素的集合$B_1, \cdots, B_m$的并的方法个数等于
\begin{equation}
C_n(k_1, k_2, \cdots, k_m) = \frac{n!}{k_1!k_2!\cdots k_m!} .
\end{equation}

\textcolor{red}{重复排列}:

若$n$个元素共有$k$种类型,第一种类型有$k_1$个元素,第二种类型有$k_2$个元素,$\cdots$,第$m$种类型有$k_m$个元素。则$n$个元素的各种不同排列的个数等于
\begin{equation}
C_n(k_1, k_2, \cdots, k_m) = \frac{n!}{k_1!k_2!\cdots k_m!} .
\end{equation}

\textcolor{red}{集合的直积}:

设有$k$个集合$A_1, A_2, \cdots, A_k$。所有形如$(a_1, a_2, \cdots, a_k)$,其中$a_1 \in A_1, a_2 \in A_2, \cdots, a_k \in A_k$的元素构成的集合叫做集合$A_1, A_2, \cdots, A_k$的直积,并记成$A_1\times A_2\times \cdots \times A_k$。

\textcolor{red}{集合直积的元素的个数}:

设$N(A)$表示集合$A$的元素个数,那么$N(A_1\times A_2\times \cdots \times A_k) = N(A_1)\cdot N(A_2) \cdots N(A_k)$。

\section{方程的整数解个数}

























































%%%%%%%%%%%%%%%%%%%%%%%%%%%%%%%%%%%%%%%%%%%%%%%%%%%%%%%%%%%%%%%%%%%%%%
\bibliographystyle{unsrt_update}
\bibliography{ref}
%%%%%%%%%%%%%%%%%%%%%%%%%%%%%%%%%%%%%%%%%%%%%%%%%%%%%%%%%%%%%%%%%%%%%%


\end{document}