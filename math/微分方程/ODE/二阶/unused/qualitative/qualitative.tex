\documentclass[12pt,a4paper]{article}
%\usepackage{fontspec, xunicode, xltxtra}  
%\setmainfont{Hiragino Sans GB}  
%\usepackage{xeCJK}
%\setCJKmainfont[BoldFont=STZhongsong, ItalicFont=STKaiti]{STSong}
%\setCJKsansfont[BoldFont=STHeiti]{STXihei}
%\setCJKmonofont{STFangsong}

%使用Xelatex编译

% 设置页面
%==================================================
\linespread{2} %行距
% \usepackage[top=1in,bottom=1in,left=1.25in,right=1.25in]{geometry}
% \headsep=2cm
% \textwidth=16cm \textheight=24.2cm
%==================================================

% 其它需要使用的宏包
%==================================================
\usepackage[colorlinks,linkcolor=blue,anchorcolor=red,citecolor=green,urlcolor=blue]{hyperref} 
\usepackage{tabularx}
\usepackage{authblk}         % 作者信息
\usepackage{algorithm}     % 算法排版
\usepackage{amsmath}     % 数学符号与公式
\usepackage{amsfonts}     % 数学符号与字体
\usepackage{amssymb}
\usepackage[framemethod=TikZ]{mdframed}

\usepackage{graphicx} 
\usepackage{graphics}
\usepackage{color}
\usepackage{xcolor}
\usepackage{tcolorbox}
\usepackage{lipsum}
\usepackage{empheq}

\usepackage{fancyhdr}       % 设置页眉页脚
\usepackage{fancyvrb}       % 抄录环境
\usepackage{float}              % 管理浮动体
\usepackage{geometry}     % 定制页面格式
\usepackage{hyperref}       % 为PDF文档创建超链接
\usepackage{lineno}          % 生成行号
\usepackage{listings}        % 插入程序源代码
\usepackage{multicol}       % 多栏排版
\usepackage{natbib}         % 管理文献引用
\usepackage{rotating}       % 旋转文字,图形,表格
\usepackage{subfigure}    % 排版子图形
\usepackage{titlesec}       % 改变章节标题格式
\usepackage{moresize}   % 更多字体大小
\usepackage{anysize}
\usepackage{indentfirst}  % 首段缩进
\usepackage{booktabs}   % 使用\multicolumn
\usepackage{multirow}    % 使用\multirow

\usepackage{wrapfig}

\usepackage{titlesec}     % 改变标题样式
\usepackage{enumitem}

\renewcommand{\vec}[1]{\boldsymbol{#1}}
\newcommand{\me}{\mathrm{e}}
\newcommand{\mi}{\mathrm{i}}
\newcommand{\dif}{\mathrm{d}}
\newcommand{\tabincell}[2]{\begin{tabular}{@{}#1@{}}#2\end{tabular}}

\def\kpc{{\rm kpc}}
\def\km{{\rm km}}
\def\cm{{\rm cm}}
\def\TeV{{\rm TeV}}
\def\GeV{{\rm GeV}}
\def\MeV{{\rm MeV}}
\def\GV{{\rm GV}}
\def\MV{{\rm MV}}
\def\yr{{\rm yr}}
\def\s{{\rm s}}
\def\ns{{\rm ns}}
\def\GHz{{\rm GHz}}
\def\muGs{{\rm \mu Gs}}
\def\arcsec{{\rm arcsec}}
\def\K{{\rm K}}
\def\microK{\mu{\rm K}}
\def\sr{{\rm sr}}
\newcolumntype{p}{D{,}{\pm}{-1}}

\renewcommand{\figurename}{Fig.}
\renewcommand{\tablename}{Tab.}

\renewcommand{\arraystretch}{1.5}

\newcounter{theo}[section]\setcounter{theo}{0}
\renewcommand{\thetheo}{\arabic{section}.\arabic{theo}}
\newenvironment{theo}[2][]{%
\refstepcounter{theo}%
\ifstrempty{#1}%
{\mdfsetup{%
frametitle={%
\tikz[baseline=(current bounding box.east),outer sep=0pt]
\node[anchor=east,rectangle,fill=blue!20]
{\strut Theorem~\thetheo};}}
}%
{\mdfsetup{%
frametitle={%
\tikz[baseline=(current bounding box.east),outer sep=0pt]
\node[anchor=east,rectangle,fill=blue!20]
{\strut Theorem~\thetheo:~#1};}}%
}%
\mdfsetup{innertopmargin=10pt,linecolor=blue!20,%
linewidth=2pt,topline=true,%
frametitleaboveskip=\dimexpr-\ht\strutbox\relax
}
\begin{mdframed}[]\relax%
\label{#2}}{\end{mdframed}}

\newcommand*\widefbox[1]{\fbox{\hspace{2em}#1\hspace{2em}}}


\title{Qualitative Theory}
\author{}
\date{\today}
\begin{document}

\maketitle

\section{Qualitative Properties of Solutions}












\section{Nonlinear Equations}

\subsection{Autonomous Systems. The Phase Plane and Its Phenomena}
\begin{equation}
\frac{\dif^2 x}{\dif t^2} = f\left(x, \frac{\dif x}{\dif t} \right)
\end{equation}
The values of $x$ (position) and $\dif x/\dif t$ (velocity), which at each instant characterize the state of the system, are called its \textcolor{red}{phases}, and the plane of the variables $x$ and $\dif x/\dif t$ is called the \textcolor{red}{phase plane}. Introduce the variable $y = \dif x/\dif t$,
\begin{equation}
\left\{
\begin{aligned}
\frac{\dif x}{\dif t} &= y \\
\frac{\dif y}{\dif t} &= f(x, y) ~.
\label{equ_sys}
\end{aligned}
\right.
\end{equation}
When $t$ is regarded as a parameter, then in general a solution of (\ref{equ_sys}) is a pair of functions $x(t)$ and $y(t)$ defining a curve in the $xy$-plane, 
\begin{equation}
\left\{
\begin{aligned}
\frac{\dif x}{\dif t} &= F(x, y) \\
\frac{\dif y}{\dif t} &= G(x, y) ~.
\label{equ_sys_ge}
\end{aligned}
\right.
\end{equation}
where $F$ and $G$ are continuous and have continuous first partial derivatives throughout the plane. A system of this kind, in which the \textcolor{orange}{independent variable $t$ does not appear in the functions $F$ and $G$} on the right, is said to be \textcolor{red}{autonomous}. 

If $t_0$ is any number and $(x_0,y_0)$ is any point in the phase plane, then there exists a unique solution
\begin{equation}
\left\{
\begin{aligned}
x &= x(t) \\
y &= y(t) ~.
\label{sulo}
\end{aligned}
\right.
\end{equation}
of (\ref{equ_sys_ge}) such that $x(t_0) = x_0$ and $y(t_0) = y_0$. If $x(t)$ and $y(t)$ are not both constant functions, then (\ref{sulo}) defines a curve in the phase plane called a path of the system. If (\ref{sulo}) is a solution of (\ref{equ_sys_ge}), then 
\begin{equation}
\left\{
\begin{aligned}
x &= x(t +c) \\
y &= y(t +c) ~.
\label{sulo_c}
\end{aligned}
\right.
\end{equation}
is also a solution for any constant $c$. Each path is represented by many solutions, which differ from one another only by a translation of the parameter. Any path through the point $(x_0,y_0)$ must correspond to a solution of the form (\ref{sulo_c}). At most one path passes through each point of the phase plane. The direction of increasing $t$ along a given path is the same for all solutions representing the path. We shall use arrows to indicate the direction in which the path is traced out as $t$ increases.

In general the path of (\ref{equ_sys_ge}) cover the entire phase plane and do not intersect one another. The only exceptions to this statement occur at points $(x_0, y_0)$ where both $F$ and $G$ vanish :
\begin{equation}
F(x_0, y_0) = 0 ~, ~~{\rm and} ~~ G(x_0, y_0) = 0 
\end{equation}
These points are called  \textcolor{red}{critical points}.










\subsection{Types of Critical Points. Stability}


\subsection{Critical Points and Stability for Linear Systems}


\subsection{Stability by Liapunov's Direct Method}


\subsection{Simple Critical Points of Nonlinear Systems}


\subsection{Nonlinear Mechanics. Conservative Systems}



\subsection{Periodic Solutions. The Poincare-Bendixson Theorem}








\end{document}