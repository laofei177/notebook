\documentclass[11pt,a4paper]{article}
%\usepackage{fontspec, xunicode, xltxtra}  
%\setmainfont{Hiragino Sans GB}  
%\usepackage{xeCJK}
%\setCJKmainfont[BoldFont=STZhongsong, ItalicFont=STKaiti]{STSong}
%\setCJKsansfont[BoldFont=STHeiti]{STXihei}
%\setCJKmonofont{STFangsong}

%使用Xelatex编译

% 设置页面
%==================================================
\linespread{1} %行距
% \usepackage[top=1in,bottom=1in,left=1.25in,right=1.25in]{geometry}
% \headsep=2cm
% \textwidth=16cm \textheight=24.2cm
%==================================================

% 其它需要使用的宏包
%==================================================
\usepackage[colorlinks,linkcolor=blue,anchorcolor=red,citecolor=green,urlcolor=blue]{hyperref} 
\usepackage{tabularx}
\usepackage{authblk}         % 作者信息
\usepackage{algorithm}     % 算法排版
\usepackage{amsmath}     % 数学符号与公式
\usepackage{amsfonts}     % 数学符号与字体
\usepackage{amssymb}
\usepackage[framemethod=TikZ]{mdframed}

\usepackage{graphicx} 
\usepackage{graphics}
\usepackage{color}
\usepackage{xcolor}
\usepackage{tcolorbox}
\usepackage{lipsum}
\usepackage{empheq}

\usepackage{fancyhdr}       % 设置页眉页脚
\usepackage{fancyvrb}       % 抄录环境
\usepackage{float}              % 管理浮动体
\usepackage{geometry}     % 定制页面格式
\usepackage{hyperref}       % 为PDF文档创建超链接
\usepackage{lineno}          % 生成行号
\usepackage{listings}        % 插入程序源代码
\usepackage{multicol}       % 多栏排版
\usepackage{natbib}         % 管理文献引用
\usepackage{rotating}       % 旋转文字,图形,表格
\usepackage{subfigure}    % 排版子图形
\usepackage{titlesec}       % 改变章节标题格式
\usepackage{moresize}   % 更多字体大小
\usepackage{anysize}
\usepackage{indentfirst}  % 首段缩进
\usepackage{booktabs}   % 使用\multicolumn
\usepackage{multirow}    % 使用\multirow

\usepackage{wrapfig}

\usepackage{titlesec}     % 改变标题样式
\usepackage{enumitem}

\renewcommand{\vec}[1]{\boldsymbol{#1}}
\newcommand{\me}{\mathrm{e}}
\newcommand{\mi}{\mathrm{i}}
\newcommand{\dif}{\mathrm{d}}
\newcommand{\tabincell}[2]{\begin{tabular}{@{}#1@{}}#2\end{tabular}}

\def\kpc{{\rm kpc}}
\def\km{{\rm km}}
\def\cm{{\rm cm}}
\def\TeV{{\rm TeV}}
\def\GeV{{\rm GeV}}
\def\MeV{{\rm MeV}}
\def\GV{{\rm GV}}
\def\MV{{\rm MV}}
\def\yr{{\rm yr}}
\def\s{{\rm s}}
\def\ns{{\rm ns}}
\def\GHz{{\rm GHz}}
\def\muGs{{\rm \mu Gs}}
\def\arcsec{{\rm arcsec}}
\def\K{{\rm K}}
\def\microK{\mu{\rm K}}
\def\sr{{\rm sr}}
\newcolumntype{p}{D{,}{\pm}{-1}}

\renewcommand{\figurename}{Fig.}
\renewcommand{\tablename}{Tab.}

\renewcommand{\arraystretch}{1.5}

\newcounter{theo}[section]\setcounter{theo}{0}
\renewcommand{\thetheo}{\arabic{section}.\arabic{theo}}
\newenvironment{theo}[2][]{%
\refstepcounter{theo}%
\ifstrempty{#1}%
{\mdfsetup{%
frametitle={%
\tikz[baseline=(current bounding box.east),outer sep=0pt]
\node[anchor=east,rectangle,fill=blue!20]
{\strut Theorem~\thetheo};}}
}%
{\mdfsetup{%
frametitle={%
\tikz[baseline=(current bounding box.east),outer sep=0pt]
\node[anchor=east,rectangle,fill=blue!20]
{\strut Theorem~\thetheo:~#1};}}%
}%
\mdfsetup{innertopmargin=10pt,linecolor=blue!20,%
linewidth=2pt,topline=true,%
frametitleaboveskip=\dimexpr-\ht\strutbox\relax
}
\begin{mdframed}[]\relax%
\label{#2}}{\end{mdframed}}

\newcommand*\widefbox[1]{\fbox{\hspace{2em}#1\hspace{2em}}}


\title{Second Order Ordinary Differential Equation}
\author{}
\date{\today}
\begin{document}

\maketitle














\section{Second Order Linear Equations}
The general second order linear differential equation is
\begin{equation}
\color{red} \frac{\dif^2 y}{\dif x^2} +P(x) \frac{\dif y}{\dif x} +Q(x)y = R(x) ~,
\label{2nd_ge}
\end{equation}
or
\begin{equation}
y^{\prime\prime} +P(x) y^\prime +Q(x) y = R(x)
\end{equation}

\begin{tcolorbox}[colback=green!15,colframe=green!40!black,title=Theorem A]
Let $P(x)$, $Q(x)$, and $R(x)$ be continuous functions on a closed interval $[a,b]$. If $x_0$ is any point in $[a,b]$, and if $y_0$ and $y^\prime_0$ are any numbers whatever, then equation (\ref{2nd_ge}) has one and only one solution $y(x)$ on the entire interval such that $y(x_0) = y_0$ and $y^\prime(x_0) = y_0$.
\end{tcolorbox}

If $R(x)$ is identically zero, then (\ref{2nd_ge}) reduces to the \textcolor{red}{homogeneous equation}
\begin{equation}
\color{red} \frac{\dif^2 y}{\dif x^2} +P(x) \frac{\dif y}{\dif x} +Q(x)y = 0 ~,
\label{2nd_red}
\end{equation}
If $R(x)$ is not identically zero, then (\ref{2nd_ge}) is said to be \textcolor{red}{nonhomogeneous}.


\begin{tcolorbox}[colback=green!15,colframe=green!40!black,title=Theorem B]
If $y_g$ is the general solution of the reduced equation (\ref{2nd_red}) and $y_p$ is any particular solution of the complete equation (\ref{2nd_ge}), then $y_g + y_p$ is the general solution of (\ref{2nd_ge}).
\end{tcolorbox}



\begin{tcolorbox}[colback=green!15,colframe=green!40!black,title=Theorem C]
If $y_1(x)$ and $y_2(x)$ are any two solutions of (\ref{2nd_red}), then
\begin{equation}
c_1 y_1(x) + c_2 y_2(x)
\label{linear_comb}
\end{equation}
is also a solution for any constants $c_1$ and $c_2$ ~.
\end{tcolorbox}
Any linear combination of two solutions of the homogeneous equation (\ref{2nd_red}) also a solution. The solution ($\ref{linear_comb}$) is commonly called a linear combination of the solutions $y_1(x)$ and $y_2(x)$. 



\subsection{The General Solution of the Homogeneous Equation}
If two functions $f(x)$ and $g(x)$ are defined on an interval $[a, b]$ and have the property that one is a constant multiple of the other, then they are said to be \textcolor{red}{linearly dependent} on $[a, b]$. Otherwise, if neither is a constant multiple of the other, they are called \textcolor{red}{linearly independent}. It is worth noting that if $f(x)$ is identically zero, then $f(x)$ and $g(x)$ are linearly dependent for every function $g(x)$, since $f(x) = 0 \cdot g(x)$.

\begin{tcolorbox}[colback=green!15,colframe=green!40!black,title=Theorem A]
Let $y_1(x)$ and $y_2(x)$ be \textcolor{red}{linearly independent solutions} of the \textcolor{red}{homogeneous equation}
\begin{equation}
y^{\prime\prime} +P(x) y^\prime +Q(x) y = 0
\label{2nd_ho}
\end{equation}
on the interval $[a, b]$. Then
\begin{equation}
c_1 y_1(x) + c_2 y_2(x)
\label{ge_solu}
\end{equation}
is the \textcolor{red}{general solution} of equation (\ref{2nd_ho}) on $[a, b]$, in the sense that every solution of (\ref{2nd_ho}) on this interval can be obtained from (\ref{ge_solu}) by a suitable choice of the arbitrary  constants $c_1$ and $c_2$.
\end{tcolorbox}

The Wronskian of $y_1$ and $y_2$ is
\begin{equation}
W(y_1, y_2) = \left[ \begin{array}{cc}
y_1(x) & y_2(x) \\
y^\prime_1(x) & y^\prime_2(x)
\end{array} \right] = y_1(x) y^\prime_2(x) -y_2(x) y^\prime_1(x)
\end{equation}

\begin{tcolorbox}[colback=green!15,colframe=green!40!black,title=Lemma 1]
If $y_1(x)$ and $y_2(x)$ are any two solutions of equation (\ref{2nd_ho}) on $[a, b]$, then their Wronskian $W = W(y_1, y_2)$ is either identically zero or never zero on $[a, b]$.
\end{tcolorbox}



\begin{tcolorbox}[colback=green!15,colframe=green!40!black,title=Lemma 2]
If $y_1(x)$ and $y_2(x)$ are two solutions of equation (\ref{2nd_ho}) on $[a, b]$, then they are linearly dependent on this interval if and only if their Wronskian $W(y_1, y_2) = y_1y^\prime_2 -y_2y_1^\prime$ is identically zero.
\end{tcolorbox}


\subsection{The Use of A Known Solution to Find Another}
\begin{equation}
y^{\prime\prime} +P(x) y^\prime +Q(x) y = 0
\label{2nd_homo}
\end{equation}
Assume that $y_1(x)$ is a known nonzero solution of (\ref{2nd_homo}), so that $cy_1(x)$ is also a solution for any constant $c$. 


二阶非齐次常微分方程的解,就是相应的齐次微分方程的通解加上非齐次方程的一个特解。
格林函数主要用于求解非齐次方程的特解。

\subsection{The Homogeneous Equation with Constant Coefficients}
If $P(x)$ and $Q(x)$ are constants $p$ and $q$:
\begin{equation}
y^{\prime\prime} + p y^\prime +q y = 0
\end{equation}


\subsection{The Method of Undetermined Coefficients}

\subsection{The Method of Variation of Parameters}


\subsection{Higher Order Linear Equations. Coupled Harmonic Oscillators}



\subsection{Operator Methods for Finding Particular Solutions}



\section{Power Series Solutions and Special Functions}



\section{Fourier Series and Orthogonal Functions}


\section{Some Special Functions of Mathematical Physics}




\section{Laplace Transforms}












\section{The Calculus of Variations}












\end{document}