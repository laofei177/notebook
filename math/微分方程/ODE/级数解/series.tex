\documentclass[12pt,a4paper]{article}
%\usepackage{fontspec, xunicode, xltxtra}  
%\setmainfont{Hiragino Sans GB}  
%\usepackage{xeCJK}
%\setCJKmainfont[BoldFont=STZhongsong, ItalicFont=STKaiti]{STSong}
%\setCJKsansfont[BoldFont=STHeiti]{STXihei}
%\setCJKmonofont{STFangsong}

%使用Xelatex编译

% 设置页面
%==================================================
\linespread{2} %行距
% \usepackage[top=1in,bottom=1in,left=1.25in,right=1.25in]{geometry}
% \headsep=2cm
% \textwidth=16cm \textheight=24.2cm
%==================================================

% 其它需要使用的宏包
%==================================================
\usepackage[colorlinks,linkcolor=blue,anchorcolor=red,citecolor=green,urlcolor=blue]{hyperref} 
\usepackage{tabularx}
\usepackage{authblk}         % 作者信息
\usepackage{algorithm}     % 算法排版
\usepackage{amsmath}     % 数学符号与公式
\usepackage{amsfonts}     % 数学符号与字体
\usepackage{amssymb}
\usepackage[framemethod=TikZ]{mdframed}

\usepackage{graphicx} 
\usepackage{graphics}
\usepackage{color}
\usepackage{xcolor}
\usepackage{tcolorbox}
\usepackage{lipsum}
\usepackage{empheq}

\usepackage{fancyhdr}       % 设置页眉页脚
\usepackage{fancyvrb}       % 抄录环境
\usepackage{float}              % 管理浮动体
\usepackage{geometry}     % 定制页面格式
\usepackage{hyperref}       % 为PDF文档创建超链接
\usepackage{lineno}          % 生成行号
\usepackage{listings}        % 插入程序源代码
\usepackage{multicol}       % 多栏排版
\usepackage{natbib}         % 管理文献引用
\usepackage{rotating}       % 旋转文字,图形,表格
\usepackage{subfigure}    % 排版子图形
\usepackage{titlesec}       % 改变章节标题格式
\usepackage{moresize}   % 更多字体大小
\usepackage{anysize}
\usepackage{indentfirst}  % 首段缩进
\usepackage{booktabs}   % 使用\multicolumn
\usepackage{multirow}    % 使用\multirow

\usepackage{wrapfig}

\usepackage{titlesec}     % 改变标题样式
\usepackage{enumitem}

\renewcommand{\vec}[1]{\boldsymbol{#1}}
\newcommand{\me}{\mathrm{e}}
\newcommand{\mi}{\mathrm{i}}
\newcommand{\dif}{\mathrm{d}}
\newcommand{\tabincell}[2]{\begin{tabular}{@{}#1@{}}#2\end{tabular}}

\def\kpc{{\rm kpc}}
\def\km{{\rm km}}
\def\cm{{\rm cm}}
\def\TeV{{\rm TeV}}
\def\GeV{{\rm GeV}}
\def\MeV{{\rm MeV}}
\def\GV{{\rm GV}}
\def\MV{{\rm MV}}
\def\yr{{\rm yr}}
\def\s{{\rm s}}
\def\ns{{\rm ns}}
\def\GHz{{\rm GHz}}
\def\muGs{{\rm \mu Gs}}
\def\arcsec{{\rm arcsec}}
\def\K{{\rm K}}
\def\microK{\mu{\rm K}}
\def\sr{{\rm sr}}
\newcolumntype{p}{D{,}{\pm}{-1}}

\renewcommand{\figurename}{Fig.}
\renewcommand{\tablename}{Tab.}

\renewcommand{\arraystretch}{1.5}

\newcounter{theo}[section]\setcounter{theo}{0}
\renewcommand{\thetheo}{\arabic{section}.\arabic{theo}}
\newenvironment{theo}[2][]{%
\refstepcounter{theo}%
\ifstrempty{#1}%
{\mdfsetup{%
frametitle={%
\tikz[baseline=(current bounding box.east),outer sep=0pt]
\node[anchor=east,rectangle,fill=blue!20]
{\strut Theorem~\thetheo};}}
}%
{\mdfsetup{%
frametitle={%
\tikz[baseline=(current bounding box.east),outer sep=0pt]
\node[anchor=east,rectangle,fill=blue!20]
{\strut Theorem~\thetheo:~#1};}}%
}%
\mdfsetup{innertopmargin=10pt,linecolor=blue!20,%
linewidth=2pt,topline=true,%
frametitleaboveskip=\dimexpr-\ht\strutbox\relax
}
\begin{mdframed}[]\relax%
\label{#2}}{\end{mdframed}}

\newcommand*\widefbox[1]{\fbox{\hspace{2em}#1\hspace{2em}}}


\title{Power Series Solutions}
\author{}
\date{\today}
\begin{document}

\maketitle










\section{Power Series Solutions and Special Functions}




\subsection{Hermite Polynomials and Quantum Mechanics}








\section{Fourier Series and Orthogonal Functions}









\subsection{Eigenvalues, Eigenfunctions, and the Vibrating String}
Seeking a nontrivial solution $y(x)$ of the equation
\begin{equation}
y^{\prime\prime} +\lambda y = 0 ~,
\label{eq:eigen_prob}
\end{equation}
that satisfies the \textcolor{red}{boundary conditions}
\begin{equation}
y(0) = 0 ~, y(\pi) = 0 ~.
\label{eq:eigen_bound}
\end{equation}
The parameter \textcolor{orange}{$\lambda$} in (\ref{eq:eigen_prob}) is \textcolor{orange}{free to assume any real value whatever}, and part of our task is to \textcolor{red}{discover the $\lambda$'s for which the problem can be solved}. In our previous work we have considered only \textcolor{red}{initial value problems}, in which the solution of a second order equation is sought that \textcolor{blue}{satisfies two conditions at a single value of the independent variable}. Here we have an entirely different situation, for we wish to \textcolor{blue}{satisfy one condition at each of two distinct values of $x$}. Problems of this kind are called \textcolor{red}{boundary value problems}. 

If $\lambda$ is negative, only the trivial solution of (\ref{eq:eigen_prob}) can satisfy (\ref{eq:eigen_bound}); and if $\lambda = 0$, then the general solution of (\ref{eq:eigen_prob}) is $y(x)=c_1 x+ c_2$, and we have the same conclusion. We are thus restricted to the case in which $\lambda$ is positive, where the general solution of (\ref{eq:eigen_prob}) is
\begin{equation*}
y(x) = c_1 \sin \sqrt{\lambda} x + c_2 \cos \sqrt{\lambda} x  ~,
\end{equation*}
and since $y(0)$ must be $0$, this reduces to
\begin{equation}
y(x) = c_1 \sin \sqrt{\lambda} x ~.
\end{equation}
For the second boundary condition $y(\pi)=0$ to be satisfied, it is clear that $\sqrt{\pi}$ must equal $n\pi$ for some positive integer $n$, so $\lambda =n^2$. In other words, $\lambda$ must equal one of the numbers $1, 4, 9, \cdots$. These \textcolor{red}{values of $\lambda$} are called the \textcolor{red}{eigenvalues} of the problem, and \textcolor{red}{corresponding solutions}
\begin{equation}
\sin x ~, \sin 2x ~, \sin 3x ~, \cdots 
\label{eq:eigenfunc}
\end{equation}
are called \textcolor{red}{eigenfunctions}. The \textcolor{yellow}{eigenvalues are uniquely determined by the problem}, but that the \textcolor{yellow}{eigenfunctions are not}; \textcolor{blue}{for any nonzero constant multiples of (\ref{eq:eigenfunc})}, say ~$a_1 \sin x, ~a_2 \sin 2x, ~a_3 \sin 3x, ~\cdots,$ will serve just as well and \textcolor{blue}{are also eigenfunctions}.  The eigenvalues form an increasing sequence of positive numbers that
approaches $\infty$; and the $n$-th eigenfunction, $\sin nx$, vanishes at the endpoints of the interval $[0, \pi]$ and has exactly $n - 1$ zeros inside this interval.










\begin{equation}
a^2 \dfrac{\partial^2 y}{\partial x^2} = \dfrac{\partial^2 y}{\partial t^2} ~,
\label{eq:wave_equ}
\end{equation}
with $a = \sqrt{T/m}$. It is called the one-dimensional wave equation. $y(x,t)$ satisfies the boundary conditions
\begin{align}
y(0, t) = 0 ~, \label{eq:wave_boud_1} \\
y(\pi, t) = 0 ~, \label{eq:wave_boud_2}
\end{align}
and the initial conditions
\begin{equation}
\dfrac{\partial y}{\partial t}\Big|_{t=0} = 0 ~,
\label{eq:wave_init_1}
\end{equation}
and
\begin{equation}
y(x,0) = f(x) ~.
\label{eq:wave_init_2}
\end{equation}
Conditions (\ref{eq:wave_boud_1}) and (\ref{eq:wave_boud_2}) express the assumption that the ends of the string are permanently fixed at the points $x= 0$ and $x=\pi$ and (\ref{eq:wave_init_1}) and (\ref{eq:wave_init_2}) assert that the string is motionless when it is released and that $y=f(x)$ is its shape at that moment.

\textcolor{red}{separation of variables} : $y(x,t) = u(x) v(t) ~,$ which are factorable into a product of functions each of which depends on only one of the independent variables.  
\begin{align}
\nonumber a^2 u^{\prime \prime}(x) v(t) &= u(x) v^{\prime \prime}(t) ~, \\
\dfrac{u^{\prime \prime}(x) }{u(x)} &= \dfrac{1}{a^2} \dfrac{v^{\prime \prime}(t)}{v(t)} ~.
\label{eq:sov}
\end{align}
Since the left side is a function only of $x$ and the right side is a function only of $t$, equation (\ref{eq:sov}) can hold only if both sides are constant. If we denote this constant by $-\lambda$, then (\ref{eq:sov}) splits into two ordinary differential equations for $u(x)$ and $v(t)$:
\begin{align}
u^{\prime \prime} +\lambda u = 0 ~, \label{eq:sov_x} \\
v^{\prime \prime} +\lambda a^2 v = 0 ~. \label{eq:sov_t}
\end{align}
It is possible to satisfy (\ref{eq:wave_boud_1}) and (\ref{eq:wave_boud_2}) by solving (\ref{eq:sov_x}) with the boundary conditions $u(0)=u(\pi)=0$. This problem has a nontrivial solution if and only if $\lambda= n^2$ for some positive integer $n$, and that corresponding solutions (the eigenfunctions) are 
\begin{equation}
u_n(x) = \sin nx ~.
\end{equation}
For these $\lambda$'s (the eigenvalues) the general solution of (\ref{eq:sov_t}) is
\begin{equation}
v(t) = c_1 \sin nat +c_2 \cos nat ~,
\end{equation}
and impose $v^\prime(0)=0$, so that (\ref{eq:wave_init_1}) is satisfied, then $c_1 =0$ and 
\begin{equation}
v_n(t) = \cos nat ~.
\end{equation}
\begin{equation*}
y_n(x,t) =\sin nx \cos nat ~.
\end{equation*}
Each of these functions, for $n = 1, 2, \cdots,$ satisfies equation (\ref{eq:wave_equ}) and conditions (\ref{eq:wave_boud_1}), (\ref{eq:wave_boud_2}), and (\ref{eq:wave_init_1}); and the same is true for any finite
sum of constant multiples of the $y_n$
\begin{equation}
b_1 \sin x \cos at +b_2 \sin 2x \cos 2at +\cdots +b_n \sin nx \cos nat ~.
\end{equation}
If proceeding formally --- that is, ignoring all questions of convergence, term-by-term differentiability, and the like --- then any infinite series of the form
\begin{equation}
\color{orange} y(x,t) = \sum_{n=1}^\infty b_n \sin nx \cos nat = b_1 \sin x \cos at +b_2 \sin 2x \cos 2at +\cdots +b_n \sin nx \cos nat + \cdots ~,
\label{eq:wave_sol}
\end{equation}
is also a solution that satisfies (\ref{eq:wave_boud_1}), (\ref{eq:wave_boud_2}), and (\ref{eq:wave_init_1}). For $t =0$ (\ref{eq:wave_init_2}) should yield the initial shape of the string
\begin{equation}
\color{orange} f(x) = b_1 \sin x+b_2 \sin 2x  +\cdots +b_n \sin nx + \cdots ~.
\label{eq:init_shape}
\end{equation}
The expressions of the form (\ref{eq:init_shape}) are valid for very wide classes of functions $f(x)$ that vanish at $0$ and $\pi$. 

The eigenfunctions $u_m(x)$ and $u_n(x)$ satisfy the equations
\begin{align*}
u^{\prime \prime}_m = -m^2 u_m ~, \\
u^{\prime \prime}_n = -n^2 u_n ~.
\end{align*}
If the first equation is multiplied by $u_n$ and the second by $u_m$, then the difference of the resulting equations is
\begin{align}
\nonumber u_n u^{\prime \prime}_m - u_m u^{\prime \prime}_n = (n^2 -m^2) u_m u_n ~, \\
(u_n u^{\prime}_m - u_m u^{\prime}_n)^\prime = (n^2 -m^2) u_m u_n ~.
\end{align}
On integrating both sides from $0$ to $\pi$ and using the fact that $u_m(x)= \sin mx$ and $u_n(x)= \sin nx$ both vanish at $0$ and $\pi$,
\begin{align}
\nonumber (n^2 -m^2) \int_0^\pi u_m(x) u_n(x) \dif x = [u_n u^{\prime}_m - u_m u^{\prime}_n]_0^\pi = 0 ~, \\
\int_0^\pi \sin mx \sin nx \dif x = 0 ~, ~{\rm when} ~ m\neq n ~.
\label{eq:sin_orthogonal}
\end{align}
Multiplying (\ref{eq:init_shape}) through by $\sin nx$ and integrating the result term by term from $0$ to $\pi$. When these operations are carried out, (\ref{eq:sin_orthogonal}) produces a wholesale disappearance of terms, leaving only
\begin{align}
\nonumber \int_0^\pi f(x) \sin nx \dif x = b_n \int_0^\pi \sin^2 nx \dif x ~, \\
b_n = \dfrac{2}{\pi} \int_0^\pi f(x) \sin nx \dif x ~.
\label{eq:b_n}
\end{align}
\textcolor{violet}{$b_n$} are called the \textcolor{violet}{Fourier coefficients of $f(x)$}. With these coefficients, (\ref{eq:init_shape}) is the \textcolor{violet}{Fourier sine series of $f(x)$} or the \textcolor{violet}{eigenfunction expansion of $f(x)$ in terms of the eigenfunctions $\sin nx$}, and (\ref{eq:wave_sol}) is called \textcolor{violet}{Bernoulli's solution of the wave equation}.

The function $f(x)$ under consideration is defined on the interval $[0,\pi]$ and vanishes at the endpoints. Suppose that \textcolor{blue}{$f(x)$ is continuous on the entire interval}, and also that \textcolor{blue}{its derivative is continuous with the possible exception of a finite number of jump discontinuities}, where the \textcolor{blue}{derivative approaches finite but different limits from the left and from the right}. In geometric language, the graph of such a function is \textcolor{blue}{a continuous curve with the property that the direction of the tangent changes continuously as it moves along the curve, except possibly at a finite number of ``corners" where its direction changes abruptly}. Under these hypotheses the expansion (\ref{eq:init_shape}) is valid; that is, if the $b_n$ are defined by (\ref{eq:b_n}), then the \textcolor{blue}{series on the right converges at every point to the value of the function at that point}. The need for a carefully constructed theory can be seen from the fact that \textcolor{yellow}{if $f(x)$ is merely assumed to be continuous, and nothing is said about its derivative, then it is known to be possible for the series on the right of (\ref{eq:init_shape}) to diverge at some points}.\footnote{\textcolor{yellow}{There even exists a continuous function whose Fourier series diverges at every rational point in $[0,\pi]$}.}

Considers the possibility of eigenfunction expansions like (18) for other boundary value problems. If putting aside the
issue of the validity of such expansions, the main problem becomes that of showing in other cases that we have an adequate supply of suitable building materials, i.e., a sequence of eigenvalues with corresponding eigenfunctions that satisfy some condition similar to (20).

Consider the vibrating string: the string is nonhomogeneous, its density $m = m(x)$ may vary from point to point. In this situation, (\ref{eq:wave_equ}) is replaced by
\begin{equation}
\dfrac{\partial^2 y}{\partial x^2} = \dfrac{m(x)}{T} \dfrac{\partial^2 y}{\partial t^2} ~.
\label{eq:vibrating_string_m(x)}
\end{equation}
Equ. (\ref{eq:vibrating_string_m(x)}) becomes 
\begin{equation*}
\dfrac{u^{\prime \prime}(x) }{m(x) u(x)} = \dfrac{1}{T} \dfrac{v^{\prime \prime}(t)}{v(t)} ~,
\end{equation*}
and we are led to the following boundary value problem:
\begin{align}
u^{\prime \prime} +\lambda m(x) u =0 ~, \\
u(0) = u(\pi) = 0 ~.
\end{align}





















\section{Some Special Functions of Mathematical Physics}








































\section{Laplace Transforms}






















\end{document}