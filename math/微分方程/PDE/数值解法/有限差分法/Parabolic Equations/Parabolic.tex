\documentclass[12pt,a4paper]{article}
%\usepackage{fontspec, xunicode, xltxtra}  
%\setmainfont{Hiragino Sans GB}  
%\usepackage{xeCJK}
%\setCJKmainfont[BoldFont=STZhongsong, ItalicFont=STKaiti]{STSong}
%\setCJKsansfont[BoldFont=STHeiti]{STXihei}
%\setCJKmonofont{STFangsong}

%使用Xelatex编译

% 设置页面
%==================================================
\linespread{2} %行距
% \usepackage[top=1in,bottom=1in,left=1.25in,right=1.25in]{geometry}
% \headsep=2cm
% \textwidth=16cm \textheight=24.2cm
%==================================================

% 其它需要使用的宏包
%==================================================
\usepackage[colorlinks,linkcolor=blue,anchorcolor=red,citecolor=green,urlcolor=blue]{hyperref} 
\usepackage{tabularx}
\usepackage{authblk}         % 作者信息
\usepackage{algorithm}     % 算法排版
\usepackage{amsmath}     % 数学符号与公式
\usepackage{amsfonts}     % 数学符号与字体
\usepackage{amssymb}
\usepackage{mathrsfs}
\usepackage[framemethod=TikZ]{mdframed}

\usepackage{graphicx} 
\usepackage{graphics}
\usepackage{color}
\usepackage{xcolor}
\usepackage{tcolorbox}
\usepackage{lipsum}
\usepackage{empheq}

\usepackage{fancyhdr}       % 设置页眉页脚
\usepackage{fancyvrb}       % 抄录环境
\usepackage{float}              % 管理浮动体
\usepackage{geometry}     % 定制页面格式
\usepackage{hyperref}       % 为PDF文档创建超链接
\usepackage{lineno}          % 生成行号
\usepackage{listings}        % 插入程序源代码
\usepackage{multicol}       % 多栏排版
%\usepackage{natbib}         % 管理文献引用
\usepackage{rotating}       % 旋转文字,图形,表格
\usepackage{subfigure}    % 排版子图形
\usepackage{titlesec}       % 改变章节标题格式
\usepackage{moresize}   % 更多字体大小
\usepackage{anysize}
\usepackage{indentfirst}  % 首段缩进
\usepackage{booktabs}   % 使用\multicolumn
\usepackage{multirow}    % 使用\multirow
\usepackage{wrapfig}

\usepackage{titlesec}     % 改变标题样式
\usepackage{enumitem}

\renewcommand{\vec}[1]{\boldsymbol{#1}}
\newcommand{\me}{\mathrm{e}}
\newcommand{\mi}{\mathrm{i}}
\newcommand{\dif}{\mathrm{d}}
\newcommand{\tabincell}[2]{\begin{tabular}{@{}#1@{}}#2\end{tabular}}

\def\kpc{{\rm kpc}}
\def\km{{\rm km}}
\def\cm{{\rm cm}}
\def\TeV{{\rm TeV}}
\def\GeV{{\rm GeV}}
\def\MeV{{\rm MeV}}
\def\GV{{\rm GV}}
\def\MV{{\rm MV}}
\def\yr{{\rm yr}}
\def\s{{\rm s}}
\def\ns{{\rm ns}}
\def\GHz{{\rm GHz}}
\def\muGs{{\rm \mu Gs}}
\def\arcsec{{\rm arcsec}}
\def\K{{\rm K}}
\def\microK{\mu{\rm K}}
\def\sr{{\rm sr}}
\newcolumntype{p}{D{,}{\pm}{-1}}

\renewcommand{\figurename}{Fig.}
\renewcommand{\tablename}{Tab.}

\renewcommand{\arraystretch}{1.5}

\newcounter{theo}[section]\setcounter{theo}{0}
\renewcommand{\thetheo}{\arabic{section}.\arabic{theo}}
\newenvironment{theo}[2][]{%
\refstepcounter{theo}%
\ifstrempty{#1}%
{\mdfsetup{%
frametitle={%
\tikz[baseline=(current bounding box.east),outer sep=0pt]
\node[anchor=east,rectangle,fill=blue!20]
{\strut Theorem~\thetheo};}}
}%
{\mdfsetup{%
frametitle={%
\tikz[baseline=(current bounding box.east),outer sep=0pt]
\node[anchor=east,rectangle,fill=blue!20]
{\strut Theorem~\thetheo:~#1};}}%
}%
\mdfsetup{innertopmargin=10pt,linecolor=blue!20,%
linewidth=2pt,topline=true,%
frametitleaboveskip=\dimexpr-\ht\strutbox\relax
}
\begin{mdframed}[]\relax%
\label{#2}}{\end{mdframed}}

\newcommand*\widefbox[1]{\fbox{\hspace{2em}#1\hspace{2em}}}


\title{Parabolic Equations}
\author{}
\date{\today}
\begin{document}

\maketitle










\subsection{Three Dimensional Schemes}
Consider the partial differential equation
\begin{equation}
v_t = \nu \nabla^2 v +F(x,y,z,t) ~,
\end{equation}
and the obvious FTCS explicit scheme for approximating the solution of equation is
\begin{equation}
u_{jkl}^{n+1} = u_{jkl}^{n} +(r_x \delta_x^2 +r_y \delta_y^2 +r_z \delta_z^2) u_{jkl}^{n} +F_{jkl}^{n} ~.
\end{equation}
The difference scheme is a $\mathcal O(\Delta t) + O(\Delta x^2) + O(\Delta y^2) + O(\Delta z^2)$ order approximation of partial differential equation. And the difference scheme is conditionally stable, with stability condition $r_x+r_y+r_z \leqslant 1/2$. The three dimensional BTCS scheme
\begin{equation}
u_{jkl}^{n+1} -(r_x \delta_x^2 +r_y \delta_y^2 +r_z \delta_z^2) u_{jkl}^{n+1} = u_{jkl}^{n} +F_{jkl}^{n} ~.
\end{equation}
and the three dimensional Crank-Nicolson scheme
\begin{equation}
u_{jkl}^{n+1} -\dfrac{1}{2} (r_x \delta_x^2 +r_y \delta_y^2 +r_z \delta_z^2) u_{jkl}^{n+1} = u_{jkl}^{n} +\dfrac{1}{2} (r_x \delta_x^2 +r_y \delta_y^2 +r_z \delta_z^2) u_{jkl}^{n} +\dfrac{1}{2} \left( F_{jkl}^{n} +F_{jkl}^{n+1} \right)
\end{equation}
are both unconditionally stable schemes which are $\mathcal O(\Delta t) + O(\Delta x^2) + O(\Delta y^2) + O(\Delta z^2)$ and $\mathcal O(\Delta t^2) + O(\Delta x^2) + O(\Delta y^2) + O(\Delta z^2)$, respectively. One approach is to use three dimensional Peaceman-Rachford scheme (along with the rationalization of using $\Delta t/3$ as time steps). But the \textcolor{red}{three dimensional Peaceman-Rachford scheme is not unconditionally stable}. And the three dimensional Peaceman-Rachford scheme is
only first order accurate in time. 

Factor the left hand side of equation of three dimensional Crank-Nicolson scheme
\begin{equation}
\left(1 - \dfrac{r_x}{2} \delta_x^2 \right) \left(1 - \dfrac{r_y}{2} \delta_y^2 \right) \left(1 - \dfrac{r_z}{2} \delta_z^2 \right) u_{jkl}^{n+1} ~,
\end{equation}
If add
\begin{equation}
\left[ \dfrac{r_x r_y}{4} \delta_x^2 \delta_y^2 +\dfrac{r_x r_z}{4} \delta_x^2 \delta_z^2 +\dfrac{r_y r_z}{4} \delta_y^2 \delta_z^2 \right] \left(u_{jkl}^{n+1} -u_{jkl}^{n} \right) -\dfrac{r_x r_y r_z}{8} \delta_x^2 \delta_y^2 \delta_z^2 \left(u_{jkl}^{n+1} +u_{jkl}^{n} \right) ~,
\end{equation}
to the left hand side of equation.
\begin{align}
\nonumber & \left(1 - \dfrac{r_x}{2} \delta_x^2 \right) \left(1 - \dfrac{r_y}{2} \delta_y^2 \right) \left(1 - \dfrac{r_z}{2} \delta_z^2 \right) u_{jkl}^{n+1}  = \\
& \left(1+\dfrac{r_x}{2} \delta_x^2 \right) \left(1+\dfrac{r_y}{2} \delta_y^2 \right) \left(1+\dfrac{r_z}{2} \delta_z^2 \right) u_{jkl}^{n} +\dfrac{1}{2} \left( F_{jkl}^{n} +F_{jkl}^{n+1} \right) ~.
\end{align}
It is equivalent to
\begin{align}
\nonumber &\left(1-\dfrac{r_x}{2} \delta_x^2 \right) \left(1-\dfrac{r_y}{2} \delta_y^2 \right) \left(1-\dfrac{r_z}{2} \delta_z^2 \right) \left(u_{jkl}^{n+1} -u_{jkl}^{n} \right) = \\
& \left(r_x \delta_x^2+r_y \delta_y^2+ r_z\delta_z^2 \right) u_{jkl}^{n} +\dfrac{r_x r_y r_z}{4} \delta_x^2 \delta_y^2 \delta_z^2 u_{jkl}^{n} +\dfrac{1}{2} \left( F_{jkl}^{n} +F_{jkl}^{n+1} \right) ~.
\end{align}
Drop the $ \delta_x^2 \delta_y^2 \delta_z^2$ term and get the following form of the Douglas-Gunn scheme
\begin{align}
& \left(1-\dfrac{r_x}{2} \delta_x^2 \right) \Delta u^\ast = \left(r_x \delta_x^2+r_y \delta_y^2+ r_z\delta_z^2 \right) u_{jkl}^{n} +\dfrac{1}{2} \left( F_{jkl}^{n} +F_{jkl}^{n+1} \right) ~, \\
& \left(1-\dfrac{r_y}{2} \delta_y^2 \right) \Delta u^{\ast\ast} = \Delta u^\ast ~, \\
& \left(1-\dfrac{r_z}{2} \delta_z^2 \right) \Delta u = \Delta u^{\ast\ast} ~, \\
& \Delta u = u_{jkl}^{n+1} -u_{jkl}^{n}
\end{align}
The Douglas-Gunn scheme is accurate of order of $\mathcal O(\Delta t^2) + O(\Delta x^2) + O(\Delta y^2) + O(\Delta z^2)$. The discrete Fourier transforms of the nonhomogeneous version of equations are
\begin{align}
& \left(1+2 r_x \sin^2 \dfrac{\xi}{2} \right) \widehat{\Delta u^\ast} = \left(-4r_x \sin^2 \dfrac{\xi}{2} -4r_y \sin^2 \dfrac{\eta}{2} -4r_z   \sin^2 \dfrac{\zeta}{2} \right) \hat{u}^{n}  ~, \\
& \left(1+2 r_y \sin^2 \dfrac{\eta}{2} \right) \widehat{\Delta u^{\ast\ast}} = \widehat{\Delta u^\ast} ~, \\
& \left(1+2 r_z \sin^2 \dfrac{\zeta}{2}  \right) \widehat{\Delta u} =  \widehat{\Delta u^{\ast\ast} } ~, \\
&  \widehat{\Delta u} =  \widehat{u}^{n+1} - \widehat{u}^{n} ~,
\end{align}
then
\begin{align}
\nonumber  \widehat{u}^{n+1} = & \left[1 -2 r_x \sin^2 \dfrac{\xi}{2} -2 r_y \sin^2 \dfrac{\eta}{2} -2 r_z \sin^2 \dfrac{\zeta}{2} +4r_x r_y \sin^2 \dfrac{\xi}{2} \sin^2 \dfrac{\eta}{2} \right. \\
\nonumber  & \left. +4r_x r_z \sin^2 \dfrac{\xi}{2} \sin^2 \dfrac{\eta}{2} + 4r_y r_z \sin^2 \dfrac{\eta}{2} \sin^2 \dfrac{\zeta}{2} +8 r_xr_yr_z \sin^2 \dfrac{\xi}{2} \sin^2 \dfrac{\eta}{2} \sin^2 \dfrac{\zeta}{2} \right] / \\
& \left[\left(1+2 r_x \sin^2 \dfrac{\xi}{2} \right) \left(1+2 r_y \sin^2 \dfrac{\eta}{2} \right) \left(1+2 r_z \sin^2 \dfrac{\zeta}{2} \right) \right] \widehat{u}^{n} = \rho(\xi, \eta, \zeta) \widehat{u}^{n} ~.
\end{align}

The above expression is in the general form
\begin{equation*}
\dfrac{1-a-b-c+d+e+f+g}{1+a+b+c+d+e+f+g} ~,
\end{equation*}
where $a, \cdots ,g$ are all positive and it is easy to see that
\begin{equation*}
-1 \leqslant \dfrac{1-a-b-c+d+e+f+g}{1+a+b+c+d+e+f+g} \leqslant 1 ~.
\end{equation*}
Likewise, $|\rho(\xi, \eta, \zeta)| \leqslant 1$. Hence, the Douglas-Gunn scheme is unconditionally stable. Since it is both consistent and unconditionally stable, the Douglas-Gunn scheme is convergent.















































































































































%%%%%%%%%%%%%%%%%%%%%%%%%%%%%%%%%%%%%%%%%%%%%%%%%%%%%%%%%%%%%%%%%%%%%%
\bibliographystyle{unsrt_update}
\bibliography{ref}
%%%%%%%%%%%%%%%%%%%%%%%%%%%%%%%%%%%%%%%%%%%%%%%%%%%%%%%%%%%%%%%%%%%%%%

\end{document}