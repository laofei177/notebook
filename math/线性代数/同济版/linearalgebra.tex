\documentclass[12pt,a4paper]{article}
%\usepackage{fontspec, xunicode, xltxtra}
%\setmainfont{Hiragino Sans GB}
\usepackage{xeCJK}
%\setCJKmainfont[BoldFont=STZhongsong, ItalicFont=STKaiti]{STSong}
%\setCJKsansfont[BoldFont=STHeiti]{STXihei}
%\setCJKmonofont{STFangsong}

%使用Xelatex编译

% 设置页面
%==================================================
\linespread{2} %行距
% \usepackage[top=1in,bottom=1in,left=1.25in,right=1.25in]{geometry}
% \headsep=2cm
% \textwidth=16cm \textheight=24.2cm
%==================================================

% 其它需要使用的宏包
%==================================================
\usepackage[colorlinks,linkcolor=blue,anchorcolor=red,citecolor=green,urlcolor=blue]{hyperref} 
\usepackage{tabularx}
\usepackage{authblk}         % 作者信息
\usepackage{algorithm}     % 算法排版
\usepackage{amsmath}     % 数学符号与公式
\usepackage{amsfonts}     % 数学符号与字体
\usepackage{mathrsfs}      % 花体
\usepackage[framemethod=TikZ]{mdframed}
\usepackage{amssymb}

\usepackage{graphicx} 
\usepackage{graphics}
\usepackage{color}
\usepackage{xcolor}
\usepackage{tcolorbox}
\usepackage{lipsum}
\usepackage{empheq}

\usepackage{fancyhdr}       % 设置页眉页脚
\usepackage{fancyvrb}       % 抄录环境
\usepackage{float}              % 管理浮动体
\usepackage{geometry}     % 定制页面格式
\usepackage{hyperref}       % 为PDF文档创建超链接
\usepackage{lineno}          % 生成行号
\usepackage{listings}        % 插入程序源代码
\usepackage{multicol}       % 多栏排版
%\usepackage{natbib}         % 管理文献引用
\usepackage{rotating}       % 旋转文字,图形,表格
\usepackage{subfigure}    % 排版子图形
\usepackage{titlesec}       % 改变章节标题格式
\usepackage{moresize}   % 更多字体大小
\usepackage{anysize}
\usepackage{indentfirst}  % 首段缩进
\usepackage{booktabs}   % 使用\multicolumn
\usepackage{multirow}    % 使用\multirow

\usepackage{wrapfig}
\usepackage{titlesec}     % 改变标题样式
\usepackage{enumitem}
\usepackage{aas_macros}

\renewcommand{\vec}[1]{\boldsymbol{#1}}
\newcommand{\me}{\mathrm{e}}
\newcommand{\mi}{\mathrm{i}}
\newcommand{\dif}{\mathrm{d}}
\newcommand{\tabincell}[2]{\begin{tabular}{@{}#1@{}}#2\end{tabular}}

\def\kpc{{\rm kpc}}
\def\km{{\rm km}}
\def\cm{{\rm cm}}
\def\TeV{{\rm TeV}}
\def\GeV{{\rm GeV}}
\def\MeV{{\rm MeV}}
\def\GV{{\rm GV}}
\def\MV{{\rm MV}}
\def\yr{{\rm yr}}
\def\s{{\rm s}}
\def\ns{{\rm ns}}
\def\GHz{{\rm GHz}}
\def\muGs{{\rm \mu Gs}}
\def\arcsec{{\rm arcsec}}
\def\K{{\rm K}}
\def\microK{\mu{\rm K}}
\def\sr{{\rm sr}}
\newcolumntype{p}{D{,}{\pm}{-1}}

\renewcommand{\figurename}{Fig.}
\renewcommand{\tablename}{Tab.}

\renewcommand{\arraystretch}{1.5}

\setlength{\parindent}{0pt}  %取消每段开头的空格

\newcounter{theo}[section]\setcounter{theo}{0}
\renewcommand{\thetheo}{\arabic{section}.\arabic{theo}}
\newenvironment{theo}[2][]{%
\refstepcounter{theo}%
\ifstrempty{#1}%
{\mdfsetup{%
frametitle={%
\tikz[baseline=(current bounding box.east),outer sep=0pt]
\node[anchor=east,rectangle,fill=blue!20]
{\strut Theorem~\thetheo};}}
}%
{\mdfsetup{%
frametitle={%
\tikz[baseline=(current bounding box.east),outer sep=0pt]
\node[anchor=east,rectangle,fill=blue!20]
{\strut Theorem~\thetheo:~#1};}}%
}%
\mdfsetup{innertopmargin=10pt,linecolor=blue!20,%
linewidth=2pt,topline=true,%
frametitleaboveskip=\dimexpr-\ht\strutbox\relax
}
\begin{mdframed}[]\relax%
\label{#2}}{\end{mdframed}}

\newcommand*\widefbox[1]{\fbox{\hspace{2em}#1\hspace{2em}}}

\title{线性代数}
\author{}
\date{\today}
\begin{document}

\maketitle







\section{矩阵及其运算}
\subsection{矩阵}
由$m\times n$个数$a_{ij}(i = 1, 2, \cdots, m; j = 1, 2, \cdots, n)$排成的$m$行$n$列的数表
\begin{table}[htp]
\begin{center}
\renewcommand{\arraystretch}{0.7}
\begin{tabular}{cccc}
$a_{11}$ & $a_{11}$ & $\cdots$ & $a_{1n}$ \\
$a_{21}$ & $a_{22}$ & $\cdots$ & $a_{2n}$ \\
$\vdots$   & $\vdots$  &            & $\vdots$ \\
$a_{m1}$ & $a_{m1}$ & $\cdots$ & $a_{mn}$
\end{tabular}
\end{center}
\label{default}
\end{table}%
称为$m$行$n$列矩阵,简称$m\times n$矩阵,记作
\begin{equation}
A = 
\renewcommand{\arraystretch}{0.7}
\begin{pmatrix}
a_{11} & a_{11} & \cdots & a_{1n} \\
a_{21} & a_{22} & \cdots & a_{2n} \\
\vdots   & \vdots  &            & \vdots \\
a_{m1} & a_{m1} & \cdots & a_{mn}
\end{pmatrix}
\end{equation}
$m\times n$个数称为矩阵$A$的元素,简称元,数$a_{ij}$位于矩阵$A$的第$i$行第$j$列,称为矩阵$A$的$(i, j)$元。以数$a_{ij}$为$(i, j)$元的矩阵简记为$(a_{ij})$或$(a_{ij})_{m\times n}$。$m\times n$矩阵$A$也记为$A_{m\times n}$。行数与列数都等于$n$的矩阵称为$n$阶矩阵或$n$阶方阵。$n$阶矩阵$A$也记作$A_n$。

对角矩阵(对角阵),记作$\Lambda = \text{diag}(\lambda_1, \lambda_2, \cdots, \lambda_n)$。




元素都是$0$的矩阵称为零矩阵,记作$\vec{0}$。





\subsection{矩阵的运算}
\subsubsection{矩阵的加法}
有两个$m\times n$矩阵$A = (a_{ij})$和$B = (b_{ij})$,矩阵$A$与$B$的和记作$A+B$,
\begin{equation*}
A + B = 
\renewcommand{\arraystretch}{0.7}
\begin{pmatrix}
a_{11} +b_{11} & a_{11} +b_{12} & \cdots & a_{1n} +b_{1n} \\
a_{21} +b_{21} & a_{22} +b_{22} & \cdots & a_{2n} +b_{2n}\\
\vdots   & \vdots  &            & \vdots \\
a_{m1} +b_{m1} & a_{m1} +b_{m2} & \cdots & a_{mn} ++b_{mn}
\end{pmatrix}
\end{equation*}
当两个矩阵是同型矩阵时,这两个矩阵才能进行加法运算。设$A, B, C$都是$m\times n$矩阵
\begin{align*}
& A + B = B +A ~; \\
& (A + B) +C = A + (B +C) ~. 
\end{align*}




\subsubsection{数与矩阵相乘}
数$\lambda$与矩阵$A$的乘积记作$\lambda A$或$A \lambda$,
\begin{equation*}
\lambda A = A \lambda =
\renewcommand{\arraystretch}{0.7}
\begin{pmatrix}
 \lambda a_{11} &  \lambda a_{11} & \cdots &  \lambda a_{1n} \\
 \lambda a_{21} &  \lambda a_{22} & \cdots &  \lambda a_{2n} \\
\vdots   & \vdots  &            & \vdots \\
 \lambda a_{m1} &  \lambda a_{m1} & \cdots &  \lambda a_{mn}
\end{pmatrix}
\end{equation*}



\subsubsection{矩阵与矩阵相乘}







\subsubsection{矩阵的转置}
把矩阵$A$的行换成同序数的列得到的矩阵,叫做$A$的转置矩阵,记作$A^T$。








设$A$为$n$阶方阵,若满足$A^T = A$,即
\begin{equation*}
a_{ij} = a_{ji} ~~(i , j = 1, 2, \cdots, n) ~,
\end{equation*}
$A$称为对称矩阵,简称对称阵,其元素以对角线为对称轴对应相等。




\subsubsection{方阵的行列式}
由$n$阶方阵$A$的元素所构成的行列式(各元素的位置不变),称为方阵$A$的行列式,记作$|A|$或$\text{det} A$。


行列式$|A|$的各个元素的代数余子式$A_{ij}$所构成的矩阵
\begin{equation*}
\color{red} A^\ast = 
\renewcommand{\arraystretch}{0.7}
\begin{pmatrix}
A_{11} & A_{21} & \cdots & A_{n1} \\
A_{12} & a_{22} & \cdots & A_{n2} \\
\vdots   & \vdots  &            & \vdots \\
A_{1n} & a_{2n} & \cdots & A_{nn}
\end{pmatrix}
\end{equation*}
称为矩阵$A$的\textcolor{red}{伴随矩阵},简称\textcolor{red}{伴随阵}。

\subsubsection{共轭矩阵}
当$A = (a_{ij})$为复矩阵时,$\bar{a}_{ij}$表示$a_{ij}$的共轭复数,记为
\begin{equation*}
\bar{A} = (\bar{a}_{ij}) ~,
\end{equation*}
$\bar{A}$称为$A$的\textcolor{red}{共轭矩阵}。设$A, B$为复矩阵,$\lambda$为复数,
\begin{align*}
& \overline{A +B} = \overline{A} +\overline{B} ~, \\
& \overline{\lambda A} = \overline{\lambda}~ \overline{A} ~, \\
& \overline{AB} =  \overline{A}~  \overline{B} 
\end{align*}

\subsection{逆矩阵}
\begin{tcolorbox}[colback=green!5,colframe=green!40!black,title= 定义]
对于$n$阶矩阵$A$,若有一个$n$阶矩阵$B$,使得
$\vec{A B} = \vec{B A} = \vec{E}$,矩阵$\vec{A}$是可逆的,矩阵$\vec{B}$称为$\vec{A}$的\textcolor{red}{逆矩阵},简称逆阵。
\end{tcolorbox}


\begin{tcolorbox}[colback=green!5,colframe=green!40!black,title= 定理]
若矩阵$A$可逆,则$|A| \neq 0$。
\end{tcolorbox}



\begin{tcolorbox}[colback=green!5,colframe=green!40!black,title= 定理]
若$|A| \neq 0$,则矩阵$\vec{A}$可逆,且
\begin{equation}
\vec{A}^{-1} = \dfrac{1}{|A|} \vec{A}^\ast ~,
\end{equation}
其中$\vec{A}^\ast$为矩阵$\vec{A}$的\textcolor{red}{伴随阵}。
\end{tcolorbox}





































\end{document}
