\documentclass[12pt,a4paper]{article}
%\usepackage{fontspec, xunicode, xltxtra}
%\setmainfont{Hiragino Sans GB}
%\usepackage{xeCJK}
%\setCJKmainfont[BoldFont=STZhongsong, ItalicFont=STKaiti]{STSong}
%\setCJKsansfont[BoldFont=STHeiti]{STXihei}
%\setCJKmonofont{STFangsong}

%使用Xelatex编译

% 设置页面
%==================================================
\linespread{2} %行距
% \usepackage[top=1in,bottom=1in,left=1.25in,right=1.25in]{geometry}
% \headsep=2cm
% \textwidth=16cm \textheight=24.2cm
%==================================================

% 其它需要使用的宏包
%==================================================
\usepackage[colorlinks,linkcolor=blue,anchorcolor=red,citecolor=green,urlcolor=blue]{hyperref} 
\usepackage{tabularx}
\usepackage{authblk}         % 作者信息
\usepackage{algorithm}     % 算法排版
\usepackage{amsmath}     % 数学符号与公式
\usepackage{amsfonts}     % 数学符号与字体
\usepackage{mathrsfs}      % 花体
\usepackage[framemethod=TikZ]{mdframed}
\usepackage{amssymb}

\usepackage{graphicx} 
\usepackage{graphics}
\usepackage{color}
\usepackage{xcolor}
\usepackage{tcolorbox}
\usepackage{lipsum}
\usepackage{empheq}

\usepackage{fancyhdr}       % 设置页眉页脚
\usepackage{fancyvrb}       % 抄录环境
\usepackage{float}              % 管理浮动体
\usepackage{geometry}     % 定制页面格式
\usepackage{hyperref}       % 为PDF文档创建超链接
\usepackage{lineno}          % 生成行号
\usepackage{listings}        % 插入程序源代码
\usepackage{multicol}       % 多栏排版
%\usepackage{natbib}         % 管理文献引用
\usepackage{rotating}       % 旋转文字,图形,表格
\usepackage{subfigure}    % 排版子图形
\usepackage{titlesec}       % 改变章节标题格式
\usepackage{moresize}   % 更多字体大小
\usepackage{anysize}
\usepackage{indentfirst}  % 首段缩进
\usepackage{booktabs}   % 使用\multicolumn
\usepackage{multirow}    % 使用\multirow

\usepackage{wrapfig}
\usepackage{titlesec}     % 改变标题样式
\usepackage{enumitem}
\usepackage{aas_macros}


\newcommand{\myvec}[1]%
   {\stackrel{\raisebox{-2pt}[0pt][0pt]{\small$\rightharpoonup$}}{#1}}  %矢量符号
\renewcommand{\vec}[1]{\boldsymbol{#1}}
\newcommand{\me}{\mathrm{e}}
\newcommand{\mi}{\mathrm{i}}
\newcommand{\dif}{\mathrm{d}}
\newcommand{\tabincell}[2]{\begin{tabular}{@{}#1@{}}#2\end{tabular}}

\def\kpc{{\rm kpc}}
\def\km{{\rm km}}
\def\cm{{\rm cm}}
\def\TeV{{\rm TeV}}
\def\GeV{{\rm GeV}}
\def\MeV{{\rm MeV}}
\def\GV{{\rm GV}}
\def\MV{{\rm MV}}
\def\yr{{\rm yr}}
\def\s{{\rm s}}
\def\ns{{\rm ns}}
\def\GHz{{\rm GHz}}
\def\muGs{{\rm \mu Gs}}
\def\arcsec{{\rm arcsec}}
\def\K{{\rm K}}
\def\microK{\mu{\rm K}}
\def\sr{{\rm sr}}
\newcolumntype{p}{D{,}{\pm}{-1}}

\renewcommand{\figurename}{Fig.}
\renewcommand{\tablename}{Tab.}

\renewcommand{\arraystretch}{1.5}

\newtheorem{theorem}{THEOREM}[section]

\setlength{\parindent}{0pt}  %取消每段开头的空格

\newcounter{theo}[section]\setcounter{theo}{0}
\renewcommand{\thetheo}{\arabic{section}.\arabic{theo}}
\newenvironment{theo}[2][]{%
\refstepcounter{theo}%
\ifstrempty{#1}%
{\mdfsetup{%
frametitle={%
\tikz[baseline=(current bounding box.east),outer sep=0pt]
\node[anchor=east,rectangle,fill=blue!20]
{\strut Theorem~\thetheo};}}
}%
{\mdfsetup{%
frametitle={%
\tikz[baseline=(current bounding box.east),outer sep=0pt]
\node[anchor=east,rectangle,fill=blue!20]
{\strut Theorem~\thetheo:~#1};}}%
}%
\mdfsetup{innertopmargin=10pt,linecolor=blue!20,%
linewidth=2pt,topline=true,%
frametitleaboveskip=\dimexpr-\ht\strutbox\relax
}
\begin{mdframed}[]\relax%
\label{#2}}{\end{mdframed}}

\newcommand*\widefbox[1]{\fbox{\hspace{2em}#1\hspace{2em}}}


\title{The Geometry of Vector Spaces}
\author{}
\date{\today}
\begin{document}

\maketitle





\section{Null Spaces, Column Spaces, and Linear Transformations}
The subspaces of $\mathbb{R}^n$ usually arise in one of two ways: (1) as the set of all solutions to a system of homogeneous linear equations or (2) as the set of all linear combinations of certain specified vectors.



\subsection{The Null Space of a Matrix}
The set of $\vec{x}$ that satisfy $A \vec{x} = \vec{0}$ is called the \textcolor{red}{null space} of the matrix $A$.

\begin{tcolorbox}[colback=green!5,colframe=green!40!black,title= Definition]
The null space of an $m\times n$ matrix $A$, written as \textcolor{red}{Nul $A$}, is the set of all solutions to the homogeneous equation $A \vec{x} = \vec{0}$. In set notation, \\
Nul A = $\{\vec{x} : \vec{x} ~\text{is in} ~\mathbb{R}^n ~\text{and} ~A \vec{x} = \vec{0}\}$
\end{tcolorbox}
A more dynamic description of Nul $A$ is the set of all $\vec{x}$ in $\mathbb{R}^n$ that are mapped into the zero vector of $\mathbb{R}^m$ via the linear transformation $\vec{x} \mapsto A\vec{x}$. 

\begin{tcolorbox}[colback=green!5,colframe=green!40!black,title= Theorem]
The null space of an $m\times n$ matrix $A$ is a subspace of $\mathbb{R}^n$. Equivalently, the set of all solutions to a system $A\vec{x} = \vec{0}$ of $m$ homogeneous linear equations in $n$ unknowns is a subspace of $\mathbb{R}^n$.
\end{tcolorbox}

\subsection{An Explicit Description of Nul $A$}
There is no obvious relation between vectors in Nul $A$ and the entries in $A$. Nul $A$ is defined implicitly, because it is defined by a condition that must be checked. No explicit list or description of the elements in Nul $A$ is given. However, solving
the equation $A \vec{x} = \vec{0}$ amounts to producing an explicit description of Nul $A$.



When Nul $A$ contains nonzero vectors, the number of vectors in the spanning set for Nul $A$ equals the number of free variables in the equation $A \vec{x} = \vec{0}$.



\subsection{The Column Space of a Matrix}

\begin{tcolorbox}[colback=green!5,colframe=green!40!black,title= Definition]
The column space of an $m\times n$ matrix $A$, written as Col $A$, is the set of all linear
combinations of the columns of A. If $A = [\vec{a}_1 \cdots \vec{a}_n]$, then \\
Col $A$ = Span $\{\vec{a}_1 \cdots \vec{a}_n \}$
\end{tcolorbox}

\begin{tcolorbox}[colback=green!5,colframe=green!40!black,title= Theorem]
The column space of an $m\times n$ matrix $A$ is a subspace of $\mathbb{R}^m$.
\end{tcolorbox}

\subsection{The Contrast Between Nul $A$ and Col $A$}





\subsection{Kernel and Range of a Linear Transformation}

\begin{tcolorbox}[colback=green!5,colframe=green!40!black,title= Definition]
A linear transformation $T$ from a vector space $V$ into a vector space $W$ is a rule that assigns to each vector $\vec{x}$ in $V$ a unique vector $T(\vec{x})$ in $W$, such that \\
(i) $T(\vec{u} + \vec{v}) = T(\vec{u}) +T(\vec{v})$ for all $\vec{u}, \vec{v} in \vec{V}$, and \\
(ii) $T(c\vec{u}) = cT(\vec{u})$ for all $\vec{u}$ in $V$ and all scalars $c$.
\end{tcolorbox}
The kernel (or null space) of such a $T$ is the set of all $\vec{u}$ in $V$ such that $T(u) = 0$ (the zero vector in $W$). The range of T is the set of all vectors in $W$ of the form $T(x)$ for some $\vec{x}$ in $V$. If $T$ happens to arise as a matrix transformation-say, $T(\vec{x}) = A\vec{x}$ for some matrix $A$-then the kernel and the range of $T$ are just the null space and the column space of $A$.

The kernel of $T$ is a subspace of $V$.


\section{Linearly Independent Sets; Bases}







\section{Coordinate Systems}

\begin{tcolorbox}[colback=green!5,colframe=green!40!black,title= The Unique Representation Theorem]
Let $\mathcal B = \{\vec{b}_1, \cdots, \vec{b}_n\}$ be a basis for a vector space $V$. Then for each $\vec{x}$ in $V$, there exists a unique set of scalars $c_1, \cdots, c_n$ such that
\begin{equation}
\vec{x} = c_1 \vec{b}_1 + \cdots + c_n \vec{b}_n ~.
\end{equation}
\end{tcolorbox}

\begin{tcolorbox}[colback=green!5,colframe=green!40!black,title= Definition]
Suppose $\mathcal B = \{\vec{b}_1, \cdots, \vec{b}_n\}$ is a basis for $V$ and $\vec{x}$ is in $V$. The coordinates of $\vec{x}$
relative to the basis $\mathcal B$ (or the  $\mathcal B$-coordinates of $\vec{x}$) are the weights $c_1, \cdots, c_n$
such that $\vec{x} = c_1 \vec{b}_1 + \cdots + c_n \vec{b}_n$.
\end{tcolorbox}


\subsection{A Graphical Interpretation of Coordinates}





\section{The Dimension of a Vector Space}



\section{Rank}



\section{Change of Basis}


\section{Applications to Difference Equations}



\section{Applications to Markov Chains}






























\end{document}
