\documentclass[12pt,a4paper]{article}
%\usepackage{fontspec, xunicode, xltxtra}
%\setmainfont{Hiragino Sans GB}
%\usepackage{xeCJK}
%\setCJKmainfont[BoldFont=STZhongsong, ItalicFont=STKaiti]{STSong}
%\setCJKsansfont[BoldFont=STHeiti]{STXihei}
%\setCJKmonofont{STFangsong}

%使用Xelatex编译

% 设置页面
%==================================================
\linespread{2} %行距
% \usepackage[top=1in,bottom=1in,left=1.25in,right=1.25in]{geometry}
% \headsep=2cm
% \textwidth=16cm \textheight=24.2cm
%==================================================

% 其它需要使用的宏包
%==================================================
\usepackage[colorlinks,linkcolor=blue,anchorcolor=red,citecolor=green,urlcolor=blue]{hyperref} 
\usepackage{tabularx}
\usepackage{authblk}         % 作者信息
\usepackage{algorithm}     % 算法排版
\usepackage{amsmath}     % 数学符号与公式
\usepackage{amsfonts}     % 数学符号与字体
\usepackage{mathrsfs}      % 花体
\usepackage[framemethod=TikZ]{mdframed}

\usepackage{graphicx} 
\usepackage{graphics}
\usepackage{color}
\usepackage{xcolor}
\usepackage{tcolorbox}
\usepackage{lipsum}
\usepackage{empheq}

\usepackage{fancyhdr}       % 设置页眉页脚
\usepackage{fancyvrb}       % 抄录环境
\usepackage{float}              % 管理浮动体
\usepackage{geometry}     % 定制页面格式
\usepackage{hyperref}       % 为PDF文档创建超链接
\usepackage{lineno}          % 生成行号
\usepackage{listings}        % 插入程序源代码
\usepackage{multicol}       % 多栏排版
%\usepackage{natbib}         % 管理文献引用
\usepackage{rotating}       % 旋转文字,图形,表格
\usepackage{subfigure}    % 排版子图形
\usepackage{titlesec}       % 改变章节标题格式
\usepackage{moresize}   % 更多字体大小
\usepackage{anysize}
\usepackage{indentfirst}  % 首段缩进
\usepackage{booktabs}   % 使用\multicolumn
\usepackage{multirow}    % 使用\multirow

\usepackage{wrapfig}
\usepackage{titlesec}     % 改变标题样式
\usepackage{enumitem}
\usepackage{aas_macros}

\renewcommand{\vec}[1]{\boldsymbol{#1}}
\newcommand{\me}{\mathrm{e}}
\newcommand{\mi}{\mathrm{i}}
\newcommand{\dif}{\mathrm{d}}
\newcommand{\tabincell}[2]{\begin{tabular}{@{}#1@{}}#2\end{tabular}}

\def\kpc{{\rm kpc}}
\def\km{{\rm km}}
\def\cm{{\rm cm}}
\def\TeV{{\rm TeV}}
\def\GeV{{\rm GeV}}
\def\MeV{{\rm MeV}}
\def\GV{{\rm GV}}
\def\MV{{\rm MV}}
\def\yr{{\rm yr}}
\def\s{{\rm s}}
\def\ns{{\rm ns}}
\def\GHz{{\rm GHz}}
\def\muGs{{\rm \mu Gs}}
\def\arcsec{{\rm arcsec}}
\def\K{{\rm K}}
\def\microK{\mu{\rm K}}
\def\sr{{\rm sr}}
\newcolumntype{p}{D{,}{\pm}{-1}}

\renewcommand{\figurename}{Fig.}
\renewcommand{\tablename}{Tab.}

\renewcommand{\arraystretch}{1.5}

\setlength{\parindent}{0pt}  %取消每段开头的空格

\newcounter{theo}[section]\setcounter{theo}{0}
\renewcommand{\thetheo}{\arabic{section}.\arabic{theo}}
\newenvironment{theo}[2][]{%
\refstepcounter{theo}%
\ifstrempty{#1}%
{\mdfsetup{%
frametitle={%
\tikz[baseline=(current bounding box.east),outer sep=0pt]
\node[anchor=east,rectangle,fill=blue!20]
{\strut Theorem~\thetheo};}}
}%
{\mdfsetup{%
frametitle={%
\tikz[baseline=(current bounding box.east),outer sep=0pt]
\node[anchor=east,rectangle,fill=blue!20]
{\strut Theorem~\thetheo:~#1};}}%
}%
\mdfsetup{innertopmargin=10pt,linecolor=blue!20,%
linewidth=2pt,topline=true,%
frametitleaboveskip=\dimexpr-\ht\strutbox\relax
}
\begin{mdframed}[]\relax%
\label{#2}}{\end{mdframed}}

\newcommand*\widefbox[1]{\fbox{\hspace{2em}#1\hspace{2em}}}



\title{Matrix Algebra}
\author{}
\date{\today}
\begin{document}

\maketitle
\section{Matrix Operations}

The diagonal entries in an $m\times n$ matrix $A= [a_{ij}]$ are $a_{11}, a_{22}, a_{33}, \cdots$, and they form the main diagonal of $A$. A diagonal matrix is a square matrix whose nondiagonal entries are zero. An example is the $n\times n$ identity matrix, $I_n$. An $m\times n$ matrix whose entries are all zero is a zero matrix and is written as $0$. The size of a zero matrix is usually clear from the context.

\subsection{Sums and Scalar Multiples}

\begin{tcolorbox}[colback=green!5,colframe=green!40!black,title= Theorem]
Let $A, B, and C$ be matrices of the same size, and let $r$ and $s$ be scalars. 
\begin{align}
& a. A+B = B+A ~, ~~ & d. r(A+B) = rA +rB \\
& b. (A+B) +C =A +(B +C) ~, ~~ & e. (r +s)A = rA +sA \\
& c. A+0 = A ~, ~~ & f. r(sA)=(rs)A
\end{align}
\end{tcolorbox}

\subsection{Matrix Multiplication}

\begin{tcolorbox}[colback=green!5,colframe=green!40!black,title= Theorem]
If $A$ is an $m\times n$ matrix, and if $B$ is an $n\times p$ matrix with columns $\vec{b}_1, \cdots, \vec{b}_p$, then the product $AB$ is the $m\times p$ matrix whose columns are $A\vec{b}_1, \cdots, A\vec{b}_p$. That is, 
\begin{equation*}
AB = A [\vec{b}_1, \vec{b}_2, \cdots, \vec{b}_p] = [A\vec{b}_1 A\vec{b}_2 \cdots A\vec{b}_p]
\end{equation*}
\end{tcolorbox}

Multiplication of matrices corresponds to composition of linear transformations.

Each column of $AB$ is a linear combination of the columns of $A$ using weights from the corresponding column of $B$.

\begin{tcolorbox}[colback=green!5,colframe=green!40!black,title= ROW-COLUMN RULE FOR COMPUTING $AB$]
If the product $AB$ is defined, then the entry in row $i$ and column $j$ of $AB$ is the sum of the products of corresponding entries from row $i$ of $A$ and column $j$ of $B$. If $(AB)_{ij}$ denotes the $(i, j)$-entry in $AB$, and if $A$ is an $m\times n$ matrix, then
\begin{equation*}
(AB)_{ij} = a_{i1} b_{1j} + a_{i2} b_{2j} +\cdots +a_{in}b_{nj} ~.
\end{equation*}
\end{tcolorbox}

\subsection{Properties of Matrix Multiplication}


\begin{tcolorbox}[colback=green!5,colframe=green!40!black,title= Theorem]
Let $A$ be an $m\times n$ matrix, and let $B$ and $C$ have sizes for which the indicated sums and products are defined.
\begin{align*}
& a. ~A(BC) = (AB)C ~~~ \text{(associative law of multiplication)} \\
& b. ~A(B + C) = AB + AC ~~~ \text{(left distributive law)} \\
& c. ~(B + C)A = BA + CA ~~~ \text{(right distributive law)} \\
& d. ~r(AB) = (rA)B = A(rB) ~\text{for any scalar r} \\
& e. ~I_m A = A = A I_n ~~~\text{(identity for matrix multiplication)}
\end{align*}
\end{tcolorbox}


\subsection{Powers of a Matrix}



\subsection{The Transpose of a Matrix}
Given an $m\times n$ matrix $A$, the \textcolor{red}{transpose} of $A$ is the $n\times m$ matrix, denoted by $A^T$, whose columns are formed from the corresponding rows of $A$.



\begin{tcolorbox}[colback=green!5,colframe=green!40!black,title= Theorem]
Let $A$ and $B$ denote matrices whose sizes are appropriate for the following sums and products.
\begin{align*}
& a. ~(A^T)^T = A \\
& b. ~(A+B)^T = A^T +B^T \\
& c. ~\text{For any scalar} r, (rA)^T = rA^T \\
& d. ~(AB)^T =B^T A^T 
\end{align*}
\end{tcolorbox}




















\section{The Inverse of a Matrix}
An $n\times n$ matrix $A$ is said to be invertible if there is an $n\times n$ matrix $C$ such that
\begin{equation*}
C A = I ~\text{and} ~ A C = I
\end{equation*}
where $I = I_n$, the $n\times n$ identity matrix. $C$ is called an inverse of $A$, denoted by $A^{-1}$, then
\begin{equation*}
A^{-1} A = A A^{-1} = I ~.
\end{equation*}
A matrix that is \textcolor{blue}{not invertible} is sometimes called a \textcolor{red}{singular matrix}, and an \textcolor{blue}{invertible matrix} is called a \textcolor{red}{nonsingular matrix}.


\begin{tcolorbox}[colback=green!5,colframe=green!40!black,title= Theorem]
Let $A= \renewcommand{\arraystretch}{0.7} \begin{pmatrix} d & -b \\ -c & a \end{pmatrix}$. If $ad -bc \neq 0$, then $A$ is invertible and 
\begin{equation*}
A^{-1} = \dfrac{1}{ad-bc} 
\renewcommand{\arraystretch}{0.7}
\begin{pmatrix}
d & -b \\
-c & a
\end{pmatrix}
\end{equation*}
If $ad - bc = 0$, then $A$ is not invertible.
\end{tcolorbox}
The quantity $ad - bc$ is called the \textcolor{red}{determinant of $A$}, i.e.
\begin{equation*}
\text{det} A = ad -bc ~.
\end{equation*}

\begin{tcolorbox}[colback=green!5,colframe=green!40!black,title= Theorem]
If $A$ is an invertible $n\times n$ matrix, then for each $\vec{b}$ in $\mathbb R^n$, the equation $A\vec{x} = \vec{b}$ has the unique solution $\vec{x} = A^{-1} \vec{b}$.
\end{tcolorbox}


\begin{tcolorbox}[colback=green!5,colframe=green!40!black,title= Theorem]
a. ~If $A$ is an invertible matrix, then $A^{-1}$ is invertible and 
\begin{equation*}
(A^{-1})^{-1} = A
\end{equation*}
b. If $A$ and $B$ are $n\times n$ invertible matrices, then so is $AB$, and the inverse of $AB$ is the product of the inverses of $A$ and $B$ in the reverse order. That is,
\begin{equation*}
(AB)^{-1} = B^{-1} A^{-1}
\end{equation*}
c. If $A$ is an invertible matrix, then so is $A^T$, and the inverse of $A^T$ is the transpose of $A^{-1}$. That is,
\begin{equation*}
(A^T )^{-1} = (A^{-1})^T
\end{equation*}
\end{tcolorbox}

The \textcolor{orange}{product of $n\times n$ invertible matrices is invertible}, and the \textcolor{orange}{inverse is the product of their inverses in the reverse order}.



\subsection{Elementary Matrices}
An \textcolor{red}{elementary matrix} is one that is obtained by \textcolor{red}{performing a single elementary row operation on an identity matrix}.


If an elementary row operation is performed on an $m\times n$ matrix $A$, the resulting matrix can be written as $EA$, where the $m\times m$ matrix $E$ is created by performing the same row operation on $I_m$.



Each elementary matrix $E$ is invertible. The inverse of $E$ is the elementary matrix of the same type that transforms $E$ back into $I$.


\begin{tcolorbox}[colback=green!5,colframe=green!40!black,title= Theorem]
An $n\times n$ matrix $A$ is invertible if and only if $A$ is row equivalent to $I_n$, and in this case, any sequence of elementary row operations that reduces $A$ to $I_n$ also transforms $I_n$ into $A^{-1}$.
\end{tcolorbox}


\subsection{An Algorithm for Finding $A^{-1}$}

\begin{tcolorbox}[colback=green!5,colframe=green!40!black,title= ALGORITHM FOR FINDING $A^{-1}$]
Row reduce the augmented matrix $[ A~ I]$. If $A$ is row equivalent to $I$, then $[A ~I]$ is row equivalent to $[I A^{-1}]$. Otherwise, $A$ does not have an inverse.
\end{tcolorbox}


\subsection{Another View of Matrix Inversion}



\section{Characterizations of Invertible Matrices}



\section{Partitioned Matrices}






\section{Matrix Factorizations}






\section{The Leontief Input–Output Model}




\section{Applications to Computer Graphics}




\section{Subspaces of $\mathbb R^n$}





\section{Dimension and Rank}




\end{document}
