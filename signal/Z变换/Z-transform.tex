\documentclass[12pt,a4paper]{article}
%\usepackage{fontspec, xunicode, xltxtra}  
%\setmainfont{Hiragino Sans GB}  
\usepackage{xeCJK}
%\setCJKmainfont[BoldFont=STZhongsong, ItalicFont=STKaiti]{STSong}
%\setCJKsansfont[BoldFont=STHeiti]{STXihei}
%\setCJKmonofont{STFangsong}

%使用Xelatex编译

% 设置页面
%==================================================
\linespread{2} %行距
% \usepackage[top=1in,bottom=1in,left=1.25in,right=1.25in]{geometry}
% \headsep=2cm
% \textwidth=16cm \textheight=24.2cm
%==================================================

% 其它需要使用的宏包
%==================================================
\usepackage[colorlinks,linkcolor=blue,anchorcolor=red,citecolor=green,urlcolor=blue]{hyperref} 
\usepackage{tabularx}
\usepackage{authblk}         % 作者信息
\usepackage{algorithm}     % 算法排版
\usepackage{amsmath}     % 数学符号与公式
\usepackage{amsfonts}     % 数学符号与字体
\usepackage{mathrsfs}      % 花体
\usepackage{amssymb}
\usepackage{graphics}
\usepackage{color}
\usepackage{fancyhdr}       % 设置页眉页脚
\usepackage{fancyvrb}       % 抄录环境
\usepackage{float}              % 管理浮动体
\usepackage{geometry}     % 定制页面格式
\usepackage{hyperref}       % 为PDF文档创建超链接
\usepackage{lineno}          % 生成行号
\usepackage{listings}        % 插入程序源代码
\usepackage{multicol}       % 多栏排版
\usepackage{natbib}         % 管理文献引用
\usepackage{rotating}       % 旋转文字,图形,表格
\usepackage{subfigure}    % 排版子图形
\usepackage{titlesec}       % 改变章节标题格式
\usepackage{moresize}   % 更多字体大小
\usepackage{anysize}
\usepackage{indentfirst}  % 首段缩进
\usepackage{booktabs}   % 使用\multicolumn
\usepackage{multirow}    % 使用\multirow
\usepackage{graphicx} 
\usepackage{wrapfig}
\usepackage{xcolor}
\usepackage{titlesec}     % 改变标题样式
\usepackage{enumitem}
\usepackage{harpoon}   %矢量符号

\newcommand{\myvec}[1]%
   {\stackrel{\raisebox{-2pt}[0pt][0pt]{\small$\rightharpoonup$}}{#1}}  %矢量符号
\renewcommand{\vec}[1]{\boldsymbol{#1}}
\newcommand{\me}{\mathrm{e}}
\newcommand{\mi}{\mathrm{i}}
\newcommand{\dif}{\mathrm{d}}
\newcommand{\tabincell}[2]{\begin{tabular}{@{}#1@{}}#2\end{tabular}}

\def\kpc{{\rm kpc}}
\def\km{{\rm km}}
\def\cm{{\rm cm}}
\def\TeV{{\rm TeV}}
\def\GeV{{\rm GeV}}
\def\MeV{{\rm MeV}}
\def\GV{{\rm GV}}
\def\MV{{\rm MV}}
\def\yr{{\rm yr}}
\def\s{{\rm s}}
\def\ns{{\rm ns}}
\def\GHz{{\rm GHz}}
\def\muGs{{\rm \mu Gs}}
\def\arcsec{{\rm arcsec}}
\def\K{{\rm K}}
\def\microK{\mu{\rm K}}
\def\sr{{\rm sr}}
\newcolumntype{p}{D{,}{\pm}{-1}}

\renewcommand{\figurename}{Fig.}
\renewcommand{\tablename}{Tab.}

\renewcommand{\arraystretch}{1.5}

\setlength{\parindent}{0pt}  %取消每段开头的空格

%从右倾的积分号变为竖直的积分号
%define a new command for \rm font of int
\DeclareSymbolFont{rmlargesymbols}{OMX}{mdbch}{m}{n}
% or \DeclareSymbolFont{rmlargesymbols}{U}{euex}{m}{n}
\DeclareMathSymbol{\rmintop}{\mathop}{rmlargesymbols}{82}
\newcommand{\rmint}{\rmintop\nolimits}
\DeclareMathSymbol{\rmointop}{\mathop}{rmlargesymbols}{72}
\newcommand{\rmoint}{\rmointop\nolimits}

%从右倾的积分号变为竖直的积分号 2
\DeclareSymbolFont{EulerExtension}{U}{euex}{m}{n}
\DeclareMathSymbol{\euintop}{\mathop} {EulerExtension}{"52}
\DeclareMathSymbol{\euointop}{\mathop} {EulerExtension}{"48}
\let\intop\euintop
\let\ointop\euointop



\title{$z$变换}
\author{}
\date{\today}
\begin{document}

\maketitle
$z$变换在离散系统中的地位与作用,类似于连续系统中的拉普拉斯变换。

$z$变换的定义可以借助抽样信号的拉氏变换引出,也可直接对离散时间信号给予$z$变换的定义。

抽样信号的拉氏变换:

若连续因果信号$x(t)$经均匀冲激抽样,则抽样信号$x_S(t)$的表示式
\begin{eqnarray*}
x_S(t) &=& x(t) \cdot \delta_T(t) \\ 
&=& \sum_{n=0}^{\infty} x(nT) \delta(t-nT)
\end{eqnarray*}
$T$为抽样间隔。对上式进行拉氏变换,
\begin{eqnarray}
\nonumber X_S(s) &=& \rmint_0^{\infty} x_S(t) {\rm e}^{-st} \dif t = \rmint_0^{\infty} \left[ \sum_{n=0}^{\infty} x(nT) \delta(t-nT) \right]  {\rm e}^{-st} \dif t \\
&=& \sum_{n=0}^{\infty} x(nT) {\rm e}^{-snT}
\label{discre_lap}
\end{eqnarray}
引入复变量$z$,令
\begin{equation*}
z = {\rm e}^{sT}
\end{equation*}
或者
\begin{equation*}
s = \frac{1}{T} \ln z 
\end{equation*}
则式(\ref{discre_lap})变成复变量$z$的函数式$X(z)$
\begin{equation}
X(z) = \sum_{n=0}^{\infty} x(nT) z^{-n}
\end{equation}
令$T = 1$,则
\begin{eqnarray*}
X(z) &=& \sum_{n=0}^{\infty} x(n) z^{-n} \\
z &=& {\rm e}^{s}
\end{eqnarray*}

\section{$z$变换定义}
与拉氏变换定义类似,$z$变换也有单边和双边之分。序列$x(n)$的\textcolor{red}{单边$z$变换}定义为
\begin{eqnarray}
\nonumber X(z) &=& \mathscr{Z}[x(n)] \\
\nonumber &=& x(0) + \frac{x(1)}{z} + \frac{x(2)}{z^2}  + \cdots \\
&=& \sum_{n=0}^{\infty} x(n) z^{-n}
\end{eqnarray}
$\mathscr{Z}$表示取$z$变换,$z$是复变量。对一切$n$值都有定义的双边序列$x(n)$,定义\textcolor{red}{双边$z$变换}为
\begin{eqnarray}
\nonumber X(z) &=& \mathscr{Z}[x(n)] \\
&=& \sum_{n=-\infty}^{\infty} x(n) z^{-n}
\end{eqnarray}
若$x(n)$是因果序列,则双边$z$变换和单边$z$变换是等同的。

\section{典型序列的$z$变换}
\subsection{单位样值函数}
$\delta(n)$定义为
\begin{equation}
\delta(n) = \left\{ 
\begin{aligned}
1 ~~(n = 0) \\
0 ~~(n \neq 0)
\end{aligned}
\right.
\end{equation}
其$z$变换为
\begin{equation*}
\mathscr{Z}[\delta(n)] = \sum_{n=0}^{\infty} \delta(n) z^{-n} = 1
\end{equation*}
单位样值函数$\delta(n)$的$z$变换等于$1$。


\subsection{单位阶跃序列}
$u(n)$定义
\begin{equation}
\delta(n) = \left\{ 
\begin{aligned}
&1 ~~&(n \geqslant 0) \\
&0 ~~&(n < 0)
\end{aligned}
\right.
\end{equation}
其$z$变换为
\begin{equation*}
\mathscr{Z}[u(n)] = \sum_{n=0}^{\infty} u(n) z^{-n} = \sum_{n=0}^{\infty} z^{-n}
\end{equation*}
若$|z| > 1$,该几何级数收敛,等于
\begin{equation}
\mathscr{Z}[u(n)] = \frac{z}{z-1} = \frac{1}{1-z^{-1}}
\end{equation}

\subsection{斜变序列}
\begin{equation}
x(n) = n u(x)
\end{equation}
其$z$变换为
\begin{equation}
\mathscr{Z}[x(n)] = \sum_{n=0}^{\infty} n z^{-n}
\end{equation}
\begin{equation*}
\sum_{n=0}^{\infty} z^{-n} = \frac{1}{1-z^{-1}}  ~~(|z| > 1)
\end{equation*}
的两边分别对$z^{-1}$求导,
\begin{equation*}
\sum_{n=0}^{\infty} n (z^{-1})^{n-1} = \frac{1}{(1-z^{-1})^2} .
\end{equation*}
两边各乘$z^{-1}$,得到
\begin{equation}
\mathscr{Z}[n x(n)] = \sum_{n=0}^{\infty} n z^{-n} = \frac{z}{(z-1)^2}  ~~(|z| > 1)
\end{equation}

\begin{eqnarray*}
\mathscr{Z}[n^2 x(n)] &=& \frac{z(z+1)}{(z-1)^3} \\
\mathscr{Z}[n^3 x(n)] &=& \frac{z(z^2+4z+1)}{(z-1)^4}
\end{eqnarray*}

\subsection{指数序列}
单边指数序列
\begin{equation}
x(n) = a^n u(n)
\end{equation}
其$z$变换为
\begin{eqnarray*}
\mathscr{Z}[a^n u(n)] &=& \sum_{n=0}^{\infty} a^n z^{-n} \\
&=& \sum_{n=0}^{\infty} (a z^{-1})^n
\end{eqnarray*}
若满足$|z| > |a|$,则可收敛为
\begin{equation}
\mathscr{Z}[a^n u(n)] = \frac{1}{1-(az^{-1})} = \frac{z}{z-a}  ~~(|z| > |a|)
\end{equation}
若令$a = {\rm e}^b$,当$|z| > |{\rm e}^b|$,则
\begin{equation*}
\mathscr{Z}[{\rm e}^{bn} u(n)] = \frac{z}{z-{\rm e}^b}
\end{equation*}


\subsection{正弦与余弦序列}
因
\begin{equation*}
\mathscr{Z}[{\rm e}^{bn} u(n)] = \frac{z}{z-{\rm e}^b}  ~~(|z| > |{\rm e}^b|)
\end{equation*}
令$b = i\omega_0$,则当$|z| > |{\rm e}^{i\omega}| = 1$时,得到
\begin{equation*}
\mathscr{Z}[{\rm e}^{i\omega_0 n} u(n)] = \frac{z}{z -{\rm e}^{i\omega_0}}
\end{equation*}
令$b = -i\omega_0$,则得到
\begin{equation*}
\mathscr{Z}[{\rm e}^{-i\omega_0 n} u(n)] = \frac{z}{z -{\rm e}^{-i\omega_0}}
\end{equation*}

\begin{eqnarray*}
\mathscr{Z}[\cos(\omega_0 n) u(n)] &=& \frac{1}{2} \left(\frac{z}{z -{\rm e}^{i\omega_0}} + \frac{z}{z -{\rm e}^{-i\omega_0}}  \right) \\
&=& \frac{z(z-\cos \omega_0)}{z^2 -2z\cos \omega_0 +1}
\end{eqnarray*}


\begin{eqnarray*}
\mathscr{Z}[\sin(\omega_0 n) u(n)] &=& \frac{1}{2} \left(\frac{z}{z -{\rm e}^{i\omega_0}} -\frac{z}{z -{\rm e}^{-i\omega_0}}  \right) \\
&=& \frac{z\sin \omega_0}{z^2 -2z\cos \omega_0 +1}
\end{eqnarray*}

\section{逆$z$变换}
若序列$x(n)$的$z$变换为
\begin{equation*}
X(z) = \mathscr{Z}[x(n)]
\end{equation*}
则$X(z)$的逆变换记作$\mathscr{Z}^{-1}[X(z)]$,并由围道积分给出
\begin{eqnarray*}
x(n) &=& \mathscr{Z}^{-1}[X(z)] \\
&=& \frac{1}{2\pi i} \rmoint_C X(z) z^{n-1} \dif z
\end{eqnarray*}
$C$是包围$X(z) z^{n-1}$所有极点之逆时针闭合积分路线,通常选择$z$平面收敛域内以原点为中心的圆。


\subsection{计算逆$z$变换的方法}

\subsubsection{围道积分法}






\subsubsection{幂级数展开法(长除法)}










\subsubsection{部分分式展开法}








一般情况下,$X(z)$表达式为
\begin{equation}
X(z) = \frac{N(z)}{D(z)} = \frac{b_0 +b_1 z +\cdots +b_{r-1} z^{r-1}  +b_r z^r}{a_0 +a_1 z +\cdots +a_{k-1} z^{k-1}  +b_k z^k}
\end{equation}



























\end{document}