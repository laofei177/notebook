\documentclass[12pt,a4paper]{article}
%\usepackage{fontspec, xunicode, xltxtra}  
%\setmainfont{Hiragino Sans GB}  
\usepackage{xeCJK}
%\setCJKmainfont[BoldFont=STZhongsong, ItalicFont=STKaiti]{STSong}
%\setCJKsansfont[BoldFont=STHeiti]{STXihei}
%\setCJKmonofont{STFangsong}

%使用Xelatex编译

% 设置页面
%==================================================
\linespread{2} %行距
% \usepackage[top=1in,bottom=1in,left=1.25in,right=1.25in]{geometry}
% \headsep=2cm
% \textwidth=16cm \textheight=24.2cm
%==================================================

% 其它需要使用的宏包
%==================================================
\usepackage[colorlinks,linkcolor=blue,anchorcolor=red,citecolor=green,urlcolor=blue]{hyperref} 
\usepackage{tabularx}
\usepackage{authblk}         % 作者信息
\usepackage{algorithm}     % 算法排版
\usepackage{amsmath}     % 数学符号与公式
\usepackage{amsfonts}     % 数学符号与字体
\usepackage{mathrsfs}      % 花体
\usepackage{graphics}
\usepackage{color}
\usepackage{fancyhdr}       % 设置页眉页脚
\usepackage{fancyvrb}       % 抄录环境
\usepackage{float}              % 管理浮动体
\usepackage{geometry}     % 定制页面格式
\usepackage{hyperref}       % 为PDF文档创建超链接
\usepackage{lineno}          % 生成行号
\usepackage{listings}        % 插入程序源代码
\usepackage{multicol}       % 多栏排版
\usepackage{natbib}         % 管理文献引用
\usepackage{rotating}       % 旋转文字,图形,表格
\usepackage{subfigure}    % 排版子图形
\usepackage{titlesec}       % 改变章节标题格式
\usepackage{moresize}   % 更多字体大小
\usepackage{anysize}
\usepackage{indentfirst}  % 首段缩进
\usepackage{booktabs}   % 使用\multicolumn
\usepackage{multirow}    % 使用\multirow
\usepackage{graphicx} 
\usepackage{wrapfig}
\usepackage{xcolor}
\usepackage{titlesec}     % 改变标题样式
\usepackage{enumitem}
\usepackage{harpoon}   %矢量符号

\newcommand{\myvec}[1]%
   {\stackrel{\raisebox{-2pt}[0pt][0pt]{\small$\rightharpoonup$}}{#1}}  %矢量符号
\renewcommand{\vec}[1]{\boldsymbol{#1}}
\newcommand{\me}{\mathrm{e}}
\newcommand{\mi}{\mathrm{i}}
\newcommand{\dif}{\mathrm{d}}
\newcommand{\tabincell}[2]{\begin{tabular}{@{}#1@{}}#2\end{tabular}}

\def\kpc{{\rm kpc}}
\def\km{{\rm km}}
\def\cm{{\rm cm}}
\def\TeV{{\rm TeV}}
\def\GeV{{\rm GeV}}
\def\MeV{{\rm MeV}}
\def\GV{{\rm GV}}
\def\MV{{\rm MV}}
\def\yr{{\rm yr}}
\def\s{{\rm s}}
\def\ns{{\rm ns}}
\def\GHz{{\rm GHz}}
\def\muGs{{\rm \mu Gs}}
\def\arcsec{{\rm arcsec}}
\def\K{{\rm K}}
\def\microK{\mu{\rm K}}
\def\sr{{\rm sr}}
\newcolumntype{p}{D{,}{\pm}{-1}}

\renewcommand{\figurename}{Fig.}
\renewcommand{\tablename}{Tab.}

\renewcommand{\arraystretch}{1.5}

\setlength{\parindent}{0pt}  %取消每段开头的空格

\title{卷积}
\author{}
\date{\today}
\begin{document}

\maketitle
对于任意两个信号$f_1(t)$和$f_2(t)$,卷积运算定义为
\begin{equation}
f(t) = f_1(t)*f_2(t) = f_1(t)\otimes f_2(t) = \int_{-\infty}^{\infty} f_1(\tau) f_2(t -\tau) \dif \tau
\end{equation}

\section{卷积代数}
\subsection{交换律}
\begin{equation}
f_1(t)*f_2(t) = f_2(t)* f_1(t)
\end{equation}
卷积积分中的次序是可以交换的;


\subsection{分配律}
\begin{equation}
f_1(t)*[f_2(t) +f_3(t)] = f_1(t)* f_2(t) + f_1(t)* f_3(t)
\end{equation}


\subsection{结合律}
\begin{equation}
[f_1(t)*f_2(t)] *f_3(t) = f_1(t)* [f_2(t)* f_3(t)]
\end{equation}



\section{卷积定理}
\subsection{时域卷积定理}
若给定两个时间函数$f_1(t)$和$f_2(t)$,已知:
\begin{eqnarray*}
\mathscr{F}[f_1(t)] &=& F_1(\omega) \\
\mathscr{F}[f_2(t)] &=& F_2(\omega)
\end{eqnarray*}
则
\begin{equation}
\mathscr{F}[f_1(t) *f_2(t)] =  F_1(\omega)F_2(\omega)
\end{equation}


\subsection{频域卷积定理}
若给定两个时间函数$f_1(t)$和$f_2(t)$,已知:
\begin{eqnarray*}
\mathscr{F}[f_1(t)] &=& F_1(\omega) \\
\mathscr{F}[f_2(t)] &=& F_2(\omega)
\end{eqnarray*}
则
\begin{equation}
\mathscr{F}[f_1(t) \cdot f_2(t)] =  \frac{1}{2\pi} F_1(\omega)*F_2(\omega)
\end{equation}

\subsection{拉氏变换卷积定理}
若$\mathscr{L}[f_1(t)] = F_1(s)$,$\mathscr{L}[f_2(t)] = F_2(s)$,则有
\begin{equation}
\mathscr{L}[f_1(t) * f_2(t)] = F_1(s) F_2(s)
\end{equation}
两原函数卷积的拉氏变换等于两函数拉氏变换的乘积。

























\end{document}