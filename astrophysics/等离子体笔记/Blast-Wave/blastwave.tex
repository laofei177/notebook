\documentclass[12pt,a4paper]{article}
%\usepackage{fontspec, xunicode, xltxtra}  
%\setmainfont{Hiragino Sans GB}  
%\usepackage{xeCJK}
%\setCJKmainfont[BoldFont=STZhongsong, ItalicFont=STKaiti]{STSong}
%\setCJKsansfont[BoldFont=STHeiti]{STXihei}
%\setCJKmonofont{STFangsong}

%使用Xelatex编译

% 设置页面
%==================================================
\linespread{2} %行距
% \usepackage[top=1in,bottom=1in,left=1.25in,right=1.25in]{geometry}
% \headsep=2cm
% \textwidth=16cm \textheight=24.2cm
%==================================================

% 其它需要使用的宏包
%==================================================
\usepackage[colorlinks,linkcolor=blue,anchorcolor=red,citecolor=green,urlcolor=blue]{hyperref} 
\usepackage{tabularx}
\usepackage{authblk}         % 作者信息
\usepackage{algorithm}     % 算法排版
\usepackage{amsmath}     % 数学符号与公式
\usepackage{amsfonts}     % 数学符号与字体
\usepackage{amssymb}
\usepackage{mathrsfs}      % 花体

\usepackage{graphicx} 
\usepackage{graphics}
\usepackage{color}
\usepackage{xcolor}

\usepackage{fancyhdr}       % 设置页眉页脚
\usepackage{fancyvrb}       % 抄录环境
\usepackage{float}              % 管理浮动体
\usepackage{geometry}     % 定制页面格式
\usepackage{hyperref}       % 为PDF文档创建超链接
\usepackage{lineno}          % 生成行号
\usepackage{listings}        % 插入程序源代码
\usepackage{multicol}       % 多栏排版
%\usepackage{natbib}         % 管理文献引用
\usepackage{rotating}       % 旋转文字,图形,表格
\usepackage{subfigure}    % 排版子图形
\usepackage{titlesec}       % 改变章节标题格式
\usepackage{moresize}   % 更多字体大小
\usepackage{anysize}
\usepackage{indentfirst}  % 首段缩进
\usepackage{booktabs}   % 使用\multicolumn
\usepackage{multirow}    % 使用\multirow

\usepackage{wrapfig}
\usepackage{titlesec}     % 改变标题样式
\usepackage{enumitem}
\usepackage{aas_macros}

\newcommand{\myvec}[1]%
   {\stackrel{\raisebox{-2pt}[0pt][0pt]{\small$\rightharpoonup$}}{#1}}  %矢量符号
\renewcommand{\vec}[1]{\boldsymbol{#1}}
\newcommand{\me}{\mathrm{e}}
\newcommand{\mi}{\mathrm{i}}
\newcommand{\dif}{\mathrm{d}}
\newcommand{\tabincell}[2]{\begin{tabular}{@{}#1@{}}#2\end{tabular}}

\def\kpc{{\rm kpc}}
\def\km{{\rm km}}
\def\cm{{\rm cm}}
\def\TeV{{\rm TeV}}
\def\GeV{{\rm GeV}}
\def\MeV{{\rm MeV}}
\def\GV{{\rm GV}}
\def\MV{{\rm MV}}
\def\yr{{\rm yr}}
\def\s{{\rm s}}
\def\ns{{\rm ns}}
\def\GHz{{\rm GHz}}
\def\muGs{{\rm \mu Gs}}
\def\arcsec{{\rm arcsec}}
\def\K{{\rm K}}
\def\microK{\mu{\rm K}}
\def\sr{{\rm sr}}
\newcolumntype{p}{D{,}{\pm}{-1}}

\renewcommand{\figurename}{Fig.}
\renewcommand{\tablename}{Tab.}

\renewcommand{\arraystretch}{1.5}

\setlength{\parindent}{0pt}  %取消每段开头的空格

\title{Blast-Wave Physics}
\author{}
\date{\today}
\begin{document}

\maketitle

\section{FIREBALLS AND RELATIVISTIC BLAST WAVES}
\cite{2009herb.book.....D} Consider an explosion that takes place in a uniform circumburst medium (CBM) with density $n_0$. Suppose that the event releases energy at a fixed rate over a timescale $\Delta_0/c$, where $\Delta_0$ is a characteristic size scale of the engine that releases the energy. 

The apparent isotropic equivalent $\gamma$-ray energy $E_0$ released by
a GRB explosion can exceed $\sim 10^{54}$ ergs, with apparent isotropic powers $\gtrsim 10^{51}$ ergs s${^-1}$. In comparison, the rest mass energy of a Solar mass of material is $\simeq 2\times 10^{54}$ ergs, and the bolometric luminosity of the universe is $\sim 10^{54}$ ergs s${^-1}$.


\subsection{Blast-Wave Deceleration}
The blast-wave deceleration occurs when the relativistic blast-wave shell from an explosive event sweeps up CBM material at an external shock. The accumulation of swept-up material causes the shell to decelerate. For a uniform spherically symmetric CBM, the mass of swept-up material at radius $x$ is $M_{\rm sw} = 4\pi \mu_0 m_p n_0 x^3/3$, where $n_0$ is the proton density and $\mu_0$ is a factor accounting for the metallicity of the CBM.

The blast wave will start to undergo significant deceleration when an amount of energy comparable to the initial energy $E_0$ in the blast wave is swept up. Looked at from the comoving frame, each proton from the CBM carries with it an amount of energy $\Gamma_0 m_p c^2$ when captured by the blast wave. After capture and isotropization, the amount of energy carried by the blast
wave from this swept-up proton is $\Gamma_0^2 m_p c^2$ as measured in the stationary frame. The condition $\Gamma_0^2 M_{sw} c^2 = E_0$ gives the deceleration radius
\begin{align}
x_d \equiv \left(\dfrac{3E_0}{4\pi \Gamma_0^2 m_p c^2 n_0} \right)^{1/3} \simeq 2.6 \times 10^{16} \left(\dfrac{E_{52}}{\Gamma_{300}^2 n_0} \right)^{1/3} ~\rm cm ~,
\end{align}
where $E_0 = E_{52}/10^{52}$ ergs is the total explosion energy including rest mass energy, $\Gamma_{300} = \Gamma_0/300$, and $n_0$ is the CBM proton density in units of cm$^{-3}$.

Differential time elements in the stationary (starred), comoving (primed), and observer (unscripted) reference frames satisfy the relations
\begin{align}
\dif x = \beta c \dif t_{\ast} = \beta \Gamma c \dif t^\prime = \beta c \dfrac{\dif t}{(1+z)(1-\beta \mu)} ~,
\end{align}
where $\theta = \arccos \mu$ is the angle between the direction of outflow and the observer, and $\dif t = (1+z)\dif t^\prime/\delta_D$. 
\begin{align}
\dif t = \dfrac{(1+z)}{c} \dif x (\beta^{-1} -\mu) \simeq \dfrac{(1+z) \dif x}{\Gamma^2 c} ~.
\end{align}
The final term in this expression applies to relativistic flows ($\Gamma \gg 1$) observed at $\theta \simeq 1/\Gamma$, assuming that the average emitting region is located at cosine angle $\mu \simeq \beta$ to the line of sight.

The \textcolor{red}{deceleration time as measured by an observer} is
\begin{align}
\nonumber t_d &\equiv (1+z) \dfrac{x_d}{\beta_0 \Gamma_0^2 c} \\ 
&= (1+z) \left(\dfrac{3E_0}{4\pi n_0 m_p c^5 \Gamma_0^8} \right)^{1/3} \simeq \dfrac{9.6(1+z)}{\beta_0} \left(\dfrac{E_{52}}{\Gamma_{300}^8 n_0} \right)^{1/3}  ~\rm s ~,
\end{align}
where $z$ is the redshift of the source, and the factor $\beta_0^{-1} = 1/\sqrt{1 -\Gamma_0^{-2}}$ generalizes the original result for mildly relativistic and nonrelativistic supernova explosions. The Sedov radius, giving the distance traveled in a uniform CBM when the shell sweeps up an amount of mass energy comparable to the explosion energy, is given by
\begin{align}
\ell_S &= \left(\dfrac{3E_0}{4\pi m_p c^2 n_0} \right)^{1/3} = \Gamma_0^{2/3} x_d \\
&\simeq 1.2 \times 10^{18} \left(\dfrac{E_{52}}{n_0} \right)^{1/3} ~{\rm cm} \simeq 6.6 \times 10^{18} \left(\dfrac{\mathcal E_\odot}{n_0} \right)^{1/3} ~\rm cm ~,
\end{align}
The final term is written in units of $\mathcal E_\odot = E_0/M_\odot c^2$, where $M_\odot$ is the mass of the Sun. For relativistic explosions, $\ell_S$ refers to the radius where the blast wave slows to mildly relativistic speeds, i.e., $\Gamma \sim 2$. The Sedov radius of a SN that ejects a $10 M_\odot$ envelope can reach several pc or more.


\subsection{Blast-Wave Equation of Motion}
The equation describing the momentum $P = \beta \gamma$ of the relativistic blast wave, which changes as a consequence of the blast wave sweeping up material from the surrounding medium and radiating internal energy, is derived.

Applying momentum conservation for the explosion and swept-up mass $m(x)$ gives
\begin{align}
\nonumber P\left[M_0 +\int_0^x \dif \tilde{x} \left(\dfrac{\dif m(\tilde{x})}{\dif \tilde{x}} \right) \Gamma(\tilde{x}) \right] & \simeq \beta \Gamma [M_0 +m(x) \Gamma(x)] \\
& \simeq \beta \Gamma [M_0 +kx^3 \Gamma] \simeq \rm const ~,
\end{align}
giving the asymptotes $\Gamma \propto x^{-3/2}$ when $\Gamma_0 \gg \Gamma \gg 1$ and $\beta \propto x^{-3}$ when $\Gamma -1 \ll 1$. For the radiative solution,
\begin{align}
\nonumber P\left[M_0 +\int_0^x \dif \tilde{x} \left(\dfrac{\dif m(\tilde{x})}{\dif \tilde{x}} \right) \right] & \simeq \beta \Gamma [M_0 +m(x) ] \\
& \simeq \beta \Gamma [M_0 +kx^3 ] \simeq \rm const ~,
\end{align}
giving the asymptotes $\Gamma \propto x^{-3}$ when $\Gamma_0 \gg \Gamma \gg 1$ and $\beta \propto x^{-3}$ when $\Gamma -1 \ll 1$.

The limits for the adiabatic solution would seem to be obtained through total energy conservation from
\begin{align}
\Gamma \left[M_0 +\int_0^x \dif \tilde{x} \Gamma(\tilde{x}) \left(\dfrac{\dif m(\tilde{x})}{\dif \tilde{x}} \right) \right] \simeq \Gamma [M_0 +\Gamma m(x) ] \simeq \Gamma(M_0 +kx^3 \Gamma) \simeq \rm const ~.
\end{align}
This gives the correct relativistic asymptote $\Gamma \propto x^{-3/2}$ when $\Gamma_0 \gg \Gamma \gg 1$, but implies that $\beta \propto x^{-3}$ when $\Gamma -1 \ll 1$. The change in internal energy due to adiabatic losses becomes important in the nonrelativistic regime so that this estimate is not valid there.









\section{RELATIVISTIC SHOCK HYDRODYNAMICS}
\cite{2009herb.book.....D} Assume idealized shock structures. 





















\section{BEAMING BREAKS AND JETS}
\cite{2009herb.book.....D} An observer will receive most emission from those portions of a GRB blast wave that are within an angle $\sim 1/\Gamma$ to the direction to the observer. As the blast wave decelerates by sweeping up material from the CBM, a break in the light curve will occur when the jet opening half angle $\theta_j$ becomes smaller than $1/\Gamma$. This is due to a change from a spherical blast-wave geometry to a geometry defined by a localized emission region. Assuming that the blast wave
decelerates adiabatically in a uniform surrounding medium, the condition $\theta_j \simeq 1/\Gamma = \Gamma_0^{-1} (x_{\rm br}/x_d)^{3/2} = \Gamma_0^{-1} (t_{\rm br}/t_d)^{3/8}$ implies
\begin{align}
t_{\rm br} \approx 45(1+z) \left(\dfrac{E_{52}}{n_0} \right)^{1/3} \theta_j^{8/3} ~ \rm days ~,
\end{align}
from which the jet angle
\begin{align}
\theta_j \approx 0.1 \left[\dfrac{t_{\rm br}(d)}{(1+z)} \right]^{3/8} \left(\dfrac{n_0}{E_{52}} \right)^{1/8}
\end{align}
can be derived. The beaming angle is only weakly dependent on $n_0$ and $E_0$.


















\section{SYNCHROTRON SELF-COMPTON RADIATION}
\cite{2009herb.book.....D} Electrons cool in the comoving fluid frame by synchrotron and Compton losses. The Compton $y$-parameter
\begin{align}
y_C \equiv \dfrac{L_C}{L_{\rm syn}} \simeq \dfrac{U_{\rm syn}}{U_B} ~,
\end{align}
gives the ratio of the (synchrotron-self) Compton and synchrotron powers, and the final expression holds for scattering in the Thomson regime.

The total internal energy $U = U_e + U_p + U_B + U_{\rm ph}$ in the shocked fluid shell is found in the form of nonthermal electron and protons/ions, magnetic field, and photons. The electron energy $U_e = \epsilon_e U$, and the magnetic field energy $U_B = \epsilon_B U$. The photon energy
\begin{align}
U_{\rm ph} = U_{\rm syn} +U_C = \eta_e \epsilon_e U = \dfrac{\eta_e \epsilon_e U_B}{\epsilon_B} ~,
\end{align}
and $\eta_e$ is the radiative efficiency to convert nonthermal electron energy into radiation. In the fast cooling regime, $\gamma_c < \gamma_{\rm min}$, and $\eta_e \simeq 1$. In the slowcooling regime, only electrons with $\gamma > \gamma_c$ cool efficiently. The fractional energy in electrons that stongly cool is $\sim (\gamma_{\rm min}/\gamma_c)^{p-2}$, so
\begin{align}
\eta_e = {\rm min} \left[1, \left(\dfrac{\gamma_{\rm min}}{\gamma_c} \right)^{p-2} \right] ~.
\end{align}
\begin{align}
\dfrac{U_{\rm syn}}{U_B} +\dfrac{U_C}{U_{\rm syn}} \dfrac{U_{\rm syn}}{U_B} = y_C +y_C^2 = \dfrac{\eta_{e} \epsilon_e}{\epsilon_B}
\end{align}
\begin{align}
y_C \equiv \dfrac{L_C}{L_{\rm syn}} = \dfrac{-1+\sqrt{1+4\eta_e \epsilon_e/\epsilon_B}}{2} \rightarrow \left\{
\begin{aligned}
& \eta_e \epsilon_e/\epsilon_B, & \eta_e \epsilon_e/\epsilon_B \ll 1 ~, \\
& \sqrt{\eta_e \epsilon_e/\epsilon_B}, & \eta_e \epsilon_e/\epsilon_B \gg 1 ~. \\
\end{aligned}
\right.
\end{align}
The inclusion of a Compton component in blast-wave afterglow modeling can be used to derive analytic and numerical spectra and light curves. A standard approximation to treat the Compton component analytically is to use the Thomson cross section truncated for scattering in the Klein-Nishina regime. 




\section{THEORY OF THE PROMPT PHASE}
\cite{2009herb.book.....D} The prompt phase of long-duration GRBs, when they are most luminous, lasts from seconds to minutes at $\sim$ 100 keV-MeV energies, as long as $90$ minutes at $\geqslant 100$ MeV energies, and up to $\sim 10^5$ after the start of a burst for keV X-ray flares found with Swift. 

In the \textcolor{red}{internal shock} model, an active central engine ejects \textcolor{red}{waves of relativistic plasma} that \textcolor{red}{overtake and collide to form shocks}. The shocks accelerate nonthermal particles that radiate high-energy photons. By contrast, in the \textcolor{red}{external shock} model, \textcolor{red}{a single relativistic wave of particles interacts with inhomogeneities in the surrounding medium} to accelerate particles that radiate the prompt $\gamma$-rays.






















%%%%%%%%%%%%%%%%%%%%%%%%%%%%%%%%%%%%%%%%%%%%%%%%%%%%%%%%%%%%%%%%%%%%%%
\bibliographystyle{unsrt_update}
\bibliography{ref}
%%%%%%%%%%%%%%%%%%%%%%%%%%%%%%%%%%%%%%%%%%%%%%%%%%%%%%%%%%%%%%%%%%%%%%

\end{document}