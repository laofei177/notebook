\documentclass[12pt,a4paper]{article}
%\usepackage{fontspec, xunicode, xltxtra}  
%\setmainfont{Hiragino Sans GB}  
\usepackage{xeCJK}
%\setCJKmainfont[BoldFont=STZhongsong, ItalicFont=STKaiti]{STSong}
%\setCJKsansfont[BoldFont=STHeiti]{STXihei}
%\setCJKmonofont{STFangsong}

%使用Xelatex编译

% 设置页面
%==================================================
\linespread{2} %行距
% \usepackage[top=1in,bottom=1in,left=1.25in,right=1.25in]{geometry}
% \headsep=2cm
% \textwidth=16cm \textheight=24.2cm
%==================================================

% 其它需要使用的宏包
%==================================================
\usepackage[colorlinks,linkcolor=blue,anchorcolor=red,citecolor=green,urlcolor=blue]{hyperref} 
\usepackage{tabularx}
\usepackage{authblk}         % 作者信息
\usepackage{algorithm}     % 算法排版
\usepackage{amsmath}     % 数学符号与公式
\usepackage{amsfonts}     % 数学符号与字体
\usepackage{graphics}
\usepackage{color}
\usepackage{fancyhdr}       % 设置页眉页脚
\usepackage{fancyvrb}       % 抄录环境
\usepackage{float}              % 管理浮动体
\usepackage{geometry}     % 定制页面格式
\usepackage{hyperref}       % 为PDF文档创建超链接
\usepackage{lineno}          % 生成行号
\usepackage{listings}        % 插入程序源代码
\usepackage{multicol}       % 多栏排版
\usepackage{natbib}         % 管理文献引用
\usepackage{rotating}       % 旋转文字,图形,表格
\usepackage{subfigure}    % 排版子图形
\usepackage{titlesec}       % 改变章节标题格式
\usepackage{moresize}   % 更多字体大小
\usepackage{anysize}
\usepackage{indentfirst}  % 首段缩进
\usepackage{booktabs}   % 使用\multicolumn
\usepackage{multirow}    % 使用\multirow
\usepackage{graphicx} 
\usepackage{wrapfig}
\usepackage{xcolor}
\usepackage{titlesec}     % 改变标题样式
\usepackage{enumitem}

\renewcommand{\vec}[1]{\boldsymbol{#1}}
\newcommand{\me}{\mathrm{e}}
\newcommand{\mi}{\mathrm{i}}
\newcommand{\dif}{\mathrm{d}}
\newcommand{\tabincell}[2]{\begin{tabular}{@{}#1@{}}#2\end{tabular}}

\def\kpc{{\rm kpc}}
\def\km{{\rm km}}
\def\cm{{\rm cm}}
\def\TeV{{\rm TeV}}
\def\GeV{{\rm GeV}}
\def\MeV{{\rm MeV}}
\def\GV{{\rm GV}}
\def\MV{{\rm MV}}
\def\yr{{\rm yr}}
\def\s{{\rm s}}
\def\ns{{\rm ns}}
\def\GHz{{\rm GHz}}
\def\muGs{{\rm \mu Gs}}
\def\arcsec{{\rm arcsec}}
\def\K{{\rm K}}
\def\microK{\mu{\rm K}}
\def\sr{{\rm sr}}
\newcolumntype{p}{D{,}{\pm}{-1}}

\renewcommand{\figurename}{Fig.}
\renewcommand{\tablename}{Tab.}

\renewcommand{\arraystretch}{1.5}

\setlength{\parindent}{0pt}  %取消每段开头的空格

\title{密度扰动的线性演化}
\author{}
\date{\today}
\begin{document}

\maketitle
\section{}

\textcolor{red}{线性转移函数}

\section{密度扰动功率谱}
把整个宇宙分割成许多个立方体,立方体之间满足周期性边界条件

密度涨落定义为
\begin{equation}
\delta(\vec{x}) = \left[\rho(\vec{x}) -\left\langle \rho \right\rangle \right] /\left\langle \rho \right\rangle 
\end{equation}
$\left\langle \rho \right\rangle$: 体积$V$内的平均密度;

把$\delta(\vec{x})$进行平面波展开
\begin{equation}
\delta(\vec{x}) = \sum_k \delta_k \exp (i\vec{k}\cdot \vec{x}) = \sum_k \delta^*_k \exp(-i\vec{k}\cdot \vec{x})
\end{equation}
周期性边界条件要求
\begin{equation}
k_x = n_x \frac{2\pi}{L}, ~k_y = n_y \frac{2\pi}{L}, ~k_z = n_z \frac{2\pi}{L}, 
\end{equation}
$n_x, n_y, n_z$为整数;
\begin{equation}
\delta_k = \frac{1}{V} \int_V \delta(\vec{x}) \exp (-i\vec{k}\cdot \vec{x}) \dif \vec{x}
\end{equation}

假设取另一个体积$V$,把该体积内的扰动同样展开成以上形式,则展开式的系数$\delta_k$可能会不同;若取大量数目的体积$V$,无论是$\delta_k$的振幅还是相位,都可能会各处不同;如果相位是随机的,此时的密度扰动场具有Gauss分布的统计特征,它的全空间平均值,即
\begin{equation}
\left\langle \delta(\vec{x}) \right\rangle \equiv \bar{\delta}(\vec{x}) = 0
\end{equation}
但方差
\begin{equation}
\sigma^2 \equiv \left\langle \delta^2(\vec{x}) \right\rangle = \sum_k \left\langle |\delta_k|^2 \right\rangle = \frac{1}{V} \sum_k \delta_k^2 \neq 0
\end{equation}
当$V\rightarrow \infty$,
\begin{equation}
\delta(\vec{x}) = \frac{V}{(2\pi)^3} \int \delta_k \exp (i\vec{k}\cdot \vec{x}) \dif \vec{k}
\end{equation}

\begin{equation}
\sigma^2 \equiv \left\langle \delta^2(\vec{x}) \right\rangle = \frac{1}{V} \int \delta^2(\vec{x}) \dif \vec{x} = \frac{V}{(2\pi)^3} \int \delta^2_k \dif \vec{k}
\end{equation}

若密度扰动场在统计上是均匀各项同性的,统计性质与方向无关,因此
\begin{equation}
\sigma^2 = \frac{V}{2\pi^2} \int_0^{\infty} \delta^2_k k^2 \dif k = \frac{V}{2\pi^2} \int_0^{\infty} \delta^2_k k^3 \dif \ln k = \frac{V}{2\pi^2} \int_0^{\infty} P(k) k^3 \dif \ln k
\end{equation}
密度扰动的\textcolor{red}{功率谱}
\begin{equation}
P(k) = \delta^2_k
\end{equation}
与空间位置无关,只是时间的函数
$\delta^2_k k^3$或者$P(k) k^3$:在$k$尺度上,单位对数间隔内密度扰动的功率大小,即在该尺度上成团强度的大小。对于CDM宇宙,演化的结果是小尺度上有较大的扰动功率,即成团先从小的质量开始,然后通过引力作用逐渐形成越来越大的结构,hierarchical、bottom-up成团模式;对于HDM宇宙,大尺度将产生较大的扰动功率,使大尺度上优先成团,再通过分裂过程产生较小尺度的结构,top-down模式。

线性演化阶段结束时的功率谱与初始功率谱的关系
\begin{equation}
P(k, t_f) \equiv \delta^2_k(t_f) = T^2(k) \delta^2_k(t_i) D^2(t_i, t_f) = T^2(k) P(k, t_i) D^2(t_i, t_f)  
\end{equation}

一个半径为$R$的球体积内的平均质量
\begin{equation}
\left\langle M \right\rangle = \left\langle \rho \right\rangle V = \frac{4\pi}{3} \left\langle \rho \right\rangle R^3
\end{equation}
质量涨落方差
\begin{equation}
\sigma_M^2 = \left\langle \frac{[M -\left\langle M \right\rangle]^2}{\left\langle M \right\rangle^2} \right\rangle = \left\langle \left( \frac{\delta M}{M} \right)^2 (R) \right\rangle 
\end{equation}
表示在全空间任意选取的、半径为$R$的球体积$V=4\pi R^3/3$内质量涨落的方均值。

由Fourier级数展开,
\begin{eqnarray}
\nonumber \sigma_M^2 &=& \left\langle \frac{\int \delta(\vec{x}) \dif \vec{x} }{V} \frac{\int \delta(\vec{x}') \dif \vec{x}' }{V} \right\rangle \\
\nonumber &=& \frac{1}{V^2}  \left\langle \int_V \int_V \sum_k \delta_k \exp (i\vec{k}\cdot \vec{x}) \sum_{k'} \delta^*_{k'} \exp (-i\vec{k'}\cdot \vec{x'}) \dif \vec{x} \dif \vec{x}' \right\rangle \\
\nonumber &=& \frac{1}{V^2}  \left\langle \sum_{k,k'} \delta_k \delta^*_{k'}  \int_V \exp (i\vec{k}\cdot \vec{x}) \dif \vec{x} \int_V \exp (-i\vec{k'}\cdot \vec{x'}) \dif \vec{x}' \right\rangle \\
\nonumber &=& \frac{1}{V^2} \int_V \dif \vec{x}_0 \left[ \sum_{k,k'} \delta_k \delta^*_{k'}  \int_V \exp [i\vec{k}\cdot (\vec{x}_0 +\vec{x})] \dif \vec{x} \int_V \exp [-i\vec{k'}\cdot (\vec{x}_0 +\vec{x'})] \dif \vec{x}'  \right] \\
\nonumber &=& \frac{1}{V^2} \int_V \dif \vec{x}_0 \exp [i(\vec{k}-\vec{k'})\cdot \vec{x}_0] \left[ \sum_{k,k'} \delta_k \delta^*_{k'}  \int_V \exp [i\vec{k}\cdot \vec{x}] \dif \vec{x} \int_V \exp [-i\vec{k'}\cdot \vec{x'}] \dif \vec{x}'  \right]
\end{eqnarray}
由于
\begin{equation}
\int_V \dif \vec{x}_0 \exp [i(\vec{k}-\vec{k'})\cdot \vec{x}_0]  = \delta_{k,k'}
\end{equation}
上式化为
\begin{equation}
\sigma_M^2 = \sum_k \delta^2_k \left[ \frac{1}{V} \int_V \exp [i\vec{k}\cdot \vec{x}] \dif \vec{x} \right]^2
\end{equation}

\begin{equation}
\frac{1}{V} \int_V \exp [i\vec{k}\cdot \vec{x}] \dif \vec{x} = \frac{3}{(kR)^3} [\sin(kR) -kR\cos(kR)] \equiv W(kR)
\end{equation}

\begin{equation}
\sigma_M^2 = \sum_k \delta^2_k W^2(kR) = \frac{V}{2\pi^2} \int_0^{\infty} \delta^2_k W^2(kR) k^2 \dif k = \frac{V}{2\pi^2} \int_0^{\infty} P(k) W^2(kR) k^3 \dif \ln k 
\end{equation}






















%%%%%%%%%%%%%%%%%%%%%%%%%%%%%%%%%%%%%%%%%%%%%%%%%%%%%%%%%%%%%%%%%%%%%%
\bibliographystyle{unsrt_update}
\bibliography{ref}
%%%%%%%%%%%%%%%%%%%%%%%%%%%%%%%%%%%%%%%%%%%%%%%%%%%%%%%%%%%%%%%%%%%%%%


\end{document}