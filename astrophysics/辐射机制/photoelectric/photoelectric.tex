\documentclass[12pt,a4paper]{article}
%\usepackage{fontspec, xunicode, xltxtra}  
%\setmainfont{Hiragino Sans GB}  
\usepackage{xeCJK}
%\setCJKmainfont[BoldFont=STZhongsong, ItalicFont=STKaiti]{STSong}
%\setCJKsansfont[BoldFont=STHeiti]{STXihei}
%\setCJKmonofont{STFangsong}

%使用Xelatex编译

% 设置页面
%==================================================
\linespread{2} %行距
% \usepackage[top=1in,bottom=1in,left=1.25in,right=1.25in]{geometry}
% \headsep=2cm
% \textwidth=16cm \textheight=24.2cm
%==================================================

% 其它需要使用的宏包
%==================================================
\usepackage[colorlinks,linkcolor=blue,anchorcolor=red,citecolor=green,urlcolor=blue]{hyperref} 
\usepackage{tabularx}
\usepackage{authblk}         % 作者信息
\usepackage{algorithm}     % 算法排版

\usepackage{amsmath}     % 数学符号与公式
\usepackage{amsfonts}     % 数学符号与字体
\usepackage{mathrsfs}      % 花体
\usepackage{amssymb}

\usepackage{graphicx} 
\usepackage{graphics}

\usepackage{xcolor}
\usepackage{color}
\usepackage{fancyhdr}       % 设置页眉页脚
\usepackage{fancyvrb}       % 抄录环境
\usepackage{float}              % 管理浮动体
\usepackage{geometry}     % 定制页面格式
\usepackage{hyperref}       % 为PDF文档创建超链接
\usepackage{lineno}          % 生成行号
\usepackage{listings}        % 插入程序源代码
\usepackage{multicol}       % 多栏排版
%\usepackage{natbib}         % 管理文献引用
\usepackage{rotating}       % 旋转文字,图形,表格
\usepackage{subfigure}    % 排版子图形
\usepackage{titlesec}       % 改变章节标题格式
\usepackage{moresize}   % 更多字体大小
\usepackage{anysize}
\usepackage{indentfirst}  % 首段缩进
\usepackage{booktabs}   % 使用\multicolumn
\usepackage{multirow}    % 使用\multirow
\usepackage{wrapfig}
\usepackage{xcolor}
\usepackage{titlesec}     % 改变标题样式
\usepackage{enumitem}
\usepackage{aas_macros}

\newcommand{\myvec}[1]%
   {\stackrel{\raisebox{-2pt}[0pt][0pt]{\small$\rightharpoonup$}}{#1}}  %矢量符号
\renewcommand{\vec}[1]{\boldsymbol{#1}}
\newcommand{\me}{\mathrm{e}}
\newcommand{\mi}{\mathrm{i}}
\newcommand{\dif}{\mathrm{d}}
\newcommand{\tabincell}[2]{\begin{tabular}{@{}#1@{}}#2\end{tabular}}

\def\kpc{{\rm kpc}}
\def\km{{\rm km}}
\def\cm{{\rm cm}}
\def\TeV{{\rm TeV}}
\def\GeV{{\rm GeV}}
\def\MeV{{\rm MeV}}
\def\GV{{\rm GV}}
\def\MV{{\rm MV}}
\def\yr{{\rm yr}}
\def\s{{\rm s}}
\def\ns{{\rm ns}}
\def\GHz{{\rm GHz}}
\def\muGs{{\rm \mu Gs}}
\def\arcsec{{\rm arcsec}}
\def\K{{\rm K}}
\def\microK{\mu{\rm K}}
\def\sr{{\rm sr}}
\newcolumntype{p}{D{,}{\pm}{-1}}

\renewcommand{\figurename}{Fig.}
\renewcommand{\tablename}{Tab.}

\renewcommand{\arraystretch}{1.5}

\setlength{\parindent}{0pt}  %取消每段开头的空格

\title{Photoelectric absorption}
\author{}
\date{\today}
\begin{document}

\maketitle

The three main processes involved in the interaction of high energy photons with atoms, nuclei and electrons are \textcolor{magenta}{photoelectric absorption}, \textcolor{magenta}{Compton scattering} and \textcolor{magenta}{electron–positron pair production}. These processes are important not only in the study of high energy astrophysical phenomena in a wide variety of different circumstances but also in the detection of high energy particles and photons. For example, \textcolor{magenta}{photoelectric absorption} is observed in the \textcolor{red}{spectra of most X-ray sources at energies $\epsilon \lesssim 1$ keV}. \textcolor{magenta}{Thomson} and \textcolor{magenta}{Compton scattering} appear in a myriad of guises from the processes occurring in \textcolor{red}{stellar interiors}, to the \textcolor{red}{spectra of binary X-ray sources}, and \textcolor{magenta}{inverse Compton scattering} figures prominently in sources in which there are \textcolor{red}{intense radiation fields and high energy electrons}. \textcolor{magenta}{Pair production} is bound to occur wherever there are significant fluxes of \textcolor{red}{high energy $\gamma$-rays} - evidence for the production of positrons by this process is provided by the \textcolor{red}{detection of the $511$ keV electron–positron annihilation line in our own Galaxy}.


At \textcolor{red}{low photon energies}, \textcolor{red}{$\hbar\omega \ll m_{\rm e}c^2$}, the dominant process by which photons interact with matter is \textcolor{red}{photoelectric}, or \textcolor{red}{bound–free}, \textcolor{red}{absorption} and is one of the principal sources of \textcolor{red}{opacity in stellar interiors}.

If the \textcolor{red}{energies of the incident photons $\epsilon = \hbar\omega$} are \textcolor{red}{greater} than the \textcolor{red}{energy of the X-ray atomic energy level $E_{\rm I}$}, \textcolor{red}{an electron can be ejected from that level}, the \textcolor{red}{remaining energy ($\hbar\omega - E_{\rm I}$) being carried away as the kinetic energy of the ejected electron}, the \textcolor{red}{photoelectric effect}. The photon energy at which $\hbar\omega = E_{\rm I}$ corresponds to an \textcolor{red}{absorption edge} in the spectrum of the radiation because ejection of electrons from this energy level is impossible if the photons are of lower energy. For photons with higher energies, the \textcolor{red}{cross section for photoelectric absorption} from this level \textcolor{red}{decreases as roughly $\omega^{-3}$}.

The analytic solution for the absorption cross section for photons with energies \textcolor{red}{$\hbar\omega \gg E_{\rm I}$ and $\hbar\omega \ll m_{\rm e}c^2$} due to the ejection of electrons from the \textcolor{red}{K-shells} of atoms, that is, from the \textcolor{red}{$1$s level}, is
\begin{equation}
\sigma_{\rm K} = 4\sqrt{2} \sigma_{\rm T} \alpha^4 Z^5 \left(\frac{m_{\rm e} c^2}{\hbar \omega} \right)^{7/2} = \frac{e^{12} m_{\rm e}^{3/2} Z^5}{192\sqrt{2} \pi^5 \epsilon_0^6 \hbar^4 c} \left(\frac{1}{\hbar \omega} \right)^{7/2}
\end{equation}
where $\alpha = e^2/4\pi\epsilon_0\hbar c$ is the fine structure constant and $\sigma_{\rm T} = 8\pi r_{\rm e}^2/3 = e^2/6\pi\epsilon_0^2m^2_{\rm e}c^4$ the Thomson cross-section. This cross-section takes account of the fact that there are $2$ K-shell electrons in all elements except hydrogen, both $1$s electrons contributing to the opacity of the material. The absorption cross section has a strong dependence upon the atomic number $Z$ and so, although heavy elements are very much less abundant than hydrogen, the combination of the $\omega^{-3}$ dependence and the fifth-power dependence upon Z means that quite rare elements can make significant contributions to the total absorption cross-section at ultraviolet and X-ray energies. 

The total absorption coefficient for X-rays, weighted by the cosmic abundance of the different elements,
\begin{equation}
\sigma_{\rm e}(\epsilon) = \frac{1}{n_{\rm H}} \sum_i n_i \sigma_i (\epsilon)
\end{equation}
The K-edges, corresponding to the ejection of electrons from the $1$s shell of the atom or ion, provide the dominant source of opacity. In low resolution X-ray spectral studies, these edges cannot be resolved individually as distinct features and a useful linear interpolation formula for the X-ray absorption coefficient, $\sigma_{\rm e}$, and the corresponding optical depth, $\tau_{\rm e}$ is
\begin{equation}
\tau_{\rm e}(\hbar \omega) = \int \sigma_{\rm e} N_{\rm H} \dif l = 2\times 10^{-26} \left( \frac{\hbar \omega}{1 \rm keV} \right)^{-8/3} \int N_{\rm H} \dif l
\end{equation}
where the \textcolor{red}{column depth} $\int N_{\rm H} \dif l$ is expressed in \textcolor{red}{particles per square metre} and $N_{\rm H}$ is the number density of hydrogen atoms in particles \textcolor{red}{per cubic metre}. For example, if the interstellar gas density were $10^6$ hydrogen atoms m$^{-3}$, the optical depth of the medium is roughly unity for a path length of $1$ kpc at $1$ keV. Thus, the spectra of many X-ray sources turn over at about $1$ keV because of interstellar photoelectric absorption. Because of the steep energy dependence of $\tau_{\rm e}$, photoelectric absorption is only important at energies \textcolor{red}{$\hbar\omega \gg 1$ keV} for sources with \textcolor{red}{large column densities of matter between the source and the observer}.

%%%%%%%%%%%%%%%%%%%%%%%%%%%%%%%%%%%%%%%%%%%%%%%%%%%%%%%%%%%%%%%%%%%%%%
\bibliographystyle{unsrt_update}
\bibliography{ref}
%%%%%%%%%%%%%%%%%%%%%%%%%%%%%%%%%%%%%%%%%%%%%%%%%%%%%%%%%%%%%%%%%%%%%%

\end{document}