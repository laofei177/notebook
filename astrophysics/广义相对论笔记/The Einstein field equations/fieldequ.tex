\documentclass[12pt,a4paper]{article}
%\usepackage{fontspec, xunicode, xltxtra}  
%\setmainfont{Hiragino Sans GB}  
%\usepackage{xeCJK}
%\setCJKmainfont[BoldFont=STZhongsong, ItalicFont=STKaiti]{STSong}
%\setCJKsansfont[BoldFont=STHeiti]{STXihei}
%\setCJKmonofont{STFangsong}

%使用Xelatex编译

% 设置页面
%==================================================
\linespread{2} %行距
% \usepackage[top=1in,bottom=1in,left=1.25in,right=1.25in]{geometry}
% \headsep=2cm
% \textwidth=16cm \textheight=24.2cm
%==================================================

% 其它需要使用的宏包
%==================================================
\usepackage[colorlinks,linkcolor=blue,anchorcolor=red,citecolor=green,urlcolor=blue]{hyperref} 
\usepackage{tabularx}
\usepackage{authblk}         % 作者信息
\usepackage{algorithm}     % 算法排版
\usepackage{amsmath}     % 数学符号与公式
\usepackage{amsfonts}     % 数学符号与字体
\usepackage{mathrsfs}      % 花体
\usepackage[framemethod=TikZ]{mdframed}

\usepackage{graphicx} 
\usepackage{graphics}
\usepackage{color}
\usepackage{xcolor}
\usepackage{tcolorbox}
\usepackage{lipsum}
\usepackage{empheq}

\usepackage{fancyhdr}       % 设置页眉页脚
\usepackage{fancyvrb}       % 抄录环境
\usepackage{float}              % 管理浮动体
\usepackage{geometry}     % 定制页面格式
\usepackage{hyperref}       % 为PDF文档创建超链接
\usepackage{lineno}          % 生成行号
\usepackage{listings}        % 插入程序源代码
\usepackage{multicol}       % 多栏排版
%\usepackage{natbib}         % 管理文献引用
\usepackage{rotating}       % 旋转文字,图形,表格
\usepackage{subfigure}    % 排版子图形
\usepackage{titlesec}       % 改变章节标题格式
\usepackage{moresize}   % 更多字体大小
\usepackage{anysize}
\usepackage{indentfirst}  % 首段缩进
\usepackage{booktabs}   % 使用\multicolumn
\usepackage{multirow}    % 使用\multirow

\usepackage{wrapfig}
\usepackage{titlesec}     % 改变标题样式
\usepackage{enumitem}
\usepackage{aas_macros}

\newcommand{\myvec}[1]%
   {\stackrel{\raisebox{-2pt}[0pt][0pt]{\small$\rightharpoonup$}}{#1}}  %矢量符号
\renewcommand{\vec}[1]{\boldsymbol{#1}}
\newcommand{\me}{\mathrm{e}}
\newcommand{\mi}{\mathrm{i}}
\newcommand{\dif}{\mathrm{d}}
\newcommand{\tabincell}[2]{\begin{tabular}{@{}#1@{}}#2\end{tabular}}

\def\kpc{{\rm kpc}}
\def\km{{\rm km}}
\def\cm{{\rm cm}}
\def\TeV{{\rm TeV}}
\def\GeV{{\rm GeV}}
\def\MeV{{\rm MeV}}
\def\GV{{\rm GV}}
\def\MV{{\rm MV}}
\def\yr{{\rm yr}}
\def\s{{\rm s}}
\def\ns{{\rm ns}}
\def\GHz{{\rm GHz}}
\def\muGs{{\rm \mu Gs}}
\def\arcsec{{\rm arcsec}}
\def\K{{\rm K}}
\def\microK{\mu{\rm K}}
\def\sr{{\rm sr}}
\newcolumntype{p}{D{,}{\pm}{-1}}

\renewcommand{\figurename}{Fig.}
\renewcommand{\tablename}{Tab.}

\renewcommand{\arraystretch}{1.5}

\setlength{\parindent}{0pt}  %取消每段开头的空格

\newcounter{theo}[section]\setcounter{theo}{0}
\renewcommand{\thetheo}{\arabic{section}.\arabic{theo}}
\newenvironment{theo}[2][]{%
\refstepcounter{theo}%
\ifstrempty{#1}%
{\mdfsetup{%
frametitle={%
\tikz[baseline=(current bounding box.east),outer sep=0pt]
\node[anchor=east,rectangle,fill=blue!20]
{\strut Theorem~\thetheo};}}
}%
{\mdfsetup{%
frametitle={%
\tikz[baseline=(current bounding box.east),outer sep=0pt]
\node[anchor=east,rectangle,fill=blue!20]
{\strut Theorem~\thetheo:~#1};}}%
}%
\mdfsetup{innertopmargin=10pt,linecolor=blue!20,%
linewidth=2pt,topline=true,%
frametitleaboveskip=\dimexpr-\ht\strutbox\relax
}
\begin{mdframed}[]\relax%
\label{#2}}{\end{mdframed}}

\newcommand*\widefbox[1]{\fbox{\hspace{2em}#1\hspace{2em}}}


\title{The Einstein field equations}
\author{}
\date{\today}
\begin{document}

\maketitle
\section{Purpose and justification of the field equations}
\cite{2009fcgr.book.....S} The Newtonian equation is
\begin{equation}
\nabla^2 \phi = 4\pi G\rho ~,
\label{Newtonian}
\end{equation}
where $\rho$ is the density of mass. Its solution for a point particle of mass $m$ is
\begin{equation}
\phi = -\frac{Gm}{r} ~,
\end{equation}
which is dimensionless in units where $c = 1$.

$\rho$ is the energy density as measured by only one observer, the MCRF (\textcolor{red}{momentarily comoving reference frame}). Other observers measure the energy density to be the component $T^{00}$ in their own reference frames. If using $\rho$ as the source of the field, then it means one class of observers is preferred above all others, namely those for whom $\rho$ is the energy density. However, \textcolor{orange}{all coordinate systems on an equal footing}. An invariant theory can avoid introducing preferred coordinate systems by using the whole of the stress-energy tensor $\textbf{T}$ as the source of the gravitational field. The generalization of Eq. (\ref{Newtonian}) to relativity would then have 
\begin{equation}
\textbf{O}(\textbf{g}) = k \textbf{T} ~,
\end{equation}
where $k$ is a constant and $\textbf{O}$ is a differential operator on the metric tensor $\textbf{g}$.


$\{\textbf{O}^{\alpha\beta} \}$ must be the components of a $\left(\begin{smallmatrix} 2\\0 \end{smallmatrix} \right)$ tensor and must be combinations of $g_{\mu\nu,\lambda \sigma}$, $g_{\mu\nu,\lambda}$, and $g_{\mu\nu}$. Ricci tensor $R^{\alpha\beta}$ satisfies these conditions. Any tensor of the form
\begin{equation}
O^{\alpha\beta} = R^{\alpha\beta} +\mu g^{\alpha\beta} R +\Lambda g^{\alpha\beta} ~,
\end{equation}
satisfies these conditions, if $\mu$ and $\Lambda$ are constants. $\mu$ can be determined by the \textcolor{blue}{Einstein equivalence principle} which demands \textcolor{blue}{local conservation of energy and momentum}
\begin{equation}
T^{\alpha\beta}{}_{;\beta} = 0 ~.
\end{equation}
This equation must be true for any metric tensor, then
\begin{equation}
O^{\alpha\beta}{}_{;\beta} = 0 ~.
\end{equation}
Since $g^{\alpha\beta}{}_{;\mu} = 0$, 
\begin{equation}
(R^{\alpha\beta} +\mu g^{\alpha\beta}R)_{;\beta} = 0 ~.
\end{equation}
$\mu = -\dfrac{1}{2}$ if it is to be an identity for arbitrary $g_{\alpha\beta}$. 
\begin{align}
& G^{\alpha\beta} +\Lambda g^{\alpha\beta}  = k T^{\alpha\beta} ~, \\
& \textbf{G} +\Lambda \textbf{g} = k \textbf{T}
\end{align}
which are called the field equations of GR, or Einstein's field equations. Constant $k$ can be determined by demanding that Newton's gravitational field equation comes out right, but that $\Lambda$ remains arbitrary.

\subsection{Geometrized units}
$c = G = 1$ and $1 = G/c^2 = 7.425 \times 10^{-28}$ m kg$^{-1}$.

\section{Einstein’s equations}
\begin{equation}
G^{\alpha\beta}  = 8\pi T^{\alpha\beta} ~,
\end{equation}
where $k = 8\pi$ and constant $\Lambda$ is called the cosmological constant. The value of $k$ is obtained by demanding that Einstein's equations predict the correct behavior of planets in the solar system. Since $\{g_{\alpha\beta}\}$ are the components of a tensor in some coordinate system, a change in coordinates induces a change in them. There are four coordinates, so there are four arbitrary functional degrees of freedom among the ten $g_{\alpha\beta}$. It should be impossible, therefore, to determine all ten $g_{\alpha\beta}$ from any initial data, since the coordinates to the future of the initial moment can be changed arbitrarily. Einstein’s equations have exactly this property: the \textcolor{red}{Bianchi identities}
\begin{equation}
G^{\alpha\beta}{}_{;\beta} = 0
\end{equation}
mean that there are four differential identities (one for each value of $\alpha$ above) among the ten $G^{\alpha\beta}$. These ten are not independent, and the ten Einstein equations are really only six independent differential equations for the six functions among the ten $g_{\alpha\beta}$ that characterize the geometry independently of the coordinates.

\section{Einstein's equations for weak gravitational fields}






%%%%%%%%%%%%%%%%%%%%%%%%%%%%%%%%%%%%%%%%%%%%%%%%%%%%%%%%%%%%%%%%%%%%%%
\bibliographystyle{unsrt_update}
\bibliography{ref}
%%%%%%%%%%%%%%%%%%%%%%%%%%%%%%%%%%%%%%%%%%%%%%%%%%%%%%%%%%%%%%%%%%%%%%

\end{document}